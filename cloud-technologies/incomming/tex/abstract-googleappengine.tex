\section{Google App Engine}
\index{Google App Engine}

%%%%%%%%%%%%%%%%%%%%%%%%%%%%%%%%%%%%%%%%%%%%%%%%%%%%%%%%%%%%%%%%%%%%%%%%%%%%%%%%%%%%
Google App Engine, generally called App Engine is a Platform as a service cloud 
solution (PaaS). It lets you build and run applications on Google's cloud 
infrastructure. In this platform the developer does not have to worry about 
infrastructure such as database administration, server configurations and 
load balancing which is done by google. Developers only job is to develop 
source codes.  It claims to be highly scalable as it can automatically 
increases capacity depending upon the
workloads~\cite{hid-sp18-415-www-scalabitity}.
 
Applications in App Engine can be run in either Flexible or Standard Environment or 
both can be used at the same time~\cite{hid-sp18-415-www-cloud-goggle}. 
Automatic scaling of apps, user customization of runtime (Eclipse Jetty 9, 
Python 2.7 and Python 3.6, Node.js, Ruby, PHP, \verb|.NET| core, and Go), operating 
system and even CPU memory  are some of the features of App Engine Flexible 
environment~\cite{hid-sp18-415-www-cloud-goggle}. Applications in Flexible 
environment run in Docker containers on Google Compute virtual machines. While 
in App Engine Standard Environment application instances are run in sandbox with 
prespecified runtime environment of supported language (Python 2.7, Java 7, 
Java 8, PHP 5.5 and GO 1.8,  1.6)~\cite{hid-sp18-415-app-engine}. 
This means if source code uses Python then its instances are run in Python runtime. 
