\section{Apache Geode}
\index{Geode}

Apache Geode is an in-memory distributed 
data management platform that provides 
real-time, consistent access to data-intensive
application through extensively distributed cloud
architectures and supports high availability and
disaster recovery in case of any node failures
~\cite{hid-sp18-514-apachegeodewiki}. 
Apache Geode initially developed by GemStone Systems
and later this framework has been renamed as GemFire.

Gemfire was first installed in 
the financial sector as the transactional,
low-latency data engine used in Wall Street 
tradingplatforms~\cite{hid-sp18-514-apachegeodewiki}.
Distributed cache servers are generalization
that define the nodes. In each cache we 
define regions, regions are 
equivalent to tables in any relational 
databases or XSD schema structure 
and manage data in the distributed environment. 
For high availability the data is replicated 
to multiple regions (same data is available 
on each cache servers) by which it ensures 
high availability as one member goes down still copy
is available on other cache member. 
Locater's responsibility to determine 
and load balance client (MapReduce, JTA, 
spring, REST service call, or API) requests 
to be processed by available cache servers. 
Locators get notifications continuously 
if there is any issue in the cluster members, 
based on this the client request will be navigated
appropriately~\cite{hid-sp18-514-apachegeodewiki}.
The main features of this framework are high performance, 
scalability, fault-tolerance for any data grid platform and 
can be integrated to other open sources
technologies - Spring Data
Gemfire~\cite{hid-sp18-514-geodespringgemfire}, Spring
Cache~\cite{hid-sp18-514-geodespringcache}, and
Python~\cite{hid-sp18-514-geodepython}.