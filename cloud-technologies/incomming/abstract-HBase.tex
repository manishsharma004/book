\section{HBase}
\index{HBase}

Apache HBase\cite{hid-sp18-515-www-hbase}is a distributed, scalable, 
big data store, Hadoop database. 
You can use Apache HBase when you need random, realtime read/write 
access to your Big Data.
The goal of HBase is hosting of very large tables -- billions of rows 
X millions of columns -- atop clusters of commodity hardware.
Apache HBase is modeled after Google's Bigtable, which is a Distributed 
Storage System for Structured Data.
HBase is an open source, non-relational distributed database that 
mimics Google's Bigtable and is written in Java. It was developed as
part of the Apache Software Foundation's Apache Hadoop project and 
runs on top of HDFS (Hadoop Distributed File System), providing 
Hadoop with similar Bigtable functionality. 
HBase provides compression, in-memory operations, and Bloom filters 
for each column listed in the original Bigtable file.
HBase does not directly replace the traditional SQL database, but 
the Apache Phoenix project provides SQL layers for HBase and JDBC 
drivers that can be integrated with a variety of analytics and 
business intelligence applications.
