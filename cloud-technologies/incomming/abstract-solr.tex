\section{Apache Solr}
\index{Lucene}
\index{search platform}
\index{restful}

Apache Solr is an open-source search platform used
to build search applications. It leverages the power of Apache
Lucene~\cite{hid-sp18-516-www-wiki-lucene}, which is a Java-based 
search library providing the core operations required by any search 
application like Indexing and 
Searching~\cite{hid-sp18-516-www-tutorialspoint-solr}. Apache Lucene and Apache
Solr were merged together in 2010 and since then, produced by the Apache
Software Foundation development team. It has an active development community 
and regular releases. Solr has RESTful API's like 
HTTP/XML~\cite{hid-sp18-516-www-wiki-xml} or 
JSON~\cite{hid-sp18-516-www-wiki-json} to communicate with it that can be used 
from most popular programming languages~\cite{hid-sp18-516-www-wiki-solr}. 
It has all capabilities required for a full-text search server such as tokens, 
spell check, wildcard, phrases and auto-complete. It is fast, highly scalable, 
reliable, fault-tolerant and enterprise-ready and can be deployed in any kind 
of systems such as standalone, distributed or 
cloud~\cite{hid-sp18-516-www-tutorialspoint-solr}. Other major features include 
hit highlighting, built-in security, distributed search through sharding, 
database integration, faceted search, rich document (e.g., Word, PDF)
handling~\cite{hid-sp18-516-www-wiki-solr}. As Hadoop can handle large amounts
of data, Solr can be used with Hadoop~\cite{hid-sp18-516-www-wiki-hadoop} to 
find the required information from a large source. Apart from search, Solr also 
has NoSQL~\cite{hid-sp18-516-www-wiki-nosql} features and it can be used
as a non-relational data storage and processing technology. The components of
Solr can be customized easily by extending and configuring its Java classes
thereby making it flexible and 
extensible~\cite{hid-sp18-516-www-tutorialspoint-solr}. Solr provides navigation
features to world's largest internet sites like Netflix, Instagram, Best Buy,
eBay etc.~\cite{hid-sp18-516-www-apacheorg-solr}. It is packaged as the built-in
search in many applications such as content management systems and enterprise
content management systems~\cite{hid-sp18-516-www-wiki-solr}.
