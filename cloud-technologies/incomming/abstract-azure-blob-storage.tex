\section{Azure Blob Storage}

Microsoft Azure BLOB storage service can be used to store and retrieve Binary
Large Objects (BLOBs), or what are more commonly known as files~\cite{hid-sp18-501-RedGate}


This service can be used to share files with clients and to off-load some of the
static content from web servers to reduce the load on them. Azure BLOB storage
also provides persistent storage. By using this service, developers get
dedicated virtual machines to run code without having to worry about managing
those virtual machines. Azure BLOB Storage can store any type of file, such as
Image files, database files, text files, or virtual hard drive files. However,
when they are uploaded to the service they are stored as either a Page BLOB or a
Block BLOB depending on how one plans on using the file or the size of the file.
Page BLOBs are optimized for random reads and writes so they are most commonly
used when storing virtual hard drive files for virtual machines. Each Page BLOB
is made up of one or more 512-byte pages of data, up to a total size limit of 1
TB per file. The majority of files would benefit from being stored as Block
BLOBs, which are written to the storage account as a series of blocks and then
committed into a single file. One can create a large file by breaking it into
blocks, which can be uploaded concurrently and then then committed together into
a single file in one operation. This allows faster upload times and better
throughput. The client storage libraries manage this process by uploading files
of less than 64 MB in size in a single operation, and uploading larger files
across multiple operations by breaking down the files and running the concurrent
uploads. A Block BLOB has a maximum size of 200 GB~\cite{hid-sp18-501-RedGate}

