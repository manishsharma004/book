\section{Exercise}

\begin{description}

\item[TechList.1:] In class you will be given an HID and you will be assigned
  a number of technologies that you need to research and create a
  summary as well as one or more relevant references to be added to the
  Web page. All technologies for TechList.1 are marked with a (1)
  behind the technology.  An example text is given for Nagios in this
  page.  Please create a pull request with your responses. You are
  responsible for making sure the request shows up and each commit is
  using gitchangelog in the commit message::

    new:usr: added paragraph about <PUTTECHHERE>

  You can create one or more pull requests for the technology and the
  references. We have created in the referens file a placeholder using
  your HID to simplify the management of the references while avoiding
  conflicts.  For the technologies you are responsible to investigate
  them and write an academic summary of the technology. Make sure to
  add your reference to refs.bib.  Many technologies may have
  additional references than the Web page. Please add the most
  important once while limiting it to three if you can. Avoid
  plagiarism and use proper quotations or better rewrite the text.

  You must look at~\ref{technologies-hw} to successfully complete the
  homework

  A video about this homework is posted at
  \url{https://www.youtube.com/watch?v=roi7vezNmfo} showing how to
  do references in emacs and jabref, it shows you how to configure
  git, it shows you how to do the fork request while asking you to add
  ``new:usr ....'' to the commit messages. As this is a homework
  related video we put a lot of information in it that is not only
  useful for beginners. We recommend you watch it.


  This homework can be done in steps. First you can collect all the
  content in an editor. Second you can create a fork. Third you can
  add the new content to the fork. Fourth you can commit. Fith you
  can push. Six if the TAs have commend improve. The commit message
  must have new:usr: at the beginning.

  While the Nagios entry is a good example (make sure grammer is ok
  the Google app engine is an example for a bad entry.

  Do Techlist 1.a 1.b 1.c first. We  will assign Techlist 1.d and
  TechList 2 in February.

\item[TechList.1.a:]  Complete the pull request with the technologies assigned to you.
  Details for the assignment are posted in Piazza. Search for TechList.

\item[TechList.1.b:] Identify how to cite. We are using \textit{scientific} citation
  formats such as IEEEtran, and ACM. We are {\bf not} using citation
  formats such as Chicago, MLA, or ALP. The later are all for non
  scientific publications and thus of no use to us. Also when writing
  about a technology do not use the names of the person, simply say
  something like. In [1] the definition of a turing machine is given
  as follows, ...  and do not use elaborate sentences such as: In his
  groundbraking work conducted in England, Allan Turing, introduced
  the turing machine in the years 1936-37 [2]. Its definition is base
  on ... The difference is clear, while the first focusses on results
  and technological concepts, the second introduces a colorful
  description that is more suitable for a magazine or a computer
  history paper.

\item[TechList 1.c:]
  Learn about plagiarism and how to avoid it.
  Many Web pages will conduct self advertisement while adding
  suspicious and subjective adjectives or phrases such as cheaper,
  superior, best, most important, with no equal, and others that you
  may not want to copy into your descriptions. Please focus on facts,
  not on what the author of the Web page claims.

\item[TechList 1.d:]
  Identify technologies from the Apache Project
  \url{https://projects.apache.org/} or other Big Data related Web pages
  and projects that are not yet listed here. Add them at the end of
  the Technologies page under the :ref:`New Technologies <new-techs>`
  section, together with a description and appropriate references just
  like you did for your list of technologies in TechList 1a-1c. As
  part of your paragraph, please suggest a section where you think is
  best to add the technologies. Once the new technologies have been
  submitted, the AIs will integrate them in the appropriate
  sections. Please, only add new techs to the last section, otherwise
  it will be easy to introduce conflicts in the file.

\item[TechList.2:]
  In this hopweork we provide you with additional technologies that
  you need to complete. They are marked with (2) in the :doc:`HID
  Assignment page <hids-techs>`.

\item[TechList.3:]
  Identify technologies that are not listed here and add   them.
  Provide a description and a reference just as you did before. Before you
  add a technology, verify that it is not on the {\bf new technologies} list
  already.  Duplicated entries will be merged.

\item[TechList.4:]
  For useful information on how to correctly create BibTeX entries,
  see and contribute to :ref:`these open discussion threads Piazza
  <bibtex-discussions>`. 

\end{description}
