\section{REST with Eve}\label{rest-with-eve}

\subsection{Overview of REST}\label{overview-of-rest}

REST stands for REpresentational State Transfer. REST is an architecture
style for designing networked applications. It is based on stateless,
client-server, cacheable communications protocol. Although not based on
http, in most cases, the HTTP protocol is used. In contrast to what some
others write or say, REST is not a \emph{standard}.

RESTful applications use HTTP requests to:

\begin{itemize}
\item
  post data: while creating and/or updating it,
\item
  read data: while making queries, and
\item
  delete data.
\end{itemize}

Hence REST uses HTTP for the four CRUD operations:

\begin{itemize}
\item
  Create
\item
  Read
\item
  Update
\item
  Delete
\end{itemize}

As part of the HTTP protocol we have methods such as GET, PUT, POST, and
DELETE. These methods can than be used to implement a REST service. As
REST introduces collections and items we need to implement the CRUD
functions for them. The semantics is explained in the Table
illustrationg how to implement them with HTTP methods.

Source:
\url{https://en.wikipedia.org/wiki/Representational_state_transfer}

\subsection{REST and eve}\label{rest-and-eve}

Now that we have outlined the basic functionality that we need, we lke
to introduce you to Eve that makes this process rather trivial. We will
provide you with an implementation example that showcases that we can
create REST services without writing a single line of code. The code for
this is located at \url{https://github.com/cloudmesh/rest}

This code will have a master branch but will also have a dev branch in
which we will add gradually more objects. Objects in the dev branch will
include:

\begin{itemize}
\tightlist
\item
  virtual directories
\item
  virtual clusters
\item
  job sequences
\item
  inventories
\end{itemize}

;You may want to check our active development work in the dev branch.
However for the purpose of this class the master branch will be
sufficient.

\subsubsection{Installation}\label{installation}

First we havt to install mongodb. The instalation will depend on your
operating system. For the use of the rest service it is not important to
integrate mongodb into the system upon reboot, which is focus of many
online documents. However, for us it is better if we can start and stop
the services explicitly for now.

On ubuntu, you need to do the following steps:

\begin{verbatim}
TO BE CONTRIBUTED BY THE STUDENTS OF THE CLASS as homework
\end{verbatim}

On windows 10, you need to do the following steps:

\begin{verbatim}
TO BE CONTRIBUTED BY THE STUDENTS OF THE CLASS as homework, if you
elect Windows 10. YOu could be using the online documentation
provided by starting it on Windows, or rinning it in a docker container.
\end{verbatim}

On OSX you can use homebrew and install it with:

\begin{verbatim}
brew update
brew install mongodb
\end{verbatim}

\begin{description}
\item[In future we may want to add ssl authentication in which case you
may]
need to install it as follows:
\end{description}

brew install mongodb --with-openssl

\subsubsection{Starting the service}\label{starting-the-service}

We have provided a convenient Makefile that currently only works for
OSX. It will be easy for you to adapt it to Linux. Certainly you can
look at the targes in the makefile and replicate them one by one.
Improtaht targest are deploy and test.

When using the makefile you can start the services with:

\begin{verbatim}
make deploy
\end{verbatim}

IT will start two terminals. IN one you will see the mongo service, in
the other you will see the eve service. The eve service will take a file
called sample.settings.py that is base on sample.json for the start of
the eve service. The mongo servide is configured in suc a wahy that it
only accepts incimming connections from the local host which will be
suffiicent fpr our case. The mongo data is written into the
\$USER/.cloudmesh directory, so make sure it exists.

To test the services you can say:

\begin{verbatim}
make test
\end{verbatim}

YOu will se a number of json text been written to the screen.

\subsection{Creating your own objects}\label{creating-your-own-objects}

The example demonstrated how easy it is to create a mongodb and an eve
rest service. Now lets use this example to creat your own. FOr this we
have modified a tool called evegenie to install it onto your system.

The original documentation for evegenie is located at:

\begin{itemize}
\tightlist
\item
  \url{http://evegenie.readthedocs.io/en/latest/}
\end{itemize}

However, we have improved evegenie while providing a commandline tool
based on it. The improved code is located at:

\begin{itemize}
\tightlist
\item
  \url{https://github.com/cloudmesh/evegenie}
\end{itemize}

You clone it and install on your system as follows:

\begin{verbatim}
cd ~/github
git clone https://github.com/cloudmesh/evegenie
cd evegenie
python setup.py install
pip install .
\end{verbatim}

This shoudl install in your system evegenie. YOu can verify this by
typing:

\begin{verbatim}
which evegenie
\end{verbatim}

If you see the path evegenie is installed. With evegenie installed its
usaage is simple:

\begin{verbatim}
$ evegenie

Usage:
    evegenie --help
    evegenie FILENAME
\end{verbatim}

It takes a json file as input and writes out a settings file for the use
in eve. Lets assume the file is called sample.json, than the settings
file will be called sample.settings.py. Having the evegenie programm
will allow us to generate the settings files easily. You can include
them into your project and leverage the Makefile targets to start the
services in your project. In case you generate new objects, make sure
you rerun evegenie, kill all previous windows in whcih you run eve and
mongo and restart. In case of changes to objects that you have designed
and run previously, you need to also delete the mongod database.

\subsection{Towards cmd5 extensions to manage eve and
mongo}\label{towards-cmd5-extensions-to-manage-eve-and-mongo}

Naturally it is of advantage to have in cms administration commands to
manage mongo and eve from cmd instead of targets in the Makefile. Hence,
we \textbf{propose} that the class develops such an extension. We will
create in the repository the extension called admin and hobe that
students through collaborative work and pull requests complete such an
admin command.

The proposed command is located at:

  \URL{https://github.com/cloudmesh/rest/blob/master/cloudmesh/ext/command/admin.py}

It will be up to the class to implement such a command. Please
coordinate with each other.

The implementation based on what we provided in the Make file seems
straight forward. A great extensinion is to load the objects definitions
or eve e.g. settings.py not from the class, but forma place in
.cloudmesh. I propose to place the file at:

\begin{verbatim}
.cloudmesh/db/settings.py
\end{verbatim}

the location of this file is used whne the Service class is initialized
with None. Prior to starting the service the file needs to be copied
there. This could be achived with a set commad.
