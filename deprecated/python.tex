\FILENAME

\section{Introduction to Python}\label{introduction-to-python}

Portions of this lesson have been adapted from the
\href{https://docs.python.org/2/tutorial/}{official Python Tutorial}
copyright \href{http://www.python.org/}{Python Software Foundation}.

Python is an easy to learn programming language. It has efficient
high-level data structures and a simple but effective approach to
object-oriented programming. Python's simple syntax and dynamic typing,
together with its interpreted nature, make it an ideal language for
scripting and rapid application development in many areas on most
platforms. The Python interpreter and the extensive standard library are
freely available in source or binary form for all major platforms from
the Python Web site, \url{https://www.python.org/}, and may be freely
distributed. The same site also contains distributions of and pointers
to many free third party Python modules, programs and tools, and
additional documentation. The Python interpreter can be extended with
new functions and data types implemented in C or C++ (or other languages
callable from C). Python is also suitable as an extension language for
customizable applications.

Python is an interpreted, dynamic, high-level programming language
suitable for a wide range of applications.

The philosophy of python is summarized in
\href{https://www.python.org/dev/peps/pep-0020/}{The Zen of Python} as
follows:

\begin{itemize}
\tightlist
\item
  Explicit is better than implicit
\item
  Simple is better than complex
\item
  Complex is better than complicated
\item
  Readability counts
\end{itemize}

The main features of Python are:

\begin{itemize}
\tightlist
\item
  Use of indentation whitespace to indicate blocks
\item
  Object orient paradigm
\item
  Dynamic typing
\item
  Interpreted runtime
\item
  Garbage collected memory management
\item
  a large standard library
\item
  a large repository of third-party libraries
\end{itemize}

Python is used by many companies (such as Google, Yahoo!, CERN, NASA)
and is applied for web development, scientific computing, embedded
applications, artificial intelligence, software development, and
information security, to name a few.

\subsection{About the Tutorial}\label{about-the-tutorial}

This tutorial introduces the reader informally to the basic concepts and
features of the Python language and system. It helps to have a Python
interpreter handy for hands-on experience, but all examples are
self-contained, so the tutorial can be read off-line as well. At the end
of this lesson you will be able to:

\begin{itemize}
\tightlist
\item
  use Python
\item
  use the interactive Python interface
\item
  understand the basic syntax of Python
\item
  write and run Python programs stored in a file
\item
  have an overview of the standard library
\item
  install Python libraries using pyenv or if it is not available
  virtualenv
\end{itemize}

This tutorial does not attempt to be comprehensive and cover every
single feature, or even every commonly used feature. Instead, it
introduces many of Python's most noteworthy features, and will give you
a good idea of the language's flavor and style. After reading it, you
will be able to read and write Python modules and programs, and you will
be ready to learn more about the various Python library modules.

In order to conduct this lesson you need

\begin{itemize}
\tightlist
\item
  A computer with Python 2.7.13 or 3.6.2
\item
  Familiarity with command line usage
\item
  A text editor such as
  \href{https://www.jetbrains.com/pycharm/}{PyCharm}, emacs, vi or
  others. You should identity which works best for you and set it up.
\end{itemize}

\subsection{Links}\label{links}

\begin{itemize}
\tightlist
\item
  \href{https://www.python.org/}{Python}
\item
  \href{https://pip.pypa.io/en/stable/}{Pip}
\item
  \href{https://virtualenv.pypa.io/en/stable/}{Virtualenv}
\item
  \href{http://www.numpy.org/}{NumPy}
\item
  \href{https://scipy.org/}{SciPy}
\item
  \href{http://matplotlib.org/}{Matplotlib}
\item
  \href{http://pandas.pydata.org/}{Pandas}
\item
  \href{https://github.com/pyenv/pyenv}{pyenv}
\item
  \href{https://github.com/pyenv/pyenv}{PyCharm}
\end{itemize}

Python module of the week is a Web site that provides a number of short
examples on how to use some elementary python modules. Not all modules
are equally useful and you should decide if there are better
alternatives. However for beginners this site provides a number of good
examples

\begin{itemize}
\tightlist
\item
  Python 2: \url{https://pymotw.com/2/}
\item
  Python 3: \url{https://pymotw.com/3/}
\end{itemize}

\subsection{Python Installation}\label{python-installation}

Python is easy to install and very good instructions for most platforms
can be found on the python.org Web page. We will be using Python 2.7.13
and/or Python 3 in our activities.

To manage python modules, it is useful to have
\href{https://pypi.python.org/pypi/pip}{pip} package installation tool
on your system.

In the tutorial, we assume that you have a computer with python
installed. However, we also recommend that for the class you use
Python's virtualenv (see below) to isolate your development Python from
the system installed Python.

\subsection{Managing custom Python
installs}\label{managing-custom-python-installs}

Often you have your own computer and you do not like to change its
environment to keep it in pristine condition. Python comes with mnay
libraries that could for example conflict with libraries that you have
installed. To avoid this it is bets to work in an isolated python we can
use tools such as virtualenv, pyenv or pyvenv for 3.6.2. Which you use
depends on you, but we highly recommend pyenv if you can.

\subsubsection{Managing Multiple Python Versions with
Pyenv}\label{managing-multiple-python-versions-with-pyenv}

Python has several versions that are used by the community. This
includes Python 2 and Python 3, but alls different management of the
python libraries. As each OS may have their own version of python
installed. It is not recommended that you modify that version. Instead
you may want to create a localized python installation that you as a
user can modify. To do that we recommend \emph{pyenv}. Pyenv allows
users to switch between multiple versions of Python
(\url{https://github.com/yyuu/pyenv}). To summarize:

\begin{itemize}
\tightlist
\item
  users to change the global Python version on a per-user basis;
\item
  users to enable support for per-project Python versions;
\item
  easy version changes without complex environment variable management;
\item
  to search installed commands across different python versions;
\item
  integrate with tox (\url{https://tox.readthedocs.io/}).
\end{itemize}

\paragraph{Instalation without pyenv}\label{instalation-without-pyenv}

If you need to have more than one python version installed and do not
want or can use pyenv, we recommend you download and install python
2.7.13 and 3.6.2 from python.org
(\url{https://www.python.org/downloads/})

\paragraph{Disabeling wrong python installs on
OSX}\label{disabeling-wrong-python-installs-on-osx}

While working with students we have seen at times that they take other
classes either at universities or online that teach them how to program
in python. Unfortuanatley, although they seem to do that they often
ignore to teach you how to properly install python. I just reachentl had
a students that had installed python 7 times on his OSX machine, while
another student had 3 different instalations, all of which confliced
with each other as they were not set up properly.

We recommend that you inspect if you have a files such as
\textasciitilde{}/.bashrc or \textasciitilde{}/.bashrc\_profile in your
ehome directory and identify if it activates various versions of python
on your computer. If so you could try to deactivate them while
outcommenting the various versions with the \# character at the
beginning of the line, start a new terminal and see if the terminal
shell still works. Than you can follow our instructions here while using
an install on pyenv.

\paragraph{Install pyenv on OSX from
git}\label{install-pyenv-on-osx-from-git}

This is our recommended way to install pyenv on OSX:

\begin{verbatim}
$ git clone https://github.com/pyenv/pyenv.git ~/.pyenv
$ git clone https://github.com/pyenv/pyenv-virtualenv.git ~/.pyenv/plugins/pyenv-virtualenv
$ git clone https://github.com/yyuu/pyenv-virtualenvwrapper.git ~/.pyenv/plugins/pyenv-virtualenvwrapper
$ echo 'export PYENV_ROOT="$HOME/.pyenv"' >> ~/.bash_profile
$ echo 'export PATH="$PYENV_ROOT/bin:$PATH"' >> ~/.bash_profile
\end{verbatim}

\paragraph{Instalation of Homebrew}\label{instalation-of-homebrew}

Before installing anything on your computer make sure you have enough
space. Use in the terminal the command:

\begin{verbatim}
$ df -h
\end{verbatim}

which gives your an overview of your file system. If you do not have
enough space, please make sure you free up unused files from your drive.

In many occasions it is beneficial to use readline as it provides nice
editing features for the terminal and xz for completion. First, make
sure you have xcode installed:

\begin{verbatim}
$ xcode-select --install
\end{verbatim}

Next install homebrew, pyenv, pyenv-virtualenv and pyenv-virtualwrapper.
Additionally install readline and some compression tools:

\begin{verbatim}
/usr/bin/ruby -e "$(curl -fsSL https://raw.githubusercontent.com/Homebrew/install/master/install)"
brew update
brew install readline xz
\end{verbatim}

\paragraph{Install pyenv on OSX with
Homebrew}\label{install-pyenv-on-osx-with-homebrew}

We describe here a mechanism of installing pyenv with homebrew. Other
mechanisms can be found on the pyenv documentation page
(\url{https://github.com/yyuu/pyenv-installer}). You must have homebrew
installed as discussed in the previous section.

To install pyenv with homebrew execute in the terminal:

\begin{verbatim}
brew install pyenv pyenv-virtualenv pyenv-virtualenvwrapper
\end{verbatim}

\paragraph{Install pyenv on Ubuntu}\label{install-pyenv-on-ubuntu}

The following steps will install pyenv in a new ubuntu 16.04
distribution.

Start up a terminal and execute in the terminal the following commands.
We recommend that you do it one command at a time so you can observe if
the command succeeds:

\begin{verbatim}
$ sudo apt-get update
$ sudo apt-get install git python-pip make build-essential libssl-dev
$ sudo apt-get install zlib1g-dev libbz2-dev libreadline-dev libsqlite3-dev
$ sudo pip install virtualenvwrapper

$ git clone https://github.com/yyuu/pyenv.git ~/.pyenv
$ git clone https://github.com/pyenv/pyenv-virtualenv.git ~/.pyenv/plugins/pyenv-virtualenv   
$ git clone https://github.com/yyuu/pyenv-virtualenvwrapper.git ~/.pyenv/plugins/pyenv-virtualenvwrapper

$ echo 'export PYENV_ROOT="$HOME/.pyenv"' >> ~/.bashrc
$ echo 'export PATH="$PYENV_ROOT/bin:$PATH"' >> ~/.bashrc
\end{verbatim}

Now that you have installed pyenv it is not yet activated in your
current terminal. The easiest thing to do is to start a new terminal and
typ in:

\begin{verbatim}
which pyenv
\end{verbatim}

If you see a response pyenv is installed and you can proceed with the
next steps.

\begin{description}
\item[Please remember whenever you modify .bashrc or]
.bash\_profile you need to start a new terminal.
\end{description}

\paragraph{Install Different Python
Versions}\label{install-different-python-versions}

Pyenv provides a large list of different python versions. To see the
entire list please use the command:

\begin{verbatim}
$ pyenv install -l
\end{verbatim}

However, for us we only need to worry about python 2.7.13 and python
3.6.2 (once 3.6.2 becomes available we will use that). You can now
install different versions of python into your local environment with
the following commands:

\begin{verbatim}
$ pyenv install 2.7.13
$ pyenv install 3.6.2
\end{verbatim}

You can set the global python default version with:

\begin{verbatim}
$ pyenv global 2.7.13
\end{verbatim}

Type the following to determine which version you activated:

\begin{verbatim}
$ pyenv version
\end{verbatim}

Type the following to determine which versions you have available:

\begin{verbatim}
$ pyenv versions
\end{verbatim}

Associate a specific environment name with a certain python version, use
the following commands:

\begin{verbatim}
$ pyenv virtualenv 2.7.13 ENV2
$ pyenv virtualenv 3.6.2 ENV3
\end{verbatim}

In the example above, ENV2 would represent python 2.7.13 while ENV3
would represent python 3.6.2. Often it is easier to type the alias
rather than the explicit version.

\paragraph{Set up the Shell}\label{set-up-the-shell}

To make all work smoothly from your terminal, you can include the
following in your .bashrc files:

\begin{verbatim}
export PYENV_VIRTUALENV_DISABLE_PROMPT=1
eval "$(pyenv init -)"
eval "$(pyenv virtualenv-init -)"

__pyenv_version_ps1() {
  local ret=$?;
  output=$(pyenv version-name)
  if [[ ! -z $output ]]; then
    echo -n "($output)"
  fi
  return $ret;
}

PS1="\$(__pyenv_version_ps1) ${PS1}"
\end{verbatim}

We recommend that you do this towards the end of your file.

\paragraph{Switching Environments}\label{switching-environments}

After setting up the different environments, switching between them is
now easy. Simply use the following commands:

\begin{verbatim}
(2.7.13) $ pyenv activate ENV2
(ENV2) $ pyenv activate ENV3
(ENV3) $ pyenv activate ENV2
(ENV2) $ pyenv deactivate ENV2
(2.7.13) $ 
\end{verbatim}

To make it even easier, you can add the following lines to your
.bash\_profile file:

\begin{verbatim}
alias ENV2="pyenv activate ENV2"
alias ENV3="pyenv activate ENV3"
\end{verbatim}

If you start a new terminal, you can switch between the different
versions of python simply by typing:

\begin{verbatim}
$ ENV2
$ ENV3
\end{verbatim}

\subsection{Instalation without
pyenv}\label{instalation-without-pyenv-1}

If you need to have more than one python version installed and do not
want or can use pyenv, we recommend you download and install python
2.7.13 and 3.6.2 from python.org
(\url{https://www.python.org/downloads/})

\subsubsection{Make sure pip is up to
date}\label{make-sure-pip-is-up-to-date}

As you will want to install other packages, make sure pip is up to date:

\begin{verbatim}
pip install pip -U
\end{verbatim}

pyenv virtualenv anaconda3-4.3.1 ANA3 pyenv activate ANA3

\subsection{Anaconda and Miniconda}\label{anaconda-and-miniconda}

\begin{description}
\item[We do not recommend that you use anaconda or miniconda as it may]
interfere with your default python interpreters and setup.
\end{description}

Please note that beginners to pyton should always use anaconda or
miniconda only afterthey have installed pyenv and use it. For this class
neither anaconda nor miniconda is required. In fact we do not recommend
it. We keep this section as we know that other classes at IU may use
anaconda. We are not aware if these classes teach you the right way to
install it, with \emph{pyenv}.

\subsubsection{Miniconda}\label{miniconda}

\begin{description}
\item[This section about miniconda is experimental and has not]
been tested. We are looking for contributors that help completing it. If
you use anaconda or miniconda we recommend to manage it via pyenv.
\end{description}

To install mini conda you can use the following commands:

\begin{verbatim}
$ mkdir ana
$ cd ana
$ pyenv install miniconda3-latest
$ pyenv local miniconda3-latest
$ pyenv activate miniconda3-latest
$ conda create -n ana anaconda
\end{verbatim}

To activate use:

\begin{verbatim}
$ source activate ana
\end{verbatim}

To deactivate use:

\begin{verbatim}
$ source deactivate
\end{verbatim}

To install cloudmesh cmd5 please use:

\begin{verbatim}
$ pip install cloudmesh.cmd5
$ pip install cloudmesh.sys
\end{verbatim}

\subsubsection{Anaconda}\label{anaconda}

\begin{description}
\item[This section about anaconda is experimental and has not]
been tested. We are looking for contributors that help completing it.
\end{description}

You can add anaconda to your pyenv with the following commands:

\begin{verbatim}
pyenv install anaconda3-4.3.1
\end{verbatim}

To switch more easily we recommend that you use the following in your
.bash\_profile file:

\begin{verbatim}
alias ANA="pyenv activate anaconda3-4.3.1"
\end{verbatim}

Once you have done this you can easily switch to anaconda with the
command:

\begin{verbatim}
$ ANA
\end{verbatim}

Terminology in annaconda could lead to confusion. Thus we like to point
out that the version number of anaconda is unrelated to the python
version. Furthermore, anaconda uses the term root not for the root user,
but for the originating directory in which the anaconda program is
installed.

In case you like to build your own conda packages at a later time we
recommend that you install the conda-build package:

\begin{verbatim}
$ conda install conda-build
\end{verbatim}

When executing:

\begin{verbatim}
pyenv versions
\end{verbatim}

you will see after the install completed the anaconda versions
installed:

\begin{verbatim}
pyenv versions
system
2.7.13
2.7.13/envs/ENV2
3.6.2
3.6.2/envs/ENV3
ENV2 
ENV3
* anaconda3-4.3.1 (set by PYENV_VERSION environment variable)
\end{verbatim}

Let us now create virtualenv for anaconda:

\begin{verbatim}
$ pyenv virtualenv anaconda3-4.3.1 ANA
\end{verbatim}

To activate it you can now use:

\begin{verbatim}
$ pyenv ANA
\end{verbatim}

However, anaconda may modify your .bashrc or .bash\_profile files and ,
may result in incompatibilities with other python versions. For this
reason we recommend not to use it. If you find ways to get it to work
reliably with other versions, please let us know and we update this
tutorial.

To install cloudmesh cmd5 please use:

\begin{verbatim}
$ pip install cloudmesh.cmd5
$ pip install cloudmesh.sys
\end{verbatim}

\paragraph{Exercise}\label{exercise}

\begin{description}
\item[Epyenv.1:]
Write installation instructions for an operating system of your choice
and add to this documentation.
\item[Epyenv.2:]
Replicate the steps above, so you can type in ENV2 and ENV3 in your
terminals to switch between python 2 and 3.
\end{description}

\subsubsection{virtualenv}\label{virtualenv}

environment while using virtualenv,. Documentation about it can be found
at:

\begin{verbatim}
* https://virtualenv.pypa.io
\end{verbatim}

The installation is simple once you have pip installed. If it is not
installed you can say:

\begin{verbatim}
$ easy_install pip
\end{verbatim}

After that you can install the virtual env with:

\begin{verbatim}
$ pip install virtualenv
\end{verbatim}

To setup an isolated environment for example in the directory
\textasciitilde{}/ENV please use:

\begin{verbatim}
$ virtualenv ~/ENV
\end{verbatim}

To activate it you can use the command:

\begin{verbatim}
$ source ~/ENV/bin/activate
\end{verbatim}

you can put this command in your .bashrc or .bash\_profile files so you
do not forget to activate it. Instructions for this can be
found in our lesson on Linux \textless{}bashrc\textgreater{}.

\subsection{Interactive Python}\label{interactive-python}

Python can be used interactively. Start by entering the interactive loop
by executing the command:

\begin{verbatim}
$ python
\end{verbatim}

You should see something like the following:

\begin{verbatim}
Python 2.7.13 (default, Nov 19 2016, 06:48:10)
[GCC 5.4.0 20160609] on linux2
Type "help", "copyright", "credits" or "license" for more information.
>>>
\end{verbatim}

The \textgreater{}\textgreater{}\textgreater{} is the prompt for the
interpreter. This is similar to the shell interpreter you have been
using.

Often we show the prompt when illustrating an example. This is to
provide some context for what we are doing. If you are following along
you will not need to type in the prompt.

This interactive prompt does the following:

\begin{itemize}
\tightlist
\item
  \emph{read} your input commands
\item
  \emph{evaluate} your command
\item
  \emph{print} the result of evaluation
\item
  \emph{loop} back to the beginning.
\end{itemize}

This is why you may see the interactive loop referred to as a
\textbf{REPL}:
\textbf{R}ead-\textbf{E}valuate-\textbf{P}rint-\textbf{L}oop.

\subsection{Python 3 Features in Python
2}\label{python-3-features-in-python-2}

As mentioned earlier, we assume you will use Python 2.7.X because there
are still some libraries that haven't been ported to Python 3. However,
there are some features of Python 3 we can and want to use in Python
2.7. Before we do anything else, we need to make these features
available to any subsequent code we write:

\begin{verbatim}
>>> from __future__ import print_function, division
\end{verbatim}

The first of these imports allows us to use the print function to output
text to the screen, instead of the print statement, which Python 2 uses.
This is simply a \href{https://www.python.org/dev/peps/pep-3105/}{design
decision} that better reflects Python's underlying philosophy.

The second of these imports makes sure that the
\href{https://www.python.org/dev/peps/pep-0238/}{division operator}
behaves in a way a newcomer to the language might find more intruitive.
In Python 2, division / is \emph{floor division} when the arguments are
integers, meaning that 5 / 2 == 2, for example. In Python 3, division /
is \emph{true division}, thus 5 / 2 == 2.5.

\subsection{Statements and Strings}\label{statements-and-strings}

Let us explore the syntax of Python. Type into the interactive loop and
press Enter:

\begin{verbatim}
>>> print("Hello world from Python!")
Hello world from Python!
\end{verbatim}

What happened: the print function was given a \textbf{string} to
process. A string is a sequence of characters. A \textbf{character} can
be a alphabetic (A through Z, lower and upper case), numeric (any of the
digits), white space (spaces, tabs, newlines, etc), syntactic directives
(comma, colon, quotation, exclamation, etc), and so forth. A string is
just a sequence of the character and typically indicated by surrounding
the characters in double quotes.

Standard output is discussed in the ../../lesson/linux/shell lesson.

So, what happened when you pressed Enter? The interactive Python program
read the line print "Hello world from Python!", split it into the print
statement and the "Hello world from Python!" string, and then executed
the line, showing you the output.

\subsection{Variables}\label{variables}

You can store data into a \textbf{variable} to access it later. For
instance, instead of:

\begin{verbatim}
>>> print('Hello world from Python!')
\end{verbatim}

which is a lot to type if you need to do it multiple times, you can
store the string in a variable for convenient access:

\begin{verbatim}
>>> hello = 'Hello world from Python!'
>>> print(hello)
Hello world from Python!
\end{verbatim}

\subsection{Data Types}\label{data-types}

\subsubsection{Booleans}\label{booleans}

A \textbf{boolean} is a value that indicates \emph{truthness} of
something. You can think of it as a toggle: either ``on'' or ``off'',
``one'' or ``zero'', ``true'' or ``false''. In fact, the only possible
values of the \textbf{boolean} (or bool) type in Python are:

\begin{itemize}
\tightlist
\item
  True
\item
  False
\end{itemize}

You can combine booleans with \textbf{boolean operators}:

\begin{itemize}
\tightlist
\item
  and
\item
  or
\end{itemize}

\begin{verbatim}
>>> print(True and True)
True
>>> print(True and False)
False
>>> print(False and False)
False
>>> print(True or True)
True
>>> print(True or False)
True
>>> print(False or False)
False
\end{verbatim}

\subsubsection{Numbers}\label{numbers}

The interactive interpreter can also be used as a calculator. For
instance, say we wanted to compute a multiple of 21:

\begin{verbatim}
>>> print(21 * 2)
42
\end{verbatim}

We saw here the print statement again. We passed in the result of the
operation 21 * 2. An \textbf{integer} (or \textbf{int}) in Python is a
numeric value without a fractional component (those are called
\textbf{floating point} numbers, or \textbf{float} for short).

The mathematical operators compute the related mathematical operation to
the provided numbers. Some operators are:

\begin{itemize}
\tightlist
\item
  * --- multiplication
\item
  / --- division
\item
  + --- addition
\item
  - --- subtraction
\item
  ** --- exponent
\end{itemize}

Exponentiation is read as x**y is x to the yth power:

\[x^y\]

You can combine \textbf{float}s and \textbf{int}s:

\begin{verbatim}
>>> print(3.14 * 42 / 11 + 4 - 2)
13.9890909091
>>> print(2**3)
8
\end{verbatim}

Note that \textbf{operator precedence} is important. Using parenthesis
to indicate affect the order of operations gives a difference results,
as expected:

\begin{verbatim}
>>> print(3.14 * (42 / 11) + 4 - 2)
11.42
>>> print(1 + 2 * 3 - 4 / 5.0)
6.2
>>> print( (1 + 2) * (3 - 4) / 5.0 )
-0.6
\end{verbatim}

\subsection{REPL (Read Eval Print
Loop)}\label{repl-read-eval-print-loop}

We have so far seen a few examples of types: \textbf{string}s,
\textbf{bool}s, \textbf{int}s, and \textbf{float}s. A \textbf{type}
indicates that values of that type support a certain set of operations.
For instance, how would you exponentiate a string? If you ask the
interpreter, this results in an error:

\begin{verbatim}
>>> "hello"**3
Traceback (most recent call last):
  File "<stdin>", line 1, in <module>
TypeError: unsupported operand type(s) for ** or pow(): 'str' and 'int'
\end{verbatim}

There are many different types beyond what we have seen so far, such as
\textbf{dictionaries}s, \textbf{list}s, \textbf{set}s. One handy way of
using the interactive python is to get the type of a value using
`type():

::

   \textgreater{}\textgreater{}\textgreater{} type(42)
   \textless{}type 'int'\textgreater{}
   \textgreater{}\textgreater{}\textgreater{} type(hello)
   \textless{}type 'str'\textgreater{}
   \textgreater{}\textgreater{}\textgreater{} type(3.14)
   \textless{}type 'float'\textgreater{}

You can also ask for help about something using help():

::

   \textgreater{}\textgreater{}\textgreater{} help(int)
   \textgreater{}\textgreater{}\textgreater{} help(list)
   \textgreater{}\textgreater{}\textgreater{} help(str)

.. tip::

   Using help()` opens up a pager. To navigate you can use the spacebar
to go down a page w to go up a page, the arrow keys to go up/down
line-by-line, or q to exit.

\subsection{Control Statements}\label{control-statements}

\subsubsection{Comparision}\label{comparision}

Computer programs do not only execute instructions. Occasionally, a
choice needs to be made. Such as a choice is based on a condition.
Python has several conditional operators:

\begin{verbatim}
>   greater than
<   smaller than
==  equals
!=  is not
\end{verbatim}

Conditions are always combined with variables. A program can make a
choice using the if keyword. For example:

\begin{verbatim}
>>> x = int(input("Guess x:"))
>>> if x == 4:
...    print('You guessed correctly!')
...    <ENTER>
\end{verbatim}

In this example, \emph{You guessed correctly!} will only be printed if
the variable x equals to four (see table above). Python can also execute
multiple conditions using the elif and else keywords.

\begin{verbatim}
>>> x = int(input("Guess x:"))
>>> if x == 4:
...     print('You guessed correctly!')
... elif abs(4 - x) == 1:
...     print('Wrong guess, but you are close!')
... else:
...     print('Wrong guess')
... <ENTER>
\end{verbatim}

\subsubsection{Iteration}\label{iteration}

To repeat code, the for keyword can be used. For example, to display the
numbers from 1 to 10, we could write something like this:

\begin{verbatim}
>>> for i in range(1, 11):
...    print('Hello!')
\end{verbatim}

The second argument to range, \emph{11}, is not inclusive, meaning that
the loop will only get to \emph{10} before it finishes. Python itself
starts counting from 0, so this code will also work:

\begin{verbatim}
>>> for i in range(0, 10):
...    print(i + 1)
\end{verbatim}

In fact, the range function defaults to starting value of \emph{0}, so
the above is equivalent to:

\begin{verbatim}
>>> for i in range(10):
...    print(i + 1)
\end{verbatim}

We can also nest loops inside each other:

\begin{verbatim}
>>> for i in range(0,10):
...     for j in range(0,10):
...         print(i,' ',j)
... <ENTER>
\end{verbatim}

In this case we have two nested loops. The code will iterate over the
entire coordinate range (0,0) to (9,9)

\subsection{Datatypes}\label{datatypes}

\subsubsection{Lists}\label{lists}

see: \url{https://www.tutorialspoint.com/python/python_lists.htm}

Lists in Python are ordered sequences of elements, where each element
can be accessed using a 0-based index.

To define a list, you simply list its elements between square brackest
`{[}{]}`:

\begin{verbatim}
>>> >>> names = ['Albert', 'Jane', 'Liz', 'John', 'Abby']
>>> names[0] # access the first element of the list
'Albert'
>>> names[2] # access the third element of the list
'Liz'
\end{verbatim}

You can also use a negative index if you want to start counting elements
from the end of the list. Thus, the last element has index \emph{-1},
the second before last element has index \emph{-2} and so on:

\begin{verbatim}
>>> names[-1] # access the last element of the list
'Abby'
>>> names[-2] # access the second last element of the list
'John'
\end{verbatim}

Python also allows you to take whole slices of the list by specifing a
beginning and end of the slice separated by a colon `::

::

  \textgreater{}\textgreater{}\textgreater{} names{[}1:-1{]} \# the middle elements, excluding first and last
  {[}'Jane', 'Liz', 'John'{]}

As you can see from the example above, the starting index in the slice
is inclusive and the ending one, exclusive.

Python provides a variety of methods for manipulating the members of a
list.

You can add elements with append`:

\begin{verbatim}
>>> names.append('Liz')
>>> names
['Albert', 'Jane', 'Liz', 'John', 'Abby', 'Liz']
\end{verbatim}

As you can see, the elements in a list need not be unique.

Merge two lists with `extend`:

\begin{verbatim}
>>> names.extend(['Lindsay', 'Connor'])
>>> names
['Albert', 'Jane', 'Liz', 'John', 'Abby', 'Liz', 'Lindsay', 'Connor']
\end{verbatim}

Find the index of the first occurrence of an element with `index`:

\begin{verbatim}
>>> names.index('Liz')
2
\end{verbatim}

Remove elements by value with `remove`:

\begin{verbatim}
>>> names.remove('Abby')
>>> names
['Albert', 'Jane', 'Liz', 'John', 'Liz', 'Lindsay', 'Connor']
\end{verbatim}

Remove elements by index with `pop`:

\begin{verbatim}
>>> names.pop(1)
'Jane'
>>> names
['Albert', 'Liz', 'John', 'Liz', 'Lindsay', 'Connor']
\end{verbatim}

Notice that pop returns the element being removed, while remove does
not.

If you are familiar with stacks from other programming languages, you
can use insert and `pop`:

\begin{verbatim}
>>> names.insert(0, 'Lincoln')
>>> names
['Lincoln', 'Albert', 'Liz', 'John', 'Liz', 'Lindsay', 'Connor']
>>> names.pop()
'Connor'
>>> names
['Lincoln', 'Albert', 'Liz', 'John', 'Liz', 'Lindsay']
\end{verbatim}

The Python documentation contains a \href{}{full list of list
operations}.

To go back to the range function you used earlier, it simply creates a
list of numbers:

\begin{verbatim}
>>> range(10)
[0, 1, 2, 3, 4, 5, 6, 7, 8, 9]
>>> range(2, 10, 2)
[2, 4, 6, 8]
\end{verbatim}

\subsubsection{Sets}\label{sets}

Python lists can contain duplicates as you saw above:

\begin{verbatim}
>>> names = ['Albert', 'Jane', 'Liz', 'John', 'Abby', 'Liz']
\end{verbatim}

When we don't want this to be the case, we can use a
\href{https://docs.python.org/2/library/stdtypes.html\#set}{set}:

\begin{verbatim}
>>> unique_names = set(names)
>>> unique_names
set(['Lincoln', 'John', 'Albert', 'Liz', 'Lindsay'])
\end{verbatim}

Keep in mind that the \emph{set} is an unordered collection of objects,
thus we can not access them by index:

\begin{verbatim}
>>> unique_names[0]
Traceback (most recent call last):
  File "<stdin>", line 1, in <module>
  TypeError: 'set' object does not support indexing
\end{verbatim}

However, we can convert a set to a list easily:

\textgreater{}\textgreater{}\textgreater{} unique\_names =
list(unique\_names) \textgreater{}\textgreater{}\textgreater{}
unique\_names {[}`Lincoln', `John', `Albert', `Liz', `Lindsay'{]}
\textgreater{}\textgreater{}\textgreater{} unique\_names{[}0{]}
`Lincoln'

Notice that in this case, the order of elements in the new list matches
the order in which the elements were displayed when we create the set
(we had set({[}'Lincoln', 'John', 'Albert', 'Liz',
'Lindsay'{]}) and now we have {[}'Lincoln', 'John', 'Albert', 'Liz',
'Lindsay'{]}). You should not assume this is the case in general. That
is, don't make any assumptions about the order of elements in a set when
it is converted to any type of sequential data structure.

You can change a set's contents using the add, remove and update methods
which correspond to the append, remove and extend methods in a list. In
addition to these, \emph{set} objects support the operations you may be
familiar with from mathematical sets: \emph{union}, \emph{intersection},
\emph{difference}, as well as operations to check containment. You can
read about this in the
\href{https://docs.python.org/2/library/stdtypes.html\#set}{Python
documentation for sets}.

\subsubsection{Removal and Testing for Membership in
Sets}\label{removal-and-testing-for-membership-in-sets}

One important advantage of a \emph{set} over a \emph{list} is that
\textbf{access to elements is fast}. If you are familiar with different
data structures from a Computer Science class, the Python list is
implemented by an array, while the set is implemented by a hash table.

We will demonstrate this with an example. Let's say we have a list and a
set of the same number of elements (approximately 100 thousand):

\begin{verbatim}
>>> import sys, random, timeit
>>> nums_set = set([random.randint(0, sys.maxint) for _ in range(10**5)])
>>> nums_list = list(nums_set)
>>> len(nums_set)
100000
\end{verbatim}

We will use the
\href{https://docs.python.org/2/library/timeit.html}{timeit} Python
module to time 100 operations that test for the existence of a member in
either the list or set:

\begin{verbatim}
>>> timeit.timeit('random.randint(0, sys.maxint) in nums', setup='import random; nums=%s' % str(nums_set), number=100)
0.0004038810729980469
>>> timeit.timeit('random.randint(0, sys.maxint) in nums', setup='import random; nums=%s' % str(nums_list), number=100)
0.3980541229248047
\end{verbatim}

The exact duration of the operations on your system will be different,
but the take away will be the same: searching for an element in a set is
orders of magnitude faster than in a list. This is important to keep in
mind when you work with large amounts of data.

\subsubsection{Dictionaries}\label{dictionaries}

One of the very important data structures in python is a dictionary also
referred to as \emph{dict}.

A dictionary represents a key value store:

\begin{verbatim}
>>> person = {'Name': 'Albert', 'Age': 100, 'Class': 'Scientist'}
>>> print("person['Name']: ", person['Name'])
person['Name']:  Albert
>>> print("person['Age']: ", person['Age'])
person['Age']:  100
\end{verbatim}

You can delete elements with the following commands:

\begin{verbatim}
>>> del person['Name'] # remove entry with key 'Name'
>>> person
{'Age': 100, 'Class': 'Scientist'}
>>> person.clear()     # remove all entries in dict
>>> person
{}
>>> del person         # delete entire dictionary
>>> person
Traceback (most recent call last):
  File "<stdin>", line 1, in <module>
  NameError: name 'person' is not defined
\end{verbatim}

You can iterate over a dict:

\begin{verbatim}
>>> person = {'Name': 'Albert', 'Age': 100, 'Class': 'Scientist'}
>>> for item in person:
...   print(item, person[item])
...   <ENTER>
Age 100
Name Albert
Class Scientist
\end{verbatim}

\subsubsection{Dictionary Keys and
Values}\label{dictionary-keys-and-values}

You can retrieve both the keys and values of a dictionary using the
keys() and values() methods of the dictionary, respectively:

\begin{verbatim}
>>> person.keys()
['Age', 'Name', 'Class']
>>> person.values()
[100, 'Albert', 'Scientist']
\end{verbatim}

Both methods return lists. Notice, however, that the order in which the
elements appear in the returned lists (Age, Name, Class) is different
from the order in which we listed the elements when we declared the
dictionary initially (Name, Age, Class). It is important to keep this in
mind: \textbf{you can't make any assumptions about the order in which
the elements of a dictionary will be returned by the keys() and values()
methods}.

However, you can assume that if you call keys() and values() in
sequence, the order of elements will at least correspond in both
methods. In the above example Age corresponds to 100, Name to 'Albert,
and Class to Scientist, and you will observe the same correspondence in
general as long as \textbf{keys() and values() are called one right
after the other}.

\subsubsection{Counting with
Dictionaries}\label{counting-with-dictionaries}

One application of dictionaries that frequently comes up is counting the
elements in a sequence. For example, say we have a sequence of coin
flips:

\begin{verbatim}
>>> import random
>>> die_rolls = [random.choice(['heads', 'tails']) for _ in range(10)]
>>> die_rolls
['heads', 'tails', 'heads', 'tails', 'heads', 'heads', 'tails', 'heads', 'heads', 'heads']
\end{verbatim}

The actual list die\_rolls will likely be different when you execute
this on your computer since the outcomes of the die rolls are random.

To compute the probabilities of heads and tails, we could count how many
heads and tails we have in the list:

\begin{verbatim}
>>> counts = {'heads': 0, 'tails': 0}
>>> for outcome in coin_flips:
...   assert outcome in counts
...   counts[outcome] += 1
...   <ENTER>
>>> print('Probability of heads: %.2f' % (counts['heads'] / len(coin_flips)))
Probability of heads: 0.70
>>> print('Probability of tails: %.2f' % (counts['tails'] / sum(counts.values())))
Probability of tails: 0.30
\end{verbatim}

In addition to how we use the dictionary counts to count the elements of
coin\_flips, notice a couple things about this example:

\begin{enumerate}
\tightlist
\item
  We used the assert outcome in counts statement. The assert statement
  in Python allows you to easily insert debugging statements in your
  code to help you discover errors more quickly. assert statements are
  executed whenever the internal Python \_\_debug\_\_ variable is set to
  True, which is always the case unless you start Python with the -O
  option which allows you to run \emph{optimized} Python.
\item
  When we computed the probability of tails, we used the built-in sum
  function, which allowed us to quickly find the total number of coin
  flips. sum is one of many built-in function you can
  \href{https://docs.python.org/2/library/functions.html}{read about
  here}.
\end{enumerate}

\subsection{Functions}\label{functions}

You can reuse code by putting it inside a function that you can call in
other parts of your programs. Functions are also a good way of grouping
code that logically belongs together in one coherent whole. A function
has a unique name in the program. Once you call a function, it will
execute its body which consists of one or more lines of code:

\begin{verbatim}
def check_triangle(a, b, c):
return \
    a < b + c and a > abs(b - c) and \
    b < a + c and b > abs(a - c) and \
    c < a + b and c > abs(a - b)

print(check_triangle(4, 5, 6))
\end{verbatim}

The def keyword tells Python we are defining a function. As part of the
definition, we have the function name, check\_triangle, and the
parameters of the function -- variables that will be populated when the
function is called.

We call the function with arguments 4, 5 and 6, which are passed in
order into the parameters a, b and c. A function can be called several
times with varying parameters. There is no limit to the number of
function calls.

It is also possible to store the output of a function in a variable, so
it can be reused.

\begin{verbatim}
def check_triangle(a, b, c):
 return \
     a < b + c and a > abs(b - c) and \
     b < a + c and b > abs(a - c) and \
     c < a + b and c > abs(a - b)

result = check_triangle(4, 5, 6)
print(result)
\end{verbatim}

\subsection{Classes}\label{classes}

A class is an encapsulation of data and the processes that work on them.
The data is represented in member variables, and the processes are
defined in the methods of the class (methods are functions inside the
class). For example, let's see how to define a Triangle class:

\begin{verbatim}
class Triangle(object):

 def __init__(self, length, width, height, angle1, angle2, angle3):
     if not self._sides_ok(length, width, height):
         print('The sides of the triangle are invalid.')
     elif not self._angles_ok(angle1, angle2, angle3):
         print('The angles of the triangle are invalid.')

     self._length = length
     self._width = width
     self._height = height

     self._angle1 = angle1
     self._angle2 = angle2
     self._angle3 = angle3

 def _sides_ok(self, a, b, c):
     return \
         a < b + c and a > abs(b - c) and \
         b < a + c and b > abs(a - c) and \
         c < a + b and c > abs(a - b)

 def _angles_ok(self, a, b, c):
     return a + b + c == 180

triangle = Triangle(4, 5, 6, 35, 65, 80)
\end{verbatim}

Python has full Aobject-oriented programming (OOP) capabilities, however
we can not cover all of them in a quick tutorial, so please refer to the
\href{https://docs.python.org/2.7/tutorial/classes.html}{Python docs on
classes and OOP}.

\subsection{Database Access}\label{database-access}

see:
\url{https://www.tutorialspoint.com/python/python_database_access.htm}

\subsection{Modules}\label{modules}

Make sure you are no longer in the interactive interpreter. If you are
you can type quit() and press Enter to exit.

You can save your programs to files which the interpreter can then
execute. This has the benefit of allowing you to track changes made to
your programs and sharing them with other people.

Start by opening a new file hello.py in the Python editor of your
choice. If you don't have a preferred editor, we recommend
\href{https://www.jetbrains.com/pycharm/}{PyCharm}.

Now write this simple program and save it:

\begin{verbatim}
from __future__ import print_statement, division
print("Hello world!")
\end{verbatim}

As a check, make sure the file contains the expected contents on the
command line:

\begin{verbatim}
$ cat hello.py
from __future__ import print_statement, division
print("Hello world!")
\end{verbatim}

To execute your program pass the file as a parameter to the python
command:

\begin{verbatim}
$ python hello.py
Hello world!
\end{verbatim}

Files in which Python code is stored are called \textbf{module}s. You
can execute a Python module form the command line like you just did, or
you can import it in other Python code using the import statement.

Let's write a more involved Python program that will receive as input
the lengths of the three sides of a triangle, and will output whether
they define a valid triangle. A triangle is valid if the length of each
side is less than the sum of the lengths of the other two sides and
greater than the difference of the lengths of the other two sides.:

\begin{verbatim}
"""Usage: check_triangle.py [-h] LENGTH WIDTH HEIGHT

Check if a triangle is valid.

Arguments:
  LENGTH     The length of the triangle.
  WIDTH      The width of the traingle.
  HEIGHT     The height of the triangle.

Options:
-h --help
"""
from __future__ import print_function, division
from docopt import docopt

if __name__ == '__main__':
  args = docopt(__doc__)
  a, b, c = int(args['LENGTH']), int(args['WIDTH']), int(args['HEIGHT'])
  valid_triangle = \
      a < b + c and a > abs(b - c) and \
      b < a + c and b > abs(a - c) and \
      c < a + b and c > abs(a - b)
  print('Triangle with sides %d, %d and %d is valid: %r' % (
      a, b, c, valid_triangle
  ))
\end{verbatim}

Assuming we save the program in a file called check\_triangle.py, we can
run it like so:

\begin{verbatim}
$ python check_triangle.py 4 5 6
Triangle with sides 4, 5 and 6 is valid: True
\end{verbatim}

Let break this down a bit.

\begin{enumerate}
\tightlist
\item
  We are importing the print\_function and division modules from Python
  3 like we did earlier in this tutorial. It's a good idea to always
  include these in your programs.
\item
  We've defined a boolean expression that tells us if the sides that
  were input define a valid triangle. The result of the expression is
  stored in the valid\_triangle variable. inside are true, and False
  otherwise.
\item
  We've used the backslash symbol \textbackslash{} to format are code
  nicely. The backslash simply indicates that the current line is being
  continued on the next line.
\item
  When we run the program, we do the check if \_\_name\_\_ ==
  '\_\_main\_\_'. \_\_name\_\_ is an internal Python variable that
  allows us to tell whether the current file is being run from the
  command line (value \_\_name\_\_), or is being imported by a module
  (the value will be the name of the module). Thus, with this statement
  we're just making sure the program is being run by the command line.
\item
  We are using the docopt module to handle command line arguments. The
  advantage of using this module is that it generates a usage help
  statement for the program and enforces command line arguments
  automatically. All of this is done by parsing the docstring at the top
  of the file.
\item
  In the print function, we are using
  \href{https://docs.python.org/2/library/string.html\#format-string-syntax}{Python's
  string formatting capabilities} to insert values into the string we
  are displaying.
\end{enumerate}

\subsection{Installing Libraries}\label{installing-libraries}

Often you may need functionality that is not present in Python's
standard library. In this case you have two option:

\begin{itemize}
\tightlist
\item
  implement the features yourself
\item
  use a third-party library that has the desired features.
\end{itemize}

Often you can find a previous implementation of what you need. Since
this is a common situation, there is a service supporting it: the
\href{https://pypi.python.org/pypi}{Python Package Index} (or PyPi for
short).

Our task here is to install the \href{}{autopep8} tool from PyPi. This
will allow us to illustrate the use if virtual environments using the
pyenv or virtualenv command, and installing and uninstalling PyPi
packages using pip.

\subsection{Using pip to Install
Packages}\label{using-pip-to-install-packages}

Let's now look at another important tool for Python development: the
Python Package Index, or PyPI for short. PyPI provides a large set of
third-party python packages. If you want to do something in python,
first check pypi, as odd are someone already ran into the problem and
created a package solving it.

In order to install package from PyPI, use the pip command. We can
search for PyPI for packages:

\begin{verbatim}
$ pip search --trusted-host pypi.python.org autopep8 pylint
\end{verbatim}

It appears that the top two results are what we want so install them:

\begin{verbatim}
$ pip install --trusted-host pypi.python.org autopep8 pylint
\end{verbatim}

This will cause pip to download the packages from PyPI, extract them,
check their dependencies and install those as needed, then install the
requested packages.

\begin{description}
\item[You can skip `--trusted-host pypi.python.org' option if you have]
patched urllib3 on Python 2.7.9.
\end{description}

\subsection{GUI}\label{gui}

\subsubsection{GUIZero}\label{guizero}

Install guizero with the following command:

\begin{verbatim}
sudo pip3 install guizero
\end{verbatim}

For a comprehensive tutorial on guizero,
\href{https://lawsie.github.io/guizero/howto/}{click here}.

\subsubsection{Kivy}\label{kivy}

You can install Kivy on OSX as followes:

\begin{verbatim}
brew install pkg-config sdl2 sdl2_image sdl2_ttf sdl2_mixer gstreamer
pip install -U Cython
pip install kivy
pip install pygame
\end{verbatim}

A hello world program for kivy is included in the cloudmesh.robot
repository. Which you can fine here

\begin{itemize}
\tightlist
\item
  \url{https://github.com/cloudmesh/cloudmesh.robot/tree/master/projects/kivy}
\end{itemize}

To run the program, please download it or execute it in cloudmesh.robot
as follows:

\begin{verbatim}
cd cloudmesh.robot/projects/kivy
python swim.py
\end{verbatim}

To create stand alone packages with kivy, please see:

\begin{verbatim}
-  https://kivy.org/docs/guide/packaging-osx.html
\end{verbatim}

\subsection{Formatting and Checking Python
Code}\label{formatting-and-checking-python-code}

First, get the bad code:

\begin{verbatim}
$ wget --no-check-certificate http://git.io/pXqb -O bad_code_example.py
\end{verbatim}

Examine the code:

\begin{verbatim}
$ emacs bad_code_example.py
\end{verbatim}

As you can see, this is very dense and hard to read. Cleaning it up by
hand would be a time-consuming and error-prone process. Luckily, this is
a common problem so there exist a couple packages to help in this
situation.

\subsection{Using autopep8}\label{using-autopep8}

We can now run the bad code through autopep8 to fix formatting problems:

\begin{verbatim}
$ autopep8 bad_code_example.py >code_example_autopep8.py
\end{verbatim}

Let us look at the result. This is considerably better than before. It
is easy to tell what the example1 and example2 functions are doing.

It is a good idea to develop a habit of using autopep8 in your
python-development workflow. For instance: use autopep8 to check a file,
and if it passes, make any changes in place using the -i flag:

\begin{verbatim}
$ autopep8 file.py    # check output to see of passes
$ autopep8 -i file.py # update in place
\end{verbatim}

If you use pyCharm you have the ability to use a similar function while
p;ressing on Inspect Code.

\subsection{Further Learning}\label{further-learning}

There is much more to python than what we have covered here:

\begin{itemize}
\tightlist
\item
  conditional expression (if, if...then,`if..elif..then`)
\item
  function definition(def)
\item
  class definition (class)
\item
  function positional arguments and keyword arguments
\item
  lambda expression
\item
  iterators
\item
  generators
\item
  loops
\item
  docopts
\item
  humanize
\end{itemize}

\subsection{Writing Python 3 Compatible
Code}\label{writing-python-3-compatible-code}

To write python 2 and 3 compatib;e code we recommend that you take a
look at: \url{http://python-future.org/compatible_idioms.html}

\subsection{Using Python on
FutureSystems}\label{using-python-on-futuresystems}

This is only important if you use Futuresystems resources.

In order to use Python you must log into your FutureSystems account.
Then at the shell prompt execute the following command:

\begin{verbatim}
$ module load python
\end{verbatim}

This will make the python and virtualenv commands available to you.

The details of what the module load command does are described in the
future lesson modules.

\subsection{Ecosystem}\label{ecosystem}

\subsubsection{pypi}\label{pypi}

Link: \href{https://pypi.python.org/pypi}{pypi}

The Python Package Index is a large repository of software for the
Python programming language containing a large number of packages
{[}link{]}. The nice think about pipy is that many packages can be
installed with the program `pip'.

To do so you have to locate the \textless{}package\_name\textgreater{}
for example with the search function in pypi and say on the commandline:

\begin{verbatim}
pip install <package_name>
\end{verbatim}

where pagage\_name is the string name of the package. an example would
be the package called cloudmesh\_client which you can install with:

\begin{verbatim}
pip install cloudmesh_client
\end{verbatim}

If all goes well the package will be installed.

\subsubsection{Alternative
Installations}\label{alternative-installations}

The basic installation of python is provided by python.org. However
others claim to have alternative environments that allow you to install
python. This includes

\begin{itemize}
\tightlist
\item
  \href{https://store.enthought.com/downloads/\#default}{Canopy}
\item
  \href{https://www.continuum.io/downloads}{Anaconda}
\item
  \href{http://ironpython.net/}{IronPython}
\end{itemize}

Typically they include not only the python compiler but also several
useful packages. It is fine to use such environments for the class, but
it should be noted that in both cases not every python library may be
available for install in the given environment. For example if you need
to use cloudmesh client, it may not be available as conda or Canopy
package. This is also the case for many other cloud related and useful
python libraries. Hence, we do recommend that if you are new to python
to use the distribution form python.org, and use pip and virtualenv.

Additionally some python version have platform specific libraries or
dependencies. For example coca libraries, .NET or other frameworks are
examples. For the assignments and the projects such platform dependent
libraries are not to be used.

If however you can write a platform independent code that works on
Linux, OSX and Windows while using the python.org version but develop it
with any of the other tools that is just fine. However it is up to you
to guarantee that this independence is maintained and implemented. You
do have to write requirements.txt files that will install the necessary
python libraries in a platform independent fashion. The homework
assignment PRG1 has even a requirement to do so.

In order to provide platform independence we have given in the class a
``minimal'' python version that we have tested with hundreds of
students: python.org. If you use any other version, that is your
decision. Additionally some students not only use python.org but have
used iPython which is fine too. However this class is not only about
python, but also about how to have your code run on any platform. The
homework is designed so that you can identify a setup that works for
you.

However we have concerns if you for example wanted to use chameleon
cloud which we require you to access with cloudmesh. cloudmesh is not
available as conda, canopy, or other framework package. Cloudmesh client
is available form pypi which is standard and should be supported by the
frameworks. We have not tested cloudmesh on any other python version
then python.org which is the open source community standard. None of the
other versions are standard.

In fact we had students over the summer using canopy on their machines
and they got confused as they now had multiple python versions and did
not know how to switch between them and activate the correct version.
Certainly if you know how to do that, than feel free to use canopy, and
if you want to use canopy all this is up to you. However the homework
and project requires you to make your program portable to python.org. If
you know how to do that even if you use canopy, anaconda, or any other
python version that is fine. Graders will test your programs on a
python.org installation and not canpoy, anaconda, ironpython while using
virtualenv. It is obvious why. If you do not know that answer you may
want to think about that every time they test a program they need to do
a new virtualenv and run vanilla python in it. If we were to run two
instals in the same system, this will not work as we do not know if one
student will cause a side effect for another. Thus we as instructors do
not just have to look at your code but code of hundreds of students with
different setups. This is a non scalable solution as every time we test
out code from a student we would have to wipe out the OS, install it
new, install an new version of whatever python you have elected, become
familiar with that version and so on and on. This is the reason why the
open source community is using python.org. We follow best practices.
Using other versions is not a community best practice, but may work for
an individual.

We have however in regards to using other python version additional
bonus projects such as

\begin{itemize}
\tightlist
\item
  deploy run and document cloudmesh on ironpython
\item
  deploy run and document cloudmesh on anaconde, develop script to
  generate a conda packge form github
\item
  deploy run and document cloudmesh on canopy, develop script to
  generate a conda packge form github
\item
  deploy run and document cloudmesh on ironpython
\item
  other documentation that would be useful
\end{itemize}

\subsubsection{Autoenv: Directory-based
Environments}\label{autoenv-directory-based-environments}

\begin{description}
\item[We do not recommend that you use autoenv. Instead we]
recommend that you use pyenv.
\end{description}

Link:
Autoenv \textless{}https://pypi.python.org/pypi/autoenv/0.2.0\textgreater{}

If a directory contains a .env file, it will automatically be executed
when you cd into it. It's easy to use and install.

This is useful for

\begin{itemize}
\tightlist
\item
  auto-activating virtualenvs
\item
  project-specific environment variables
\end{itemize}

To use it add the ENV you created with virtualenv into .env file within
your project directory:

\begin{verbatim}
$ echo "source ~/ENV/bin/activate" > yourproject/.env
$ echo "echo 'whoa'" > yourproject/.env
$ cd project
whoa
\end{verbatim}

To install it on Mac OS X use Homebrew:

\begin{verbatim}
$ brew install autoenv
$ echo "source $(brew --prefix autoenv)/activate.sh" >> ~/.bash_profile
\end{verbatim}

To install it using pip use:

\begin{verbatim}
$ pip install autoenv
$ echo "source `which activate.sh`" >> ~/.bashrc
\end{verbatim}

To install it using git use:

\begin{verbatim}
$ git clone git://github.com/kennethreitz/autoenv.git ~/.autoenv
$ echo 'source ~/.autoenv/activate.sh' >> ~/.bashrc
\end{verbatim}

Before sourcing activate.sh, you can set the following variables:

\begin{itemize}
\tightlist
\item
  `AUTOENV\_AUTH\_FILE`: Authorized env files, defaults to
  \textasciitilde{}/.autoenv\_authorized
\item
  `AUTOENV\_ENV\_FILENAME`: Name of the .env file, defaults to .env
\item
  `AUTOENV\_LOWER\_FIRST`: Set this variable to flip the order of .env
  files executed
\end{itemize}

Autoenv overrides cd. If you already do this, invoke autoenv\_init
within your custom cd after sourcing activate.sh.

\begin{description}
\item[Autoenv can be disabled via unset cd if you experience I/O issues]
with certain file systems, particularly those that are FUSE-based (such
as smbnetfs).
\end{description}

\subsection{Resources}\label{resources}

If you are unfamiliar with programming in Python, we also refer you to
some of the numerous online resources. You may wish to start with
\href{https://www.learnpython.org}{Learn Python} or the book
\href{http://learnpythonthehardway.org/book/}{Learn Python the Hard
Way}. Other options include
\href{http://www.tutorialspoint.com/python/}{Tutorials Point} or
\href{http://www.codecademy.com/en/tracks/python}{Code Academy}, and the
Python wiki page contains a long list of
\href{https://wiki.python.org/moin/BeginnersGuide/Programmers}{references
for learning} as well. Additional resources include:

\begin{itemize}
\tightlist
\item
  \url{https://virtualenvwrapper.readthedocs.io}
\item
  \url{https://github.com/yyuu/pyenv}
\item
  \url{https://amaral.northwestern.edu/resources/guides/pyenv-tutorial}
\item
  \url{https://godjango.com/96-django-and-python-3-how-to-setup-pyenv-for-multiple-pythons/}
\item
  \url{https://www.accelebrate.com/blog/the-many-faces-of-python-and-how-to-manage-them/}
\item
  \url{http://ivory.idyll.org/articles/advanced-swc/}
\item
  \url{http://python.net/~goodger/projects/pycon/2007/idiomatic/handout.html}
\item
  \url{http://www.youtube.com/watch?v=0vJJlVBVTFg}
\item
  \url{http://www.korokithakis.net/tutorials/python/}
\item
  \url{http://www.afterhoursprogramming.com/tutorial/Python/Introduction/}
\item
  \url{http://www.greenteapress.com/thinkpython/thinkCSpy.pdf}
\end{itemize}

A very long list of useful information are also available from

\begin{itemize}
\tightlist
\item
  \url{https://github.com/vinta/awesome-python}
\item
  \url{https://github.com/rasbt/python_reference}
\end{itemize}

This list may be useful as it also contains links to data visualization
and manipulation libraries, and AI tools and libraries. Please note that
for this class you can reuse such libraries if not otherwise stated.

\subsection{Jupyter Notebook
Tutorials}\label{jupyter-notebook-tutorials}

A Short Introduction to Jupyter Notebooks and NumPy To view the
notebook, open this link in a background tab
\textless{}\url{https://nbviewer.jupyter.org/}\textgreater{} and copy
and paste the following link in the URL input area
\textless{}\url{https://cloudmesh.github.io/classes/lesson/prg/Jupyter-NumPy-tutorial-I523-F2017.ipynb}\textgreater{}
Then hit Go!

\subsection{Exercises}\label{exercises}

\begin{description}
\item[EPython.1:]
Write a python program called iterate.py that accepts an integer n from
the command line. Pass this integer to a function called iterate.

The iterate function should then iterate from 1 to n. If the ith number
is a multiple of three, print ``multiple of 3'', if a multiple of 5
print ``multiple of 5'', if a multiple of both print ``multiple of 3 and
5'', else print the value.
\item[EPython.2:]
\begin{enumerate}
\tightlist
\item
  Create a pyenv or virtualenv \textasciitilde{}/ENV
\item
  Modify your \textasciitilde{}/.bashrc shell file to activate your
  environment upon login.
\item
  Install the docopt python package using pip
\item
  Write a program that uses docopt to define a commandline program.
  Hint: modify the iterate program.
\item
  Demonstrate the program works and submit the code and output.
\end{enumerate}
\end{description}
