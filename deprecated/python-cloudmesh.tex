\chapter{Enhanced Cloudmesh}

In this chapter we will be using some advanced Python features to
enhance Cloudmesh that is supposed to easily manage multiple
clouds. We will be explicitly using python 3 and do
not worry about backwards compatibility.

We will be developing it as community so that new features can be
integrated and loaded on demand while adding an extensible package
management system based on pythons shared namespace. To do so we will
rely on \verb|cmd5| that includes a generate command to add new
papackes on demand. We will not use all features of cmd5.

We are trying to develop the following:

\begin{description}

\item[Configuration] so that we can easily configure and add various
  clouds to our multi-cloud environment.

\item[Database] of virtual machines and clouds so that they can be
  managed across different clouds in a multi cloud environment

\item[API Classes] so that we can use python as a convenient
  programming environment.

\item[Context] libraries so that in python we can easily apply context
  for clouds and virtual machines on a block of statements

\item[Command Shell] so that we can similar to matlab
  and other shells execute multiple commands

\item[REST Services]  so that we can access the features
form other programming environments and different programming
languages.

\item[Parallel Services] so that we can issue commands in parallel and
  manage virtual machines in a multi cloud environment.

\end{description}



\section{Configuration}

As we are developing a multi-cloud environment, we need some mechanism
to define the clouds easily. To make our development effort simpler,
we like to point out that the configuration file must be stored in a
particular location relative to the home directory. We store the file
in \verb|~/.cloudmesh/class.yaml|. Additionally we store our cloud
passwords in this file in cleartext and thus we must make sure our
machine is not compromized and that we properly protect the file. On a
unix system you do this with:

\begin{verbatim}
mkdir ~/.cloudmesh
touch ~/.cloudmesh/class.yaml
chmod go-rw ~/.cloudmesh
chmod go-rw ~/.cloudmesh/class.yaml
\end{verbatim}

In that directory we store a file similar to the following file:

\begin{verbatim}
version: 5.0
profile:
  firstname: Gregor
  lastname: von Laszewski
  email: laszewski@gmail.com
cloudmesh:
  default:
  - chameleon 
  active:
  - chameleon
clouds:
  uc:
    name: Chameleon UC
    host: chameleoncloud.org
    type: openstack
    version: liberty
    credentials:
      OS_AUTH_URL: https://openstack.uc.chameleoncloud.org:5000/v3
      OS_PASSWORD: TBD
      OS_TENANT_NAME: CH-818664
      OS_TENANT_ID: CH-818664
      OS_PROJECT_NAME: CH-818664
      OS_PROJECT_DOMAIN_ID: default
      OS_USER_DOMAIN_ID: default
      OS_USERNAME: TBD
      OS_VERSION: liberty
      OS_REGION_NAME: RegionOne
    default:
      flavor: m1.small
      image: Ubuntu-Server-14.04-LTS
  tacc:
    name: Chameleon TACC
    host: chameleoncloud.org
    type: openstack
    version: liberty
    credentials:
      OS_AUTH_URL: https://openstack.tacc.chameleoncloud.org:5000/v3
      OS_PASSWORD: TBD
      OS_TENANT_NAME: CH-818664
      OS_TENANT_ID: CH-818664
      OS_PROJECT_NAME: CH-818664
      OS_PROJECT_DOMAIN_ID: default
      OS_USER_DOMAIN_ID: default
      OS_USERNAME: TBD
      OS_VERSION: liberty
      OS_REGION_NAME: RegionOne
    default:
      flavor: m1.small
      image: Ubuntu-Server-14.04-LTS
\end{verbatim}

Important to note is that this file defines multiple clouds and uses
the attribute value TBD for password and username which you may want
to change. However, we also would like to support a mode that when the
password is defined to be TBD that it is asked from the terminal. This
way we do not necessarily have to store the password here. In future
we will enhance this file to be encrypted and decrypted with a
password protected ssh key.

The file is ayaml file as the typical configuration in python is not
suitable to easily store hierarchical data. YAML is also more readable
than json so it provides a really good way of defining the
configuration data. Problematic with yaml readers however are that
they typically do not preserver the read order. Your task will be to
write a \textit{short} yaml configuration reader that preserves the
order. You are encouraged to reuse methods. What your are not supposed
to do is to reimplement yaml.

There are some special properties of this file that we need to
discuss. 

\begin{itemize}

\item clouds are listed in the clouds section
\item the credentials section to each cloud defines how to connect to
  the cloud with python libraries such as libcloud. Each cloud type will
  have different parameters.
\item 


\end{itemize}



\section{Storage}

As we need to store some of the data we must identify a suitable
database for storing information about virtual machines and other
information related to the clouds. Although shelve comes in mind, we
found out that it is not compatible between python 2 and 3 which may
be an issue in future. Also when considering services such as mongodb
they have to be started and properly secured. THis naturally can be
done with containers. We also do not want to use large frameworks such
as django which come with build in object models as they are not
lightwight. Hence, we start we just use a file based sql
database as provided with sqlite3.

\subsection{sqlite3}

While we keep the configuration in the configuration yaml file we
intend to create a database entry for virtual machines we start in the
cloud. In order to store hierarchical information that we may obtain
in dict format from a virtual machine we can easily create flattened
out data structures, by simple connecting the attribute names and
separate them by \verb|_|.

Let us assome we want to store an object of the following form:

\begin{verbatim}
element = {
   'id': 1,
   'cloud': 'chameleon',
   'name': 'vm1',
   'data': {
       "image": 'ubuntu',
       "flavor": "small"
    }
}
\end{verbatim}

A table that could store such an object could be

\begin{verbatim}
create table element (
    id           integer primary key,
    name         text
    cloud        text,
    data_image:  text,
    data_flavor: text
);
\end{verbatim}

Obviously, we could create the table automatically from recursively
iterating through the dict to make our approach generalized for any
dict. As for the primary key, we simply assume it is always the id
which is an integer that always increases and is stored in the database.
as a separate element.

Thus we probably want a table generator such as

\begin{lstlisting}{python}
class Database (object):
   @staticmethod
   def generate (dictionary):
      # implement me
\end{lstlisting}

Additionally we want to create convenience methods for adding,
deleting, and searching information


\subsection{Context}

Python provides the feature of a context that we are well familiar
with from file management. An example is:

\begin{lstlisting}{python}
with open('/tmp/gregor.txt', 'wt') as f:
    f.write('Hello Gregor')
# after this, the file is automatically closed
\end{lstlisting}

If we look at this example it is desirable to develop at least two
context for multicloud envireonments. The first is to manage virtual
machines on named clouds and issue action on it, such as \textit{start, stop,
  suspend, resume, delete}. In the other context we like to issue
such asction on named virtual machines.

To illustrate what we have in mind, please take a look at our initial
examples.

\subsubsection{Cloud Context}

When defining the following cloud context 

\begin{lstlisting}{python}
class Cloud (object):
    
    def __init__(self, name):
        self.name = name

    def __enter__(self):
        print ('Running on:', self.name)
        return self

    def __exit__(self, exc_type, exc_val, exc_tb):
        print('__exit__()')

    def machine(self, name, action):
        print (name, action)
\end{lstlisting}

we can issue conveniently commands such as the following 

\begin{lstlisting}{python}
cloud = 'chameleon'
with Cloud(cloud) as c:
    vm = c.machine('vm1', 'start')
\end{lstlisting}

It is obvious that through this abstraction we can formulate a
tempalted bahavior such as starting a virtual machine and through the
switch of a ingle variable (\verb|cloud|) issue the command on other
clouds.


