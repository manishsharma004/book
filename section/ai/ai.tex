\chapter{Draft: Artifical Intelligence}
\label{c:ai}
\index{Artificial-Intelligence}

\FILENAME\

%
% Miao will be developing leture material for the rest of the semester
% for this. Every week lecture material and labs for e222
%

\section{Unsupervised Learning}

Keywords: clustering, kNN, Markov Model

\slides{AI}{40}{Unsupervised Learning}{https://drive.google.com/file/d/1r62DpK-yK0L_v_KEBnmP6tdLfQFD7Lok/view?usp=sharing}

%\video{Container}{11:01}{AI}{}

%\url{https://onedrive.live.com/view.aspx?cid=9384db89dd1152fa&page=view&resid=9384DB89DD1152FA!1564&parId=9384DB89DD1152FA!1551&app=PowerPoint}

\section{Lab:Pracitice on AI}
Keywords: Docker, REST Service, Spark

 \slides{Practice on AI}{40}{REST services}{https://drive.google.com/file/d/1pD4zbrFKkS7d6SsxIw33NIoDHQLIedXn/view?usp=sharing}
% \video{AI}{11:01}{REST services}{}

\begin{comment}
\section{Draft: Deep Learning}

\section{Draft: Time Series}
\subsection{Overview of Time-Series Data}
This section will explore the make up of time-series (TS) data, and
the fundamental concepts needed to analyze TS date. More importantly
why simple linear regression models cannot be used. 
\subsection{Why is TS Data Important?}
This section will explore the application and relavence of TS data
with repspect to big-data. 
\subsection{Forecasting TS data}
This section will explore TS forecasting methods such as the ARIMA
model, ARCH/GARCH model, Vector Autoregression model, LSTM model and
NARX Networks...time permitting ELMAN and JORDAN netowrks. 
 
\end{comment}
