\chapter{Draft: Artifical Intelligence}
\label{c:ai}
\index{Artificial-Intelligence}

\FILENAME\

%
% Miao will be developing leture material for the rest of the semester
% for this. Every week lecture material and labs for e222
%

\section{Unsupervised Learning}

Keywords: clustering, kNN, Markov Model

\slides{AI}{40}{Unsupervised Learning}{https://drive.google.com/file/d/1r62DpK-yK0L_v_KEBnmP6tdLfQFD7Lok/view?usp=sharing}

%\video{Container}{11:01}{AI}{}

%\url{https://onedrive.live.com/view.aspx?cid=9384db89dd1152fa&page=view&resid=9384DB89DD1152FA!1564&parId=9384DB89DD1152FA!1551&app=PowerPoint}

\section{Lab:Pracitice on AI}
Keywords: Docker, REST Service, Spark

 \slides{Practice on AI}{40}{REST services}{https://drive.google.com/file/d/1pD4zbrFKkS7d6SsxIw33NIoDHQLIedXn/view?usp=sharing}
% \video{AI}{11:01}{REST services}{}

\section{Draft: k-NN}

k-NN is a non-paramteric statistics, which means there is no
assumption made about the distribution of the data. Additionally the
distribution is not assumed to be fixed i.e. the distribution may
change through time. These realxed assumptions make non-paramteric
tests exteremly valuable when applied to real-world data as a vast
majority of real world data have dynamic distributions though
time--climate data comes to mind. Non-paramtric data is often ordinal
which means the variables have an inherhent categorigcal order with
unknown distances between the categories. A common example of a
non-parametric statistical test is the sign test where values are
assigned a positive or negative sign based on being above or below the
median. In k-NN predictions are made about unknown values by matching
the unknown values with simialar known values. Naturally the
determination of 'similar' is of fundamental importance. This is done
through the application of the euclidian distance calculation. 

${d(\mathbf{i},\mathbf{j})} = {d(\mathbf{j},\mathbf{i})} =
\sqrt{{(i_1-j_2)^2} +} $
${(a+b)}^2 = a^2 + 2ab + b^{c+2}$

 

We will an idepth example using a dataset from the Natiaonal
Basketball Associateed (NBA) from 2014, naturally there are more up to
date datasets but as sports analytics becomes a significant market
more and more data is becoming propitary. 

\begin{comment}
\section{Draft: Deep Learning}

\section{Draft: Time Series}
\subsection{Overview of Time-Series Data}
This section will explore the make up of time-series (TS) data, and
the fundamental concepts needed to analyze TS date. More importantly
why simple linear regression models cannot be used. 
\subsection{Why is TS Data Important?}
This section will explore the application and relavence of TS data
with repspect to big-data. 
\subsection{Forecasting TS data}
This section will explore TS forecasting methods such as the ARIMA
model, ARCH/GARCH model, Vector Autoregression model, LSTM model and
NARX Networks...time permitting ELMAN and JORDAN netowrks. 
 
\end{comment}
