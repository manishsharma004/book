\FILENAME

\section{Interactive Python}\label{interactive-python}
\index{Python!REPL}
\index{Python!interactive}

Python can be used interactively. Start by entering the interactive loop
by executing the command:

\begin{verbatim}
$ python
\end{verbatim}

You should see something like the following:

\begin{verbatim}
Python 2.7.13 (default, Nov 19 2016, 06:48:10)
[GCC 5.4.0 20160609] on linux2
Type "help", "copyright", "credits" or "license" for more information.
>>>
\end{verbatim}

The \textgreater{}\textgreater{}\textgreater{} is the prompt for the
interpreter. This is similar to the shell interpreter you have been
using.

Often we show the prompt when illustrating an example. This is to
provide some context for what we are doing. If you are following along
you will not need to type in the prompt.

This interactive prompt does the following:

\begin{itemize}
\tightlist
\item
  \emph{read} your input commands
\item
  \emph{evaluate} your command
\item
  \emph{print} the result of evaluation
\item
  \emph{loop} back to the beginning.
\end{itemize}

This is why you may see the interactive loop referred to as a
\textbf{REPL}:
\textbf{R}ead-\textbf{E}valuate-\textbf{P}rint-\textbf{L}oop.

\section{REPL (Read Eval Print Loop)}\label{repl-read-eval-print-loop}
\index{Python!REPL}

We have so far seen a few examples of types: \textbf{string}s,
\textbf{bool}s, \textbf{int}s, and \textbf{float}s. A \textbf{type}
indicates that values of that type support a certain set of operations.
For instance, how would you exponentiate a string? If you ask the
interpreter, this results in an error:

\begin{verbatim}
>>> "hello"**3
Traceback (most recent call last):
  File "<stdin>", line 1, in <module>
TypeError: unsupported operand type(s) for ** or pow(): 'str' and 'int'
\end{verbatim}

There are many different types beyond what we have seen so far, such as
\textbf{dictionaries}s, \textbf{list}s, \textbf{set}s. One handy way of
using the interactive python is to get the type of a value using
`type():

::

   \textgreater{}\textgreater{}\textgreater{} type(42)
   \textless{}type 'int'\textgreater{}
   \textgreater{}\textgreater{}\textgreater{} type(hello)
   \textless{}type 'str'\textgreater{}
   \textgreater{}\textgreater{}\textgreater{} type(3.14)
   \textless{}type 'float'\textgreater{}

You can also ask for help about something using help():

::

   \textgreater{}\textgreater{}\textgreater{} help(int)
   \textgreater{}\textgreater{}\textgreater{} help(list)
   \textgreater{}\textgreater{}\textgreater{} help(str)

.. tip::

   Using help()` opens up a pager. To navigate you can use the spacebar
to go down a page w to go up a page, the arrow keys to go up/down
line-by-line, or q to exit.

\section{Python 3 Features in Python 2}\label{python-3-features-in-python-2}
\index{Python!2 and 3}


As mentioned earlier, we assume you will use Python 2.7.X because there
are still some libraries that haven't been ported to Python 3. However,
there are some features of Python 3 we can and want to use in Python
2.7. Before we do anything else, we need to make these features
available to any subsequent code we write:

\begin{verbatim}
>>> from __future__ import print_function, division
\end{verbatim}

The first of these imports allows us to use the print function to output
text to the screen, instead of the print statement, which Python 2 uses.
This is simply a \href{https://www.python.org/dev/peps/pep-3105/}{design
decision} that better reflects Python's underlying philosophy.

The second of these imports makes sure that the
\href{https://www.python.org/dev/peps/pep-0238/}{division operator}
behaves in a way a newcomer to the language might find more intruitive.
In Python 2, division / is \emph{floor division} when the arguments are
integers, meaning that 5 / 2 == 2, for example. In Python 3, division /
is \emph{true division}, thus 5 / 2 == 2.5.
