\FILENAME

\section{Introduction to Python}\label{introduction-to-python}

Portions of this lesson have been adapted from the
\href{https://docs.python.org/2/tutorial/}{official Python Tutorial}
copyright \href{http://www.python.org/}{Python Software Foundation}.

Python is an easy to learn programming language. It has efficient
high-level data structures and a simple but effective approach to
object-oriented programming. Python's simple syntax and dynamic typing,
together with its interpreted nature, make it an ideal language for
scripting and rapid application development in many areas on most
platforms. The Python interpreter and the extensive standard library are
freely available in source or binary form for all major platforms from
the Python Web site, \url{https://www.python.org/}, and may be freely
distributed. The same site also contains distributions of and pointers
to many free third party Python modules, programs and tools, and
additional documentation. The Python interpreter can be extended with
new functions and data types implemented in C or C++ (or other languages
callable from C). Python is also suitable as an extension language for
customizable applications.

Python is an interpreted, dynamic, high-level programming language
suitable for a wide range of applications.

The philosophy of python is summarized in
\href{https://www.python.org/dev/peps/pep-0020/}{The Zen of Python} as
follows:

\begin{itemize}

\item
  Explicit is better than implicit
\item
  Simple is better than complex
\item
  Complex is better than complicated
\item
  Readability counts
\end{itemize}

The main features of Python are:

\begin{itemize}

\item
  Use of indentation whitespace to indicate blocks
\item
  Object orient paradigm
\item
  Dynamic typing
\item
  Interpreted runtime
\item
  Garbage collected memory management
\item
  a large standard library
\item
  a large repository of third-party libraries
\end{itemize}

Python is used by many companies (such as Google, Yahoo!, CERN, NASA)
and is applied for web development, scientific computing, embedded
applications, artificial intelligence, software development, and
information security, to name a few.

The material collected here introduces the reader to the basic concepts and
features of the Python language and system. After you have worked
tthrough the material you will be able to:

\begin{itemize}

\item
  use Python
\item
  use the interactive Python interface
\item
  understand the basic syntax of Python
\item
  write and run Python programs stored in a file
\item
  have an overview of the standard library
\item
  install Python libraries using pyenv or if it is not available
  virtualenv
\end{itemize}

This tutorial does not attempt to be comprehensive and cover every
single feature, or even every commonly used feature. Instead, it
introduces many of Python's most noteworthy features, and will give you
a good idea of the language's flavor and style. After reading it, you
will be able to read and write Python modules and programs, and you will
be ready to learn more about the various Python library modules.

In order to conduct this lesson you need

\begin{itemize}

\item
  A computer with Python 2.7.13 or 3.6.2
\item
  Familiarity with command line usage
\item
  A text editor such as
  \href{https://www.jetbrains.com/pycharm/}{PyCharm}, emacs, vi or
  others. You should identity which works best for you and set it up.
\end{itemize}

\section{Refernces}

Some important additional information can be found on the following
Web pages.

\begin{itemize}
\item \href{https://www.python.org/}{Python}
\item \href{https://pip.pypa.io/en/stable/}{Pip}
\item \href{https://virtualenv.pypa.io/en/stable/}{Virtualenv}
\item \href{http://www.numpy.org/}{NumPy}
\item \href{https://scipy.org/}{SciPy}
\item \href{http://matplotlib.org/}{Matplotlib}
\item \href{http://pandas.pydata.org/}{Pandas}
\item \href{https://github.com/pyenv/pyenv}{pyenv}
\item \href{https://github.com/pyenv/pyenv}{PyCharm}
\end{itemize}

Python module of the week is a Web site that provides a number of short
examples on how to use some elementary python modules. Not all modules
are equally useful and you should decide if there are better
alternatives. However for beginners this site provides a number of good
examples

\begin{itemize}

\item
  Python 2: \url{https://pymotw.com/2/}
\item
  Python 3: \url{https://pymotw.com/3/}
\end{itemize}
