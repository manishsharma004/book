\FILENAME\

\section{Draft: Plotting with matplotlib}\label{matplotlib}
\index{matplotlib} %make new index

A brief overview of plotting with matplotlib along with examples is
provided. First matplotlib must be installed, which can be
accomplished with pip install as follows:

\begin{lstlisting}
 pip install matplotlib
\end{lstlisting}

We will start by plotting a simple line graph using built in numpy
functions for sine and cosine. This first step is to import the proper
libraries shown below.

\begin{lstlisting}
import numpy as np
import matplotlib.pyplot as plt
\end{lstlisting}

Next we will define the values for the x axis, we do this with the
linspace option in numpy. The first two paramters are the starting and
ending points, these must be scalors. The third paramter is optional
and defines the number of samples to be generated between the starting
and ending points, this value must be an interger. Additional
paramters for the linespace utility can be found here:
\URL{https://docs.scipy.org/doc/numpy/reference/generated/numpy.linspace.html}

\begin{lstlisting}
x = np.linspace(-np.pi, np.pi, 16)
\end{lstlisting}

Now we will use the sine and cosine functions in order to generate y
values, for this we will use the values of x for the argument of both
our sine and cosine fucntions i.e. $cos(x)$.

\begin{lstlisting}
cos = np.cos(x)
sin = np.sin(x)
#cos, sin = np.cos(x), np.sin(x) will produce the same as the above
#two lines
\end{lstlisting} 

You can display the values of the three paramters we have defined by
typing them in a python shell.

\begin{lstlisting}
x
array([-3.14159265, -2.72271363, -2.30383461, -1.88495559, -1.46607657,
       -1.04719755, -0.62831853, -0.20943951,  0.20943951,  0.62831853,
        1.04719755,  1.46607657,  1.88495559,  2.30383461,  2.72271363,
        3.14159265])
\end{lstlisting}

Having defined x and y values we can generate a line plot and since we
imported matplotlib.pyplot as plt we simply use plt.plot.

\begin{lstlisting}
plt.plot(x,cos)
\end{lstlisting}
We can display the plot using plt.show() which will pop up a figure
displaying the plot defined. 

\begin{lstlisting}
plt.show()
\end{lstlisting}

Additionally we can add the sine line to out line graph by entering
the following. 

\begin{lstlisting}
plt.plot(x,sin)
\end{lstlisting}

Invoking plt.show() now will show a figure with both sine and
cosine lines displayed. Now that we have a figure generated it would
be useful to label the x and y axis and provide a title. This is done
by the following three commands below: 

\begin{lstlisting}
plt.xlabel("X - label (units)")
plt.ylabel("Y - label (units)")
plt.title("A clever Title for your Figure")
\end{lstlisting}

Along with axis labels and a title another useful figure feature may
be a legend. In order to create a legend you must first designate a
label for the line, this label will be what shows up in the
legend. The label is defined in the intial plt.plot(x,y) instance,
below is an example. 

\begin{lstlisting}
plt.plot(x,cos, label="cosine")
\end{lstlisting}

Then in order to display the legen the following command is issued:

\begin{lstlisting}
plt.legend(loc='upper right')
\end{lstlisting}

The location is specified by using upper or lower and left or
right. Naturally all these commands can be combined and put in a file
with the .py extension and run from the command line. 

\begin{lstlisting}
import numpy as np
import matplotlib.pyplot as plt

x = np.linspace(-np.pi, np.pi, 16)
cos = np.cos(x)
sin = np.sin(x)
plt.plot(x,cos, label="cosine")
plt.plot(x,sin, label="sine")

plt.xlabel("X - label (units)")
plt.ylabel("Y - label (units)")
plt.title("A clever Title for your Figure")

plt.legend(loc='upper right')

plt.show()

\end{lstlisting}

An example of a barchart is procided below using data from
~\ref{T:fast-cars}. 

\begin{lstlisting}
import matplotlib.pyplot as plt

x = [' Toyota Prius', 'Tesla Roadster ', ' Bugatti  Veyron', ' Honda Civic ', ' Lamborghini Aventador ']
horse_power = [120, 288, 1200, 158, 695]

x_pos = [i for i, _ in enumerate(x)]

plt.bar(x_pos, horse_power, color='green')
plt.xlabel("Car Model")
plt.ylabel("Horse Power (Hp)")
plt.title("Horse Power for Selected Cars")

plt.xticks(x_pos, x)

plt.show()
\end{lstlisting}

You can customize plots further by using plt.style.use(), in pyhton
3. If you provide the following command inside a python command shell
you will see a list of available styles.

\begin{lstlisting}
print(plt.style.available)
\end{lstlisting}
 
An example of using a predefined style is shown below. 

\begin{lstlisting}
print(plt.style.use('seaborn'))
\end{lstlisting}

Up to this point we have only showcased how to display figures through
python output, however web browsers are a popular way to display
figures. One example is Bokeh, the following lines can be entered in a
python shell and the figure is outputted to a browser. 

\begin{lstlisting}
from bokeh.plotting import figure

x_values = [1, 2, 3, 4, 5]
y_values = [6, 7, 2, 3, 6]

p = figure()
p.circle(x=x_values, y=y_values)
\end{lstlisting}
