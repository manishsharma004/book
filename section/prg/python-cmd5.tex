\chapter{Cloudmesh Command Shell}
\label{C:python-cmd5}

\FILENAME\

\section{CMD5}\label{cmd5}

Python's CMD (\url{https://docs.python.org/2/library/cmd.html}) is a
very useful package to create command line shells. However it does not
allow the dynamic integration of newly defined commands. Furthermore,
additions to CMD need to be done within the same source tree. To
simplify developing commands by a number of people and to have a
dynamic plugin mechanism, we developed cmd5. It is a rewrite on our
earlier efforts in cloudmesh client and cmd3.

\subsection{Resources}

The source code for cmd5 is located in github:

\begin{itemize}

\item
  \url{https://github.com/cloudmesh/cmd5}
\end{itemize}

\subsection{Creating a Python Development
Environment}\label{creating-a-python-development-environment}

We recommend that you use a virtualenv either with virtualenv or
pyenv.  This is in detail documented in the
Section~\ref{S:managing-multiple-python-versions-with-pyenv}.

\subsection{Installation from source}\label{installation-from-source}

Cmd5 can be easily deployed with pip:

\begin{verbatim}
pip install cloudmesh.cmd5
\end{verbatim}

In case you would like to generate easily new cmd5 commands we also recommend
you install the cloudmesh sys command with:

\begin{verbatim}
pip install cloudmesh.sys
\end{verbatim}

In case you like to work with the source please clone the following
directories from github:

\begin{verbatim}
mkdir -p ~/github
cd ~/github

git clone https://github.com/cloudmesh/cloudmesh.common.git
git clone https://github.com/cloudmesh/cloudmesh.cmd5.git
git clone https://github.com/cloudmesh/cloudmesh.sys.git  

cd ~/github/cloudmesh.common
python setup.py install
pip install .

cd ~/github/cloudmesh.cmd5
python setup.py install
pip install .

cd ~/github/cloudmesh.sys
python setup.py install
pip install .
\end{verbatim}

The common directory contains some useful libraries, the cmd5 repository
contains the shell, while the sys directory contains a command to
generate extensions to cloudmesh.

\subsection{Execution}\label{execution}

To run the shell you can activate it with the cms command. cms stands
for cloudmesh shell:

\begin{verbatim}
(ENV2) $ cms
\end{verbatim}

It will print the banner and enter the shell:

\begin{verbatim}
+-------------------------------------------------------+
|   ____ _                 _                     _      |
|  / ___| | ___  _   _  __| |_ __ ___   ___  ___| |__   |
| | |   | |/ _ \| | | |/ _` | '_ ` _ \ / _ \/ __| '_ \  |
| | |___| | (_) | |_| | (_| | | | | | |  __/\__ \ | | | |
|  \____|_|\___/ \__,_|\__,_|_| |_| |_|\___||___/_| |_| |
+-------------------------------------------------------+
|                  Cloudmesh CMD5 Shell                 |
+-------------------------------------------------------+

cms>
\end{verbatim}

To see the list of commands you can say:

\begin{verbatim}
cms> help
\end{verbatim}

To see the manual page for a specific command, please use:

\begin{verbatim}
help COMMANDNAME
\end{verbatim}

\subsection{Create your own Extension}\label{create-your-own-extension}

One of the most important features of CMD5 is its ability to extend it
with new commands. This is done via packaged name spaces. We recommend
you name is cloudmesh.mycommand, where mycommand is the name of the
command that you like to create. This can easily be done while using the
\emph{sys} command:

\begin{verbatim}
cms sys command generate mycommand
\end{verbatim}

It will download a template from cloudmesh called cloudmesh.bar and
generate a new directory cloudmesh.mycommand with all the needed files
to create your own command and register it dynamically with cloudmesh.
All you have to do is to cd into the directory and install the code:

\begin{verbatim}
cd cloudmesh.mycommand
python setup.py install
pip install .
\end{verbatim}

Adding your own command is easy. It is important that all objects are
defined in the command itself and that no global variables be use in
order to allow each shell command to stand alone. Naturally you should
develop API libraries outside of the cloudmesh shell command and reuse
them in order to keep the command code as small as possible. We place
the command in:

\begin{verbatim}
cloudmsesh/mycommand/command/mycommand.py
\end{verbatim}

An example for the bar command is presented at:

\begin{itemize}

\item
  \url{https://github.com/cloudmesh/cloudmesh.bar/blob/master/cloudmesh/bar/command/bar.py}
\end{itemize}

It shows how simple the command definition is (bar.py):

\begin{verbatim}
from __future__ import print_function
from cloudmesh.shell.command import command
from cloudmesh.shell.command import PluginCommand

class BarCommand(PluginCommand):

    @command
    def do_bar(self, args, arguments):
        """
        ::
          Usage:
                command -f FILE
                command FILE
                command list
          This command does some useful things.
          Arguments:
              FILE   a file name
          Options:
              -f      specify the file
        """
        print(arguments)
\end{verbatim}

An important difference to other CMD solutions is that our commands can
leverage (besides the standard definition), docopts as a way to define
the manual page. This allows us to use arguments as dict and use simple
if conditions to interpret the command. Using docopts has the advantage
that contributors are forced to think about the command and its options
and document them from the start. Previously we did not use but argparse
and click. However we noticed that for our contributors both systems
lead to commands that were either not properly documented or the
developers delivered ambiguous commands that resulted in confusion and
wrong usage by subsequent users. Hence, we do recommend that you use
docopts for documenting cmd5 commands. The transformation is enabled by
the @command decorator that generates a manual page and creates a proper
help message for the shell automatically. Thus there is no need to
introduce a separate help method as would normally be needed in CMD
while reducing the effort it takes to contribute new commands in a
dynamic fashion.

\subsection{Exercises}

\begin{exercise}
\label{E:CMD5.1:}
Install cmd5 on your computer.
\end{exercise}

\begin{exercise}
\label{E:CMD5.2:}
Write a new command with your firstname as the command name.
\end{exercise}

\begin{exercise}
\label{E:CMD5.3:}
Write a new command and experiment with docopt syntax and argument
interpretation of the dict with if conditions.
\end{exercise}

\begin{exercise}
\label{E:CMD5.4:}
If you have useful extensions that you like us to add by default, please
work with us.
\end{exercise}

\begin{exercise}
\label{E:CMD5.5}
At this time one needs to quote in some commands the \verb|"| in the shell command
line. Develop and test code that fixes this.
\end{exercise}
