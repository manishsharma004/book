
\section{Getting Started}
\label{C:cc-start}

\FILENAME

We describe how you can get access to chameleon cloud under the
assumption that you are a student or a researcher that joins an
existing project on Chameleon cloud. You will need to follow the
following steps:

\subsection{Step 1: Create~a Chameleon~account}

To get started using Chameleon you will need to
\href{https://www.chameleoncloud.org/register}{create a user account}.

You will be asked to agree to the
\href{https://www.chameleoncloud.org/terms/view/chameleon-user-terms/}{Chameleon
  terms and conditions} which, among others, ask you to acknowledge
the use of Chameleon in your publications. 

Acknowledgement of support from the Chameleon project and the
National Science Foundation should appear in any publication of
material, whether copyrighted or not, that describes work which
benefited from access to Chameleon cyberinfrastructure resources. The
suggested acknowledgement is as follows: ``Results presented in this
paper were obtained using the Chameleon testbed supported by the
National Science Foundation''.

\begin{IU}
  As part of creating an account you may request PI status. However
  you are not a PI as you will be joining a project.
\end{IU}

\subsection{Step 2: Create or join a project}

To use Chameleon, you will need to be associated with a
\href{https://www.chameleoncloud.org/docs/user-faq/\#toc-how-do-i-apply-for-a-chameleon-project-}{project}
that is assigned an
\href{https://www.chameleoncloud.org/docs/user-faq/\#toc-what-are-the-project-allocation-sizes-and-limits-}{allocation}.
This means that you either need to

\begin{enumerate}

\item \textbf{\href{https://www.chameleoncloud.org/user/projects/new/}{apply
for a new project}} or 

\item
\textbf{\href{https://www.chameleoncloud.org/docs/user-faq/\#toc-my-pi-professor-colleague-already-has-a-chameleon-project-how-do-i-get-added-as-a-user-on-the-project-}{ask
the PI of an existing Chameleon project to add you}.}

\end{enumerate}

A project is headed by a project PI, typically
\href{https://www.chameleoncloud.org/docs/user-faq/\#toc-who-is-eligible-to-be-chameleon-pi-and-how-do-i-make-sure-that-my-pi-status-is-reflected-in-my-profile-}{a
faculty member or researcher scientist at a scientific institution}. If
you are a student we recommend that you ask your professor to work with
you on creating a project. Please note that you must not create a
project by yourself and that you indeed need to work with your
professor. 

In case you need to do a project application typically consists of
about one paragraph description of the intended research and takes one
business day to process.

Enrolling you into an existing research or class project depends on
the time availability of the project lead or professor of your
class. It is important that you communicate your chameleon cloud
account name to the project lead so they can easily add you. Make sure
you really give them only your chameleon account name and potentially
your organizational e-mail, Firstname, and Lastname so they can check
you are eligible to get access. 

\begin{IU}
Indiana University students that take the e516 and e616 classes will
have to fill out a google form in which they communicate the chameleon
cloud name. You can already apply for an account name, but do not
apply for a project. If you nevertheless apply for a project, we will
hear from the administrators and you will receive a point deduction.
\end{IU}

\subsection{Step 3: Start using Chameleon!}\label{using-chameleon}

Now that you have enrolled and once you are added to the project by your
project lead you can start using chameleon cloud. However be reminded
that you ought to shut down the resources/VMs whenever they are not in
use to avoid unnecessary charging. Remember the project has imited
time on chameleon and any unused time will be charged against the project.

Chameleon provides two types of resources with links to their respective
users guides below:

\textbf{\href{https://www.chameleoncloud.org/docs/bare-metal-user-guide-old/}{Bare
Metal User Guide}} will tell you how to use Chameleon bare metal
resources which provide strong isolation and allow you maximum control
(reboot to new operating system, reboot the kernel, etc.)

\textbf{\href{https://www.chameleoncloud.org/docs/user-guides/openstack-kvm-user-guide/}{OpenStack
KVM User Guide}} will tell you how to get started with Chameleon's
OpenStack KVM cloud which is a multi-tenant environment providing weak
performance isolation. 

If you have any questions or encounter any problems, you can check out
our \href{https://www.chameleoncloud.org/docs/user-faq/}{User FAQ},
or \href{https://www.chameleoncloud.org/user/help/}{submit a ticket}.

\begin{IU}
As part of the classes you will need to first pass a cloud \textit{security}
drivers licence test.  The test is designed so that you think about
gaining access to a VM securely and how to properly secure the
VM. Once passed, access is typically provided by midterm time. You are
not allowed to constantly run VM's and must shut them down if not in
use. You will get point deductions if we detect you do not obey by
this rule. We have access to log files about your VM usage.
\end{IU}

