\section{Charge Rates}
\label{C:cc-charge}

It is important to fully understand the charge rates of your VM and storage use.

Chameleon has two types of limitations, introduced to promote fair resource
usage to all:

\begin{description}

\item  [Allocation:] Chameleon projects are limited to a per-project allocation
  currently set to~20,000 service units for 6 months. Allocations can be
  renewed or extended (see
  \href{index.html\#toc-project-and-allocation-management}{Project and
  Allocation Management} section for more details on Chameleon
allocations.)

  \item [Lease:] To ensure fairness to all users, resource reservations (leases)
  are limited to a duration of 7 days. However, an active lease within
  48 hours of its end time can be prolonged by up to 7 days from the
  moment of request if resources are available. To prolong a lease,
  click on the ``Update Lease'' button in the Reservations panel of the
  CHI OpenStack dashboard, and enter the additional duration requested
  in the ``Prolong for'' box including the unit suffix, e.g. ``5d'' for
  5 days or ``30m'' for 30 minutes. If there is an advance reservation
  blocking your lease prolongation that could potentially be moved,
  you can interact through the users mailing list to coordinate with
  others  users.~Additionally, if you know from the start that your
  lease will  require longer than a week and can justify it, you can
  \href{https://www.chameleoncloud.org/user/help/ticket/new/}{contact
    Chameleon staff via the ticketing system} to request a one-time
  exception to create a longer lease. The lease must be requested by
  the PI.

\end{description}

\subsection{Service Units}

Chameleon allocations can consist of several components of the system.
Users can request allocation of individual compute nodes, storage
servers, or complete Scalable Compute Units (SCUs) which contain compute
servers, storage nodes, and an open flow switch.

Compute servers are allocated in Service Units (SUs), which equates to
one hour of wall clock time on a single server (for virtual machines, an
SU is 24 cores with up to 128GB of RAM). Note this unit differs from
traditional HPC or cloud service units that are charged in core-hours; a
Chameleon SU is a full server, as the type of experiments and
performance measurements users may wish to do may be contaminated by
sharing nodes.

Storage servers are also charged in SUs, at 2x the rate of compute
servers (i.e., 1 hour allocation of 1 storage server == 2 SUs). SCUs are
charged at the rate of 50 SUs per wall clock hour (42 compute servers, 4
storage nodes, plus one OpenFlow switch).

An allocation may make use of multiple SCUs, up to the size of the full
testbed.

For example, a user wishing to provision a 10 node cluster +1 storage
server for a 1 week experiment should budget
\texttt{{[}(10\ +\ 2)\ SUs\ per\ hour{]}\ *\ {[}7\ days\ *\ 24\ hours/day{]}\ =\ 2,016\ SUs}
for that experiment.

SUs are charged the same regardless of use case. Hence, whether asking
for bare metal access, virtual machine access, or use of default images,
the charge is the same --- you are charged for the fraction of the
resource your experiment occupies, regardless of the type of the
experiment.

The basic principle for charging service units for Chameleon resources
is to evaluate the amount of time a fraction of the resource is
unavailable to other users. If a reservation is made through the portal
for a particular date/time in the future, the user will be charged for
this time regardless of whether the reservation is actually used, as the
Chameleon scheduling system will have to drain the appropriate part of
the system to satisfy the reservation, even if the nodes requested are
not actually used. A reservation request may be cancelled in which case
no charges will apply.

\subsection{Project Allocation Size}

CUrrently Chameleon is operating on a ``soft allocation
model'' where each project, if approved, will receive a startup
allocation of 20,000 SUs for six months that can be both recharged
(i.e., more SUs can be added) and renewed (i.e., the duration can be
extended) via submitting a renew/recharge request. This startup
allocation value has been designed to respond to both PI needs (i.e.,
cover an amount of experimentation needed to obtain a significant
result) and balance fairness to other users (it represents roughly 1\%
of testbed six months' capacity). Requests for these startup projects
will receive a fast track internal review (i.e., users can expect them
to be approved within a few days).

A PI can apply for multiple projects/allocations; however, the number of
held allocations will be taken into account during review.

As our understanding of user need grows we expect the Chameleon
allocation model to evolve towards closer reflection of those needs in
the form of more differentiated allocations that will allow us to give
larger allocations to users for longer time.

\begin{IU}
  Please be mindful to shutting down your VMS when not in use as even VMs 
  that do not do any calculations get charged. In past classes we had students 
  that did not shut down their VMs and within 2 weeks used up all SUs 
  for the entire class of 70 students. We like to avoid this. In future cases 
  we will assign the grade ``F'' to such students, as is customary also 
  in other universities.
\end{IU}