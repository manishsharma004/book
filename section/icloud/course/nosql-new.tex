\chapter{NoSQL}
\label{c:nosql}

\FILENAME\

%  11 Video lectures (1 hour 26 minutes 8 seconds)

\section{RDBMS vs. NoSQL}

A discussion of relational database management systems (RDBMS)
compared to NoSQL data storage systems is presented. This discussion
includes an evolution of data storage systems, limitations with RDBMS
in terms of scalability and how NoSQL fits into the picture in terms
of big data. 
  
\video{Cloud}{9:22}{RDBMS vs. NoSQL}{https://www.youtube.com/watch?v=dJunqER9lb8}

\slides{Cloud}{Page 1}{RDBMS vs. NoSQL}{https://drive.google.com/open?id=0B88HKpainTSfaDFNbjNiMm44bnc}

\slides{Cloud}{Page 1}{RDBMS vs. NoSQL - pptx}{https://drive.google.com/open?id=0B88HKpainTSfNnQ5SEVKTm1tRk0}

\section{NoSQL Characteristics}

Clouds have arisen as an answer to the data demands of social media.
Three major programs for NoSQL are BigTable, Dynamo, and CAP theory.
More recently mongoDB has emerged as a leader in NoSQL databases and
will also be discussed. NoSQL is not meant to replace SQL, but to
tackle the large-data problems SQL is not well equipped to handle. SQL
ACID transactions are Atomic, Consistent, Isolated, and
Durable. Consistency can be either strong (ACID) or weak (BASE). CAP
theorem offers Consistency, Availability, and Partition tolerance,
only two of which can coexist for a shared-data system. NoSQL comes in
two varieties, each with pros and cons: Key-Value or
schema-less. Common advantages of NoSQL include their being open
source and fault tolerant. The number of NoSQL databases makes it
impossible to simply split them into Key-Value and schema-less
categories, instead they can generally be placed in one of the three
categories being document model, graph model and key-value and wide
column models. 


\video{Cloud}{10:31}{NoSQL Characteristics}{https://www.youtube.com/watch?v=BjtTDiKhqk8}

\slides{Cloud}{Page 11}{NoSQL Characteristics}{https://drive.google.com/open?id=0B88HKpainTSfaDFNbjNiMm44bnc}

\slides{Cloud}{Page 11}{NoSQL Characteristics - pptx}{https://drive.google.com/open?id=0B88HKpainTSfNnQ5SEVKTm1tRk0}

\subsection{Document Model}
Relational databases store data in rows and columns, the class of
NoSQL databases that fall into the document model store data as
documents. Typically these documents are in JSON format and provide a
straight forward way to model data that parallels object oriented
programing where each document is an object. Each document stores one
or more fields and within the field a value is stored in the form of
an array, string date etc. In comparison to relational databases that
distribute the records across columns that are connected with keys the
document model sotres records and their associated data in a single
document. A key advantage to this model is that is reduces the need
for JOIN operations that can be computationally expensive. Two
examples of document databases are MongoDB and CouchDB. 

\subsection{Graph Model}
As the name implies graph databases exploit the properties of graph
structures and represent data through nodes and edges. Graph databases
are useful for exploring the relationships of the components that make
up an application by representing the data as a network of
relationships. Graph databases can be useful for exploring social
network connections, topologies of networks and supply chains. Two
examples of graph databases are Neo4j and Graph.  

\subsection{Key-Value and Wide Column Models} 
Key-value stores are the most basic type on NoSQL databases as every
item is stored as an attribute name or key paired with its value. In
this model the value is not interpretable by the system as data can
only be queried by the key. Key-value type databases do not enforce a
set schema for key-value pairs making these database useful for
unstructured and polymorphic data. In contrast to key-value stores,
wide-column store data in distributed variable-dimensional sorted
map where each record is stored as columns. These columns can be
grouped into families and or columns can be distributed to several
column families. Each column family has a primary key that can be
queried in order to retrieve the data.  

\URL{https://www.mongodb.com/collateral/top-5-considerations-when-evaluating-nosql-databases}


\section{BigTable}

Big Table is a key-value NoSQL model with data arranged in rows and
columns. It is composed of Data File System, Chubby, and SSTable. A
tablet is a range of rows in BigTable. The master node assigns tablets
to tablet servers and manages these servers. Memory is conserved by
making SSTables and memtables compact. BigTable is used in features of
Google like their search engine and Google Earth.

\video{Cloud}{6:55}{BigTable}{https://www.youtube.com/watch?v=JAlz9AI5I-M}

\slides{Cloud}{Page 28}{BigTable}{https://drive.google.com/open?id=0B88HKpainTSfaDFNbjNiMm44bnc}

\slides{Cloud}{Page 28}{BigTable - pptx}{https://drive.google.com/open?id=0B88HKpainTSfNnQ5SEVKTm1tRk0}

\section{HBase}

HBase is a NoSQL core component of the Hadoop Distributed File System.
It is a scalable distributed data store. A timeline of HBase and
Hadoop is shown. BigTable still has its uses but does not scale well
to large amounts of analytic processing. HBase has a row-column
structure similar to BigTable as well as master and slave nodes. Its
place in the architecture of HDFS is shown in a diagram.

\video{Cloud}{7:37}{HBase}{https://www.youtube.com/watch?v=i-ibhuVs-ck}

\slides{Cloud}{Page 44}{HBase}{https://drive.google.com/open?id=0B88HKpainTSfaDFNbjNiMm44bnc}

\slides{Cloud}{Page 44}{HBase - pptx}{https://drive.google.com/open?id=0B88HKpainTSfNnQ5SEVKTm1tRk0}

\section{HBase Coding}

This video gives an overview of the code used in the installation of
HBase and connecting to it.

\video{Cloud}{HBase Coding}{4:30}{https://www.youtube.com/watch?v=KbFMpYRBTtU}

\slides{Cloud}{Page 60}{HBase Coding}{https://drive.google.com/open?id=0B88HKpainTSfaDFNbjNiMm44bnc}

\slides{Cloud}{Page 60}{HBase Coding -
  pptx}{https://drive.google.com/open?id=0B88HKpainTSfNnQ5SEVKTm1tRk0}

\section{Draft: MongoDB}
MongoDB belongs to the documnet model described above. Within a
typical MongoDB server there are often multiple databases. Each
database is a physical container of collections with its own set of
files within the file system. A collection is a set of MongoDB
documents similar to a typical relational database table. In MongoDB
databases the collections are not required to follow a schema allowing
documents in the same collection to have different fields. Some of
the main advantages of MongoDB are its ease of scalability, lack of
complex joins, schema-less, use of internal memory and the support for
dynamic queriers. 

\URL{https://www.tutorialspoint.com/mongodb/mongodb_quick_guide.htm}



\section{Indexing Applications}

The setup of a search engine is briefly discussed using a diagram that
contains the core components of a search engine. Google's search engine
contains three key technologies: Google File System, BigTable, and
MapReduce. However, research into big data remains difficult owing to
the scope of its size. Social media data in particular is a huge source
of data with numerous subsets, all of which demands specific approaches
in terms of search queries. There are three stages to this approach:
query, analysis, and visualization.

\video{Cloud}{9:33}{Indexing Applications}{https://www.youtube.com/watch?v=MxgabfoGH-M}

\slides{Cloud}{Page 1}{Indexing Applications}{https://drive.google.com/open?id=0B88HKpainTSfWUh6dVNHcXloSnc}

\slides{Cloud}{Page 1}{Indexing Applications - pptx}{https://drive.google.com/open?id=0B88HKpainTSfZkJpLTNIbDJ1dVU}

\section{Related Work}

Indexing improves efficiency in querying data subsets and analysis.
Indices can be single (B+, Hash) or multi-dimensional (R, Quad). Four
databases which utilize indexing are HBase, Cassandra, Riak, and
MongoDB. Current indexing strategies have limits; for instance, they
cannot support range queries or only retrieve Top `n' most relevant
topics. Customizability of indexing among NoSQL databases is desirable.

\video{Cloud}{5:56}{Related Work}{https://www.youtube.com/watch?v=NDjAdFSVzxo}

\slides{Cloud}{Page 11}{Related Work}{https://drive.google.com/open?id=0B88HKpainTSfWUh6dVNHcXloSnc}

\slides{Cloud}{Page 11}{Related Work - pptx}{https://drive.google.com/open?id=0B88HKpainTSfZkJpLTNIbDJ1dVU}

\section{Indexamples}

Mapping between metadata and raw index data is the essential issue
with indexing. Examples are shown for HBase, Riak, and MongoDB. An
abstract index structure contains index keys, entry IDs among multiple
entries, and additional fields. Index configuration allows for
customizability through choice of fields, which can be anything from
timestamps, text, or retweet status.

\video{Cloud}{8:35}{Indexamples}{https://www.youtube.com/watch?v=Ec3VFeTGuo8}

\slides{Cloud}{Page 15}{Indexamples}{https://drive.google.com/open?id=0B88HKpainTSfWUh6dVNHcXloSnc}

\slides{Cloud}{Page 15}{Indexamples - pptx}{https://drive.google.com/open?id=0B88HKpainTSfZkJpLTNIbDJ1dVU}

\section{Indexing 101}

User-defined index allows a user to select the fields used in their
search. Data records are indexed or un-indexed. Index structure is
made up of key, entry ID, and entry fields. A walk-through customized
index creation is shown on HBase, called IndexedHBase. HBase is suited
to accommodate the creation of index tables. A performance test of
IndexedHBase is done on the Truthy Twitter repository, displaying the
various tables that can be created with different criteria. Loading
time for large-scale historical data can be reduced by adding nodes.
Streaming data can be handled by increasing loaders. A comparison of
query evaluation is made between IndexedHBase and Riak, with Riak
being more efficient with small data loads but IndexedHBase proving
superior for large-scale data.

\video{Cloud}{9:53}{Indexing 101}{https://www.youtube.com/watch?v=eKQaLkw-HBU}

\slides{Cloud}{Page 20}{Indexing 101}{https://drive.google.com/open?id=0B88HKpainTSfWUh6dVNHcXloSnc}

\slides{Cloud}{Page 20}{Indexing 101 - pptx}{https://drive.google.com/open?id=0B88HKpainTSfZkJpLTNIbDJ1dVU}

\section{Social Media Searches}

The Truthy Project archives social media data by way of metadata memes.
Some problems faced in analyzing this data include its large volume,
sparsity of information in tweets, and attempting to arrange streaming
tweets. Apache Open Stack upgrades Hadoop 2.0 with YARN and a new HDFS.
A diagram displays an indexing setup for social media data with YARN.

\video{Cloud}{6:19}{Social Media Searches}{https://www.youtube.com/watch?v=a3tcL-Qw9to}

\slides{Cloud}{Page 28}{Social Media Searches}{https://drive.google.com/open?id=0B88HKpainTSfWUh6dVNHcXloSnc}

\slides{Cloud}{Page 28}{Social Media Searches - pptx}{https://drive.google.com/open?id=0B88HKpainTSfZkJpLTNIbDJ1dVU}

\section{Analysis Algorithms}

Another method of use for inverted indices is in analysis algorithms.
The mathematics involved in this is explored, as well as how it
relates to index data, mapping, and reducing. Rather than scanning all
raw data present, indices allow for searching only the relevant
data. An example is given illustrating how this decreases the time
needed to search hashtags in Twitter.

\video{Cloud}{6:57}{Analysis Algorithms}{https://www.youtube.com/watch?v=MxoMd4mdshE}

\slides{Cloud}{Page 35}{Analysis Algorithms}{https://drive.google.com/open?id=0B88HKpainTSfWUh6dVNHcXloSnc}

\slides{Cloud}{Page 35}{Analysis Algorithms - pptx}{https://drive.google.com/open?id=0B88HKpainTSfZkJpLTNIbDJ1dVU}

