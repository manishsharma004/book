

\section{Contributing}\label{S:contributing}
\index{Contributing}

\FILENAME

The document is managed in \LaTeX and is stored on github. Thus
creating of the document is rather simple.

\subsection{Requirements}

We assume you are using a an OS on which you have installed and can
execute pdflatex. Operating systems such as Linux, OSX, or Windows can
be set up for it. On WIndows we recommend that you also install
bitbash and have  make installed.

We assume that you are familiar with git and have git installed.

\subsection{Setting up Git}

These instructions assume that you are using git from the commandline.
For Linux or OSX, please follow our instructions on how to install it
which is provided as part of the lessons.

If you use windows you can install
git for windows \url{https://git-for-windows.github.io/}
and than you can open up Gitbash in your classes directory.

It is important that you have set up git correctly, otherwise your
pull requeste may not show up properly.  First you need to initialize
your directory for use with Git:

\begin{verbatim}
git init
\end{verbatim}

Next you need to set your user name and email to get credit for your
work:

\begin{verbatim}
git config --global user.name "Firstname Lastname"
git config --global user.email yourusername@iu.edu
\end{verbatim}


Where you need to replace the firstname, lastname and the email with
your information. Next you need to pull the information down from your
forked repository.  You will need to enter your Github username and
password when prompted:


\subsection{Adding Content}

First, you have to \textbf{fork} the repository while going to the link:

  \URL{https://github.com/cloudmesh/book}

and press the fork icon. Now you can clone or directly manipulate your
fork from the web browser. 

\begin{verbatim}
git clone https://github.com/cloudmesh/book
\end{verbatim}

Next we need to swithc into the development branch called \textit{dev}

Using the GITUSERNAME will gurantee that you first If you clone, you need to make sure you clone
from your fork.

Once you have cloned and forked, you will find in the directory
\verb|book| the file:

\begin{verbatim}
Makefile
\end{verbatim}

This makefiles has som convenient targets include that allow you to
create the handbook. This is achieved most easily with

\begin{verbatim}
make g
\end{verbatim}


\begin{verbatim}
git pull https://github.com/username/classes
\end{verbatim}

With all the files downloaded to your local directory, you can begin
editing the rst files with a plain text editor.

When you have changes to commit to the repository, you will first have
to set the origin for the changes. You only need to do this the first
time you commit changes, so do not worry about this step every time you
push your changes:

\begin{verbatim}
git remote add origin https://github.com/username/classes.git
\end{verbatim}

Now you are ready to add the files you changed:

\begin{verbatim}
git add -A
\end{verbatim}

Then commit the changes with a meaningful comment explaining what you
did:

\begin{verbatim}
git commit -m "A message indicating what you changed"
\end{verbatim}

Finally you can push your changes up to Github:

\begin{verbatim}
git push -u origin master
\end{verbatim}

Once you have done that, open up a browser and go to your forked project
on Github. When you have verified that the changes are there, you can
issue a pull request for your work to be integrated into the original
repository by clicking the \emph{Pull Request} field in the right hand
corner beneath the topics.

\subsection{Adding bibligraphic references}\label{adding-bibligraphic-references}

We use a single refrence file formated in \emph{bibtex}. The file is
located in docs/source/refs.bib. We use sphinx-bibtex to manage the
refernces in rst pages. However, it has the disadvantage that we can
only have on refrence section per bib file. In order to simplify
management as we may cite the same refernces in differnt files it is not
useful to develop a bib file for each rst file. INstead we developed the
following pragmatic approach.

Here is the process that we automatically apply in the \emph{Makefile}

\begin{itemize}

\item
  We identify all rst files that have the cite command in it.
\item
  We copy the refs.bib file into the same directory as the originating
  rst file. Lets assume the file is called \emph{filename.rst}
\item
  We than rename in the file for this directory to
  \emph{filename-tmp.bib} indicating it is not checked into github
\item
  In gitignore we have a rule that excludes them
\end{itemize}

The only thing you have to do in order to use refernces is to make the
index unique and to add a section such as the following to your
document:

\begin{verbatim}
References
----------

.. bibliography:: filename-tmp.bib
   :style: unsrt
   :cited:
   :labelprefix: filename
\end{verbatim}

Please note the unique filename. To properly geberate the citations we
remommend you use:

\begin{verbatim}
make clean; make
\end{verbatim}

\subsection{Exercise}\label{exercise}

To do this assignment you need to learn about making small changes and
how to document them with `gitchangelog`:

\begin{itemize}
\item
  if you fix a spelling error: your commit message must:

\begin{verbatim}
chg:dev: corrected spelling error
\end{verbatim}
\item
  if you try to add a larger text you added real content, you can use
  ``chg:usr: added section about how to do this and that and the other''
  where you replace this and that and the other with what you actually
  contributed
\end{itemize}

\begin{description}
\item[do always create small pull request. They are easier to]
accept. If you create many different changes in many different locations
in files the likelihood is that we reject the pull request and you have
to split it up and redu, costing time on your side. This is part of
learning how we use git.
\end{description}

\begin{description}
\item[EContrib.0:]
If you find an md file and an rst file with the same prefix, which filed
should you edit?
\item[EContrib.1:]
Identify a spelling error on the web page or another item to improve.
Fork the Web page, fix the error and create a pull request.
\item[EContrib.2:]
Identify a section that is not covered by this material, but could be
useful. Add such a section and create a pull request so your
contribution can be added. Work with others that review your section
before submitting so we make sure no one else is working on this
already. If they do we bring you in contact with them.
\item[EContrib.3:]
How do you clone from your fork? What is the difference between your
fork and the main repository? How do you identify it is your fork you
clone from?
\item[EContrib.4:]
We use the creation of the class Web site on your computer to benchmark
your machine. This benchmark will be used as part of some class
assignments. To do so execute the following and write down/copy the
times you get:

\begin{verbatim}
make clean
time make
\end{verbatim}

You will see something like:

\begin{verbatim}
real   2m36.662s
user   2m34.473s
sys    0m1.467s
\end{verbatim}

Now we want you to run it again after you touched a file:

\begin{verbatim}
touch docs/source/faq.rst
\end{verbatim}

Now rerun the timed make. You will see an output such as:

\begin{verbatim}
real   0m27.853s
user   0m27.394s
sys    0m0.334s
\end{verbatim}

The only thing we are interested in is the time behind real, as well as
some information about your computer, e.g.:

\begin{verbatim}
computer: MacBook Pro, 15-in, 2016, 2.9GHz, 16GB, 2133Mhz, LPDDR3
make clean: 2m36.662s
make update: 0m27.853s
python: 3.6.2
\end{verbatim}

We will post a form in which you can enter your information. We found
that we can use this information to check if you may have an issue with
your computer or your setup.
\item[EContrib.5:]
Why do we ask you to do multiple pull requests? What could be the
consequence if you make hundrets of changes in one pull request?
\end{description}
