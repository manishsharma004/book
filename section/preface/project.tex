
\section{Tutorials, Topic Paper, Term Paper, Project Report}

\FILENAME

Dependendt on the class you need to do different assignments. The
assignments will be clearly posted in this document and updated in
case clarification is needed. 

We use the following terminology:

\begin{description}

\item[Tutorials:] Tutorials are written in markdown, RST, or LaTeX and
  include information on a particular technical issue that is in
  general helpful for other students. Tutorials can be small, but sume
  may need to be substential. As we expect that the tutorials can be
  included in the Handbook, please be careful of plagiarizm and do not
  just copy the tutorial from elswhere. 

\item[Topic Paper:] A topic paper, or short paper is a smapp paper
  about a technology, application, or useful information that provides
  an overview of what you are trying to describe and analyses its
  relatinship to the class topic. Be mindful about plagiarizm. The
  paper is written in \LaTeX and uses jabref for bibliography management. 

\item[Term Paper:] A term paper is an enhanced topic paper. The
  difference is in length. Comparative or review papers can also be
  term papers.  Term papers should have the quality to be publishable
  either in a workshop or as part of the handbook.

\item[Project Paper:] A project reportis an enhanced topic paper that
  includes not just the analysis of a topic, but an actuall code, with
  benchmark or demosntarted appliction use. Obviously it is longer
  than a paper and includes descriptions about reproducability of the
  application. Term papers should have the quality to be publishable
  either in a workshop or as part of the handbook.

\end{description}

Examples from prior classes are avalable in the class proceedings.

\TODO{include link to Results section}

Dependent on the class you have to fulfill different
requirements. Please make sure you understand which requirement you
will have.

\begin{description}

\item[E516 and E616] In these classes you will need to produce
  tutorials, topic papers and a project report with real code.

\item[I524] same as E516 and E616, but you have the choide to
  substitute the project report with a term paper. There is a penalty
  though and we encourage studnets to do the project.

\end{description}

Please be aware that the project or term paper constitute to
\textbf{60\%} of your class grade. You have plenty of time to make
this choice and if you find you struggle with programming you may want
to consider a term paper instead of a project.

In case you chose a project paper your maximum grade for the entire class
could be an A+. However, an A+ project must be truly outstanding and
include an exceptional project report. Such a project and report will
have the potential quality of being able to be published in a
conference.

In case you chose a term Paper your maximum grade for the {\em entire} class
will be an A-.

Please note that a project includes writing a project paper.  However
the length is a bit lower than for a term paper.

\subsection{Team}

Software projects and term papers can be conducted with one, two or
three class members. We do not allow more than three members in a
project. It will be up to you to determine a team, but we recommend
that you chose wisely. Naturally if a team member does not contribute
to the project you need to address this early on. Please do not come
to us a week before the deadline is due and say a team member has not
contributed, this is far to late to do any adjustment to the team. It
is in your responsibility to manage the team.

\subsection{Common Deleiverables}

Both Projects and Term paper have the following common deliverables

\begin{description}
\item[Work Breakdown:]
This is an appendix to the document that describes in detail who did
what in the project. This section comes in a new page after the
references. It does not count towards the page length of the document.
It also includes explicit URLs to the the git history that documents the
statistics to demonstrate not only one student has worked on the
project. If you can not provide such a statistic or all checkins have
been made by a single student, the project has shown that they have not
properly used git. Thus points will be deducted from the project.
Furthermore, if we detect that a student has not contributed to a
project we may invite the student to give a detailed presentation of the
project.
\item[Bibliography:]
All bibliography has to be provided in a jabref/bibtex file. This is
regardless if you use LaTeX or Word. There is \textbf{NO EXCEPTION} to
this rule. PLease be advised doing references right takes some time so
you want to do this early. Please note that exports of Endnote or other
bibliography management tools do not lead to properly formatted bibtex
files, despite they claiming to do so. You will have to clean them up
and we recommend to do it the other way around. Manage your bibliography
with jabref, and if you like to use it import them to endnote or other
tools. Naturally you may have to do some cleanup to. If you use LaTeX
and jabref, you have naturally much less work to do. What you chose is
up to you.
\item[Report Format:]
All reports will be using the our common format. This format is not the
same as the ACM format, so if you use systems such as overleaf or
sharelatex, you need to upload it and use it there.

The format for LaTeX and Word found here:

  \URL{https://github.com/bigdata-i523/sample-hid000/tree/master/paper1}

\end{description}

There will be \textbf{NO EXCEPTION} to this format. In case you are in a
team, you can use either github while collaboratively developing the
LaTeX document or use MicrosoftOne Drive which allows collaborative
editing features. All bibliographical entries must be put into a
bibliography manager such as jabref, endnote, or Mendeley. This will
guarantee that you follow proper citation styles. You can use either ACM
or IEEE reference styles. Your final submission will include the
bibliography file as a separate document.

Documents that do not follow the ACM format and are not accompanied by
references managed with jabref or endnote or are not spell checked will
be returned without review.

\TODO{Integrate the format from the class web page into the LateX
  section More details about the format can be found at
  \url{https://cloudmesh.github.io/classes/lesson/doc/report.html} }

\subsection{Project Paper}

\subsubsection{Systems Usage}

Projects may be executed on your local computer, a cloud or other
resources you may have access to. This may include:

\begin{itemize}
\item chameleoncloud.org
\item furturesystems.org
\item AWS (you will be responsible for charges)
\item Azure (you will be responsible for charges)
\item virtualbox if you have a powerful computer and like to prototype
\item other clouds, please confirm with us.
\end{itemize}

Access to clouds must be scripted and a cmd5 extension must be
developed as part of your project to receive full credit.

\subsubsection{Deliverables}

The following artifacts are part of the deliverables for a project

\begin{description}
\item[Code:]
You must deliver the \textbf{source code} in github. The code must be
compilable and a TA may try to replicate to run your code. You MUST
avoid lengthy install descriptions and everything must be installable
from the command line. We will check submission. All team members must
be responsible for one or all parts of the project.

Code repositories are for code, if you have additional libraries that
are needed you need to develop a script or use a DevOps framework to
install such software. Thus zip files and .class, .o files are not
permissible in the project. Each project must be reproducible with a
simple script. An example is:

\begin{verbatim}
git clone ....
make install
make run
make view
\end{verbatim}

Which would use a simple make file to install, run, and view the
results.  You are expected to integrate cmd5, which we teach in
class. In addition you can use or are expected to us DOCKERFILES,
ansible, or shell scripts. It is not permissible to use GUI based
DevOps preinstalled frameworks. Everything must be installable and
reproducable form the command line.

\item[Data:] Data is to be hosted on IUs google drive if needed. If
  you have larger data, it should be downloaded from the internet. It
  is in your responsibility to develop a download program,
\item[Project Report:] A report must be produced while using the
  format discussed in the Report Format section. The following length
  is required:

\begin{itemize}
\item
  6 pages, one student in the project
\item
  8 pages, two students in the project
\item
  10 pages, three students in the project
\end{itemize}

\item[License:] All projects are developed under an open source
  license such as Apache 2.0 License. You will be required to add a
  LICENCE.txt file and if you use other software identify how it can
  be reused in your project. If your project uses different licenses,
  please add in a README.rst file which packages are used and which
  license these packages have.
\end{description}

\subsection{Term Paper}

In case you chose the term paper, you or your team will pick a topic
relevant for the class. You will write a high quality scholarly paper
about this topic. The following artifacts are part of the deliverables
for a term paper. A report must be produced while using the format
discussed in the Report Format section. The following length is
required:

\begin{itemize}
\item 8 pages, one student in the project
\item 10 pages, two student in the project
\item 12 pages, three student in the project
\end{itemize}


\section{Grading}\label{grading}

Grading for homework will be done within reasonable time of the
submission if the submission was on time. This however could still
take multiple weeks. Students that miss the deadline will be graded on
best effort, which could mean at the end of the semester as TAs will
grade first papers that have been handed in on time.  A 10\% grade
reduction will be given for residential students if the project is
late.  Some homework can not be delivered late (which will be clearly
marked and 0 points will be given if late; these are mostly related to
setting up your account and communicating to us your account names.)

It is the student's responsibility to upload submissions well ahead of
the deadline to avoid last minute problems with network connectivity,
browser crashes, cloud issues, etc. It is a very good idea to make early
submissions and then upload updates as the deadline approaches; we will
grade the last submission received before the deadline.

Note that the term paper or project paper will take a considerable
amount of time and doing proper time management is a must for this
class. Avoid starting your project late. Procrastination does not pay
off. Starting a paper a day or even in the week before the deadline
will allow you not to achieve your best. Late Projects or term papers
will receive a 10\% grade reduction.

For E516 I524, E616 the grading scene is

\begin{itemize}
\item 5\% Tutorial 1
\item 5\% Tutorial 2
\item 20\% Paper 1
\item 60\% Term Paper or Project
\item 10\% Participation/Discussion
\end{itemize}

All other assignments are pass/fail assignments and are geared towards
you being able to explore rather than you being bound by small
assignments.

\subsection{Grades on Canvas}

The final grade for your class is {\bf NOT} accurately posted in
CANVAS. Please visit the registrar for your final grade.
We have run many times into issues with CANVAS thus we try to stay as
much as possible away form it. We know that the total grade will not
be accurately reported in CANVAS.

Furthermore we are using only a letter grading scheme that
distinguishes qualitatively between grades. Typically we do not engage
in arguing if you get a point more or less. Instead we look at the
artifact and decide if it is an A or A- and so on.

Once the due date is passed all incomplete assignments will appear as
an {\em F} in CANVAS. Once we receive and review the submission this grade will
be changed. Please, do not cintact us if you submitted late and you
see an F in CANVAS temportrarily.

\subsection{Discussion about Grades}

Should it be necessary a discussion about a grade must not be taking
place via e-mail nor via piazza. YOu must use the CANVAS message
feature as we want to make sure that by accident you post information
to others that you do not intend to. FOr this reason we will not read
such messages on pizza and in our e-mail and deleted them without reading.
