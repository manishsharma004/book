\FILENAME

\section{About}\label{about}

This document is managed on \verb|github.com| at 

\URL{https://github.com/cloudmesh/book/}

The current release version is held in the master branch.
Development versions will be held under a number of branches:

\begin{description}
\item[e516] Branch with contributions from students of the e616
  class. Merges to and from the {\em latex} branch will be conducted
  on a daily bases by TA's.
\item[e616] Branch with contributions from students of the e616
  class. Merges to and from the {\em latex} branch will be conducted
  on a daily bases by TA's.
\item[latex] Branch managed by Gregor and the TA's
\item[master] Branch that contains the current released version. This
  version is updated once a week from the branch {\em latex}.
\end{description}

Contributions are to be conducted as pull requests. It is important to
keep the pull requests small and on a section or even paragraph
base. This helps avoiding conflicts at time of checkin and is a common
practice in large communities. It is not a good practice to work for
weeks on improvements and than issue the pullrequest. For sure things
will have changed and it will take you a long time to catch up.

The document is based on selected material published at the following
Web page

\URL{https://cloudmesh.github.io/classes/}


It is part of a classes taught at Indiana University. The class
communication takes place in piazza and you need to enroll in it via
CANVAS where you will find a panel for it.

The PDF version will be made in future available at 

\URL{https://github.com/laszewski/laszewski.github.io/raw/master/papers/vonLaszewski-bigdata.pdf}

This PDF document will be updated based on feedback from the students
and once we have now material available. 

\section{Citation}

The bibtex entry for this document is

\begin{verbatim}
@TechReport{las17handbook,
  author =       {Gregor von Laszewski and Geoffrey C. Fox and Judy Qui},
  title =        {Handbook of Clouds and Big Data -- Theory and Practice},
  institution =  {Indiana University},
  year =         {2018},
  OPTtype =      {Draft},
  address =      {Smith Research Center, Bloomington, IN 47408},
  month =        jan,
  url={https://github.com/laszewski/laszewski.github.io/raw/master/papers/vonLaszewski-bigdata.pdf},
}
\end{verbatim}

\section{Contributing}
\index{Contributing}

\subsection{Pull requests}

It is easy to contribute to this document and we invite everyone to
improve the material. To do so you need to fork the repository from 

\URL{https://github.com/cloudmesh/book/}

and clone it. Then you can modify information in the different files
or add new sections. It is important that you make changes based on
sections and than for them create a new pull request. This simplifies
the review process. We will typically want only one file to be
changed. Aslo before you issue your pull request make sure that no one
else has already made changes. In that case we ask you to integrate
them into your document.


\section{Results}\label{S:results}

In this section we list documents about BigData technologies and
Projects that have been delivered by students of past classes. As we
can not post any grades it is importan that you yourself identify good
examples from the published list. It is important not to make the
mistake to work towards minimal fulfillment of the class requirements,
but instead work towards achieving your best. All documents produced
by this class will be made available on github.com and through this
document. This helps other students to learn from your experience and
to counteract plagiarism. 

\begin{NOTE}
  Examples provided here may not necessarily meet the requirements for
  your current class as the content and requirements have changed
  since the other classes were thought. This includes format of the
  paper, paper length, as well as the topic for projects.
\end{NOTE}

\subsection{Introduction to Big Data Applications and Analytics}

This class was known under the name I523.

% this document needs to be deleted as it is temporary
% \URL{https://github.com/laszewski/laszewski.github.io/raw/master/papers/vonLaszewski-i523.pdf}

Application and Technologies (Vol.1):
\URL{https://github.com/laszewski/laszewski.github.io/raw/master/papers/vonLaszewski-i523-v1.pdf}

Application and Technologies (Vol. 2):
\URL{https://github.com/laszewski/laszewski.github.io/raw/master/papers/vonLaszewski-i523-v2.pdf}

Project: First 500 pages:
\URL{https://github.com/laszewski/laszewski.github.io/raw/master/papers/vonLaszewski-i523-v3-1.pdf}

Project: Second part, starting past page number501:
\URL{https://github.com/laszewski/laszewski.github.io/raw/master/papers/vonLaszewski-i523-v3-2.pdf}

\subsection{Big Data Applications and Open Source Software}

This class was known under the name I524

Big Data Software Vol 1.:
\URL{https://github.com/cloudmesh/sp17-i524/blob/master/paper1/proceedings.pdf}

Big Data Software Vol 2.:
\URL{https://github.com/cloudmesh/sp17-i524/blob/master/paper2/proceedings.pdf}

Big Data Projects:
\URL{https://github.com/cloudmesh/sp17-i524/blob/master/project/projects.pdf}


\subsection{Contributors}

We like to acknowledge the following contributors that helped on this
document. Please notify us with your name and a brief commend on what
you contributed:

Descriptions provided in Section \ref{s:} wer contributed by the
following people that are either listed by full name or their
github.com id:

\begin{quotation}{\em
Abhijit Thakre, Abhishek Gupta, Abhishek Naik, Ajit Balaga, Anurag
Kumar Jain, Avadhoot Agasti, Badi' Abdul-Wahid, Cmbays, DIKSHA,
Dimitar Nikolov, Govind, Govind Mishra, Grace Li, Gregor von
Laszewski, Harshit Krishnakumar, Hyungro Lee, Jerome Mitchell, Jimmy
Ardiansyah, Jon, Jon Montgomery, Jordan Simmons, Juliette Zerick,
Karthik, Kumar Satyam, Mark McCombe, Matthew Lawson, Methkupalli
Vasanth, Miao Jiang, Miao Zhang, Milind Suryawanshi,
MilindSuryawanshi, Nandita Sathe, Naveen, Niteesh01, Piyush Rai,
Piyush Shinde, Prashanth, Pratik Jain, Rahul Raghatate, Rahul Singh,
Ribka Rufael, Ronak Parekh, Saber Sheybani, Sabyasachi Roy Choudhury,
Sagar Vora, Sahiti Korrapati, Scott McClary, Sean Shiverick,
SilviaKarim14, Sivaprasad Sushmita, Snehal Chemburkar, Sowmya Ravi,
Srikanth Ramanam, Sunanda Unni, SushmitaSivaprasad, Tony Liu, Vasanth
Methkupalli, Veera Marni, Vibhatha Abeykoon, Vibhatha Lakmal Abeykoon,
Vishwanath Kodre, William H Knapp III, acastrob, ak.15, alyez,
anveling, argetlam115, athakre, bhavesh37, cacoulte, cglmoocs,
elenadesigner, eunosm3, harkrish1, jemitchell, justbbusy, jzerick,
kartanba, karthick, karthick venkatesan, karthik-anba, kpvenkat,
ksrivatsav, lmundia, miaozhan, michaelsmith1983, mmccombe, nsathe,
piyurai, pratik11jain, ronak1182, sabyasachi087,
shah0112, sriramsitharaman, suunni, tifabi, tonythomascn, vasanth,
vibhatha, vkodre, vlabeyko, xl41, yatinsharma7
}\end{quotation}


\section{Conventions}
\index{Convention}

\subsection{Videos}

Videos to the class are referred to with embedded links into the PDF
document as follows: 

\video{About}{25:36}{Test Video}{https://www.youtube.com/watch?v=yC3PNkb_9mI}

An index will also be available in the index page
that lists on which page the video has been added.

\subsection{Slides}

Sides
\slides{About}{10}{Test slides}{PUT URL HERE}

\subsection{Images}

The video icon was copied from \url{http://www.freeiconspng.com/img/8039}.

\subsection{URLs}

The online version of this document contains a significant number of
links. The links are either embedded or are directly visible. The
color of the links is blue.

\begin{description}
\item[Direct URL:] This is an example for a
  \url{https://github.com/cloudmesh/book/}
\item[Embedded URL:] This is an example for an embedded URL that
  points to the \href{https://github.com/cloudmesh/book/}{source on github}

\end{description}

\section{Exercise}

\begin{description}
\item[Excersise.1:] Inspect the PDF documents produced by brevious
classes. Note the differences between technology and application
reviews and projects. 
\end{description}