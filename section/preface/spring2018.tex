
\chapter{Course Policies}\label{C:course-2018}

\FILENAME
\
We describe briefly how we manage different classes and how to
interact with us. 

\section{Communication and Use of CANVAS}

In the past we have found numerous limitations to CANVAS that makes it
not possible for us to use it effectively for our classes. Such
limitations were not overcome even after consulting our professional
course support team at IU although we spend days and weeks trying to
fix the issues we encountered. From the professional staff we have
been given the advice to avoid using CANVAS and use a better appoach
that we outline as follows:

\begin{itemize}

\item we use CANVAS notifications only for the initial class setup and
  notify you how to gain access to piazza where all class discussions
  take place;

\item any discussion about grades is however only done in CANVAS;

\item we will DELETE all e-mail send to us on personal or IU e-mail
  addresses. All discussions need to take place either in private or
  in public posts on piazza (other than grades which need to be send
  via CANVAS as the previous point explained);

\item TA's are strictly forbidden to answer any e-mail that is not send
  via piazza. Their answers will only be posted within piazza;

\item any post to piazza to a TA must be not done to the individual TA
  but to {\em instructors};

\item any post from piazza by a TA must be not done to the individual
  student only, but also must include {\em instructors};

\end{itemize}

\section{Online  and Office Hours}

To support you we have established an open policy of sharing
information not only as part of the class material, but also as part
of how we conduct support. We establish the following principals:

\begin{itemize}

\item in case of doubt how to communicate address this early in class
  and attend online hours;

\item all office hours if not of personal nature are open office hours
  meaning that any student in class can be joined by other students of the
  class and  all meeting times are posted publicly. 

\item it is in your responsibility to attend in person classes and
  online hours as we found that those that do get better grades. For
  residential students participation in the residential classes is
  mandatory due to the same reason.

\item instructors of this class will attempt within reason to find
  suitable times for you to attend an online hour in case you are an
  online student. Residential students can attend the class on Friday
  and ask any question. 

\end{itemize}

\section{Class Material}

\begin{WARNING}
  As the class material will evolve during the semester it is obvious
  that some content will be improved and material will be added. This
  benefits all classes. To stay up to date, please, revisit this
  document on weekly basis. This is obvious, as we will adapt content
  based on your feedback. 

  The requirement of the classes to become an expert in cloud and/or
  Big Data applications and technologies stays unchanged.
\end{WARNING}

\section{HID}

You will be assigned an hid (Homework IDentifier) which allows us to
easily communicate with you and doe allow us to not use your
university ID to communicate with you. 

You will receive the HID within the first week of the semester by the
TA's.

\section{Notebook}\label{notebook}

All students are required to maintain a \emph{class notebook} in
github in which they summarize their weekly activities for this course
in bullet form. This includes a self maintained list of which lecture
material they viewed.

The notebook is maintained in the class github.com in your hid project
folder. It is a file called notebook.md that uses markdown as format.
While using md, you can either edit it locally and upload to github, or
directly edit it via the git hub Web editor. Notebooks are expected to
be set up as soon as the git repository was created.

You will be responsible to set up and maintaining the notebook.md and
update it accordingly. We suggest that you prepare sections such as:
Logistic, Theory, Practice, Writing and put in bullet form what you have
done into these sections during the week. We can see from the github
logs when you changed the notbook.md file to monitor progress. The
management of the notebook will be part of your discussion grade.

The format of the notebook is very simple markdown format and must
follow these rules:

\begin{itemize}
\item use headings with the \# character and have a space after the \#
\item use bullets in each topic.
\item each bullet \textbf{must} have an individual date that is of the
  form yyyy/mm/dd. Please do not lump bullet points under s single
  date. Have each bullet point its own date
\item if you have done the activity in a period than add the second
  date to it yyyy/mm/dd - yyyy/mm/dd
\item If you refer to section numbers in your notebook, please aslo
  add the section title as the section numbers may change in case we
  need to add content
\end{itemize}

Please examine carefully the sample note book is available at:

  \URL{https://raw.githubusercontent.com/bigdata-i523/sample-hid000/master/notebook.md}

This will render in git as:

  \URL{https://github.com/bigdata-i523/sample-hid000/blob/master/notebook.md}

The notebook.md is not a blog and should only contain a summary of
what you have done. 

\section{Blog}

Naturally you can maintain your own blog, but it is optional. If you
like to maintain your own blog, you can create yourself also a blog.md
file. However do not include sensitive information in there. A blog is
not a replacement for the notebook. The blog will not be used for
grading. If something does not go so well, do not focus on the nective
things, but focus on how that experience can be overcome and how you
turn it to a positive experience.

\section{Calendar I524, E616, E516}\label{S:calendar}

This class is a full term class of 16 weeks.

\begin{IU}

The semester calendar is posted at 

\URL{http://registrar.indiana.edu/official-calendar/official-calendar-spring.shtml}

The class beginns Mon, Jan 8th and ends Fri, May 4th

\end{IU}



\section{Incomplete}\label{incomplete}

Incompletes have to be explicitly requested in piazza. All incompletes
have to be filed by May 1st.

Incomplete's will receive a fractional Grade reduction: A will
become A-, A- will become B+, and so forth. There is enough time in the
course to complete all assignments without getting an incomplete.

Why do we have such a policy? As we teach state-of-the-art software this
software is subject to change, not only within the course, but also
after the course. As we may offer some services and only have access to
the TA's during the semester it is obvious that we like all class
projects and homework assignments to be completed within a semester.
Services that were offered during the semester may no longer be
available after the semester is over and could adversely effect your
planing. It will be in the students responsibility to identify such
services and provide alternatives if they become unavailable. We try
hard to avoid this but we can not guarantee it.

Furthermore, once an incomplete is requested, you will have 10 month to
complete it. We will need 2 month to grade. No grading will be conducted
over breaks. This may effect those that require student loans. Please
plan ahead.

The incomplete request needs to be off the following format in piazza:

\begin{verbatim}
Subject: 
    INCOMPLETE REQUEST: HID000: Lastname, Firstname

Body:
    Firstname: TBD
    Lastname: TBD
    HID: TBD
    Semester: TBD
    Course: TBD
    Online: yes/no

    URL notebook: TBD    
    URL assignment1:
    URL assignment2: TBD
    ....
    URL paper1: TBD
    URL project: TBD

    URL other1: TBD
\end{verbatim}

Please make sure that the links ar e clickable in piazza. Also as
classes have different assignments, make sure to include whatever is
relevant for that class and add the appropriate artifacts.

\begin{IU}
Here is the process for how to deal with incomplets at IU are documented:

\URL{http://registrar.indiana.edu/grades/grade-values/grade-of-incomplete.shtml}
\end{IU}


\section{Waitlist}\label{waitlist}

The waitlist contains students that are unable to enroll in a section of
a course. Students choose to add themselves to the waitlist. They are
not automatically added, but choose to do so intentionally based on the
status of the course. There are two reasons for students to be on the
waitlist. The first, and primary, reason is that the class is already at
the scheduled, maximum capacity. Since there are no seats available, the
student can elect to add themselves to the waitlist. The second reason
is that the students' own schedule has a time conflict. This occurs when
they are trying to enroll in a class that overlaps with the time of a
class they are already enrolled in.

Students are moved from the waitlist to the regular section during a
daily batch process, and not in real time. The process is not in
realtime because the registrar receives many requests to increase
capacity, decrease capacity, and change rooms. If the process were real
time there would be a catastrophe of conflicts.

Students are moved from the waitlist in chronological order that they
added themselves to the waitlist. If you are still on the waitlist there
are no spaces free, the batch process has not run for the day, or the
student in question has a schedule conflict.

Faculty are not able to selectively choose students from the waitlist.

How long does the waitlist process stay active?: The automated
processing of the waitlist ends on Thursday of the first week of class
At this time the waitlist will no longer be processed. 
As the residential class starts on Friday, this may cause
issues. Either talk to the department on Thursday or show up on
Friday. Most likeley there will be spaces left. 
Students on the waitlist at that time will remain on the waitlist, but remain there
until the student decides to change their registration. Students may not
do that, because they get assessed a change schedule fee.

Students tell me they still want to enroll after the first week of
classes. How do they do this?

Beginning Monday, after the first week of class students begin to use
the eAdd process to do a late addition of the course. The request is
routed to the professor of record on an eDoc and the faculty will be
notified via email. Faculty can deny or approve based on whatever
criteria they wish to apply. If the faculty member approves, the eDoc
is electronically forwarded to the Academic Operations office and we
will approve the late add \textbf{if the room capacity} allows the
addition, otherwise we must deny the addition because of fire marshal
regulations. Many times, there are seats in a
classroom/discussion/lab, but because other students have not
\emph{officially} dropped, enrollment is still at capacity.

After everything, a student that was unable to enroll in the class
attended all year and completed all course work as if they had enrolled.
Can the student get credit and can I give the student a grade?

Yes. There is a provision for a late registration - contact our office
if this occurs. Students will be assessed a tuition fee at the time of
late or retroactive registration.

\section{Registration}\label{S:register-mandatory}
\index{IU!Register}

The Executive Associate Dean for Academic Affairs requires starting
Spring 2018 that students that are not officilly enrolled, can not
register at the end of the class if they inofficially took the class.
Please make sure that within the first month you have enrolled. If we
do not see in CANVAS, you are not in the class. In case you are on a
waitlist it is your responsibility to work with the administration
after the waitlist is over to be added to the class by getting
permission from the School. 

\section{Auditing the class}\label{auditing-the-class}

We no longer allow students to audit E222, I524, E516, and E616. The
motivation to not offer these classes for auditing are:

\begin{itemize}
\item Seating in the lecture room is limited and we want foster
  students that enroll full time first.

\item The best way to take the class is to conduct a project. As this
  can not be achieved without taking the class full time and as
  auditing the class does not provide the full value of the class,
  e.g. not more than 10\% of the class. Hence, we do not think it is
  useful to audit the class.

\item  Accounts and services have to be set up and require
  considerable resources that are not accessible to students that
  audit the class.

\end{itemize}


\section{Resource restrictions}

\begin{itemize}
\item It is not allowed to use our services for profit (e.g. just
  enrolling in the class to use our clouds).
\item In case of abuse of available compute time on our clouds the
  student is aware that we will terminate the computer account on our
  clouds and the student may have to conduct the project on a public cloud or
  his own computer under own cost. There will be no guarantee that
  cloud services we offer will be available after the semester is
  over.  Projects can be conducted as part of the class that do not
  require access to the cloud.
\end{itemize}

\section{Office Hours}\label{office-hours}

\begin{description}

\item[Online Students:] Online hours are prioritized for online students,
  residential students should attend the residential meetings. 

\item[Residential Students:] Residential students participate in the
  official meeting times. If additional times are required, they have
  to be done by appointment. Office hours will be announced
  publically. All technical office hours are public and can be
  attended by any student.

  Online houres are not an excusenot to come to the residential class.

  However Residential students can in addition to the residential
  class use the online student meeting times.  However, in that case
  online students will be served first. It is probably good to check
  into the zoom meeting and identify if the TA has time. They will be
  in zoom.

\end{description}

We suggest that you let the TA's know in piazza before you come, in order to make
sure they are at the office.

\TODO{TA: set up doodle for meetings. Must be completed in first week by
  next week Monday.}
\TODO{TA: set up initial times. mark they are temporary.} 

\begin{itemize}
\item Mon 6:00pm-7:00pm, 7:00pm-8:00pm, Gregor (online)
\item Tue TBD, Smith Research Center
\item Wed TBD, Smith Research Center
\item Thu TBD, Smith Research Center
\item Fri TBD, Smith Research Center
\item Sat TBD, Smith Research Center
\end{itemize}


If a meeting is needed with Gregor, this is done upon appointment
Tue-Thu 10am - 2:30pm. However, TA's will figure out if a meeting is needed.
Please prepare your technical questions ahead of time, and place them in Piazza
first. TA's and the class will try to answer them if possible

The link for joining the meeting on Zoom is posted in Piazza.

% \URL{https://iu.zoom.us/j/235405252}

\URL{TBD}

For more up-to-date details, refer to Piazza.

\section{Plagiarizm}

In teh first week of class you will need to read the information about
plagiarizm. If there are any questions about plagiarizm we require you
to take a course offered from the IU educational department.

\begin{WARNING}
  If we find cheating or plagirizm, your assignment will be receiving
  an {\em F}. This especially includes copying text without proper
  attribution. In addition you will be receiving an {\em F} for the
  appropriate time for the discussion points an assignment was issued,
  e.g. If a paper duration assignment is 4 weeks, you get for these
  four weeks no discussion points, meaning an {\em F}. Furthermore, we
  will follow IU policy and report your case to the dean of students
  who may elect to expell you form the university. Please understand
  that it is your doing and the instructors have no choice as to
  follow university policies. Thus, please do not blame the
  instructors for your actions. Excuses such as ``I missed the lecture
  on plagiarizm'', ``I forgit to include the original refrence as I
  ran out of time'', ``I did not understand what plagiarizm is'' do
  not count obvioulsy as we explicitly make the policies clear. This
  applies to all material prepared for class including assignments,
  excercises, code, tutorials, papers, and projects. If there is no
  time, do not submit and instead of an F ask for an incomplete. In
  fact if you know you have plagerized, do not even have us review
  your paper. 
\end{WARNING}

For more information on this topic please see:

\URL{https://studentaffairs.indiana.edu/student-conduct/misconduct-charges/academic-misconduct.shtml}