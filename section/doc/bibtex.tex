% chktex-file 36
% chktex-file 9
% chktex-file 8

\FILENAME\

\section{Integrating Bibliographies}
\label{S:bibliographies}
\index{Bibliography}
\index{bibtex}
\index{biber}

Bibliography management in \LaTeX\ is one of the many features that
motivate many researchers to make \LaTeX\ the tool of choice to write
academic papers. There are numerous bibliography styles available that
allow easy adaptation to differen journal and proceedings style. As
such, it includes styles for bibliographies formated in ACM and IEEE
style.

\subsection{biber}

\TODO{Describe use of biber instead of bibtex}

\subsection{bibtex}

This section assumes we use the standard bibtex program to integrate
citations into a \LaTeX\ paper.

An example to use the IEEE style for a paper includes to set
the style to IEEEtran before you add the reference:

\begin{verbatim}
\bibliographystyle{IEEEtran}
\bibliography{references.bib}
\end{verbatim}

To properly create PDF papers which using bibtex, we have to make sure
that all indexes, citations, references and labels are updated. This
is done with the help of the following commands assuming your file is
called \verb|file.tex|

\begin{verbatim}
pdflatex  file
bibtex file
pdflatex  file
pdflatex  file
\end{verbatim}


\begin{IU}
At IU you are required to use our template, which includes a Makefile
and either calls the above three commands or used latexmk
\end{IU}

The reason for the multiple execution of the \LaTeX\ program is to update
all cross-references correctly. In case you are not interested in
updating the library every time in the writing progress just postpone it
till the end. Missing citations are viewed as \verb|[?]|.

Two programs stand out when managing bibliographies: emacs and jabref:

  \URL{http://www.jabref.org/}

Other programs such as Mendeley, Zotero, and even endnote integrate with
bibtex. However their support is limited, so we recommend that you just
use jabref as it is free and runs on all platforms.

\subsection{jabref}
\label{s:jabref}

\TODO{Tyler: include cite for jabref, bibtex, biber at appropriate
  location in this section. Add them to the bib file.}

Practical experience with many generations of students shows that
\textit{jabref} is a very simple to use bibliography manager for \LaTeX\ and
other systems. It can create a multitude of bibliography file formats
and allows upload in other online bibliography managers.

For information on how to install and use jabref please go to 
\url{http://www.jabref.org/}. There you will find the appropriate
download link.

We provide also a convenient video on how to use jabref and show
several methods on how to add entries easily to your bibliography database.

\video{Bibliography}{1:41}{jabref}{https://youtu.be/QVbifcLgMic}



\section{Entry types}

In this section we will explain how to find and properly generate
bibliographic entries. We are using bibtex for this as it is easy to
use and generates reasonable entries that can be included in
papers. What we like to achieve in this section is not to just show
you a final entry, but to document the process on how that entry was
derived. This will allow you to replicate or learn from the process to
apply to your own entries. As part of this we copy and paste
information found via Web searches.

We will address a number of important reference types which includes:

\begin{itemize}

\item
  wikipedia entries
\item
  github (see Section~\ref{s:e:source-code-references})
\item
  books
\item
  articles in a scientific journal (see Section~\ref{s:e:article-in-a-journal})
\item
  articles in a conference (see Section~\ref{s:e:article-in-a-conference-proceedings})
\item
  articles in magazines (non scientific)
\item
  blogs
\end{itemize}

\subsection{Source Code References}
\label{s:e:source-code-references}
\index{Bibtex!Source code}

Often, we need to cite a source code from a publicly hosted
repository. Such repositories are frequently used and include, for
example github, bitbucket, sourceforge, or your Universities code
repository as long as it is publicly reachable. As changes can occur on
these repositories, it is important that the date of access is listed in
the entry or even the release version of the source code.

Let us without bias chose a random source dode entry that has been
contributed by a student as follows:

\begin{verbatim}
@Misc{gonzalez_2015,
  Title =  {Buildstep},
  Author =     {Gonzalez, Jose and Lindsay, Jeff},
  HowPublished = {Web Page},
  Month =  {Jul},
  Note =   {Accessed: 2017-1-24},
  Year =   2015,
  Key =        {www-buildstep},
  Url =        {https://github.com/progrium/buildstep}
}
\end{verbatim}

Is this entry correct? Let us analyse. But first we need to undersrand
the semantics of the fields.

\subsubsection{What are the Different entry Types and Fields}

You see that the entry contains a number of fields. An extensive
explanation of these fields can be found at 

Please see \url{https://en.wikipedia.org/wiki/BibTeX}

We provide for this example a comprehensive discussions of the fields
used. For other examples we suggest you refer to the document to
identify what needs to be filled out.

\subsubsection{Entry type Misc}\label{s:e:entry-type-misc}
\index{Bibtex!Field!Misc}

First, it seems appropriate to use a \emph{@misc} entry. We correctly
identify this is a misc entry as it is online available. More recent
version of bibtex include also the type \emph{@online} for it. However,
in order to maintain compatibility to older formats we chose simply Misc
here and if we really would need to we could replace it easily.

\subsubsection{Label}\label{s:e:label}
\index{Bibtex!Field!Label}

Typically the bibliography label should contain 3 letters from an
author name, short year and the short name of the publication to
provide maximum information regarding the publication. However in this
case the project is hosted on github, so creating the label based on
just the github location seems more logical. Thus our label that we
propose is given by 

\verb|github-progrium-buildstep|


Under no circumstances should you use underscores as they can have unintended
consequences in programs we use to create papers for our classes. Just
use a minus sign instead. As it is hosted on github we also want the
githubname and the projectname. As you can see we just derived it from
the URL.

\begin{IU}
When managing bibliography entries with large numbers of collaborators
it is advisable that all bibliography labels be initially be prefixed
with an id for the collaborator. In our case we use the HID. Thus in
our case we will have the following label.

\verb|hid-sp18-000-github-progrium-buildstep|

\end{IU}


\subsubsection{Author}\label{s:e:author}
\index{Bibtex!Field!Author}

Normally we can write in the author fiels the names of the authors
separated by the work and. IT is important tu use the word \verb|and|
but not use a comma. However, this only works if the lastname does not
contain a space in it. In order to avoid confusing the system, we
recommend therefore to write all names in the form
\verb|Lastname, Firstname Middle Initials|

Thus we find in our example

\verb|  Author =     {Gonzalez, Jose and Lindsay, Jeff},|

\begin{WARNING}
Please note the word and between authors and not a comma. Commas are
only used to distinguish between lastname and firstname.
\end{WARNING}


\subsubsection{Key}\label{s:e:key}
\index{Bibtex!Field!Key}

In this case the key field can be removed as the entry has an author
field entry. If there was no author field, we use the key to specify
the alphabetical ordering based on the specified key. Note that a key is
not the label. In fact in our original entry the key field was wrongly
used and the student did not understand that the key is used for
sorting.

\subsubsection{Howpublished}\label{s:e:howpublished}
\index{Bibtex!Field!Howpublished}

Since the source is a github project repository, the howpublished field
shall hold the value \verb|{Code Repository}|. If the
url specified was a normal webpage, the \verb|{Web Page}| entry would be
valid.

\subsubsection{Month}\label{s:e:month}
\index{Bibtex!Field!Month}

To allow internationalization of the month we use the first 3 letters
in the english language for the month. Thus it is 
\verb|month = jul|. Note that there are no brackets around the month.

\subsubsection{Owner}\label{s:e:owner}
\index{Bibtex!Field!Owner}

In class we introduced the convention to put the student HID in it. If
multiple students contributed, add them with space separation.

\subsubsection{Accessed}\label{s:e:accessed}
\index{Bibtex!Field!Accessed}

As some styles do not support the accessed field, we simply include it in
the note field. This is absolutely essential as code can change and when
we read the code we looked at a particular snapshot in time. In addition
it is often necessary to record the actual version of the code or the branch.
Typically for github entries, it is best to just use the month and
year field as some styles check for it.

\subsubsection{Final Entry}

Filling out as many fields as possible with information for this entry
we get:

\begin{verbatim}
@Misc{github-progrium-buildstep,
  Title =  {Buildstep},
  Author =     {Jose Gonzalez and Jeff Lindsay},
  HowPublished = {Code Repository},
  Year =   {2015},
  Month =  jul,
  Note =   {Accessed: 2017-1-24, master branch},
  Url =    {https://github.com/progrium/buildstep},
  Owner =  {S17-IO-3025},
}
\end{verbatim}

We are using the release date in the year and month field as this
project uses this for organizing releases. However, other project may
have release versions so you would have in addition to using the data
also to include the version in the note field such as:

\begin{verbatim}
Note =     {Version: 1.2.3, Accessed: 2017-1-24},
\end{verbatim}


\subsection{Pedigree}

Often it is advantageous to document the pedigree of the bibtex
entries. Eg, how did you derive the final entry? To do so you can
simply add a field such as \verb|bibsource| and include in it all
links you used to gather the final entry. However, we believe that the
effort to curate an entry is sufficient to manage your own
bibliography which you should freely distribute and constitute an own
contribution. 

\subsection{Article in a Journal}
\label{s:e:article-in-a-journal}
\index{Bibtex!Article}

Many online bibtex entries that you will find are wrong or
incomplete. Often you may find via google a bibtex entry that may need
some more research. Let us assume your first google query returns a
publication and you cite it such as this:

\begin{verbatim}
@Unpublished{unpublished-google-sawzall,
    Title = {{Interpreting the Data: Parallel Analysis with Sawzall}},
    Author = {{Rob Pike, Sean Dorward, Robert Griesemer, Sean Quinlan}},
    Note = {accessed 2017-01-28},
    Month = {October},
    Year = {2005},
    Owner = {for the purpose of this discussion removed},
    Timestamp = {2017.01.31}
}
\end{verbatim}

Could we improve this entry to achieve your best? We observe:

\begin{enumerate}
\item
  The author field has a wrong entry as the \verb|,| is to be replaced by an
  \verb|and|.
\item
  The author field has authors and thus must not have a \{\{ \}\}
\item
  The url is missing, as the simple google search actually finds a PDF
  document.
\end{enumerate}

Let us investigate a bit more while searching for the title. We find

\begin{enumerate}
\def\labelenumi{\Alph{enumi}}
\item
  \url{https://www.google.com/url?sa=t\&rct=j\&q}=\&esrc=s\&source=web\&cd=1\&ved=0ahUKEwj\_ytSA-PDRAhUH8IMKHaomC-oQFggaMAA\&url=https\%3A\%2F\%2Fresearch.google.com\%2Farchive\%2Fsawzall-sciprog.pdf\&usg=AFQjCNHSSfKBwbxVAVPQ0td4rTjitKucpA\&sig2=vbiVzi36B3gGFjIzlUKBDA\&bvm=bv.146073913,d.amc
\item
  \url{https://research.google.com/pubs/pub61.html}
\item
  \url{http://dl.acm.org/citation.cfm?id=1239658}
\end{enumerate}

Let us look at (A)

As you can see from the url this is actually some redirection to a google
web page which probably is replaced by B as its from google research. So
let us look at (B)

Now when you look at the link we find the url
\url{https://research.google.com/archive/sawzall-sciprog.pdf} which
redirects you to the PDF paper.

When we go to (B) we find surprisingly a bibtex entry as follows:

\begin{verbatim}
@article{61,
  title = {Interpreting the Data: Parallel Analysis with Sawzall},
  author = {Rob Pike and Sean Dorward and Robert Griesemer and Sean Quinlan},
  year = 2005,
  URL = {https://research.google.com/archive/sawzall.html},
  journal = {Scientific Programming Journal},
  pages = {277--298},
  volume = {13}
}
\end{verbatim}

Now we could say let us be satisfied, but (C) seems to be even more
interesting as its from a major publisher. So lats just make sure we
look at (C)

If you go to (C), you find under the colored box entitled Tools and
Resources a link called \textbf{bibtex}. Thus it seems a good idea to
click on it. This will give you:

\begin{verbatim}
@article{Pike:2005:IDP:1239655.1239658,
    author = {Pike, Rob and Dorward, Sean and Griesemer, Robert and Quinlan, Sean},
    title = {Interpreting the Data: Parallel Analysis with Sawzall},
    journal = {Sci. Program.},
    issue_date = {October 2005},
    volume = {13},
    number = {4},
    month = oct,
    year = {2005},
    issn = {1058-9244},
    pages = {277--298},
    numpages = {22},
    url = {http://dx.doi.org/10.1155/2005/962135},
    doi = {10.1155/2005/962135},
    acmid = {1239658},
    publisher = {IOS Press},
    address = {Amsterdam, The Netherlands, The Netherlands},
}
\end{verbatim}

Now we seem to be at a position to combine our search result as neither
entry is sufficient. As the doi number properly specifies a paper (look
up what a doi is) we can replace the url with one that we find online,
such as the one we found in (A) Next we see that all field sin B are
already covered in C, so we take (C) and add the url. Now as the label is
great and uniform for ACM, but for us a bit less convenient as its
difficult to remember, we just change it while for example using
authors, title, and year information. Let us also make sure to do mostly
lowercase in the label just as a convention. Thus our entry looks like:

\begin{verbatim}
@article{pike05swazall,
    author = {Pike, Rob and Dorward, Sean and Griesemer, Robert and Quinlan, Sean},
    title = {Interpreting the Data: Parallel Analysis with Sawzall},
    journal = {Sci. Program.},
    issue_date = {October 2005},
    volume = {13},
    number = {4},
    month = oct,
    year = {2005},
    issn = {1058-9244},
    pages = {277--298},
    numpages = {22},
    url = {https://research.google.com/archive/sawzall-sciprog.pdf},
    doi = {10.1155/2005/962135},
    acmid = {1239658},
    publisher = {IOS Press},
    address = {Amsterdam, The Netherlands, The Netherlands},
}
\end{verbatim}

As you can see properly specifying a reference takes multiple google
queries and merging of the results you find from various returns. As
you still have time to correct things I advise that you check your
references and correct them. If the original reference would have been
graded it would have been graded with a ``fail'' instead of a ``pass''.

Naturally you need to judge if you can integrate the URL or not, often
papers exist in a prepublication and you must make sure to cite the
version you used. IN fact if you have the prepublicAtion, you should
obtain the final manuscript as it could contain significant
corrections to the previous draft. In other cases the prepublication
may just be fine and you could for your own references keep the
url. FOr the final publication you probably want to make sure that the
doi is used. FOr the collection of your references, adding the url
could be useful.

\subsection{Article in a Conference Proceedings}
\label{s:e:article-in-a-conference-proceedings}
\index{Bibtex!InProceedings}

Now let us look at another obvious example that needs improvement:

\begin{verbatim}
@InProceedings{wettinger-any2api,
  Title      = {Any2API - Automated APIfication},
  Author     = {Wettinger, Johannes and
                Uwe Breitenb{\"u}cher
                and Frank Leymann},
  Booktitle  = {Proceedings of the 5th International
                Conference on Cloud Computing and
                Services Science},
  Year       = {2015},
  Pages      = {475­486},
  Publisher  = {SciTePress},
  ISSN       = {2326-7550},
  Owner      = {S17-IO-3005},
  Url        = {https://pdfs.semanticscholar.org/1cd4/4b87be8cf68ea5c4c642d38678a7b40a86de.pdf}
}
\end{verbatim}

As you can see this entry seems to define all required fields, so we
could be tempted to stop here. But its good to double check. Let us do
some queries against ACM and google scholar. Let us just type in the
title, and if this is in a proceedings they should return hopefully a
predefined bibtex record for us.

Let us query:

\begin{verbatim}
google: googlescholar Any2API Automated APIfication
\end{verbatim}

We get:

\begin{itemize}

\item
  \url{https://scholar.google.de/citations?view_op=view_citation\&hl=en\&user=j6lIXt0AAAAJ\&citation_for_view=j6lIXt0AAAAJ:8k81kl-MbHgC}
\end{itemize}

On that page we see
\href{https://scholar.google.com/scholar_lookup?title=Automated+drug+dispensing+system+reduces+medication+errors+in+an+intensive+care+setting\&author=Chapuis\&publication_year=2010\#}{Cite}

So we find a PDF at
\url{https://pdfs.semanticscholar.org/1cd4/4b87be8cf68ea5c4c642d38678a7b40a86de.pdf}

Let us click on this and the document includes a bibtex entry such as:

\begin{verbatim}
@inproceedings{Wettinger2015, 
  author= {Johannes Wettinger and Uwe Breitenb{\"u}cher and Frank
       Leymann},
  title = {Any2API - Automated APIfication},
  booktitle = {Proceedings of the 5th International Conference on Cloud
       Computing and Service Science (CLOSER)},
  year = {2015},
  pages = {475--486},
  publisher = {SciTePress}
} 
\end{verbatim}

Now let us add the URL and owner:

\begin{verbatim}
@inproceedings{Wettinger2015, 
  author= {Johannes Wettinger and Uwe Breitenb{\"u}cher and Frank
       Leymann},
  title = {Any2API - Automated APIfication},
  booktitle = {Proceedings of the 5th International Conference on Cloud
       Computing and Service Science (CLOSER)},
  year = {2015},
  pages = {475--486},
  publisher = {SciTePress},
  url ={https://pdfs.semanticscholar.org/1cd4/4b87be8cf68ea5c4c642d38678a7b40a86de.pdf},
  owner = {S17-IO-3005},
} 
\end{verbatim}

Should we be satisfied? No, even our original information we gathered
provided more information. So let us continue. Let us issue additional
google searches with ACM or IEEE and the title. When doing the IEEE in the
example we find an entry called

\href{http://dblp.uni-trier.de\%2Fpers\%2Fl\%2FLeymann\%3AFrank\&usg=AFQjCNHCu-66qxWH0zRlPLr4DA8jIo5V-g\&sig2=1vYdnGOEiMcLBEMpbeBA7g}{dlp:
Frank Leyman}

Let us look at it and we find two entries:

\begin{verbatim}
@inproceedings{DBLP:conf/closer/WettingerBL15,
  author    = {Johannes Wettinger and
       Uwe Breitenb{\"{u}}cher and
       Frank Leymann},
  title     = {{ANY2API} - Automated APIfication - Generating APIs for Executables
       to Ease their Integration and Orchestration for Cloud Application
       Deployment Automation},
  booktitle = {{CLOSER} 2015 - Proceedings of the 5th International Conference on
       Cloud Computing and Services Science, Lisbon, Portugal, 20-22 May,
       2015.},
  pages     = {475--486},
  year      = {2015},
  crossref  = {DBLP:conf/closer/2015},
  url       = {http://dx.doi.org/10.5220/0005472704750486},
  doi       = {10.5220/0005472704750486},
  timestamp = {Tue, 04 Aug 2015 09:28:21 +0200},
  biburl    = {http://dblp.uni-trier.de/rec/bib/conf/closer/WettingerBL15},
  bibsource = {dblp computer science bibliography, http://dblp.org}
}

@proceedings{DBLP:conf/closer/2015,
  editor    = {Markus Helfert and
       Donald Ferguson and
       V{\'{\i}}ctor M{\'{e}}ndez Mu{\-{n}}oz},
  title     = {{CLOSER 2015 - Proceedings of the 5th International Conference on
       Cloud Computing and Services Science, Lisbon, Portugal, 20-22 May,
       2015}},
  publisher = {SciTePress},
  year      = {2015},
  isbn      = {978-989-758-104-5},
  timestamp = {Tue, 04 Aug 2015 09:17:34 +0200},
  biburl    = {http://dblp.uni-trier.de/rec/bib/conf/closer/2015},
  bibsource = {dblp computer science bibliography, http://dblp.org}
}
\end{verbatim}

So let us look at the entry and see how to get a better one for our
purpose and combine them. When using jabref, you see optional and
required fields, we want to add as many as possible, regardless if
optional or required, so Let us do that. We write it here in ASCII as it
is easier to document and can also be done in emacs:

\begin{verbatim}
@InProceedings{,
  author =   {},
  title =    {},
  OPTcrossref =  {},
  OPTkey =   {},
  OPTbooktitle = {},
  OPTyear =      {},
  OPTeditor =    {},
  OPTvolume =    {},
  OPTnumber =    {},
  OPTseries =    {},
  OPTpages =     {},
  OPTmonth =     {},
  OPTaddress =   {},
  OPTorganization = {},
  OPTpublisher = {},
  OPTnote =      {},
  OPTannote =    {},
  url = {}
}
\end{verbatim}

Now we copy and fill out the \textbf{form} from our various searches:

\begin{verbatim}
@InProceedings{Wettinger2015any2api,    
  author    = {Johannes Wettinger and
     Uwe Breitenb{\"{u}}cher and
     Frank Leymann},
  title     = {{ANY2API - Automated APIfication - Generating APIs for Executables
     to Ease their Integration and Orchestration for Cloud Application
     Deployment Automation}},
  booktitle = {{CLOSER 2015 - Proceedings of the 5th International Conference on
       Cloud Computing and Services Science}},
  year =     {2015},
  editor    = {Markus Helfert and
       Donald Ferguson and
       V{\'{\i}}ctor M{\'{e}}ndez Mu{\-{n}}oz},
  publisher = {SciTePress},
  isbn      = {978-989-758-104-5},
  pages = {475--486},
  month = {20-22 May},
  address =      {Lisbon, Portugal},
  doi       = {10.5220/0005472704750486},
  url ={https://pdfs.semanticscholar.org/1cd4/4b87be8cf68ea5c4c642d38678a7b40a86de.pdf},
  owner = {S17-IO-3005},
}
\end{verbatim}


For the rest of the section we provide just some simple examples.

\subsection{InProceedings}\label{s:e:inproceedings}
\index{Bibtex!InProceedings}

Please fill out

\begin{verbatim}
@InProceedings{,
  author =       {},
  title =        {},
  OPTcrossref =  {},
  OPTkey =       {},
  OPTbooktitle = {},
  OPTyear =      {},
  OPTeditor =    {},
  OPTvolume =    {},
  OPTnumber =    {},
  OPTseries =    {},
  OPTpages =     {},
  OPTmonth =     {},
  OPTaddress =   {},
  OPTorganization = {},
  OPTpublisher = {},
  OPTnote =      {},
  OPTannote =    {},
  url = {}
}
\end{verbatim}

\begin{verbatim}
@inproceedings{vonLaszewski15tas,
  author =     {DeLeon, Robert L. and Furlani, Thomas R. and Gallo,
                  Steven M. and White, Joseph P. and Jones, Matthew
                  D. and Patra, Abani and Innus, Martins and Yearke,
                  Thomas and Palmer, Jeffrey T. and Sperhac, Jeanette
                  M. and Rathsam, Ryan and Simakov, Nikolay and von
                  Laszewski, Gregor and Wang, Fugang},
  title =  {{TAS View of XSEDE Users and Usage}},
  booktitle =  {Proceedings of the 2015 XSEDE Conference: Scientific
                  Advancements Enabled by Enhanced
                  Cyberinfrastructure},
  series =     {XSEDE '15},
  year =   2015,
  isbn =   {978-1-4503-3720-5},
  location =   {St. Louis, Missouri},
  pages =  {21:1--21:8},
  articleno =  21,
  numpages =   8,
  url =        {http://doi.acm.org/10.1145/2792745.2792766},
  doi =        {10.1145/2792745.2792766},
  acmid =  2792766,
  publisher =  {ACM},
  address =    {New York, NY, USA},
  keywords =   {HPC, SUPReMM, TAS, XDMoD, XSEDE usage, XSEDE users},
}
\end{verbatim}

\subsection{TechReport}\label{s:e:techreport}
\index{Bibtex!TechReport}

Please fill out

\begin{verbatim}
@TechReport{,
  author =       {},
  title =        {},
  institution =  {},
  year =         {},
  OPTkey =       {},
  OPTtype =      {},
  OPTnumber =    {},
  OPTaddress =   {},
  OPTmonth =     {},
  OPTnote =      {},
  OPTannote =    {},
  url = {}    
}
\end{verbatim}

\begin{verbatim}
@TechReport{las05exp,
  title =  {{The Java CoG Kit Experiment Manager}},
  Author =     {von Laszewski, Gregor},
  Institution =    {Argonne National Laboratory},
  Year =   2005,
  Month =  jun,
  Number =     {P1259},
  url = {https://laszewski.github.io/papers/vonLaszewski-exp.pdf}
}
\end{verbatim}

\subsection{Article}
\index{Bibtex!Article}

Please fill out

\begin{verbatim}
@Article{,
  author =       {},
  title =        {},
  journal =      {},
  year =         {},
  OPTkey =       {},
  OPTvolume =    {},
  OPTnumber =    {},
  OPTpages =     {},
  OPTmonth =     {},
  OPTnote =      {},
  OPTannote =    {},,
  url = {}
}
\end{verbatim}

\begin{verbatim}
@Article{las05gridhistory,
  title =  {{The Grid-Idea and Its Evolution}},
  author =     {von Laszewski, Gregor},
  journal =    {Journal of Information Technology},
  year =   2005,
  month =  jun,
  number =     6,
  pages =  {319-329},
  volume =     47,
  doi =        {10.1524/itit.2005.47.6.319},
  url = {https://laszewski.github.io/papers/vonLaszewski-grid-idea.pdf}
}
\end{verbatim}

\subsection{Proceedings}\label{s:e:proceedings}
\index{Bibtex!Proceedings}

Please fill out

\begin{verbatim}
@Proceedings{,
  title =        {},
  year =         {},
  OPTkey =       {},
  OPTbooktitle = {},
  OPTeditor =    {},
  OPTvolume =    {},
  OPTnumber =    {},
  OPTseries =    {},
  OPTaddress =   {},
  OPTmonth =     {},
  OPTorganization = {},
  OPTpublisher = {},
  OPTnote =      {},
  OPTannote =    {},
  url = {}
}
\end{verbatim}

\begin{verbatim}
@Proceedings{las12fedcloud-proc,
  title =  {{FederatedClouds '12: Proceedings of the 2012
                  Workshop on Cloud Services, Federation, and the 8th
                  Open Cirrus Summit}},
  year =   2012,
  address =    {New York, NY, USA},
  editor =     {vonLaszewski, Gregor and Robert Grossman and Michael
                  Kozuchand Rick McGeerand Dejan Milojicic},
  publisher =  {ACM},
  iSBN =   {978-1-4503-1754-2},
  location =   {San Jose, California, USA},
  url =
                  {http://dl.acm.org/citation.cfm?id=2378975&picked=prox&cfid=389635474&cftoken=32712991}
}
\end{verbatim}

\subsection{Wikipedia Entry}\label{s:e:wikipedia-entry}
\index{Bibtex!Wikipedia Entry}

Please fill out

\begin{verbatim}
@Misc{,
  OPTkey =       {},
  OPTauthor =    {},
  OPTtitle =     {},
  OPThowpublished = {},
  OPTmonth =     {},
  OPTyear =      {},
  OPTnote =      {},
  OPTannote =    {},
  url = {}
}
\end{verbatim}

\begin{verbatim}
@Misc{www-ode-wikipedia,
  Title =  {Apache ODE},
  HowPublished = {Web Page},
  Note =   {Accessed: 2017-2-11},
  Key =        {Apache ODE},
  Url =        {https://en.wikipedia.org/wiki/Apache_ODE}
}
\end{verbatim}

\subsection{Blogs}\label{blogs}
\index{Bibtex!Blog}

Please fill out

\begin{verbatim}
@Misc{,
  OPTkey =       {},
  OPTauthor =    {},
  OPTtitle =     {},
  OPThowpublished = {},
  OPTmonth =     {},
  OPTyear =      {},
  OPTnote =      {},
  OPTannote =    {},
  OPTurl = {}
}
\end{verbatim}

\begin{verbatim}
@Misc{www-clarridge-discoproject-blog,
  title =  {Disco - A Powerful Erlang and Python Map/Reduce
                  Framework},
  uthor =  {Clarridge, Tait},
  howpublished = {Blog},
  month =  may,
  note =   {Accessed: 25-feb-2017},
  year =   2014,
  url =  {http://www.taitclarridge.com/techlog/2014/05/disco-a-powerful-erlang-and-python-mapreduce-framework.html}
}
\end{verbatim}

\subsection{Web Page}\label{s:e:web-page}
\index{Bibtex!Web Page}

Please fill out

\begin{verbatim}
@Misc{, 
  OPTkey =       {}, 
  OPTauthor =    {}, 
  OPTtitle =     {}, 
  OPThowpublished = {}, 
  OPTmonth =     {}, 
  OPTyear =      {}, 
  OPTnote =      {},
  OPTannote =    {},
  url = {}
}
\end{verbatim}

\begin{verbatim}
@Misc{www-cloudmesh-classes,
  OPTkey =       {},
  author =    {von Laszewski, Gregor},
  title =     {Cloudmesh Classes},
  howpublished = {Web Page},
  OPTmonth =     {},
  OPTyear =      {},
  OPTnote =      {},
  OPTannote =    {},
  url = {https://cloudmesh.github.io/classes/}
}
\end{verbatim}

\begin{verbatim}
@Misc{www-awslambda,
  title =  {AWS Lambda},
  author =     {{Amazon}},
  key =        {AWS Lambda},
  howpublished = {Web Page},
  url =        {https://aws.amazon.com/lambda/faqs/}
}
\end{verbatim}

\subsection{Book}\label{s:e:book}
\index{Bibtex!book}

Given the following entry. What is the proper entry for this book.
Provide rationale:

\begin{verbatim}
@Book{netty-book,
    Title = {Netty in Action},
    Author = {Maurer, Norman and Wolfthal, Marvin},
    Publisher = {Manning Publications},
    Year = {2016},
}
\end{verbatim}

To obtain the record of a book you can look at many information sources.
The can include:

\begin{itemize}

\item
  \url{https://www.manning.com/books/netty-in-action}
\item
  \url{https://www.amazon.com/Netty-Action-Norman-Maurer/dp/1617291471}
\item
  \url{http://www.barnesandnoble.com/w/netty-in-action-norman-maurer/1117342155?ean=9781617291470\#productInfoTabs}
\item
  \url{http://www.powells.com/book/netty-in-action-9781617291470/1-0}
\end{itemize}

Furthermore, we need to consider the entry of a book, we simply look it
up in emacs where we find the following but add the owner and the url
field:

\begin{verbatim}
@Book{,
  ALTauthor =      {},
  ALTeditor =      {},
  title =      {},
  publisher =      {},
  year =   {},
  OPTkey =     {},
  OPTvolume =      {},
  OPTnumber =      {},
  OPTseries =      {},
  OPTaddress =     {},
  OPTedition =     {},
  OPTmonth =   {},
  OPTnote =    {},
  OPTannote =      {},
  ownwer =       {},
  url = {}
}
\end{verbatim}

In summary we find the following fields:

\begin{description}
\item[Required fields:]
author/editor, title, publisher, year
\item[Optional fields:]
volume/number, series, address, edition, month, note, key
\end{description}

We apply the following to fill out the fields which is the standard
definition as defined by \LaTeX.

\begin{description}
\item[address:]
The address is the Publisher's address. Usually just the city, but can
be the full address for lesser-known publishers.
\item[author:]
The name(s) of the author(s) (in the case of more than one author,
separated by and) Names can be written in one of two forms: Donald E.
Knuth or Knuth, Donald E. or van Halen, Eddie. Please note that Eddie
van Halen would result in a wrong name. For our purpose we keep nobelity
titles part of the last name.
\item[edition:]
The edition of a book, long form (such as ``First'' or ``Second'')
\item[editor:]
The name(s) of the editor(s)
\item[key:]
A hidden field used for specifying or overriding the alphabetical order
of entries (when the ``author'' and ``editor'' fields are missing). Note
that this is very different from the key that is used to cite or
cross-reference the entry.
\item[label:]
The label field should contain three letters from the auth field, a
short year reference and a short name of the publication to provide the
maximum information regarding the publication. Underscores should be
replaced with dashes or removed completely.
\item[month:]
The month of publication or, if unpublished, the month of creation. Use
three-letter abbreviations for this field in order to account for
languages that do not capitalize month names. Additional information for
the day can be included as follows: \verb|aug#~``10,''|
\item[publisher:]
The publisher's name
\item[series:]
The series of books the book was published in (e.g. ``The Hardy Boys''
or ``Lecture Notes in Computer Science'')
\item[title:]
The title of the work. As the capitalization depends on the bibliography
style and the language used we typically use camel case. To force
capitalization of a word or its first letter you can use the curly
braces, `\{ \}'. To keep the title in camel case simple use title =
\{\{My Title\}\}
\item[type:]
The field overriding the default type of publication (e.g. ``Research
Note'' for techreport, ``\{PhD\} dissertation'' for phdthesis,
``Section'' for inbook/incollection) volume The volume of a journal or
multi-volume book year The year of publication (or, if unpublished, the
year of creation)
\end{description}

While applying the above rules and tips we summarize what we have done
for this entry:

\begin{enumerate}
\def\labelenumi{\arabic{enumi}.}
\item
  Search for the book by title/Author on ACM (\url{http://dl.acm.org/})
  or Amazon or barnesandnoble or upcitemdb (\url{http://upcitemdb.com}).
  These services return bibtex entrie that you can improve.
\item
  Hence one option is t get the ISBN of the book. For ``Mesos in
  action'' from upcitemdb we got the ISBN as ``9781617 292927''. This is
  the 13 digit ISBN. The first 3 digits (GS1 code) can be skipped. Using
  the rest of 10 digits ``1617 292927'', Add in JabRef in Optional
  Fields-\textgreater{}ISBN.

  However it is fine to youst specify the full number.

  We can also return a bibtex entry generated while using Click on the
  ``Get BibTex from ISBN''.

  Now we get more information on this book entry from ISBN. We can opt
  either the original or newly searched entry for the below bibtex
  fields or merge as appropriate. URL may not match from where we
  initially read the book, however there is option to put your original
  url or newly searched url. EAN, Edition, Pages,url,published date etc.
  Do a search on amazon for ``ASIN''. Can skip if not available.
  Sometime we get ASIN for a different publication, maybe a paperback
  ASIN=\{B01MT311CU\} We can add it as it becomes easier to search
\end{enumerate}

\begin{description}
\item[doi:]
If you can find a doi numer you should also add it. IN this case we
could not locate one.
\end{description}

As a result we obtain the entry:

\begin{verbatim}
@Book{netty-book,
  title = {Netty in Action},
  publisher = {Manning Publications Co.},
  year = {2015},
  author = {Maurer, Norman and Wolfthal, Marvin Allen},
  address = {Greenwich, CT, USA},
  edition = {1st},
  isbn = {1617291471},
  asin = {1617291471},
  date = {2015-12-23},
  ean = {9781617291470},
  owner = {S17-IO-3022 S17-IO-3010 S17-IO-3012},
  pages = {296},
  url = {http://www.ebook.de/de/product/21687528/norman_maurer_netty_in_action.html},
}
\end{verbatim}

\section{Integrating Bibtex entries into Other Systems}

We have not tested any of this

\subsection{jabref and MSWord}
\index{Bibtex!MSWord}

According to others it is possible to integrate jabref references
directly into MSWord. 
For more information please see:
\URL{https://www.paulkiddie.com/2009/07/jabref-exports-to-word-2007-xml/}


\subsection{Bibtex import to MSWord}\label{bibtex-import-to-msword}

\subsubsection{XML import}
\index{Bibtex!MSWord}

Please give feedback if you used this.

see:
\URL{http://blog.pengyifan.com/using-bibtex-in-ms-word-2015-mac-os/}

\begin{enumerate}

\item  In JabRef, export the bibliography in MS Word 2008 xml format

\item  Name the file Sources.xml (case sensitive)
\item   In OSX with MS Word 2015: Go to
  \verb|/Library/Containers/com.microsoft.word/Data/Library/Application Support/Microsoft/Office.|
\item  Rename the original Sources.xml file to Sources.xml.bak
\item  Copy the generated Sources.xml in this folder
\item  Restart MS Word.

\end{enumerate}

We do not know what needs to be done in case you need to make changes to
the references. Please report back your experiences. To avoid issues we
recommend that you use \LaTeX~and not MSWord.

\subsubsection{BibTex4Word}
\index{bibtex4word}

We have not tried this:

\URL{http://www.ee.ic.ac.uk/hp/staff/dmb/perl/index.html}


You are highly recommended to use Jabref for bibliography management in
this class. Here is an introductory video on Jabref:
\url{https://youtu.be/roi7vezNmfo?t=8m6s}

\section{Other Reference Managers}

Please note that you should first decide which reference manager you
like to use. In case you for example install zotero and mendeley, that
may not work with word or other programs.

\subsection{Endnote}
\index{Endnote}

Endnote os a reference manager that works with Windows. Many people use
Endnote. However, in the past, Endnote has caused complications when
dealing with collaborative management of references. Its price is
considerable. We have lost many hours of work because of instability of
Endnote in some cases. As a student, you may be able to use Endnote for
free at Indiana University.

\URL{http://endnote.com/}


\subsection{Mendeley}
\index{Mendeley}

Mendeley is a free reference manager compatible with Windows Word 2013,
Mac Word 2011, LibreOffice, BibTeX. Videos on how to use it are
available at:

\URL{https://community.mendeley.com/guides/videos}


Installation instructions are available at

\URL{https://www.mendeley.com/features/reference-manager/}


When dealing with large databases, we found the integration of Mendeley
into word slow.

\subsection{Zotero}
\index{Zotero}

Zotero is a free tool to help you collect, organize, cite, and share
your research sources. Documentation is available at

\URL{https://www.zotero.org/support/}

The download link is available from

\URL{https://www.zotero.org/}


We have limited experience with Zotero

\subsection{Paperpile}
\index{Paperpile}

Paper pile is a Web based reference management tool that integrates
with Google docs. It can export the database as bibtex. Paperpile is a
commercial tool costing about \$36 a year for academic users. For
others it is about 3 times as expensive.

\URL{https://paperpile.com}