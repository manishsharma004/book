\FILENAME\

\section{Plagiarism}\label{S:plagiarism}

We start with the review of a most important topic.

\subsection{Plagiarism Definition}

In academic life it is important to understand and avoid plagiarism.
The dictionary defines plagiarism as follows \url{dictionary.com}:

% \textipa{['pl\=aj@""riz@m]}

\begin{description}
\item[pla$\cdot$gia$\cdot$rism] ``the practice of taking someone else's work or ideas and passing them
off as one's own.''
\end{description}



\subsection{Plagiarism Policies}
Organizations and universities will have policies in place do address
plagiarism. An example is provided for Indiana University
\cite{www-iu-plagiarism}. We quote:

\begin{quotation}
``Honesty requires that any ideas or materials taken from
another source for either written or oral use must be fully
acknowledged. Offering the work of someone else as one's own is
plagiarism. The language or ideas thus taken from another may range
from isolated formulas, sentences, or paragraphs to entire articles
copied from books, periodicals, speeches, or the writings of other
students. The offering of materials assembled or collected by others
in the form of projects or collections without acknowledgment also is
considered plagiarism. Any student who fails to give credit for ideas
or materials taken from another source is guilty of plagiarism. 

(Faculty Council, May 2, 1961; University Faculty Council, March 11,
1975; Board of Trustees, July 11, 1975)''
\end{quotation}

Faculty members at Universitys are also bound by policies that mandate
reporting. At Indiana University the following policy applies (for a
complete policy see the Web page):

\begin{quotation}
``Should
the faculty member detect signs of plagiarism or cheating, it is his
or her most serious obligation to investigate these thoroughly, to
take appropriate action with respect to the grades of students, and
\textit{in any event} to report the matter to the Dean for Student Services [or
equivalent administrator]. The necessity to report every case of
cheating, whether or not further action is desirable, arises
particularly because of the possibility that this is not the student's
first offense, or that other offenses may follow it. Equity also
demands that a uniform reporting practice be enforced; otherwise, some
students will be penalized while others guilty of the same actions
will go free.

(Faculty Council, May 2, 1961)''
\end{quotation}

Naturally if a student has any questions about understanding
plagiarism the University can provide assistance. If a student is in
doubt and asks for help this is not considered at that time
plagiarism. 

As you can see from the previous policies, the faculty do not have any
choice but reporting real cases of plagiarism to the university
administration.  Thus you must not hold them personally responsible as
this is part of the tasks they are required to do if they like it or
not. Instead, it is {\bf the responsibility of the authors of the
  document} to assure no plagiarism occurs. If you are a student of a
class that writes a paper or project report this naturally also all
applies to you. In addition, if you work in a team you need to assure
the entire team addresses plagiarism appropriately.

In practice this means that the teachers of a course expect yo know
plagiarism and you need to be informed about it. This is typically
done in other courses. However, as it is often overlooked by the
student we are pointing it out here so we can make sure you contribute
to courses that require you to write papers and reports. This also
means you can not claim you did not know what plagiarism is. You are
required to know what it is, know how to detect it and know how to
avoid it. The resources provided next will give you the necessary
tools and background.

\subsection{Plagiarism Resources}

The \href{http://education.indiana.edu/}{School of Education at Indiana
University} has a significant set of resources to get educated about
plagiarism. These resources are intended to ``preparing educators,
advancing knowledge, and improving education''~\cite{www-iu-plagiarism}
\url{https://www.indiana.edu/~istd/patterns.html}

The content here is copied from the Web Page

  \URL{https://www.indiana.edu/~istd/patterns.html}

As such we have not included quotes but refer to their Web page for the
original source which may also include updates. Naturally we do not want
to be accused of plagiarize in a chapter about plagiarism.  Thus assume
the content for the rest of this chapter are copied from that Web
page. The resources in particular include:

\begin{itemize}
\item
  \href{https://www.indiana.edu/~istd/definition.html}{IU Definition} of
  Plagiarism from Student Code of Conduct
\item
  \href{https://www.indiana.edu/~istd/overview.html}{Overview} How to
  give proper credit, steps.
\item
  \href{https://www.indiana.edu/~istd/cases.html}{Cases} of Plagiarism
  in the US, in the news, and elsewhere
\item
  \href{https://www.indiana.edu/~istd/examples.html}{Examples} Word for
  word, paraphrasing
\item
  \href{https://www.indiana.edu/~istd/practice.html}{Practice} with
  feedback on word-for-word and paraphrasing plagiarism
\item
  \href{https://www.indiana.edu/~istd/test.html}{Test} 10 questions on
  recognizing plagiarism
\item
  \href{https://www.indiana.edu/~istd/sitemap.html}{Tutorial Site Map}
  Expanded table of contents
\item
  \href{https://www.indiana.edu/~istd/resources.html}{Resources}
  Websites, books, dictionary links, references for learning more about
  plagiarism
\end{itemize}

\subsection{Tutorials}\label{S:ptutorial}

A number of tutorials are offerd by Indiana University \href{http://education.indiana.edu/graduate/programs/instructional-systems/index.html}{Instructional
  Systems Technology Department}
 Web pages ealing with plagiarism. Thes include:

\begin{itemize}
\item
  \href{https://www.indiana.edu/~academy/firstPrinciples/choice.html}{Plagiarsim
    Tutorial}
\item
  \href{https://www.indiana.edu/~tedfrick/plagiarism/}{Understanding
  Plagiarism}
\end{itemize}

\subsection{How to Recognize Plagiarism}

We are listing fifteen patterns of plagiarism that are defined on the Web
pages itedntified in Section~\ref{S:ptutorial} as part of the
tutorials taht we recommend you take:

\begin{tabular}{p{4cm}p{4cm}p{6cm}}
Name & Plagiarism Type & Reason \\
\toprule
  \href{patternCluelessQuote.html}{Clueless Quote} &  word-for-word & no
  quotes, no citation, no reference 
\\
  \href{patternCraftyCoverUp.html}{Crafty Cover-up} &  proper paraphrase
  but word-for-word  &  also present
\\
  \href{patternCunningCoverUp.html}{Cunning Cover-up} &  paraphrasing & no
  citation, no reference
\\
  \href{patternDeceptiveDupe.html}{Deceptive Dupe} &  word-for-word  &  no
  quotes, no citation, but has reference
\\
  \href{patternDisconnectedDupe.html}{Delinked Dupe} &  word-for-word  &  no
  reference, even though quotes and citation
\\
  \href{patternDeviousDupe.html}{Devious Dupe} &  correct quote but word-for-word  & 
  also present
\\
  \href{patternDippyDupe.html}{Dippy Dupe} &  word-for-word  &  quotes
  missing, even though full citation and reference
\\
  \href{patternDisguisedDupe.html}{Disguised Dupe} &  looks like proper
  para, but actually word-for-word  &  no quotes, no locator
\\
  \href{patternDoubleTrouble.html}{Double Trouble} & word-for-word  and
  paraphrasing  & although has reference
\\
  \href{patternLostLoser.html}{Linkless Loser} &  word-for-word  &  citation
  and reference lacking, although has quotes and locator
\\
  \href{patternLostLocator.html}{Lost Locator} &  word-for-word  &  missing
  locator, although has quotes, citation, and reference
\\
  \href{patternPointlessParaphrase.html}{Placeless Paraphrase} &  paraphrasing  & 
  no reference, although citation present
\\
  \href{patternSeveredCite.html}{Severed Cite} &  paraphrasing  &  reference
  but no citation
\\
  \href{patternShirkingCite.html}{Shirking Cite} &  word-for-word  &  lacks
  locator and reference, although quotes and citation present
\\
  \href{patternTripleD.html}{Triple D--Disguised Disconnected Dupe} & 
  word-for-word  & looks like proper para, but no quotes, no reference, no locator
\end{tabular}

In addition they do specify  three patterns of
non-plagiarism:

\begin{tabular}{p{4cm}p{4cm}p{6cm}}
Name & Type & Description \\
\toprule
\\
  \href{patternCorrectQuote.html}{Correct Quote} &  non-plagiarizm & takes another's words
  verbatim and acknowledges with quotation marks, full in-text citation
  with locator, and reference
\\
  \href{patternProperParaphrase.html}{Proper Paraphrase} &  non-plagiarizm & summarizes
  another's words and acknowledges with in-text citation and reference
\\
  \href{patternMindlessParaphrase.html}{Parroted Paraphrase} &  non-plagiarizm & appears to
  be paraphrasing, and technically may not be plagiarism \ldots{}
\end{tabular}
