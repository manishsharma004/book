\FILENAME

\section{Markdown}\label{S:markdown}

The content form this section originates from see:
\url{https://en.wikipedia.org/wiki/Markdown}.

Markdown is a simple markup language, however there is no precise
standard defined for it and implementations may have features not
supported by other implementations. Nevertheless, it provieds as imple
and easy way to quicly develop clean looking documents.

There are severla tools that make markdown attractive allowing to
write the text in one window while at the same time seeing the
rendered out put in another.

This includes

\begin{description}

\item[Macdown] An editor for mardown targeted on OSX

\end{description}

To convert the markdown to other formats with \verb|pandoc|

\begin{verbatim}
# Heading

## Sub-heading

### Another deeper heading
 
Paragraphs are separated
by a blank line.

Two spaces at the end of a line leave a  
line break.

Text attributes _italic_, *italic*, __bold__, **bold**, `monospace`.

Horizontal rule:

---

Bullet list:

  * apples
  * oranges
  * pears

Numbered list:

  1. apples
  2. oranges
  3. pears

A [link](http://example.com).

\end{verbatim}

\subsection{Tools}

\begin{description}
\item [Dilinger] \URL{https://dillinger.io/}. A HTML5 based cloud
  enabled editor. It allows to download the created Markdown.
\item[Macdown] \URL{https://macdown.uranusjr.com/} ``MacDown is an
  open source Markdown editor for macOS''
\end{description}