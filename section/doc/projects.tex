\chapter{Projects}
\label{c:projects}
\index{Projects}
\FILENAME\

\section{Class: E516}\label{s:e516-project}

\subsection{Scope of Project}

The objective of the project is to define a clear problem statement
and create a framework to address that problem. This framework must
include a docker packaged service that includes all necessary
dependencies necessary to perform the analysis. For example if you are
using a machine learning algorithm your docker packaged service will
include the machine learning algorithm which works on a particular
dataset providing clustered ,classified or regression outputs to the
user. The final objective is to obtain results in form of graphs or
tabular mode. In selecting a dataset, you can use a dataset from a
public dataset like 
\URL{https://archive.ics.uci.edu/ml/datasets.html}
and pick a suitable dataset according to your problem statement. Final
results must be exposed via a REST service. For instance if it is a
classification problem, the should have the capability of entering a
new test data set or single data point (meaning a single record) via a
REST API endpoint and get the expected output in form of a JSON
object. UI creation is an optional task. This is the overall
expectation of the project.

\subsection{Deliverables}

\begin{itemize}
\item Finding a Data Set Size(A good judgement of the datasize would
  be 100 MB < datasize < 500MB): (Estimated Time 1 hour)
\item Cleaned Up Data Set: (Estimated Time 0.5 Week)
\item Spark Mllib or Scikit Learn ML Algorithm Application: (Estimated
  Time 0.5 Week) (You can use any other library you prefer)
\item Clustering or Regression or Classification results in Graphical
  Format (Matplotlib) (GUI optional, save graphs as png). Otherwise
  you can do some analysis on the dataset the way you propose to do
  and this way must be explicitly discussed with Professor Gregor. :
  (Estimated Time 5 days)
\item Swagger or Flask  Rest Service  to send  a sample data  set and  get output
  (terminal output is enough): (Estimated Time 1 day) (Completing with Swagger Bonus points)
\item Take results in a constant environment (ex: Chameleon Cloud or
  your PC) provide system configuration: (Estimated Time 2 days)
\item Create a Makefile with running, setting up, sample test case
  running commands: (Estimated Time 1 hour)
\item requirement.txt file for pip installation: (Estimated Time < 1 hour)
\item Package with Docker: (Estimated Time 10 hour)
\item 8 Page Report including (Abstract, Introduction, Project
  Procedure, Technology Usage, Results(Training Time breakdown, Data
  cleaning time breakdown, system set up time, time taken to system
  launch, define the performance of the machine you're taking
  results), Conclusion ,Work Distribution (if you have team members): (40 hours)
\item Dockerfile to run your project. (Your project must run in the
  docker environment, working project within docker is required).
\end{itemize}

Total Estimated Time : 3 weeks.

\subsubsection{Grading Scheme}

\begin{itemize}
\item Working Project Inside Docker + Dockerfile : 60\%
\item Project Report : 40\%
\end{itemize}

\subsection{Helpful Chapters}

You can learn how to use docker, referring to section
\ref{S:docker-local}.  Using docker in Chameleon clouds can be learn
via section \ref{S:docker-fg}.  Need more clarification regarding
steps in data cleaning and data processing, refer the section
\ref{s:e222-proj-exp}.


\section{Class: E616}\label{s:e616-project}

\subsection{Scope of Project}

The objective of the project is to define a clear problem statement
and create a framework to address that problem. This framework must
include a docker packaged service that includes all necessary
dependencies necessary to perform the analysis. For example if you are using a
machine learning algorithm your docker packaged service will include the
machine learning algorithm which works on a particular dataset
providing clustered ,classified or regression outputs to the user. The
final objective is to obtain results in form of graphs or tabular
mode. In selecting a dataset, you can use a dataset from a public
dataset like \URL{https://archive.ics.uci.edu/ml/datasets.html} and
pick a suitable dataset according to your problem statement. Final
results must be exposed via a REST service. For instance if it is a
classification problem, the should have the capability of entering a
new test data set or single data point (meaning a single record) via a
REST API endpoint and get the expected output in form of a JSON
object. UI creation is an optional task. This is the overall
expectation of the project. 

\subsection{Deliverables}

\begin{itemize}
\item Finding a Data Set Size(A good judgement of the datasize would
  be 100 MB < datasize < 500MB): (Estimated Time 1 hour)
\item Cleaned Up Data Set: (Estimated Time 0.5 Week)
\item Spark Mllib or Scikit Learn ML Algorithm Application: (Estimated
  Time 0.5 Week) (You can use any other library you prefer)
\item Clustering or Regression or Classification results in Graphical
  Format (Matplotlib) (GUI optional, save graphs as png). Otherwise
  you can do some analysis on the dataset the way you propose to do
  and this way must be explicitly discussed with Professor Gregor. :
  (Estimated Time 5 days)
\item Swagger or Flask  Rest Service  to send  a sample data  set and  get output
  (terminal output is enough): (Estimated Time 1 day) (Completing with
  Swagger Bonus points)
\item Take results in a constant environment (ex: Chameleon Cloud or
  your PC) provide system configuration: (Estimated Time 2 days)
\item Create a Makefile with running, setting up, sample test case
  running commands: (Estimated Time 1 hour)
\item requirement.txt file for pip installation: (Estimated Time < 1 hour)
\item Package with Docker: (Estimated Time 10 hour)
\item 8 Page Report including (Abstract, Introduction, Project
  Procedure, Technology Usage, Results(Training Time breakdown, Data
  cleaning time breakdown, system set up time, time taken to system
  launch, define the performance of the machine you're taking
  results), Conclusion ,Work Distribution (if you have team members): (40 hours)
\item Dockerfile to run your project. (Your project must run in the
  docker environment, working project within docker is required).
\end{itemize}

Total Estimated Time : 3 weeks.

\subsection{Helpful Chapters}

You can learn how to use docker, referring to section
\ref{S:docker-local}.  Using docker in Chameleon clouds can be learn
via section \ref{S:docker-fg}.  Need more clarification regarding
steps in data cleaning and data processing, refer the section
\ref{s:e222-proj-exp}.

\subsubsection{Grading Scheme}

\begin{itemize}
\item Working Project Inside Docker + Dockerfile : 60\%
\item Project Report : 40\%
\end{itemize}

\section{Class: i524}\label{s:i524-project}

\subsection{Scope of Project}

The objective of the project is to define a clear problem statement
and create a framework to address that problem. This framework must
include a docker packaged service that includes all necessary
dependencies necessary to perform the analysis. For example if you are
using a machine learning algorithm your docker packaged service will
include the machine learning algorithm which works on a particular
dataset providing clustered ,classified or regression outputs to the
user. The final objective is to obtain results in form of graphs or
tabular mode. In selecting a dataset, you can use a dataset from a
public dataset like \URL{https://archive.ics.uci.edu/ml/datasets.html}
and pick a suitable dataset according to your problem statement. Final
results must be exposed via a REST service. For instance if it is a
classification problem, the should have the capability of entering a
new test data set or single data point (meaning a single record) via a
REST API endpoint and get the expected output in form of a JSON
object. UI creation is an optional task. This is the overall
expectation of the project.

\subsection{Deliverables}

\begin{itemize}
\item Finding a Data Set Size(A good judgement of the datasize would
  be 100 MB < datasize < 500MB): (Estimated Time 1 hour)
\item Cleaned Up Data Set: (Estimated Time 0.5 Week)
\item Spark Mllib or Scikit Learn ML Algorithm Application: (Estimated
  Time 0.5 Week) (You can use any other library you prefer)
\item Clustering or Regression or Classification results in Graphical
  Format (Matplotlib) (GUI optional, save graphs as png). Otherwise
  you can do some analysis on the dataset the way you propose to do
  and this way must be explicitly discussed with Professor Gregor. :
  (Estimated Time 5 days)
\item Swagger or Flask  Rest Service  to send  a sample data  set and  get output
  (terminal output is enough): (Estimated Time 1 day) (Completing with Swagger Bonus points)
\item Take results in a constant environment (ex: Chameleon Cloud or
  your PC) provide system configuration: (Estimated Time 2 days)
\item Create a Makefile with running, setting up, sample test case
  running commands: (Estimated Time 1 hour)
\item requirement.txt file for pip installation: (Estimated Time < 1 hour)
\item Package with Docker: (Estimated Time 10 hour)
\item 8 Page Report including (Abstract, Introduction, Project
  Procedure, Technology Usage, Results(Training Time breakdown, Data
  cleaning time breakdown, system set up time, time taken to system
  launch, define the performance of the machine you're taking
  results), Conclusion ,Work Distribution (if you have team members): (40 hours)
\item Dockerfile to run your project. (Your project must run in the
  docker environment, working project within docker is required).
\end{itemize}

Total Estimated Time : 3 weeks.

\subsection{Helpful Chapters}

You can learn how to use docker, referring to section
\ref{S:docker-local}.  Using docker in Chameleon clouds can be learn
via section \ref{S:docker-fg}.  Need more clarification regarding
steps in data cleaning and data processing, refer the section
\ref{s:e222-proj-exp}.

\subsubsection{Grading Scheme}

\begin{itemize}
\item Working Project Inside Docker + Dockerfile : 60\%
\item Project Report : 40\%
\end{itemize}

\section{Class: E222}\label{s:classe22-proj}

For the final project in this class you need to do the following.

\begin{itemize}
\item Find a dataset : Sample Data Repositories
  \URL{https://archive.ics.uci.edu/ml/datasets.html}
\item Select a suitable dataset.
\item Use Scikit Learn Library to run Machine Learning algorithm
  depending on your problem.
\item Do some classifications, clustering or a regression calculation
  using a machine learning algorithm.
\item Use the results from classification to draw charts. You can use
  matplotlib to do this.
\item Use Flask Rest API to expose the data to the viewers. So people
  can send a data set and get the outputs as a json object.
\end{itemize}


\subsection{Simple Explained Project}\label{s:e222-proj-exp}

First select a simple data set like Adult data set from the above Data
Repository. If you read through the description in the data set, you
can find out that this data set is a classifiable data set. And also
it says there are 14 attributes per each data sample and there are
48842 data points with multiple classes. For the classification
problem, you can use an algorithm like Support Vector Machines and do
the classification using scikit learn SVM
\URL{http://scikit-learn.org/stable/modules/svm.html}. Or you can view
this problem as a regression problem if you want to predict the age of
a particular person
\URL{http://scikit-learn.org/stable/modules/generated/sklearn.linear_model.LinearRegression.html}.

First you need to clean up the data. For instance consider the following data point in the data set. 

\begin{lstlisting}
32,  Private, 205019, Assoc-acdm, 12, Never-married, Sales, Not-in-family, Black, Male, 0, 0, 50, United-States, <=50K
\end{lstlisting}

The expected output for this set of features is the age 32.

In your programme first you need to separate age column to a separate
file like y.csv and keep features in a file called, x.csv.
\begin{lstlisting}
Features: Private, 205019, Assoc-acdm, 12, Never-married, Sales, Not-in-family, Black, Male, 0, 0, 50, United-States, <=50K
Expected Output : 32
\end{lstlisting}

So you will be feeding x.csv as feature data and y.csv as
corresponding outputs. We do this because, the ML algorithms only
understand numbers.

But there is an important thing that you need to do when cleaning up
the data in the early stages of the project. Choose numerical values
for string fields.  For instance workclass or the first attribute in
the feature vector.  workclass: Private, Self-emp-not-inc,
Self-emp-inc, Federal-gov, Local-gov, State-gov, Without-pay,
Never-worked.  You can pick values as Private = 0, Self-emp-not-inc =
1, etc. Data preprocessing will take a longer time, but that has to be done. 


Then we need to show these data to the user in form of tabular or
chart representation. For this purpose you should use Python Pandas
and Matplotlib.  Now you have completed 80 percent of your
project. Then you must give a functionality to users. Now you need to
feed these data to the Linear Regression algorithm and get results and
then use matplotlib to show the regression curve. User submits a datapoint
in form of the original data format.

\begin{lstlisting}
User Input: Private, 205019, Assoc-acdm, 12, Never-married, Sales, Not-in-family, Black, Male, 0, 0, 50, United-States, <=50K
Expected Output : 12
Predicted Output : 12.5
\end{lstlisting}

This way you can compare the accuracy of your training model.


\subsection{Deliverables}

\begin{itemize}
\item Finding a Data Set (A good judgement of datasize would be 10MB < datasize < 100MB): (Estimated Time 1 hour)
\item Cleaned Up Data Set: (Estimated Time 0.5 Week)
\item Scikit Learn ML Algorithm Application: (Estimated Time 1 Week)
\item Clustering or Regression or Classification results in Graphical
  Format (Matplotlib) (GUI not needed, save graphs as png): (Estimated
  Time 10 hours)
\item Flask  Rest Service  to send  a sample data  set and  get output
  (terminal output is enough): (Estimated Time 1 day) (Optional : Bonus
  points)
\item Create a Makefile with running, setting up, sample test case
  running commands: (Estimated Time 1 hour)
\item requirement.txt file for pip installation: (Estimated Time < 1 hour)
\item Package with Docker: (Estimated Time 1 hour)
\item 4 Page Report including (Abstract, Introduction, Project
  Procedure, Technology Usage, Results, Work Distribution (if you have
  team members): (30 hours)
\end{itemize}

Total Estimated Time : 3 weeks.

\subsubsection{Grading Scheme}

\begin{itemize}
\item Working Project Inside Docker + Dockerfile : 60\%
\item Project Report : 40\%
\end{itemize}

