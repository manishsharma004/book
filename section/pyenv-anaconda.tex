\FILENAME\
\section{Anaconda}\label{anaconda}

\begin{description}
\item[We do not recommend that you use anaconda as it may]
interfere with your default python interpreters and setup.
\end{description}

\begin{description}
\item[This section about anaconda is experimental and has not]
been tested.
\end{description}

You can add anaconda to your pyenv with the following commands:

\begin{verbatim}
pyenv install anaconda2-4.3.1
pyenv install anaconda3-4.3.1
\end{verbatim}

Here we install both the version 2 and version 3 python environments
from anavconda. Please be aware that the install may tacke several
minutes. Make sure to install the latest release which you can find out
if you leave of the version after the 2 or 3.

When executing:

\begin{verbatim}
pyenv versions
\end{verbatim}

you will see after the install completed the anaconda versiosn
installed:

\begin{verbatim}
pyenv versions
system
2.7.13
2.7.13/envs/ENV2
3.6.1
3.6.1/envs/ENV3
* ENV2 (set by PYENV_VERSION environment variable)
ENV3
anaconda2-4.3.1
anaconda3-4.3.1
\end{verbatim}

Let us now create virtualenv for anaconda:

\begin{verbatim}
$ pyenv virtualenv anaconda2-4.3.1 ANA2
$ pyenv virtualenv anaconda3-4.3.1 ANA3
\end{verbatim}

\section{Excersise}\label{excersise}

\begin{description}
\item[Econda.1:]
Write installation instructions for an operating system of your choice
and add to this documentation.
\item[Econda.2:]
Replicate the steps above, so you can type in ENV2 and ENV3 in your
terminals to switch between python 2 and 3.
\end{description}
