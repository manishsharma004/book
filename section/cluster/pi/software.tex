\chapter{Operatiing Systems}

In this chapter we will provide information in how to install additional software that may not be provided with the default operating system

You can run different operationg systems on the PI, this includes 
rasbian, NOOBS, Dexter (for Grove PI) but aslo Windows 10.

We like to get feedback on what OS is best suited for which task.

\subsubsection{Rasbian}

\begin{exercise}
provide a mini tutorial 
\end{exercise}

\subsubsection{NOOBS}

\begin{exercise}
provide a mini tutorial 
\end{exercise}

\subsubsection{Dexter}

\begin{exercise}
provide a mini tutorial 
\end{exercise}

\subsubsection{Other Linux}

\begin{exercise}
provide a mini tutorial 
\end{exercise}

\subsection{Operatiing Systems}

You can run different operationg systems on the PI, this includes 
rasbian, NOOBS, Dexter (for Grove PI) but aslo Windows 10.

We like to get feedback on what OS is best suited for which task.

\subsubsection{Rasbian}

\begin{exercise}
provide a mini tutorial 
\end{exercise}

\subsubsection{NOOBS}

\begin{exercise}
provide a mini tutorial 
\end{exercise}

\subsubsection{Dexter}

\begin{exercise}
provide a mini tutorial 
\end{exercise}

\subsubsection{Other Linux}

\begin{exercise}
provide a mini tutorial 
\end{exercise}

\subsubsection{Windows IOT}

Previously we played a bit with Windows on Raspberry, but it was not
very well supported. This may have changed by now. Your task will be to
evaluate its feature and contrast it to other OSes.

There seems to be support on developing PI application on WIndows 10
itself, but as we do not have windows machines, we have not tried
this. You could escplore and update us.

\begin{exercise}
provide a mini tutorial 
\end{exercise}

\chapter{PI Software}

\section{Editors}

\subsection{emacs}

\begin{exercise}
provide a mini tutorial 
\end{exercise}

\subsection{vim}

\begin{exercise}
provide a mini tutorial 
\end{exercise}

\subsection{gedit}

\begin{exercise}
provide a mini tutorial 
\end{exercise}


\subsection{ssh}

The most iomportant program for us. describe ho to enable without GUI

point to our other ssh section.

\begin{exercise}
make sure to describe how to not copy private keys !!! and instead use
ssh keygen on each machine, gather all pub keys and distribute

write a cloudmesh command for this (may already exist) 
\end{exercise}



\begin{exercise}
provide a mini tutorial 
\end{exercise}




\section{Python 3}

\begin{exercise}
provide a mini tutorial 
\end{exercise}

sudo apt-get install python3

update to current version and try out. measure time it thakes to do that, it may take a long time use time commands before and after

make sure you use alst install

\URL{https://liftcodeplay.com/2017/06/30/how-to-install-python-3-6-on-raspbian-linux-for-raspberry-pi/}

can we remove the source after install to safe space?

\section{Python IDLE}

Click Menu > Programming > Python 3 (IDLE), and you’ll get a new window called ‘Python 3.4.2 Shell:’. This Shell works just like Python on the command line. Enter print(“Hello World”) to see the message.

\begin{exercise}
provide a mini tutorial 
\end{exercise}


\section{Docker}

see go, maybe there are other better resources.

\begin{exercise}
provide a mini tutorial 
\end{exercise}

\section{Go}

\url{https://blog.alexellis.io/golang-docker-rpi/}

\begin{exercise}
provide a mini tutorial 
\end{exercise}


\begin{exercise}
provide a mini tutorial 
\end{exercise}

\subsection{Eclipse}

\begin{exercise}
provide a mini tutorial 
\end{exercise}


\subsection{Adafruit Web IDE}

\URL{https://learn.adafruit.com/webide/overview}

\begin{exercise}
provide a mini tutorial 
\end{exercise}

\subsection{Coder}

\URL{https://googlecreativelab.github.io/coder/}

\begin{exercise}
provide a mini tutorial 
\end{exercise}

\chapter{Computing}

\section{Numpy}

Refer to other section in book and describe what is different

\begin{exercise}
provide a mini tutorial 
\end{exercise}

\section{Scipy}

Refer to other section in book and describe what is different

\begin{exercise}
provide a mini tutorial 
\end{exercise}

\section{Image Processing}

Refer to other section in book and describe what is different

\begin{exercise}
provide a mini tutorial 
\end{exercise}