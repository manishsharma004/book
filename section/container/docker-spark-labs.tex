\section{Apache Spark with Docker}\label{apache-spark-with-docker}

\subsection{Pull Image from Docker
Repository}\label{pull-image-from-docker-repository}

We use a Docker image from Docker Hub:
(https://hub.docker.com/r/sequenceiq/spark/) This repository contains a
Docker file to build a Docker image with Apache Spark and Hadoop Yarn.

\begin{lstlisting}
docker pull sequenceiq/spark:1.6.0
\end{lstlisting}

\subsection{Running the Image}\label{running-the-image}

In this step, we will launch a Spark container.

\subsubsection{Running interactively}\label{running-interactively}

\begin{lstlisting}
docker run -it -p 8088:8088 -p 8042:8042 -h sandbox sequenceiq/spark:1.6.0 bash
\end{lstlisting}

\subsubsection{Running in the
background}\label{running-in-the-background}

\begin{lstlisting}
docker run -d -h sandbox sequenceiq/spark:1.6.0 -d
\end{lstlisting}

\subsection{Run Spark}\label{run-spark}

After a container is launched, we can run Spark in the following two
modes: 1) yarn-client and 2) yarn-cluster. The differences between the
two modes can be found here:
https://spark.apache.org/docs/latest/running-on-yarn.html

\subsubsection{Run Spark in Yarn-Client
Mode}\label{run-spark-in-yarn-client-mode}

\begin{lstlisting}
spark-shell --master yarn-client --driver-memory 1g --executor-memory 1g --executor-cores 1
\end{lstlisting}

\subsubsection{Run Spark in Yarn-Cluster
Mode}\label{run-spark-in-yarn-cluster-mode}

\begin{lstlisting}
spark-submit --class org.apache.spark.examples.SparkPi --master yarn-client --driver-memory 1g --executor-memory 1g --executor-cores 1 $SPARK_HOME/lib/spark-examples-1.6.0-hadoop2.6.0.jar
\end{lstlisting}

\subsection{Observe Task Execution from Running Logs of Spark
Pi}\label{observe-task-execution-from-running-logs-of-spark-pi}

Let us observe Spark task execution by adjusting the parameter of
SparkPi and the Pi result from the following two commands.

\begin{lstlisting}
spark-submit --class org.apache.spark.examples.SparkPi --master yarn-client --driver-memory 1g --executor-memory 1g --executor-cores 1 $SPARK_HOME/lib/spark-examples-1.6.0-hadoop2.6.0.jar 10



spark-submit --class org.apache.spark.examples.SparkPi --master yarn-client --driver-memory 1g --executor-memory 1g --executor-cores 1 $SPARK_HOME/lib/spark-examples-1.6.0-hadoop2.6.0.jar 10000
\end{lstlisting}

\subsection{Write a Word-Count Application with Spark
RDD}\label{write-a-word-count-application-with-spark-rdd}

Let us write our own word-count with Spark RDD. After the shell has been started, copy and paste the following code in console line by line.

\subsubsection{Launch Spark Interactive
Shell}\label{launch-spark-interactive-shell}

\begin{lstlisting}
spark-shell --master yarn-client --driver-memory 1g --executor-memory 1g --executor-cores 1
\end{lstlisting}

\subsubsection{Program in Scala}\label{program-in-scala}

\begin{lstlisting}
val textFile = sc.textFile("file:///etc/hosts")
val words = textFile.flatMap(line => line.split("\\s+"))
val counts = words.map(word => (word, 1)).reduceByKey(_ + _)
counts.values.sum()
\end{lstlisting}

\subsubsection{Launch PySpark Interactive
Shell}\label{launch-pyspark-interactive-shell}

\begin{lstlisting}
pyspark --master yarn-client --driver-memory 1g --executor-memory 1g --executor-cores 1
\end{lstlisting}

\subsubsection{Program in Python}\label{program-in-python}

\begin{lstlisting}
textFile = sc.textFile("file:///etc/hosts")
words = textFile.flatMap(lambda line:line.split())
counts = words.map(lambda word:(word, 1)).reduceByKey(lambda x,y: x+y)
counts.map(lambda x:x[1]).sum()
\end{lstlisting}

