\MDNAME\
%%%%%%%%%%%%%%%%%%%%%%%%%%%%%%%%%%%%%%%%%%%%%%%%%%%%%%%%%%%%%%%%%%%%%%%%%%%%%%%
% DO NOT MODIFY THIS FILE
%%%%%%%%%%%%%%%%%%%%%%%%%%%%%%%%%%%%%%%%%%%%%%%%%%%%%%%%%%%%%%%%%%%%%%%%%%%%%%%

\section{Introduction to Docker Swarm}

\subsection{Swarm clusters}

A swarm is a group of machines that are running Docker and joined into a
cluster. Docker commands are executed on a cluster by a swarm manager.
The machines in a swarm can be physical or virtual. After joining a
swarm, they are referred to as \textbf{nodes}.

\subsection{Set up your swarm}

A swarm is made up of multiple nodes, which can be either physical or
virtual machines. The basic steps are:

\begin{enumerate}
\def\labelenumi{\arabic{enumi}.}
\item
  run \texttt{docker\ swarm\ init} to enable swarm mode and make your
  current machine a swarm manager,
\item
  then run \texttt{docker\ swarm\ join} on other machines to have them
  join the swarm as workers. Choose a tab below to see how this plays
  out in various contexts. We use VMs to quickly create a two-machine
  cluster and turn it into a swarm.
\end{enumerate}

\subsection{Create a cluster with VirtualBox}

To create virtual machines on your local machine, first create a couple
of VMs using docker-machine, using the VirtualBox driver:

\begin{lstlisting}
docker-machine create --driver virtualbox myvm1
docker-machine create --driver virtualbox myvm2
\end{lstlisting}

To list the VMs and get their ip addresses. Use this command to list the
machines and get their IP addresses.

\begin{lstlisting}
docker-machine ls
\end{lstlisting}

\subsection{Initialize the Swarm Manager Node and Add Worker Nodes}

The first machine acts as the manager, which executes management
commands and authenticates workers to join the swarm, and the second is
a worker.

\subsubsection{\texorpdfstring{Instruct \texttt{myvm1} to become a swarm manager:}{Instruct myvm1 to become a swarm manager:}}

\begin{lstlisting}
docker swarm init
\end{lstlisting}

output should be like this:

\begin{lstlisting}
$ docker-machine ssh myvm1 "docker swarm init --advertise-addr <myvm1 ip>"
Swarm initialized: current node <node ID> is now a manager.

To add a worker to this swarm, run the following command:

  docker swarm join \
  --token <token> \
  <myvm ip>:<port>

To add a manager to this swarm, run 'docker swarm join-token manager' and follow the instructions.
\end{lstlisting}

\subsubsection{\texorpdfstring{To construct \texttt{myvm2} as a worker node.}{To construct myvm2 as a worker node.}}

Copy this command, and send it to \texttt{myvm2} via docker-machine ssh
to have \texttt{myvm2} join your new swarm as a worker:

\begin{lstlisting}
$ docker-machine ssh myvm2 "docker swarm join \
--token <token> \
<ip>:2377"
\end{lstlisting}

The output should be like this:

\begin{lstlisting}
This node joined a swarm as a worker.
\end{lstlisting}

Run \texttt{docker-machine\ ls} to verify that \texttt{myvm1} is now the
active machine, as indicated by the asterisk next to it.

The output should be like this:

\begin{lstlisting}
$ docker-machine ls
NAME    ACTIVE   DRIVER       STATE     URL                         SWARM   DOCKER        ERRORS
myvm1   *        virtualbox   Running   tcp://192.168.99.100:2376           v17.06.2-ce   
myvm2   -        virtualbox   Running   tcp://192.168.99.101:2376           v17.06.2-ce   
\end{lstlisting}

\subsection{Deploy the application on the swarm manager}

To deploy an application, run the following command to deploy on
\texttt{myvm1}.

\begin{lstlisting}
docker stack deploy -c docker-compose.yml getstartedlab
\end{lstlisting}

To verify the services (and associated containers) have been distributed
between both \texttt{myvm1} and \texttt{myvm2}.

\begin{lstlisting}
$ docker stack ps getstartedlab

ID            NAME                  IMAGE                   NODE   DESIRED STATE
jq2g3qp8nzwx  getstartedlab_web.1   john/get-started:part2  myvm1  Running
88wgshobzoxl  getstartedlab_web.2   john/get-started:part2  myvm2  Running
vbb1qbkb0o2z  getstartedlab_web.3   john/get-started:part2  myvm2  Running
ghii74p9budx  getstartedlab_web.4   john/get-started:part2  myvm1  Running
0prmarhavs87  getstartedlab_web.5   john/get-started:part2  myvm2  Running
\end{lstlisting}

