% chktex-file 8

\chapter{Assignments}
\label{c:assignments}

\FILENAME\

The assignments are listed in chronological order. All assignments
posted here are supposed to be conducted by you. There is a slight
delay between assignments posted in piazza and the assignments in this
section as TAs need to be integrating them. Typical delay is one
business day. Business days are defined as Mon-Fri 9am-5pm
EST. Thus, it is often better for those working on the weekends to
visit piazza.com instead. The assignments are ordered by class. Please
focus on conducting the assignments listed either here on in
piazza. Just as any textbook has many exercises, we are providing
selected exercises for you as assignments. Not every exercise
mentioned in a chapter has to be done.

\begin{IU}
  There can not be any confusion which assignments have been issued,
  as they are all pinned in piazza and you can visit piazza.com to
  find out which are assigned and listed there.
\end{IU}

If you use a calendar system, it is in your responsibility
to manage it in such a system. This could include Google, Outlook,
CANVAS, and may others. Naturally you can also use to do lists that
you can manage as part of your github repository issues, once you have
access to. This is probably the preferred method as it allows you to
add the tasks yourself and you can check them on and off as you have
conducted the assignments. 

This way we teach you how to use open source technologies to
coordinate your own work with your own time management tools and
constraints in mind. This is in contrast to just using CANVAS as
CANVAS does not support open source developments in teams. Furthermore
it is unlikely that you will ever use CANVAS after you graduate. We
rather like to have you use systems that you use after you graduate.
However, if you still like to use CANVAS for alerting you, that is
entirely up to you as you can add assignments yourself to CANVAS. 

\smallskip

\begin{IU}
  All assignments are due Monday morning at 9:00 am est. No exceptions
  unless otherwise specified.
\end{IU}

\section{Assignments E222}
\label{s:e222-assignment}
\label{s:e222-assignments}
\index{Assignments!E222}

All assignments posted here are supposed to be conducted by you. 

\subsection{Bio Post}
\label{E:e222-bio}

\begin{exercise}\label{E:e222-bio-piazza}
{\bf Bio Post on Piazza.} Please post a formal bio to piazza
\end{exercise}


\begin{exercise}\label{E:e222-bio-googledocs}

  {\bf Bio Post in Google doc.} due Fed 5, 2018. 
  
  After you have posted it to piazza
  copy your updated formal bios into the following document.  Make
  sure you use 3rd person and stay formal. This is a formal
  bio. Comment on the effectiveness of using the cloud service for
  this task. A the end of the document. This assignment does not
  replace the post of the bio to piazza, it is used to gather all bios
  in one document and to evaluate if google docs is a good tool for
  this kind of task. Remember we have lots of students and google is
  used often just with small groups.
 
 \smallskip

 {\hfill
   \href{https://docs.google.com/document/d/1pNK94qoRfZkill_JrGAzjd8aQ6Aar0pEXhU_Tgog0W0/edit?usp=sharing}{E222
     Link to google doc $\mapsto$}}


 \end{exercise}


\subsection{Cloud Accounts}
\label{E:e222-iu-google-services}

\begin{exercise}\label{E:e222-iu-google}

  {\bf IU Google Services.} due Feb 5, 2018
  
  This assignment is only for those that do
  not yet have access to our google documents This assignment does not
  have to be conducted for anyone that has access to our google
  documents for bios, and the technology list

  \begin{itemize}
 
  \item What is the difference between umail.iu.edu and iu.edu? Tip:
    the answer is provided in the IU knowledge base

  \item Login via the iu.edu account and not the umail.iu.edu account
    to google and open the document for the bio. Paste the bio into
    the document.

  \item Explain why IU has two different google services and
    logins. As we use cloud in this class, it is important to
    understand this and what implication this has. This is not just an
    assignment to give you access to the service, but to make you
    think why this works like this.

  \item Can you imagine a different way this ought to work?

  \end{itemize}

\end{exercise}


\subsection{Account Creation}

\begin{exercise}

  {\bf Account Creation: github.com}
  
  If you do not have a github.com
  account, go to github.com and apply for a \url{https://github.com}
  account. Write down your account name and remember the password. You
  will need the account for upcoming assignments.

\end{exercise}

\begin{exercise}

  {\bf Account Creation: futuresystems.org}
  
  If you do not have a futuresystems.org account go to
  \url{https://portal.futuresystems.org/user/register} and apply for an
  account. Write down your account name and remember the password. You
  will need the account for upcoming assignments.

\end{exercise}

\begin{exercise}
  {\bf Account Creation: chameleoncloud.org}. 
  
  If you do not have a chameleon cloud account please go to 
  \url{https://www.chameleoncloud.org} and apply for an
  account only. Do not apply for a project. Write down your account
  name and remember the password. You will need the account for
  upcoming assignments.
\end{exercise}

\begin{exercise}
   {\bf Account Collection Form } due Fed 5, 2018
 
 Fill out the form so we can activate your accounts. You will need the account for upcoming assignments.
 
  {\hfill \href{https://goo.gl/forms/W0MdgoJoY8F6Vt9Q2}{E222 Account Collection Form $\mapsto$}}
 
\end{exercise}

\subsection{Entry Survey}
\begin{exercise}
    {\bf Entry Survey}
    
 Please fill out the following survey ASAP as it will determine some of the class material we prepare based on your feedback. The survey is really simple and can be finished in under 5 minutes. \url{https://goo.gl/forms/Q04FW9eBM7eyL0Lv1}
\end{exercise}

\subsection{Account Setup}

\begin{exercise} {\bf Github Setup} due Feb, 2018

This exercise will assist you with your github account setup. Please
find the README.yml file in the link below and copy it into your
github repository and fill out with your
information. \url{https://github.com/cloudmesh-community/hid-sample/blob/master/README.yml}
Remove the README.md file. After your \verb|.yml| file is correctly setup you
now need to put your bio in your repository, naming it like so
\verb|bio-lastname-firstname.tex| where lastname is your lastname and so
on. Put the \verb|.tex| file in your repository and use the latex function
subsection\{lastname,firstname\} followed by the text of your bio.


\end{exercise}

\subsection{Eve REST Service}
\begin{exercise}
Read the Sections Overview of Rest (Section~\ref{c:rest} and Eve~\ref{s:eve-intro}).
\end{exercise}

\begin{exercise}
Assignment catch up: please look at all previous assignments and do
them. 
\end{exercise}

\begin{exercise}

AI cloud service: This assignment will evolve, therefore we just
provide the link to it in piazza. Please visit it:

\url{https://piazza.com/class/jc9dcfnbi045kv?cid=34}
\end{exercise}

\section{Assignments E516, I524, E616}
\label{s:616-assignments}
\index{Assignments!E516}
\index{Assignments!E616}
\index{Assignments!I524}

\subsection{Bio Post}\label{a:616-bio}

\begin{exercise}\label{E:616-bio-piazza}
{\bf Bio Post on Piazza.} Please post a formal bio to piazza
\end{exercise}

\begin{exercise}\label{E:616-bio-googledocs}

  {\bf Bio Post in Google doc.} due Feb 5, 2018
  
  After you have posted it to piazza
  copy your updated formal bios into the following document.  Make
  sure you use 3rd person and stay formal. This is a formal
  bio. Comment on the effectiveness of using the cloud service for
  this task. A the end of the document. This assignment does not
  replace the post of the bio to piazza, it is used to gather all bios
  in one document and to evaluate if google docs is a good tool for
  this kind of task. Remember we have lots of students and google is
  used often just with small groups.
 
 \smallskip

 {\hfill
   \href{https://docs.google.com/document/d/1ejzlKYqC3dLac8WXVpcPQsJh1j4BDqRxxgGg1cFQbeQ/edit?usp=sharing}{E516
     Link to google doc $\mapsto$}}


 \end{exercise}

\subsection{IU Google Services}
\label{E:e616-iu-google-services}

\begin{exercise}\label{E:616-iu-google}

  {\bf IU Google Services:} due Feb 5, 2018
  
  This assignment is only for those that do
  not yet have access to our Google documents This assignment does not
  have to be conducted for anyone that has access to our Google
  documents for bios, and the technology list

  \begin{itemize}
 
  \item What is the difference between umail.iu.edu and iu.edu? Tip:
    the answer is provided in the IU knowledge base

  \item Login via the iu.edu account and not the umail.iu.edu account
    to google and open the document for the bio. Paste the bio into
    the document.

  \item Explain why IU has two different google services and
    logins. As we use cloud in this class, it is important to
    understand this and what implication this has. This is not just an
    assignment to give you access to the service, but to make you
    think why this works like this.

  \item Can you imagine a different way this ought to work?

  \end{itemize}

\end{exercise}

\subsection{Big Data Collaboration}
\label{E:616-bigdata-collab}

\begin{exercise}\label{E:616-big-data-and-collaboration} {\bf Big
    data and collaboration.} due Feb 2018
    
  The purpose of this assignment is
  multifold; test the ability of Google docs to be used in
  collaborative fashion by more than a small group and report on the
  experience. Good Things and bad things, learn on how to use Google
  docs with headings and table of contents learn how to gather
  resources quickly with hyperlinks to web resources or articles and
  translate them into formal academic references. Most importantly
  convey some very important feature of big data.Contribute this into
  the handbook for everyone's benefit (done by TAs).  \smallskip

  \noindent {\bf Task:} Your task is to identify Big Data size related
  articles and Web resources and produce a historical development of
  the growth of this data

  {\hfill
    \href{https://docs.google.com/document/d/1ZHNdhX_Jx7uBQo0kthSYQ6TQR8_KNbgOwH2EuqBQcjY/edit?usp=sharing}{E516
      Link to google doc $\mapsto$}}


\end{exercise}

\subsection{New Technology List}
\label{E:616-new-tech}

\begin{exercise} {\bf Technology List} due Jan 29, 2018

 The handbook contains a large number of technologies to which an
 abstract is provided. Your task is to identify FIRST not to do an abstract but to
 collaboratively gather a LIST of new technologies that are important
 in Cloud and Big Data. We suggest doing this in a google docs document
 first. Write Lastname, Firstname, class id behind the technology so we
 know who contributed it. Indicate also if commercial, or open source,
 We are mostly interested in open source activities. Keep the list
 sorted by alphabet. Use a bullet so formatting is preserved

\smallskip

{\hfill \href{https://docs.google.com/document/d/1LeHGHTSBbaPXYVor0efhmi5W7JJjS7EQHABHqgRAPuU/edit?usp=sharing}{New Technology List $\mapsto$}}

\smallskip

Example: OpenWhisk, \url{https://openwhisk.apache.org/}, open source, Gregor von Laszewski, e616

\end{exercise}

\subsection{New Technology Abstract}
\label{E:616-new-tech-abstract}

\begin{exercise} {\bf Technology Abstract} due Feb 5, 2018

We have gathered with the technology list
\url{https://piazza.com/class/jbkvbp3ed3m2ez?cid=50} a number of
technologies that are not yet covered in the handbook or
need improvement in the handbook.


The TAs will be selecting about 5 technologies for each student. Each
student will write high-quality non-plagiarized abstracts which bibtex
references.
 
Learning outcomes:

\begin{itemize}

\item Identify how to not plagiarize
\item Work in a large team (with coordination by TAs)
\item Use bibtex and jabref for reference management which you will be using for your final paper
\item Find new trends in big data and cloud computing

\end{itemize}

\end{exercise}

\begin{exercise}{\bf Technology Abstract upload to GitHub} due Feb 26,
  2018


Please review the plagiarism and quoting guidelines chapter in the
handbook.

\WHERE{\YES}{S:plagiarism}{Week 1} 

The following is an example of how you upload your technology
abstracts to github.
\smallskip

{\hfill \href{https://github.com/cloudmesh-community/hid-sample/tree/master/technology}{Upload
    Tech Abstract to github $\mapsto$}}

\smallskip

Please direct any questions toward the TAs, additionally there is a
README available below. 

{\hfill
  \href{https://github.com/cloudmesh-community/hid-sample/blob/master/technology/README.md}{README
    $\mapsto$}}

\smallskip

The report will be generated on Mondays at 9:00 am est and made
available by 12:00 pm est of the same day.
\smallskip

\URL{https://drive.google.com/open?id=1h6_ZRmlCRIFMHG861wSyriPzn9rXxgKT}

\end{exercise}

\subsection{Cloud Accounts}

\begin{exercise}

  {\bf Account Creation: github.com}. If you do not have a github.com
  account, go to github.com and apply for a \url{https://github.com}
  account. Write down your account name and remember the password. You
  will need the account for upcoming assignments.

\end{exercise}

\begin{exercise}

  {\bf Account Creation: futuresystems.org}. 
  
  If you do not have a
  futuresystems.org account go to
  \url{https://portal.futuresystems.org/user/register} and apply for an
  account. Write down your account name and remember the password. You
  will need the account for upcoming assignments.

\end{exercise}

\begin{exercise}
  {\bf Account Creation: chameleoncloud.org}. 
  
  If you do not have an account on chameleon cloud please go to 
  \url{https://www.chameleoncloud.org} and apply for an
  account only. Do not apply for a project. Write down your account
  name and remember the password. You will need the account for
  upcoming assignments.
\end{exercise}

\begin{exercise}
Fill out the form so we can activate the accounts for you
\url{https://goo.gl/forms/W0MdgoJoY8F6Vt9Q2}
You will need the account for
  upcoming assignments.
\end{exercise}

\subsection{Entry Survey}
\begin{exercise}
Please fill out the following survey ASAP as it will determine some of the class material we prepare based on your feedback. The survey is really simple and can be finished in under 5 minutes.

\url{https://goo.gl/forms/Q04FW9eBM7eyL0Lv1}
\end{exercise}

\subsection{Account Setup}

\begin{exercise} {\bf Github Setup} due Feb, 2018

This exercise will assist you with your github account setup. Please
find the README.yml file in the link below and copy it into your
github repository and fill out with your
information. \url{https://github.com/cloudmesh-community/hid-sample/blob/master/README.yml}
Remove the README.md file. After your \verb|.yml| file is correctly setup you
now need to put your bio in your repository, naming it like so
\verb|bio-lastname-firstname.tex| where lastname is your lastname and so
on. Put the \verb|.tex| file in your repository and use the latex function
subsection\{lastname,firstname\} followed by the text of your bio.


\end{exercise}

\subsection{REST}
\begin{exercise}
Read the Sections Overview of Rest Section~\ref{c:rest} and Eve~\ref{s:eve-intro}.
\end{exercise}

\begin{exercise} {\bf Develop an Eve REST Service} due Feb, 2018

See: \url{https://piazza.com/class/jbku81aeli95rz?cid=55}

In this exercise, you will be developing an Eve REST service related
to Cloud Services. We will enhance this assignment throughout the
semester once we have spoken about cloud services in more detail. At
this time you are expected to write a rest service that exposes
information form your computer, such as, processor name, RAM,
Disk. Please identify what information would be useful to have and how
to obtain that information related to your operating
system. Additionally identify how to integrate dynamic data.


\end{exercise}

\begin{exercise}
Assignment catch up: please look at all previous assignments and do
them. 
\end{exercise}

\begin{exercise}

AI cloud service: This assignment will evolve, therefore we just
provide the link to it in piazza. Please visit it:

\url{https://piazza.com/class/jc9dcfnbi045kv?cid=34}
\end{exercise}

\subsection{Swagger REST Services}

\subsubsection{PART A: Elementary Swagger Server}
 
\begin{exercise}
  This is for residential students (The swagger preparation was
  mentioned last week in class):

 \begin{itemize}

 \item (A) you are expected to have done major portions of the Swagger
   code gen assignment this includes

  \begin{itemize}
  \item (1) pick of a resource you like to implement a rest service
    for
  \item (2) read the swagger spec
  \item (3) complete the spec for the resource as much as possible
  \item (4) generate the code via swagger codegen
  \end{itemize}

  This naturally means you need swagger, python and other needed
  libraries set up on your machine. We will help refining your
  resource. We will help working with you on a simple back-end
  implementation that you will improve. You will be using github for
  all of this

\item (B) In addition to this I recommend that you

  \begin{itemize}

  \item (1) find on the network how to put Rasbian on an SD card,
    create your self an md file for this as this may be different for
    different operating systems.

 
  \item (2) download rasbian on your machine so you can burn it as you
    will get 5 sd cards
 
    In the lab, those that finish most of the swagger service will
    switch to building a cloud cluster.
 
  \item (3) in the lab you will be first thing change the password to
    each of them before you put them on the network. Maybe there is a
    way to do this directly from your computer after you have put
    rasbian on the sd card. But I do not know if that's possible.

 
  \item (4) figure out how to configure Rasbian without a monitor,
    while just using an SDcard and your laptop, write an md file

  \end{itemize}
 \end{itemize}
 
In case of questions, lets engage in a discussion. If you have md
files for information already in your repo, please post URLs. This
assignment is about openly sharing. Naturally, you should not wait
till someone else does it, you should take leadership yourself.

 
Naturally, focus on the swagger service before starting to get more
involved with the cluster

\end{exercise}

\begin{exercise} {\bf Cloud and Big Data Rest Service with Swagger}
  due March 5, 2018 Spec before Feb 15, 2018

\begin{itemize}
     
\item (A) Read the Chapters about REST and Swagger. This includes
  Swagger specification and Swagger codegen. There is also a video
  that introduces you to Swagger \URL{https://youtu.be/0_Ub13py_K8}
 
\item (B) Read the Document
  \URL{https://laszewski.github.io/papers/NIST.SP.1500-8-draft.pdf}
 
  We will be collaboratively developing a new version of this document
  while not using examples as in the previous document but swagger
  OpenAPI 2.0 specifications.
 
\item (C) You will pick one of the existing resources or identify a
  new resource that you would like to specify. Simply go to:

\URL{https://docs.google.com/document/d/12FUtHlEzQwxc3hjyki3RbU0iMfBrcyoSMKeq3aEDPk/edit?usp=sharing
} 

and add your name to one of the resource. Make sure there is only one
name for each resource even if you work in a team.
 
\item (D) For the resource, you chose you will be developing with
  Swagger a useful REST service related to cloud computing and Big
  Data.

 
 \end{itemize}

 TA's will provide more details as we will avoid that everyone
 develops the same service.
 
 You will use this specification create a swagger python service and
 implement functions to enable a real implementation of the service
 that is useful.

 
Examples:

\begin{itemize}
\smallskip
\item (A) develop a service to upload files to a file system with REST calls.

\item (B) start a cluster of virtual machines on a supercomputer

\item (C) start a cluster of virtual machines on OpenStack

\end{itemize}


Additional resources are listed in the instructor answer. 
\end{exercise}

\subsubsection{PART B: Reproducable Swagger Services}

\begin{exercise}
It is important to be able to reproduce the Services and not just
create a code that you can run. In order to do that we will only store
the absolute minimum information in github and autogenerate the code
via the yaml file, your controller and a Makefile that you will
design.

Naturally we needed you to have understood the Swagger codegen tool
first, so you need to be familiar on how to create a swagger service.
Now that you are we can continue with this step. THis will even
include removing code that you uploaded to github.

Thus, We like you to review the following and engage with us in online
meetings if this is not clear. It is actually rather simple. This is
used to prepare you for the way we expect you to deliver the final
project. It also replicates the way we have taught you to compile your
latex pdf document. Wait -- what has latex to do with swagger services?
Technically nothing conceptional we do

 
\begin{itemize}
\item make in paper dir -> produces latex document

\item make in swagger dir-> produces swagger service

\item and in future make in project dir -> produces project services
\end{itemize}
 

So we just use the same framework, which is very convenient!

Now let us get back to the swagger service generation:
 
One of the goals of this project is to create a REPRODUCABLE framework
for generating the services. It is not enough to just develop a
swagger service. You will need to generate the service from a shell
script and/or makefile. As both technologies are available on any
computer including Windows it is your responsibility to make sure you
have make and bash installed. And use them.

Swagger-codegen applied to your yaml file will create a directory
structure. THis directory structure is not to be checked into
github. Instead you will check in the makefile or shell script (bash)
that creates it.

You will see in the controller dir a number of \verb|controller_*.py|
files, you will copy them into your github dir. The names of the
controller files can be automatically specified based on the content
in your yaml file.

 

Thus you will have a directory with the following contents, replace
000 with the id you have

 
\begin{lstlisting}
hid-sp18-000/swagger/Makefile
hid-sp18-000/swagger/conroller_a.py
hid-sp18-000/swagger/conroller_b.py
\end{lstlisting}
 

We assume that swagger-codegen is a shell variable allowing you to run
swagger-codegen.

This could be different on different systems. You will be documenting
this for your system. On OSX this is trivial as you just use brew to
install and

\begin{lstlisting}
export SWAGGER-CODEGEN=swagger-codegen
\end{lstlisting}
 
will typically work. On other systems you may have to specify the jar
file. You will be using an environment variable regardless which OS
you are on

If you use a Makefile you will be defining the following tags

 
\begin{itemize}
\item make clean -- removes the code generated

\item make service -- creates the swagger service from the yaml file
  and places the controllers in the appropriate directory

\item make start  -- starts the service

\item make stop -- stops the service

\item make test -- executes a number of tests against the service
\end{itemize}

You are allowed as part of this use nose or any other unit test. 

\end{exercise}

\subsubsection{Swagger Container}

\begin{exercise}

STEP 3 generating Swagger RESt Containers

After you have finished the STEP 2 in the Swagger series of exercises,
YOu will now generate a container. However, you will neither check in
the container in docker hub nor into github.

 

Instead, you will generate a container with a Dockerfile. You will add
to your makefile the tag

 
\begin{lstlisting}
make container -- which will generate the container form the
                  Dockerfile. Your container will be named cloudmesh-<YOURTOPIC>
\end{lstlisting}

Where topic is how you named your service

As we will start multiple services from multiple students you need to
have a proper namespace in the yaml specification file THis may be
different for different people. For example

 
\begin{lstlisting}
cloudmesh/var/<id>
cloudmesh/aws/vm/<id>
cloudmesh/google/vm/<id>
cloudmesh/openstack/vm/<id>
cloudmesh/filter/<id>
\end{lstlisting}

 

Naturally, if you have better url suggestions please integrate,
however, we need a unique prefix for each service so if we were to
combine them we can do that.

Reply to this followup discussion
\end{exercise}

\subsection{Technology Paper}
\label{E:616-tech-paper}

\begin{exercise} {\bf Technology Paper} due March 20, 2018

\begin{itemize}
Read the following two points to assist you in starting your paper 
\item (A) pick one of the technologies identified by you or if you see a
  hid that has already picked a technology for this assignment you can
  also pick one form that students list also

\item (B) write a LaTeX paper about the technology. Make sure not to
  plagiarize. The maximum number of quotes is about 25\% if
  needed. Please see the scientific writing section

\end{itemize}
 WE RECOMMEND YOU GET STARTED ON THIS RIGHT AWAY AS YOU ALSO WILL HAVE
 TO DO A TUTORIAL AND THE FINAL PROJECT.

Additional details can be found here:
\url{https://github.com/cloudmesh-community/hid-sample/blob/master/paper-instructions.md}

\end{exercise}

\subsection{Tutorial}
\label{E:616-tutorial}

\begin{exercise} {\bf Tutorial} due March 26, 2018

{\bf This is for online and residential students.}

Residential students with a substantial tutorial such as install a
kubernetes cluster on PI or docker swarm or some other larger topic
are exempt from this assignment. However, if you are working towards
an A+ consider adding an additional tutorial.

 
We like you to pick a technology and develop for this technology a
tutorial. The tutorial can be written either in markdown or in
LaTeX. However, when we like that you do not use enumerations for
steps that you document. Instead use sentences such as
 
First, we do 
 
Second, we need to
 
Next, we implement
 
Please use sections, subsections and so on to the
structure. Additionally please create a seperate directory for images
called {\bf images} where all the images are stored like the structure
below. 

\begin{lstlisting}
hid-sp18-000/tutorial/images
hid-sp18-000/tutorial/tutorial.tex
\end{lstlisting}
 
DO NOT USE SCREENSHOTS FOR CODE EXAMPLES OR CAPTURE OF THE TERMINAL
OUTPUT.  USE ASCII and put it either in lstlisting in latex, or in an
indented codeblock in markdown. Please do not add python, bash or
other markings to your codeblock, as we need real simple markdown if
you chose that. Make sure text in LaTeX.

 
We will make additional suggestions for tutorial topics in the
document in which we collect the tutorial list. We will also indicate
for these suggestings a maximum number of students able to work
together on that tutorial.

{\hfill
  \href{https://docs.google.com/document/d/1L2-wYc7S7hb5u6ZNtKpTvlXqKMkqq-B38hlaBCw-eww/edit?usp=sharing}{Tutorial
    List$\mapsto$}}



\end{exercise} 

\subsection{Project}

\TODO{Add the project assignment that was posted to piazza}

\begin{exercise} 

TBD

\end{exercise} 

%\end{comment}