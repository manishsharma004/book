\chapter{Assignments}\label{c:assignments}

\section{Assignments E222}
\label{s:e222-assignments}
\index{Assignments!E222}

All assignments are due Monday morning at 9:00 am est. No exceptions unless otherwise specified.  

\subsection{Bio Post}

\begin{exercise}\label{a:e222-bio-piazza}

{\bf Bio Post on Piazza.} Please post a formal bio to piazza

\end{exercise}
 
\begin{exercise} \label{a:e222-bio-googledocs}

{\bf Bio Post in Google doc.} Due Jan XX

After you have posted it to piazza copy your updated formal bios into the following document.  Make sure you use 3rd person and stay formal. This is a formal bio. Comment on the effectiveness of using the cloud service for this task. A the end of the document. This assignment does not replace the post of the bio to piazza, it is used to gather all bios in one document and to evaluate if google docs is a good tool for this kind of task. Remember we have lots of students and google is used often just with small groups.
 
 \smallskip

 {\hfill \href{https://docs.google.com/document/d/1ejzlKYqC3dLac8WXVpcPQsJh1j4BDqRxxgGg1cFQbeQ/edit?usp=sharing}{E222 Link to google doc $\mapsto$}}

\end{exercise}

<<<<<<< HEAD
\section{Assignments E516}\label{s:e516-assignments}
=======
\section{Assignments E516, I524, E616}
\label{s:e516/524/616-assignments}
>>>>>>> c665577c0fee6db15175c03403a5adaeef13b724
\index{Assignments!E516}

All assignments are due Monday morning at 9:00 am est. No exceptions unless otherwise specified.

\subsection{Bio Post}

\begin{exercise} \label{a:e516-bio-piazza}

{\bf Bio Post on Piazza.} Please post a formal bio to piazza

\end{exercise}

\begin{exercise} \label{a:e516-bio-googledocs}

 {\bf Bio Post in Google doc.} Due Jan XX
 
 After you have posted it to piazza copy your updated formal bios into the following document.  Make sure you use 3rd person and stay formal. This is a formal bio. Comment on the effectiveness of using the cloud service for this task. A the end of the document. This assignment does not replace the post of the bio to piazza, it is used to gather all bios in one document and to evaluate if google docs is a good tool for this kind of task. Remember we have lots of students and google is used often just with small groups.

 \smallskip

 {\hfill \href{https://docs.google.com/document/d/1ejzlKYqC3dLac8WXVpcPQsJh1j4BDqRxxgGg1cFQbeQ/edit?usp=sharing}{E516 Link to google doc $\mapsto$}}

 \end{exercise}

\subsection{Big Data Collaboration}

\begin{exercise} \label{a:e516-big-data-and-collaboration}

{\bf Big data and collaboration.} Due Jan 22

The purpose of this assignment is multifold; test the ability of Google docs to be used in collaborative fashion by more than a small group and report on the experience. Good Things and bad things, learn on how to use Google docs with headings and table of contents learn how to gather resources quickly with hyperlinks to web resources or articles and translate them into formal academic references. Most importantly convey some very important feature of big data.Contribute this into the handbook for everyone's benefit (done by TAs). Your task is to identify Big Data size related articles and Web resources and produce a historical development of the growth of this data

  {\hfill \href{https://docs.google.com/document/d/1ZHNdhX_Jx7uBQo0kthSYQ6TQR8_KNbgOwH2EuqBQcjY/edit?usp=sharing}{E516 Link to google doc $\mapsto$}}
 
\end{exercise}

\subsection{New Technology List}

\begin{exercise} \label{a:e516-new-tech-list}

{\bf New Technology List} Due: Jan 29

The handbook contains a large number of technologies to which an abstract is provided. Your task is to identify FIRST not to do an abstract but to collaboratively gather a LIST of new technologies that are important in Cloud and Big Data. We suggest doing this in a google docs document first. Write Lastname, Firstname, class id behind the technology so we know who contributed it. Indicate also if commercial, or open source, We are mostly interested in open source activities. Keep the list sorted by alphabet. Use a bullet so formatting is preserved

 
{\hfill \href{https://docs.google.com/document/d/1LeHGHTSBbaPXYVor0efhmi5W7JJjS7EQHABHqgRAPuU/edit?usp=sharing}{E516 Link to google doc $\mapsto$}}
 
Example: OpenWhisk,https://openwhisk.apache.org/, open source, Gregor von Laszewski, e516
 
\end{exercise}

\subsection{New Technology Abstract}

\begin{exercise} \label{a:e516-new-tech-abstracts}

<<<<<<< HEAD
{\bf New Technology Abstract}
Due date: Feb 5th

We have gathered the technology list \href{https://piazza.com/class/jbkvbp3ed3m2ez?cid=50}{tech list document} a number of technologies that are not yet covered in the handbook or need improvement in the handbook.\\

The TAs will be selecting about 5 technologies for each student. Each student will write high-quality non-plagiarized abstracts which bibtex references.\\
 
Details will be announced by the TAs\\
 
Learning outcomes:\\

\noindent Identify how to not plagiarize\\
Work in a large team (with coordination by TAs)\\
Use bibtex and jabref for reference management which you will be using for your final paper\\
Find new trends in big data and cloud computing\\

\end{exercise}

\section{Assignments I524} \label{s:i524-assignments}
\index{Assignments!I524}

All assignments are due Monday morning at 9:00 am est. No exceptions unless otherwise specified.

\subsection{Bio Post}

\begin{exercise} \label{a:i524-bio-piazza}

{\bf Bio Post on Piazza.} Please post a formal bio to piazza

\end{exercise}

\begin{exercise} \label{a:i524-bio-googledocs}

 {\bf Bio Post in Google doc.} Due Jan XX
 
 After you have posted it to piazza copy your updated formal bios into the following document.  Make sure you use 3rd person and stay formal. This is a formal bio. Comment on the effectiveness of using the cloud service for this task. A the end of the document. This assignment does not replace the post of the bio to piazza, it is used to gather all bios in one document and to evaluate if google docs is a good tool for this kind of task. Remember we have lots of students and google is used often just with small groups.

 \smallskip

{\hfill \href{https://docs.google.com/document/d/1ejzlKYqC3dLac8WXVpcPQsJh1j4BDqRxxgGg1cFQbeQ/edit?usp=sharing}{I524 Link to google doc $\mapsto$}}

 \end{exercise}

\subsection{Big Data Collaboration}

\begin{exercise} \label{a:i524-big-data-and-collaboration}

{\bf Big data and collaboration.} Due: Jan 22

The purpose of this assignment is multifold; test the ability of Google docs to be used in collaborative fashion by more than a small group and report on the experience. Good Things and bad things, learn on how to use Google docs with headings and table of contents learn how to gather resources quickly with hyperlinks to web resources or articles and translate them into formal academic references. Most importantly convey some very important feature of big data.Contribute this into the handbook for everyone's benefit (done by TAs). Your task is to identify Big Data size related articles and Web resources and produce a historical development of the growth of this data

\smallskip

{\hfill \href{https://docs.google.com/document/d/1ZHNdhX_Jx7uBQo0kthSYQ6TQR8_KNbgOwH2EuqBQcjY/edit?usp=sharing}{I524 Link to google doc $\mapsto$}}

\end{exercise}

\subsection{New Technology List}

\begin{exercise} \label{a:i524-new-tech-list}
{\bf New Technology List} Due: Jan 29

The handbook contains a large number of technologies to which an abstract is provided. Your task is to identify FIRST not to do an abstract but to collaboratively gather a LIST of new technologies that are important in Cloud and Big Data. We suggest doing this in a google docs document first. Write Lastname, Firstname, class id behind the technology so we know who contributed it. Indicate also if commercial, or open source, We are mostly interested in open source activities. Keep the list sorted by alphabet. Use a bullet so formatting is preserved

{\hfill \href{https://docs.google.com/document/d/1LeHGHTSBbaPXYVor0efhmi5W7JJjS7EQHABHqgRAPuU/edit?usp=sharing}{I524 Link to google doc $\mapsto$}}

\smallskip
 
Example: OpenWhisk,https://openwhisk.apache.org/, open source, Gregor von Laszewski, i524
 
\end{exercise}
=======
\subsection{New Technology List}

\begin{exercise} \label{E:new-tech}
Due: Jan 29

The handbook contains a large number of technologies to which an
abstract is provided.

Your task is to identify FIRST not to do an abstract but to
collaboratively gather a LIST of new technologies that are important
in Cloud and Big Data. We suggest doing this in a google docs document
first. Write Lastname, Firstname, class id behind the technology so we
know who contributed it. Indicate also if commercial, or open source,
We are mostly interested in open source activities. Keep the list
sorted by alphabet. Use a bullet so formatting is preserved
>>>>>>> c665577c0fee6db15175c03403a5adaeef13b724

\subsection{New Technology Abstract}

<<<<<<< HEAD
\begin{exercise} \label{a:i524-new-tech-abstracts}
 
{\bf New Technology Abstract} Due date: Feb 5th

We have gathered with the technology list (https://piazza.com/class/jbkvbp3ed3m2ez?cid=50) a number of technologies that are not yet covered in the handbook or need improvement in the handbook.\\

The TAs will be selecting about 5 technologies for each student. Each student will write high-quality non-plagiarized abstracts which bibtex references.\\
 
Details will be announced by the TAs\\
 
Learning outcomes:\\

\noindent Identify how to not plagiarize\\
Work in a large team (with coordination by TAs)\\
Use bibtex and jabref for reference management which you will be using for your final paper\\
Find new trends in big data and cloud computing\\

\end{exercise}


\section{Assignments E616} \label{s:e616-assignments}
\index{Assignments!E616}

All assignments are due Monday morning at 9:00 am est. No exceptions unless otherwise specified.

\subsection{Bio Post}

\begin{exercise}\label{a:e616-bio-piazza}
{\bf Bio Post on Piazza.} Please post a formal bio to piazza
\end{exercise}

\begin{exercise} \label{a:e616-bio-googledocs}

 {\bf Bio Post in Google doc.} Due Date: Jan XX
 
 After you have posted it to piazza copy your updated formal bios into the following document. Make sure you use 3rd person and stay formal. This is a formal bio. Comment on the effectiveness of using the cloud service for this task. A the end of the document. This assignment does not replace the post of the bio to piazza, it is used to gather all bios in one document and to evaluate if Google docs is a good tool for this kind of task. Remember we have lots of students and Google is used often just with small groups.
 
 \smallskip

 {\hfill \href{https://docs.google.com/document/d/1ejzlKYqC3dLac8WXVpcPQsJh1j4BDqRxxgGg1cFQbeQ/edit?usp=sharing} {E616 Link to google doc $\mapsto$}}

 \end{exercise}

\subsection{Big Data Collaboration}

\begin{exercise} \label{a:e616-big-data-and-collaboration}

{\bf Big data and collaboration.} Due Date: Jan XX

The purpose of this assignment is multifold; test the ability of Google docs to be used in collaborative fashion by more than a small group and report on the experience. Good Things and bad things, learn on how to use Google docs with headings and table of contents learn how to gather resources quickly with hyperlinks to web resources or articles and translate them into formal academic references. Most importantly convey some very important feature of big data.Contribute this into the handbook for everyone's benefit (done by TAs). Your task is to identify Big Data size related articles and Web resources and produce a historical development of the growth of this data

  {\hfill \href{https://docs.google.com/document/d/1ZHNdhX_Jx7uBQo0kthSYQ6TQR8_KNbgOwH2EuqBQcjY/edit?usp=sharing}{E516 Link to google doc $\mapsto$}}

\end{exercise}

\subsection{New Technology List}
=======
Example: 

OpenWhisk, \url{https://openwhisk.apache.org/}, open source, Gregor von Laszewski, e616

\end{exercise}

>>>>>>> c665577c0fee6db15175c03403a5adaeef13b724

\begin{exercise} \label{a:e6161-new-tech-list}

{\bf New Technology List} Due: Jan 29 

<<<<<<< HEAD
The handbook contains a large number of technologies to which an abstract is provided. Your task is to identify FIRST not to do an abstract but to collaboratively gather a LIST of new technologies that are important in Cloud and Big Data. We suggest doing this in a google docs document first. Write Lastname, Firstname, class id behind the technology so we know who contributed it. Indicate also if commercial, or open source, We are mostly interested in open source activities. Keep the list sorted by alphabet. Use a bullet so formatting is preserved
=======
\subsection{New Technology Abstract}
>>>>>>> c665577c0fee6db15175c03403a5adaeef13b724

\begin{exercise} \label{E:new-tech-abstract}
 
<<<<<<< HEAD
{\hfill \href{https://docs.google.com/document/d/1LeHGHTSBbaPXYVor0efhmi5W7JJjS7EQHABHqgRAPuU/edit?usp=sharing}{E616 Link to google doc $\mapsto$}}
 
Example: OpenWhisk,https://openwhisk.apache.org/, open source, Gregor von Laszewski, e616
 
\end{exercise}


\subsection{New Technology Abstract}

\begin{exercise}\label{a:e6161-new-tech-abstracts}
 
{\bf New Technology Abstract} Due date: Feb 5th

We have gathered with the technology list (https://piazza.com/class/jbkvbp3ed3m2ez?cid=50) a number of technologies that are not yet covered in the handbook or need improvement in the handbook.\\

The TAs will be selecting about 5 technologies for each student. Each student will write high-quality non-plagiarized abstracts which bibtex references.\\
 
Details will be announced by the TAs\\
 
Learning outcomes:

\noindent Identify how to not plagiarize\\
Work in a large team (with coordination by TAs)\\
Use bibtex and jabref for reference management which you will be using for your final paper\\
Find new trends in big data and cloud computing\\

\end{exercise}
=======
Due date: Feb 5th

We have gathered with the technology list 

\url{https://piazza.com/class/jbkvbp3ed3m2ez?cid=50}

a number of technologies that are not yet covered in the handbook or
need improvement in the handbook.

The TAs will be selecting about 5 technologies for each student. Each
student will write high-quality non-plagiarized abstracts which bibtex
references.
 
Learning outcomes:

\begin{itemize}

\item Identify how to not plagiarize
\item Work in a large team (with coordination by TAs)
\item Use bibtex and jabref for reference management which you will be using for your final paper
\item Find new trends in big data and cloud computing

\end{itemize}
>>>>>>> c665577c0fee6db15175c03403a5adaeef13b724

\end{exercise}
