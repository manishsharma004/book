\section{Installation of Hadoop}\label{s:hadoop-local-installation}

/FILENAME/

In this section we use Hadoop 3.0.0 and we install Hadoop locally in
Ubuntu 16.04. We also describe the instalation of the Yarn
resource manager. We assume that you have ssh, and rsync installed and
use emacs as editor. 

\subsection{Prerequisits}

\begin{lstlisting}
sudo apt-get install ssh
sudo apt-get install rsync
sudo apt-get install emacs
\end{lstlisting}

\subsection{User and User Group Creation}

For security reasons we will install hadoop in a particular user and
user group. We will use the following

\begin{lstlisting}
sudo addgroup hadoop_group
sudo adduser --ingroup hadoop_group hduser
sudo adduser hduser sudo
\end{lstlisting}

These steps will provide sudo privileges to the created hduser
user and add the user to the group \verb|hadoop_group|.

\subsection{Configuring SSH}

\begin{NOTE}
  Here we configure SSH key for the local user to install hadoop with a ssh-key.
  This is different from the ssh-key you used for Github, FutureSystems, etc.
  Follow this section to configure it for Hadoop installation. 
\end{NOTE}

The ssh content is included here because, we are making a
ssh key for this specific user. 
Next, we have to configure ssh to be used by the hadoop user.

%SUDO USER
\begin{lstlisting}
su - hduser
\end{lstlisting}
%SUDO USER

\begin{lstlisting}
ssh-keygen -t rsa
\end{lstlisting}

Follow the instructions as provided in the commandline.
When you see the following console input, press ENTER. Here only we
will create password less keys. IN general this is not a good idead,
but for this case we make an exception.

\begin{lstlisting}
Enter file in which to save the key (/home/hduser/.ssh/id_rsa):
\end{lstlisting}

Next you will be asked to enter a password for ssh configuration,

\begin{lstlisting}
Enter passphrase (empty for no passphrase):
\end{lstlisting}

Here enter the same password

\begin{lstlisting}
Enter same passphrase again:
\end{lstlisting}

Finally you will see something like this after these steps are finished.

\begin{lstlisting}
Generating public/private rsa key pair.
Enter file in which to save the key (/home/hduser/.ssh/id_rsa):
Created directory '/home/hduser/.ssh'.
Enter passphrase (empty for no passphrase):
Enter same passphrase again:
Your identification has been saved in /home/hduser/.ssh/id_rsa.
Your public key has been saved in /home/hduser/.ssh/id_rsa.pub.
The key fingerprint is:
SHA256:0UBCPd6oYp7MEzCpOhMhNiJyQo6PaPCDuOT48xUDDc0 hduser@computer
The key's randomart image is:
+---[RSA 2048]----+
|    .+ooo        |
| .   oE.oo       |
|+  .. ...+.      |
|X+=  .  o..      |
|XX.o  o.S        |
|Bo+ + .o         |
|*o * +.          |
|*.. *.           |
| +.o..           |
+----[SHA256]-----+
\end{lstlisting}
%$

You have successfully configured ssh.

\subsection{Installation of Java}

If you are already logged into su, you can skip the next command:

\begin{lstlisting}
su - hduser
\end{lstlisting}

Now execute the following commands to download and install java

\begin{lstlisting}
mkdir -p ~/cloudmesh/bin
cd ~/cloudmesh/bin
wget -c --header "Cookie: oraclelicense=accept-securebackup-cookie" "http://download.oracle.com/otn-pub/java/jdk/8u161-b12/2f38c3b165be4555a1fa6e98c45e0808/jdk-8u161-linux-x64.tar.gz"
tar xvzf jdk-8u161-linux-x64.tar.gz
\end{lstlisting}

\subsection{Installation of Hadoop}\label{installation-of-hadoop}

First we will take a look on how to install Hadoop 2.6.0 on Ubuntu
16.04. We may need a prior folder structure to do the installation
properly.

\begin{lstlisting}
cd ~/cloudmesh/bin/
wget http://mirrors.sonic.net/apache/hadoop/common/hadoop-3.0.1/hadoop-3.0.1.tar.gz
tar -xzvf hadoop-3.0.1.tar.gz
\end{lstlisting}

\subsection{Hadoop Environment Variables}

In Ubuntu the environmental variables are setup in a file called bashrc
at it can be accessed the following way

\begin{lstlisting}
emacs ~/.bashrc
\end{lstlisting}

\begin{lstlisting}
export JAVA_HOME=~/cloudmesh/bin/jdk1.8.0_161
export HADOOP_HOME=~/cloudmesh/bin/hadoop-3.0.1
export YARN_HOME=$HADOOP_HOME
export HADOOP_CONF_DIR=$HADOOP_HOME/etc/hadoop
export PATH=$HADOOP_HOME/bin:$JAVA_HOME/bin:$PATH
\end{lstlisting}
%$

In Emacs to save the file \verb|Ctrl-X-S| and \verb|Ctrl-X-C| to exit. After editing
you must update the variables in the system.

\begin{lstlisting}
source ~/.bashrc
java -version
\end{lstlisting}

If you have installed things properly there will be no errors. It will
show the version as follows,

\begin{lstlisting}
java version "1.8.0_161"
Java(TM) SE Runtime Environment (build 1.8.0_161-b12)
Java HotSpot(TM) 64-Bit Server VM (build 25.161-b12, mixed mode)
\end{lstlisting}

And verifying the hadoop installation,

\begin{lstlisting}
hadoop
\end{lstlisting}

If you have successfully installed this, there must be a message shown
as below.

\begin{lstlisting}
Usage: hadoop [--config confdir] COMMAND
       where COMMAND is one of:
  fs                   run a generic filesystem user client
  version              print the version
  jar <jar>            run a jar file
  checknative [-a|-h]  check native hadoop and compression libraries availability
  distcp <srcurl> <desturl> copy file or directories recursively
  archive -archiveName NAME -p <parent path> <src>* <dest> create a hadoop archive
  classpath            prints the class path needed to get the
  credential           interact with credential providers
                       Hadoop jar and the required libraries
  daemonlog            get/set the log level for each daemon
  trace                view and modify Hadoop tracing settings
 or
  CLASSNAME            run the class named CLASSNAME

Most commands print help when invoked w/o parameters.
\end{lstlisting}
