\section{Spark Streaming}\label{s:spark-streaming}

\FILENAME\

\subsection{Streaming Concepts}

Spark Streaming is one of the components extending from Core Spark.
Spark streaming provides a scalable fault tolerant system with high
throughput. For streaming data into spark, there are many libraries
like Kafka, Flume, Kinesis, etc.

\subsection{Simple Streaming Example}

In this section, we are going to focus on making a simple streaming
application using the network in your computer. Here we are going to
expose a particular port and from that port we are going to
continously stream data by user entries and the word count is being
calculated as the output.

First, create a Makefile

\begin{lstlisting}
  mkdir -p ~/cloudmesh/spark/examples/streaming
  cd ~/cloudmesh/spark/examples/streaming
  emacs Makefile
\end{lstlisting}

Then add the following content to Makefile. 
\begin{NOTE}
  Please add a tab when adding the corresponding command for a given instruction
  in Makefile. In pdf mode the tab is not clearly shown. 
\end{NOTE}
\begin{lstlisting}
  SPARKHOME = ${SPARK_HOME}
  run-streaming:
	${SPARKHOME}/bin/spark-submit streaming.py localhost 9999
\end{lstlisting}
      
Now we need to create file called streaming.py

\begin{lstlisting}
  emacs streaming.py
\end{lstlisting}

Then add the following content. 

\begin{lstlisting}
from pyspark import SparkContext
from pyspark.streaming import StreamingContext

# Create a local StreamingContext with two working thread and batch interval of 1 second
sc = SparkContext("local[2]", "NetworkWordCount")

log4jLogger = sc._jvm.org.apache.log4j
LOGGER = log4jLogger.LogManager.getLogger(__name__)
LOGGER.info("Pyspark script logger initialized")

ssc = StreamingContext(sc, 1)


# Create a DStream that will connect to hostname:port, like localhost:9999
lines = ssc.socketTextStream("localhost", 9999)
# Split each line into words
words = lines.flatMap(lambda line: line.split(" "))
# Count each word in each batch
pairs = words.map(lambda word: (word, 1))
wordCounts = pairs.reduceByKey(lambda x, y: x + y)

# Print the first ten elements of each RDD generated in this DStream to the console
wordCounts.pprint()
ssc.start()             # Start the computation
ssc.awaitTermination(100)  # Wait for the computation to terminate

\end{lstlisting}

To run the code, we need to open up two terminals.

\verb|Terminal 1 :|

First use netstat to open up a port to start communication.

\begin{lstlisting}
  nc -lk 9999
\end{lstlisting}


\verb|Terminal 2 :|

Now run the Spark programme in the second terminal.

\begin{lstlisting}
  make run-streaming
\end{lstlisting}

In this terminal you can see an script running trying to read
the stream coming from the port 9999. You can enter texts in the
Terminal 1 and these texts will be tokenized and the word count is
calculated and the result is shown in the Terminal 2. 