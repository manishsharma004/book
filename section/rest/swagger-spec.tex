\section{Swagger Specification}
\label{S:swagger-specification}

Swagger provides through its specification the definition of REST
services through a YAML or JSON document. 

When following the API-specification-first approach to define and
develop a RESTful service, the first and foremost step is to define
the API conforming to the OpenAPI specification, and then using
codegen tools to conveniently generate server side stub code, client
code, documentations, in the language you desire. In Section
\ref{c:swagger-codegen} the we have introduced the codegen tool and how to
use that to generate server side and client side code and documentation.
In this Section~\ref{S:vcluster-spec} we will use a slightly more complex
example to show how to define an API following the OpenAPI 2.0 specification.
The example is to retrieve virtual cluster (VC) object from the server.

The OpenAPI Specification is formerly known as Swagger RESTful API
Documentation Specification. It defines a specification to describe
and document a RESTful service API. It is also known under version 3.0
of swagger. However, as the tools for 3.0 are not yet completed, we
will continue for now to use version swagger 2.0, till the transition
has been completed. THis is especially of importance, as we need to
use the swagger codegen tool, which currently support only up to
specification v2. Hence we are at this time using OpenAPI/Swagger v2.0
in our example. There are some structure and syntax changes in v3,
while the essence is very similar. For more details of the
changes between v3 and v2, please refer to A document published on the
Web titled
\href{https://blog.readme.io/an-example-filled-guide-to-swagger-3-2/}{Difference
  between OpenAPI 3.0 and Swagger 2.0}.

You can write the API definition in json for yaml format. Let us
discuss this format briefly and focus on yaml as it is easier to read
and maintain.

On the root level of the yaml document we see fields like
\emph{swagger}, \emph{info}, and so on. Among these fields,
\textbf{\emph{swagger}}, \textbf{\emph{info}}, and \textbf{\emph{path}}
are \textbf{required}. Their meaning is as follows:

\begin{description}
\item[swagger] specifies the version number. IN our case a string
  value `2.0' is used as we are writing the definition conforming to
  the v2.0 specification.
\item [info] defines metadata information related to the API.  E.g.,
  the API \emph{version}, \emph{title} and \emph{description},
  \emph{termsOfService} if applicable, \emph{contact} information and
  \emph{license}, etc. Among these attributes, \textbf{\emph{version}}
  and \textbf{\emph{title}} are required while others are optional.
\item[path] defines the actual endpoints of the exposed RESTful API
  service. Each endpoint has a \emph{field pattern} as the key, and a
  \emph{Path Item Object} as the value. In this example we have defined
  \emph{/vc} and \emph{/vc/\{id\}} as the two service endpoints. They
  will be part of the final service URL, appended after the service
  \emph{host} and \emph{basePath}, which will be explained later.
\end{description}

Let us focus on the \emph{Path Item Object}. It contains one
or more supported \emph{operations} on the service endpoint. An
\emph{operation} is keyed by a valid HTTP operation verb, e.g., one of
\textbf{get}, \textbf{put}, \textbf{post}, \textbf{delete}, or
\textbf{patch}. It has a value of
\emph{Operation Object} that describes the operations in more detail.

The \emph{Operation Object} will always \textbf{require} a
\emph{Response Object}. A \emph{Response Object} has a \emph{HTTP status
code} as the key, e.g., \textbf{200} as successful return; \textbf{40X}
as authentication and authorization related errors; and \textbf{50x} as
other server side servers. It can also has a default response keyed by
\textbf{default} for undeclared http status return code. The
\emph{Response Object} value has a \textbf{required} \emph{description}
field, and if anything is returned, a \emph{schema} indicating the
object type to be returned, which could be a primitive type, e.g.,
\emph{string}, or an \emph{array} or customized \emph{object}. In case
of \emph{object} or an \emph{array} of \emph{object}, use \emph{\$ref}
to point to the definition of the object. In this example, we have

\begin{lstlisting}
  $ref: "#/definitions/VC"
\end{lstlisting}
%$

to point to the \emph{VC} definition in the \emph{definitions} section
in the same specification file, which will be explained later.

Besides the required field, the \emph{Operation Object} \textbf{can}
have \emph{summary} and \emph{description} to indicate what the
operation is about; and \emph{operationId} to uniquely identify the
operation; and \emph{consumes} and \emph{produces} to indicate what MIME
types it expects as input and for returns, e.g., \emph{application/json}
in most modern RESTful APIs. It can further specify what input parameter
is expected using \emph{parameters}, which requires a \emph{name} and
\emph{in} fields. \emph{name} specifies the name of the parameter, and
\emph{in} specifies from where to get the parameter, and its possible
values are \emph{query}, \emph{header}, \emph{path}, \emph{formData} or
\emph{body}. In this example in the \emph{/vc/\{id\}} path we obtain the
\emph{id} parameter from the URL path wo it has the \emph{path} value.
When the \emph{in} has \emph{path} as its value, the \emph{required}
field is required and has to be set as \emph{true}; when the \emph{in}
has value other than \emph{body}, a \emph{type} field is required to
specify the type of the parameter.

While the three root level fields mentioned above are required, in most
cases we will also use other optional fields. 

\begin{description}
\item[host] to indicate where the service is to be deployed, which
  could be \emph{localhost} or a valid IP address or a DNS name of the
  host where the service is to be deployed. If other port number other
  than \emph{80} is to be used, write the port number as well, e.g.,
  \emph{localhost:8080}.

\item[schemas]
to specify the transfer protocol, e.g, \emph{http} or \emph{https}. 
\item[basePath] to specify the common base URL to be append after the
\emph{host} to form the base path for all the endpoints, e.g.,
\emph{/api} or \emph{/api/1.0/}. In this example with the values
specified we would have the final service endpoints
\emph{http://localhost:8080/api/vcs} and
\emph{http://localhost:8080/api/vc/\{id\}} by combining the
\emph{schemas}, \emph{host}, \emph{basePath} and \emph{paths} values. 

\item[consumes and produces] can also be specified on the top
level to specify the default MIME types of the input and return if most
\emph{paths} and the defined operations have the same.

\item[definitions] as used in in the
\emph{paths} field, in order to point to a customized object definition
with a \emph{\$ref} keyword.

\end{description}

The \emph{definitions} field really contains the object definition of
the customized objects involved in the API, similar to a class
definition in any Object Oriented programming language. In this
example, we defined a \emph{VC} object, and hierarchically a
\emph{Node} object. Each object defined is a type of \emph{Schema
  Object} in which many field could be used to specify the object (see
details in the REF link at top of the document), but the most
frequently used ones are:

\begin{description}
\item[type] to specify the type, and in the customized definition case
  the value is mostly \emph{object}.
\item[required] field to list the names of the required attributes of
  the object.
\item[properties] has the detailed information of each
  attribute/property of the object, e.g, \emph{type},
  \emph{format}. It also supports hierarchical object definition so a
  property of one object could be another customized object defined
  elsewhere while using \emph{schema} and \emph{\$ref} keyword to
  point to the definition. In this example we have defined a
  \emph{VC}, or virtual cluster, object, while it contains another
  object definition of
\item[Node] as part of a cluster.
\end{description}

\subsection{The Virtual Cluster example API Definition}
\label{S:vcluster-spec}

\subsubsection{Terminology}

\begin{description}
\item[VC] A virtual cluster, which has one Front-End (FE) management node and multiple compute nodes. A VC object also has \emph{id} and \emph{name} to identify the VC, and \emph{nnodes} to indicate how many compute nodes it has.
\item[FE] A management node from which to access the compute nodes. The FE node usually connects to all the compute nodes via private network.
\item[Node] A computer node object that the info \emph{ncores} to indicate number of cores it has, and \emph{ram} and \emph{localdisk} to show the size of RAM and local disk storage.
\end{description}

\subsubsection{Specification}

\begin{lstlisting}
swagger: "2.0"
info: 
  version: "1.0.0"
  title: "A simple Virtual Cluster"
  description: "A simple service for a Virtual CLuster as a test of using swagger-2.0 specification and codegen"
  termsOfService: "http://swagger.io/terms/"
  contact: 
    name: "IU ISE software and system team"
  license: 
    name: "Apache"
host: "localhost:8080"
basePath: "/api"
schemes: 
  - "http"
consumes: 
  - "application/json"
produces: 
  - "application/json"
paths: 
  /vcs: 
    get: 
      description: "Returns all VCs from the system that the user has access to"
      produces: 
        - "application/json"
      responses: 
        "200":
          description: "A list of VCs."
          schema: 
            type: "array"
            items: 
              $ref: "#/definitions/VC"
  /vcs/{id}:
    get: 
      description: "Returns all VCs from the system that the user has access to"
      operationId: getVCById
      parameters:
        - name: id
          in: path
          description: ID of VC to fetch
          required: true
          type: string
      produces: 
        - "application/json"
      responses: 
        "200":
          description: "The vc with the given id."
          schema: 
            $ref: "#/definitions/VC"
        default:
          description: unexpected error
          schema:
            $ref: '#/definitions/Error'
definitions:
  VC: 
    type: "object"
    required: 
      - "id"
      - "name"
      - "nnodes"
      - "FE"
      - "computes"
    properties: 
      id: 
        type: "string"
      name: 
        type: "string"
      nnodes:
        type: "integer"
        format: "int64"
      FE:
        type: "object"
        schema:
          $ref: "#/definitions/Node"
      computes:
        type: "array"
        items:
          $ref: "#/definitions/Node"
      tag: 
        type: "string"
  Node:
    type: "object"
    required:
      - "ncores"
      - "ram"
      - "localdisk"
    properties:
      ncores:
        type: "integer"
        format: "int64"
      ram:
        type: "integer"
        format: "int64"
      localdisk:
        type: "integer"
        format: "int64"
  Error:
    required:
    - code
    - message
    properties:
      code:
        type: integer
        format: int32
      message:
        type: string
\end{lstlisting}
%$ 

\subsection{Refernces}

\href{https://github.com/OAI/OpenAPI-Specification/blob/master/versions/2.0.md}{The
official OpenAPI 2.0 Documentation}
