\MDNAME\
%%%%%%%%%%%%%%%%%%%%%%%%%%%%%%%%%%%%%%%%%%%%%%%%%%%%%%%%%%%%%%%%%%%%%%%%%%%%%%%
% DO NOT MODIFY THIS FILE
%%%%%%%%%%%%%%%%%%%%%%%%%%%%%%%%%%%%%%%%%%%%%%%%%%%%%%%%%%%%%%%%%%%%%%%%%%%%%%%

\section{Swagger-bravado, a fork of Swagger-py maintained by Yelp}

In this section, we are discussing a Yelp maintained fork of Swagger.
Bravado is a python client library for Swagger 2.0 services and it aims
to replace Swagger-Codegen.

We assume that you already understand the concept of REST API service
and you have some knowledge on Swagger-Codegen.

\subsection{Some important features afforded by Bravado}

\begin{itemize}
\item
  Swagger models as Python types (no need to deal with JSON).
\item
  Dynamically generated client - no code generation needed!
\item
  Swagger Schema is v2.0 compatible.
\end{itemize}

To understand the features and advantages of Bravado, it is better to
start with some practice.

\subsubsection{Installation}

You can start easily in a python environment with this command to
install the latest version of Bravado:

\begin{lstlisting}
$ pip install --upgrade bravado
\end{lstlisting}

\subsubsection{First try}

After you have installed the bravado packages, here is a simple example
for you to make the first try, you could run you code in a python
environment:

\begin{lstlisting}
from bravado.client import SwaggerClient
client = SwaggerClient.from_url("http://petstore.swagger.io/v2/swagger.json")
pet = client.pet.getPetById(petId=42).result()
\end{lstlisting}

This piece of code shows an example that how to setup a bravado client
and test its property. If you were lucky, and pet Id with 42 was
present, you will get back a result. It will be a dynamically created
instance of bravado.model.Pet with attributes category, etc. You can
even try pet.category.id or pet.tags{[}0{]}. A sample respnse should be:

\begin{lstlisting}
Pet(category=Category(id=0L, name=u''), status=u'', name=u'', tags=[Tag(id=0L, name=u'')], photoUrls=[u''], id=2)
\end{lstlisting}

\subsubsection{Try to make a POST call}

Here we will demonstrate how bravado hides all the JSON handling from
the user.

\begin{lstlisting}
Pet = client.get_model('Pet')
Category = client.get_model('Category')
pet = Pet(id=42, name="tommy", category=Category(id=24))
client.pet.addPet(body=pet).result()
\end{lstlisting}

\subsubsection{Example with Basic Authentication}

Here is example code on how to request authentication from client side:

\begin{lstlisting}
from bravado.requests_client import RequestsClient
from bravado.client import SwaggerClient

http_client = RequestsClient()
http_client.set_basic_auth(
     'api.yourhost.com',
    'username', 'password'
)
client = SwaggerClient.from_url(
     'http://petstore.swagger.io/v2/swagger.json',
      http_client=http_client,
)
pet = client.pet.getPetById(petId=42).result()
\end{lstlisting}

\subsubsection{Asynchronous client}

Bravado also provides an out of the box asynchronous http client with an
optional timeout parameter. Before you could utilize the function as an
asynchronous http client, you need to install your libray as following
command:

\begin{lstlisting}
$ pip install bravado[fido]
\end{lstlisting}

Then here is an example on how to configure a timeout option: {]} from
bravado.client import SwaggerClient from bravado.fido\_client import
FidoClient

\begin{lstlisting}
client = SwaggerClient.from_url(
     'http://petstore.swagger.io/v2/swagger.json',
    FidoClient()
)

result = client.pet.getPetById(petId=42).result(timeout=4)
\end{lstlisting}

\subsubsection{Simple Returning without any model}

If you want to get the result without specifying any models, you could
try the following code:

\begin{lstlisting}
from bravado.client import SwaggerClient
from bravado.fido_client import FidoClient

client = SwaggerClient.from_url(
     'http://petstore.swagger.io/v2/swagger.json',
    config={'use_models': False}
)

result = client.pet.getPetById(petId=42).result(timeout=4)
\end{lstlisting}

You can get the result as this:

\begin{lstlisting}
{
    'category': {
        'id': 0L,
        'name': u''
    },
    'id': 2,
    'name': u'',
    'photoUrls': [u''],
    'status': u'',
    'tags': [
        {'id': 0L, 'name': u''}
    ]
}
\end{lstlisting}

\subsection{Configuration}

\subsubsection{Configuration on Client Side}

You can configure certain behaviours when creating a SwaggerClient. Here
is a sample skelecton code for configuration:

\begin{lstlisting}
from bravado.client import SwaggerClient, SwaggerFormat

mysuperformat = SwaggerFormat(...)

config = {
    # === bravado config ===

    #Determines what is returned by the service call.
    'also_return_response': False,

    # === bravado-core config ====

    #  validate incoming responses
    'validate_responses': True,

    # validate outgoing requests
    'validate_requests': True,

    # validate the swagger spec
    'validate_swagger_spec': True,

    # Use models (Python classes) instead of dicts for #/definitions/{models}
    'use_models': True,

    # List of user-defined formats
    'formats': [my_super_duper_format],

}

client = SwaggerClient.from_url(..., config=config)
\end{lstlisting}

