\chapterimage{pigeon.jpeg}
\chapter{MQTT}
\label{c:mqtt}

\FILENAME\

With the increase importance of cloud computing and the increased
number of edge devices and their applications, such as sensor
networks, it is crucial to enable fast communication between the
sensing devices and actuators, which may not be directly connected, as
well as cloud services that analyze the data. To allow services that
are built on different software and hardware platforms to communicate,
a data agnostic, fast and service is needed. In addition to
communication, the data generated by these devices, services, and
sensors must be analyzed. Security aspects to relay this data is
highly important. We will introduce a service called MQTT, which is a
common, easy to use, queuing protocol that helps meet these
requirements.  


\section{Introduction}

As Cloud Computing and Internet of Things (IoT) applications and
sensor networks become commonplace and more and more devices are being
connected, there is an increased need to allow these devices to
communicate quickly and securely. In many cases these edge devices
have very limited memory and need to conserve power. The computing
power on some of these devices is so limited that the sensory data
need to be analyzed remotely. Furthermore, they may not even have
enough computing capacity to process traditional HTTP web requests
efficiently~\cite{mqtt-vs-http}\cite{hivemq-website} or these traditional
Web-based services are too resource hungry. Monitoring the state of a
remotely located sensor using HTTP would require sending requests and
receiving responses to and from the device frequently, which may not
be efficient on small circuits or embedded chips on edge computing
sensors~\cite{mqtt-vs-http}.

Message Queue Telemetry Transport (MQTT) is a lightweight machine to
machine (M2M) messaging protocol, based on a client/server 
publish-subscribe model. It provides a a simple service allowing us to
communicate between sensors, and services based on a subscription
model.

MQTT was first developed in 1999 to connect oil pipelines
\cite{hivemq-website}. The protocol has been designed to be used on
top of TCP/IP protocol in situations where network bandwidth, and
available memory are limited allowing low power usage. However, as it
is implemented on top of TCP/IP it is reliable in contrast to other
protocols such as UDP. It allows efficient transmission of
data to various devices listening for the same event, and is scalable
as the number of devices increase
\cite{mqtt-wiki}\cite{mqtt-official}.

The current support for MQTT is conducted as part of the Eclipse Paho
project~\cite{eclipse-mosquitto}.  As MQTT is a protocol many
different clients in various languages exist.  This includes languages
such as Python, C, Java, Lua, and many more.

The current standard of MQTT is available at 

\url{http://docs.oasis-open.org/mqtt/mqtt/v3.1.1/os/mqtt-v3.1.1-os.pdf}

\section{Publish Subscribe Model}

MQTT works via a publish-subscribe model that contains 3 entities: (1) a
publisher, that sends a message, (2) a broker, that maintains queue of all
messages based on topics and (3) multiple subscribers that subscribe to
various topics they are interested in~\cite{how-mqtt-works}.

This allows for decoupling of functionality at various levels. The
publisher and subscriber do not need to be close to each other and do
not need to know each others identity. They need only to know the
broker, as the publisher and the subscribers do not have to be running
either at the same time nor on the same hardware
\cite{hivemq-details}.

Ready to use implementation exist to be deployed as brokers in the
users application frameworks. A broker is a service that relays
information between the client and servers. Common brokers include the
open source Mosquito broker~\cite{mqtt-official} and the Eclipse Phao
MQTT Broker~\cite{eclipse-mosquitto}.


\subsection{Topics}

MQTT implements a hierarchy of topics that are relates
to all messages. These topics are recognised by strings separated by a
forward-slash (/), where each part represents a different topic
level. This is a common model introduced in file systems but also in
internet URLs. 

A topic looks therefore as follows: \textit{topic-level0/topic-level1/topic-level2}.



Subscribers can subscribe to different topics via the broker.
Subscribing to \textit{topic-level0} allows the subscriber to receive all
messages that are associated with topics that start with {\em
  topic-level0}. This allows subscribers to filter what messages to
receive based on the topic hierarchy. Thus, when a publisher publishes
a message related to a topic to the broker, the message is forwarded
to all the clients that have subscribed to the topic of the message or
a topic that has a lower depth of hierarchy~\cite{hivemq-details}
\cite{how-mqtt-works}.

This is different from traditional point-to-point message queues as
the message is forwarded to multiple subscribers, and allows for
flexibility of dealing with subscribed topics not only on the server
but also on the subscriber side~\cite{hivemq-details}. The basic steps
in a MQTT client subscriber application include to (1) connect to the broker,
(2) subscribe to some topics, (3) wait for messages and (4) perform the
appropriate action when a certain message is received
\cite{mqtt-wiki}.
 
\subsection{Callbacks}

One of the advantages of using MQTT is that it supports
asynchronous behaviour with the help of callbacks. Both the publisher
and subscriber can use non-blocking callbacks to act upon message exchanges.
\cite{hivemq-details}\cite{python-paho-mqtt}.

For example, the paho-mqtt package for python provides callbacks
methods including \verb|on-connect()|, \verb|on-message()| and
\verb|on-disconnect()|, which are invoked when the connection to the
broker is complete, a message is received from the broker, and when
the client is disconnected from the broker respectively. These methods
are used in conjunction with the \verb|loop-start()| and \verb|loop-end()| methods
which start and end an asynchronous loop that listens for these events
invoking the relevant callbacks. Hence it frees the services to perform
other tasks~\cite{python-paho-mqtt} when no messages are available,
thus reducing overhead.

\subsection{Quality of Service}

MQTT has been designed to be flexible allowing for the change of
quality of service (QoS) as desired by the application. Three basic
levels of QoS are supported by the protocol:  Atmost-once (QoS level
0), Atleast-once (QoS level 1) and Atmost-once (QoS level 2)
\cite{hivemq-qos,python-paho-mqtt}.

\begin{description}

\item[QoS level 0:] The QoS level of 0 is used in applications where
  some dropped messages may not affect the application. Under this QoS
  level, the broker forwards a message to the subscribers only once
  and does not wait for any
  acknowledgement~\cite{hivemq-qos}\cite{python-paho-mqtt}.

\item[QoS Level 1:] The QoS level of 1 is used in situations where the
  delivery of all messages is important and the subscriber can handle
  duplicate messages. Here the broker keeps on resending the message
  to a subscriber after a certain timeout until the first
  acknowledgement is received.

\item[QoS Level 3:] A QoS level of 3 is used in cases where all
  messages must be delivered and no duplicate messages should be
  allowed. In this case the broker sets up a handshake with the
  subscriber to check for its availability before sending the
  message~\cite{hivemq-qos,python-paho-mqtt}.

\end{description}

The various levels of quality of service allow the use of this
protocol with different service level expectations.

\section{Secure MQTT Services}

MQTT specification uses TCP/IP to deliver the messaged to the
subscribers, but it does not provide security by default to enable
resource constrained IoT devices. ``It allows the use of username and
password for authentication, but by default this information is sent
as plain text over the network, making it susceptible to man-in-the
middle attacks''
\cite{iot-design-mqtt-security,mqtt-sec-ssl}. Therefore, to support
sensitive applications additional security measures need to be
integrated through other means. This may include for example the use
of Virtual Private Networks (VPNs), Transport Layer Security, or
application layer security~\cite{mqtt-sec-ssl}.

\subsection{Using TLS/SSL}

Transport Layer Security (TLS) and Secure Sockets Layer (SSL) are
cryptographic protocols that establish a the identity of the server
and client with the help of a handshake mechanism which uses trust
certificates to establish identities before encrypted communication
can take place~\cite{ibm-mqtt-security}. If the handshake is not
completed for some reason, the connection is not established and no
messages are exchanged~\cite{mqtt-sec-ssl}. ``Most MQTT brokers
provide an option to use TLS instead of plain TCP and port 8883 has
been standardized for secured MQTT connections''
\cite{iot-design-mqtt-security}.

Using TLS/SSL security however comes at an additional cost. If the
connections are short-lived then most of the time is spent in
verifying the security of the handshake itself, which in addition to
using time for encryption and decryption, may take up few
kilobytes of bandwidth. In case the connections are short-lived,
temporary session IDs and session tickets can be used as alternative to resume a
session instead of repeating the handshake process. If the connections
are long term, the overhead of the handshake is negligible and TLS/SSL
security should be used
\cite{iot-design-mqtt-security,mqtt-sec-ssl}.

\subsection{Using OAuth}

OAuth is an open protocol that allows access to a resource without
providing unencrypted credentials to the third party. Although MQTT
protocol itself does not include authorization, many MQTT brokers
include authorization as an additional feature
\cite{ibm-mqtt-security}. OAuth2.0 uses JSON Web Tokens which contain
information about the token ans the user and are signed by a trusted
authorization server~\cite{hivemq-security-oauth}.

When connecting to the broker this token can be used to check whether
the client is authorised to connect at this time or not. Additionally
the same validations can be used when publishing or subscribing to the
broker. The broker may use a third party resource such as LDAP
(lightweight directory access protocol) to look up authorizations for
the client~\cite{hivemq-security-oauth}. Since there can be a large
number of clients and it can become impractical to authorize everyone,
clients may be grouped and the authorizations may be checked for each
group~\cite{ibm-mqtt-security}.
  
\section{Integration with Other Services}

As the individual IoT devices perform their respective functions in
the sensor network, a lot of data is generated which needs to be
processed. MQTT allows easy integration with other services, that have
been designed to process this data. Let us provide some examples of
MQTT integration into other Services.

\begin{description}

\item[Apache Storm.] Apache storm is a distributed processing system
  that allows real time processing of continuous data streams, much
  like Hadoop works for batch processing~\cite{apache-storm}. Apache
  storm can be easily integrated with MQTT as shown
  in~\cite{apache-storm-mqtt} to get real time data streams and allow
  analytics and online machine learning in a fault tolerant
  manner~\cite{apache-storm-wiki}.

\item [ELK stack.]  ELK stack (elastic-search, logstash and kibana) is
  an opensource project designed for scalability which contains three
  main software packages, the \textit{elastic-search} search and
  analytics engine, \textit{logstash} which is a data collection
  pipeline and \textit{kibana} which is a visualization
  dashboard~\cite{elk-stack}. Data from an IoT network can be
  collected, analysed and visualized easily with the help of the ELK
  stack as shown in~\cite{mqtt-elasticsearch-setup} and~\cite{kibana-mqtt-analysis}.


\end{description}

\section{MQTT in Production}

When using optimized MQTT broker services, MQTT can be utilized for
enterprise and production environments. A good example is the use of
EMQ (Erlang MQTT Broker) that provides a highly scalable, distributed
and reliable MQTT broker for enterprise-grade applications
\cite{erlang-mqtt-broker}.


\section{Simple Usecase}

In this example we are demonstrating how to set up a MQTT broker, a
client and a subscriber.

% We are running the program in three
%terminals to showcase the log or output of each server.

\URL{https://github.com/bigdata-i523/sample-hid000/tree/master/experiment/mqtt}

A test program that starts a MQTT broker and client is provided next,
showcasing how simple the interactions are while using a higher level
API such as provided through the python client library of Paho.

\begin{lstlisting}
import paho.mqtt.client as mqtt 
import time


def on_message(client, userdata, message):
    print("message received ", str(message.payload.decode("utf-8")))
    print("message topic=", message.topic)
    print("message qos=", message.qos)
    print("message retain flag=", message.retain)

def on_log(client, userdata, level, buf):
    print("log: ",buf)

broker_address="localhost"    
# broker_address="test.mosquitto.org"
# broker_address="broker.hivemq.com"
# broker_address="iot.eclipse.org"

print("creating new instance")
client = mqtt.Client("i523") #create new instance
client.on_log=on_log
client.on_message=on_message #attach function to callback

print("connecting to broker")
client.connect(broker_address) #connect to broker
client.loop_start() #start the loop

print("Subscribing to topic","robot/leds/led1")
client.subscribe("robot/leds/led1")

print("Publishing message to topic","robot/leds/led1")
client.publish("robot/leds/led1","OFF")

time.sleep(4) # wait
client.loop_stop() #stop the loop
\end{lstlisting}

\section{IoT Use Case}

MQTT can be used in a variety of applications. This section explores a
particular use case of the protocol. A small network was set up with
three devices to simulate an IoT environment, and actuators were
controlled with the help of messages communicated over MQTT.

The code for the project is available at 

\URL{https://github.com/bigdata-i523/hid201/tree/master/experiment/mqtt}

\subsection{Requirements and Setup}

The setup used three different machines. A laptop or a desktop running
the MQTT broker, and two raspberry pis configured with raspbean
operating system. Eclipse Paho MQTT client was setup on each of the
raspberry pis~\cite{python-paho-mqtt}. Additionally all three devices
were connected to an isolated local network.

GrovePi shields for the raspberry pis, designed by Dexter Industries
were used on each of the raspberry pis to connect the actuators as
they allow easy connections ot the raspberry pi board~\cite{grovepi}.
The actuators used were Grove relays~\cite{grove-relay} and Grove LEDs
\cite{grove-led} which respond to the messages received via MQTT.

To control the leds and relays, the python library cloudmesh.pi
\cite{cloudmesh.pi}, developed at Indiana University was used. The
library consists of interfaces for various IoT sensors and actuators
and can be easily used with the grove modules.

%sma controlled dendrites
%install instructions
%block diagram figure

%actual picture of setup ?

\subsection{Results}

The two raspberry pis subscribe connect to the broker and subscribe
with different topics. The raspberry pis wait for any messages from
the broker. A publisher program that connects to the broker publishes
messages to the broker for the topics that the two raspberry pis had
registered. Each raspberry pi receives the corresponding message and
turns the LEDs or relays on or off as per the message.

On a local network this process happens in near real time and no
delays were observed. Eclipse IoT MQTT broker (\textit{iot.eclipse.org})
was also tried which also did not result in any significant delays.

Thus it is observed that two raspberry pis can be easily controlled
using MQTT. This system can be extended to incude arbitrary number of
raspberry pis and other devices that subscribe to the broker. If a
device fails, or the connection from one device is broken, other
devices are not affected and continue to perform the same.

This project can be extended to include various other kinds of sensors
and actuators. The actuators may subscribe to topics to which various
sensors publish their data ans respond accordingly. The data of these
sensors can be captured with the help of a data collector which may
itself be a different subscriber, that performs analytics or
visualizations on this data.

%code link

\section{Conclusion}

We see that as the number of connected devices increases and their
applications become commonplace, MQTT allows different devices to
communicate with each other in a data agnostic manner. MQTT uses a
publish-subscribe model and allows various levels of quality of
service requirements to be fulfilled. Although MQTT does not provide
data security by default, most brokers allow the use of TLS/SSL to
encrypt the data. Additional features may be provided by the broker to
include authorization services. MQTT can be easily integrated with
other services to allow collection and analysis of data. A small
environment was simulated that used MQTT broker and clients running on
raspberry pis to control actuators

\section{Exercises}

\begin{exercise}
Develop a temperature broker, that collects the temperature from a
number of machines and clients can subscribe to the data and visualize it.
\end{exercise}

\begin{exercise}
Develop a CPU load broker, that collects the cpu load from a
number of machines and clients can subscribe to the data and visualize it.
\end{exercise}

\begin{exercise}
Develop a broker with a variety of topics that collects data from a
Raspberry Pi or Raspberry PI cluster and visualize it.
\end{exercise}

\begin{exercise}
Explore hosted services for MQTT while at the same time remembering
that they could pose a security risk. Can any of the online services
be used to monitor a cluster safely?
\end{exercise}



