


\chapter{Technology for Big Data Applications and
Analytics}\label{technology-for-big-data-applications-and-analytics}

\FILENAME

We use the K-means Python code in SciPy package to show real code for
clustering. After a simple example we generate 4 clusters of distinct
centers and various choice for sizes using Matplotlib tor visualization.
We show results can sometimes be incorrect and sometimes make different
choices among comparable solutions. We discuss the \textit{hill} between
different solutions and rationale for running K-means many times and
choosing best answer. Then we introduce MapReduce with the basic
architecture and a homely example. The discussion of advanced topics
includes an extension to Iterative MapReduce from Indiana University
called Twister and a generalized Map Collective model. Some measurements
of parallel performance are given. The SciPy K-means code is modified to
support a MapReduce execution style. This illustrates the key ideas of
mappers and reducers. With appropriate runtime this code would run in
parallel but here the \emph{parallel} maps run sequentially. This simple
2 map version can be generalized to scalable parallelism. Python is used
to Calculate PageRank from Web Linkage Matrix showing several different
formulations of the basic matrix equations to finding leading
eigenvector. The unit is concluded by a calculation of PageRank for
general web pages by extracting the secret from Google.

\section{Technologypi: K-means}\label{technologypi-k-means}

We use the K-means Python code in SciPy package to show real code for
clustering. After a simple example we generate 4 clusters of distinct
centers and various choice for sizes using Matplotlib tor visualization.
We show results can sometimes be incorrect and sometimes make different
choices among comparable solutions. We discuss the \emph{hill} between
different solutions and rationale for running K-means many times and
choosing best answer.

Files:

\sourcecode{K-means}{xmean.py}{examples/python/k-means/xmean.py}

\sourcecode{K-means}{sample.csv}{examples/python/k-means/sample.csv}

\sourcecode{K-means}{parallel-kmeans.py}{examples/python/k-means/parallel-kmeans.py}

\sourcecode{K-means}{kmeans-extra.py}{examples/python/k-means/kmeans-extra.py}


\subsection{K-means in Python}\label{k-means-in-python}

We use the K-means Python code in SciPy package to show real code for
clustering and applies it a set of 85 two dimensional vectors --
officially sets of weights and heights to be clustered to find T-shirt
sizes. We run through Python code with Matplotlib displays to divide
into 2-5 clusters. Then we discuss Python to generate 4 clusters of
varying sizes and centered at corners of a square in two dimensions. We
formally give the K means algorithm better than before and make
definition consistent with code in SciPy.

\subsection{Analysis of 4 Artificial
Clusters}\label{analysis-of-4-artificial-clusters}

We present clustering results on the artificial set of 1000 2D points
described in previous lesson for 3 choices of cluster sizes \emph{small}
\emph{large} and \emph{very large}. We emphasize the SciPy always does
20 independent K means and takes the best result -- an approach to
avoiding local minima. We allow this number of independent runs to be
changed and in particular set to 1 to generate more interesting erratic
results. We define changes in our new K means code that also has two
measures of quality allowed. The slides give many results of clustering
into 2 4 6 and 8 clusters (there were only 4 real clusters). We show
that the \emph{very small} case has two very different solutions when
clustered into two clusters and use this to discuss functions with
multiple minima and a hill between them. The lesson has both discussion
of already produced results in slides and interactive use of Python for
new runs.

\section{Technology: MapReduce}\label{technology-mapreduce}

We describe the basic architecture of MapReduce and a homely example.
The discussion of advanced topics includes extension to Iterative
MapReduce from Indiana University called Twister and a generalized Map
Collective model. Some measurements of parallel performance are given.

\subsection{Introduction}\label{introduction}

This introduction uses an analogy to making fruit punch by slicing and
blending fruit to illustrate MapReduce. The formal structure of
MapReduce and Iterative MapReduce is presented with parallel data
flowing from disks through multiple Map and Reduce phases to be
inspected by the user.

\subsection{Advanced Topics}\label{advanced-topics}

This defines 4 types of MapReduce and the Map Collective model of Qiu.
The Iterative MapReduce model from Indiana University called Twister is
described and a few performance measurements on Microsoft Azure are
presented.

\section{Technology: Kmeans and MapReduce
Parallelism}\label{technology-kmeans-and-mapreduce-parallelism}

We modify the SciPy K-means code to support a MapReduce execution style
and runs it in this short unit. This illustrates the key ideas of
mappers and reducers. With appropriate runtime this code would run in
parallel but here the \emph{parallel} maps run sequentially. We stress
that this simple 2 map version can be generalized to scalable
parallelism.

Files:

\sourcecode{K-means}{ParallelKmeans}{examples/python/k-means/parallel-kmeans.py}

\subsection{MapReduce Kmeans in
Python}\label{mapreduce-kmeans-in-python}

We modify the SciPy K-means code to support a MapReduce execution style
and runs it in this short unit. This illustrates the key ideas of
mappers and reducers. With appropriate runtime this code would run in
parallel but here the \emph{parallel} maps run sequentially. We stress
that this simple 2 map version can be generalized to scalable
parallelism.

\section{Technology: PageRank}\label{technology-pagerank}

We use Python to Calculate PageRank from Web Linkage Matrix showing
several different formulations of the basic matrix equations to finding
leading eigenvector. The unit is concluded by a calculation of PageRank
for general web pages by extracting the secret from Google.

Files:

\sourcecode{K-means}{pagerank1.py}{examples/python/page-rank/pagerank1.py}

\sourcecode{K-means}{pagerank2.py}{examples/python/page-rank/pagerank2.py}


\subsection{Calculate PageRank from Web Linkage
Matrix}\label{calculate-pagerank-from-web-linkage-matrix}

We take two simple matrices for 6 and 8 web sites respectively to
illustrate the calculation of PageRank.

\subsection{Calculate PageRank of a Real
Page}\label{calculate-pagerank-of-a-real-page}

This tiny lesson presents a Python code that finds the Page Rank that
Google calculates for any page on the web.
