

\section{Radar Case Study}\label{radar-case-study}
\FILENAME

The changing global climate is suspected to have long-term effects on
much of the world's inhabitants. Among the various effects, the rising
sea level will directly affect many people living in low-lying coastal
regions. While the ocean-s thermal expansion has been the dominant
contributor to rises in sea level, the potential contribution of
discharges from the polar ice sheets in Greenland and Antarctica may
provide a more significant threat due to the unpredictable response to
the changing climate. The Radar-Informatics unit provides a glimpse in
the processes fueling global climate change and explains what methods
are used for ice data acquisitions and analysis.

\slides{Radar}{Radar}{58}{https://drive.google.com/open?id=0B8936_ytjfjmZ0VzZ0ZIenpUMTQ}

\subsection{Introduction}\label{introduction}

This lesson motivates radar-informatics by building on previous
discussions on why X-applications are growing in data size and why
analytics are necessary for acquiring knowledge from large data. The
lesson details three mosaics of a changing Greenland ice sheet and
provides a concise overview to subsequent lessons by detailing
explaining how other remote sensing technologies, such as the radar, can
be used to sound the polar ice sheets and what we are doing with radar
images to extract knowledge to be incorporated into numerical models.

\video{Radar}{3:31}{Radar Informatics}{https://youtu.be/LXOncC2AhsI}

\subsection{Remote Sensing}\label{remote-sensing}

This lesson explains the basics of remote sensing, the characteristics
of remote sensors and remote sensing applications. Emphasis is on image
acquisition and data collection in the electromagnetic spectrum.

\video{Radar}{6:43}{Remote Sensing}{https://youtu.be/TTrm9rmZySQ}

\subsection{Ice Sheet Science}\label{ice-sheet-science}

This lesson provides a brief understanding on why melt water at the base
of the ice sheet can be detrimental and why it's important for sensors
to sound the bedrock.

\video{Radar}{1:00}{Ice Sheet Science}{https://youtu.be/rDpjMLguVBc}

\subsection{Global Climate Change}\label{global-climate-change}

This lesson provides an understanding and the processes for the
greenhouse effect, how warming effects the Polar Regions, and the
implications of a rise in sea level.

\video{Radar}{2:51}{Global Climate Change}{https://youtu.be/f9hzzJX0qDs}

\subsection{Radio Overview}\label{radio-overview}

This lesson provides an elementary introduction to radar and its
importance to remote sensing, especially to acquiring information about
Greenland and Antarctica.

\video{Radar}{4:16}{Radio Overview}{https://youtu.be/PuI7F-RMKCI}

\subsection{Radio Informatics}\label{radio-informatics}

This lesson focuses on the use of sophisticated computer vision
algorithms, such as active contours and a hidden markov model to support
data analysis for extracting layers, so ice sheet models can accurately
forecast future changes in climate.

\video{Radar}{3:35}{Radio Informatics}{https://youtu.be/q3Pwyt49syE}
