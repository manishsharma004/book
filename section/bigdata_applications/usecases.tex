

\section{Big Data Use Cases Survey}\label{big-data-use-cases-survey}

\FILENAME

This section covers 51 values of X and an overall study of Big data that
emerged from a NIST (National Institute for Standards and Technology)
study of Big data. The section covers the NIST Big Data Public Working
Group (NBD-PWG) Process and summarizes the work of five subgroups:
Definitions and Taxonomies Subgroup, Reference Architecture Subgroup,
Security and Privacy Subgroup, Technology Roadmap Subgroup and the
Requirements andUse Case Subgroup. 51 use cases collected in this
process are briefly discussed with a classification of the source of
parallelism and the high and low level computational structure. We
describe the key features of this classification.

\subsection{Overview of NIST Big Data Public Working Group (NBD-PWG)
Process and1
Results}\label{overview-of-nist-big-data-public-working-group-nbd-pwg-process-and-results}

This unit covers the NIST Big Data Public Working Group (NBD-PWG)
Process and summarizes the work of five subgroups: Definitions and
Taxonomies Subgroup, Reference Architecture Subgroup, Security and
Privacy Subgroup, Technology Roadmap Subgroup and the Requirements and
Use Case Subgroup. The work of latter is continued in next two units.


\slides{Usecases}{Overview}{45}{https://drive.google.com/open?id=0B8936_ytjfjmODIxNGttU1pveWc}


\subsubsection{Introduction to NIST Big Data Public Working Group
(NBD-PWG)
Process}\label{introduction-to-nist-big-data-public-working-group-nbd-pwg-process}

The focus of the (NBD-PWG) is to form a community of interest from
industry, academia, and government, with the goal of developing a
consensus definitions, taxonomies, secure reference architectures, and
technology roadmap. The aim is to create vendor-neutral, technology and
infrastructure agnostic deliverables to enable big data stakeholders to
pick-and-choose best analytics tools for their processing and
visualization requirements on the most suitable computing platforms and
clusters while allowing value-added from big data service providers and
flow of data between the stakeholders in a cohesive and secure manner.


\video{Usecases}{13:02}{Introduction}{https://www.youtube.com/watch?v=3oKdmuH0N3k} 



\subsubsection{Definitions and Taxonomies
Subgroup}\label{definitions-and-taxonomies-subgroup}

The focus is to gain a better understanding of the principles of Big
Data. It is important to develop a consensus-based common language and
vocabulary terms used in Big Data across stakeholders from industry,
academia, and government. In addition, it is also critical to identify
essential actors with roles and responsibility, and subdivide them into
components and sub-components on how they interact/ relate with each
other according to their similarities and differences.

For Definitions: Compile terms used from all stakeholders regarding the
meaning of Big Data from various standard bodies, domain applications,
and diversified operational environments. For Taxonomies: Identify key
actors with their roles and responsibilities from all stakeholders,
categorize them into components and subcomponents based on their
similarities and differences. In particular data Science and Big Data
terms are discussed.


\video{Usecases}{7:42}{Taxonomies}{https://www.youtube.com/watch?v=7eOtuBV8udo} 


\subsubsection{Reference Architecture
Subgroup}\label{reference-architecture-subgroup}

The focus is to form a community of interest from industry, academia,
and government, with the goal of developing a consensus-based approach
to orchestrate vendor-neutral, technology and infrastructure agnostic
for analytics tools and computing environments. The goal is to enable
Big Data stakeholders to pick-and-choose technology-agnostic analytics
tools for processing and visualization in any computing platform and
cluster while allowing value-added from Big Data service providers and
the flow of the data between the stakeholders in a cohesive and secure
manner. Results include a reference architecture with well defined
components and linkage as well as several exemplars.


\video{Usecases}{10:05}{Architecture}{https://www.youtube.com/watch?v=h4ylW0vztDw} 


\subsubsection{Security and Privacy
Subgroup}\label{security-and-privacy-subgroup}

The focus is to form a community of interest from industry, academia,
and government, with the goal of developing a consensus secure reference
architecture to handle security and privacy issues across all
stakeholders. This includes gaining an understanding of what standards
are available or under development, as well as identifies which key
organizations are working on these standards. The Top Ten Big Data
Security and Privacy Challenges from the CSA (Cloud Security Alliance)
BDWG are studied. Specialized use cases include Retail/Marketing, Modern
Day Consumerism, Nielsen Homescan, Web Traffic Analysis, Healthcare,
Health Information Exchange, Genetic Privacy, Pharma Clinical Trial Data
Sharing, Cyber-security, Government, Military and Education.


\video{Usecases}{9:51}{Security}{https://www.youtube.com/watch?v=dHrHk-GvruY} 


\subsubsection{Technology Roadmap
Subgroup}\label{technology-roadmap-subgroup}

The focus is to form a community of interest from industry, academia,
and government, with the goal of developing a consensus vision with
recommendations on how Big Data should move forward by performing a good
gap analysis through the materials gathered from all other NBD
subgroups. This includes setting standardization and adoption priorities
through an understanding of what standards are available or under
development as part of the recommendations. Tasks are gather input from
NBD subgroups and study the taxonomies for the actors' roles and
responsibility, use cases and requirements, and secure reference
architecture; gain understanding of what standards are available or
under development for Big Data; perform a thorough gap analysis and
document the findings; identify what possible barriers may delay or
prevent adoption of Big Data; and document vision and recommendations.


\video{Usecases}{4:14}{Technology}{https://www.youtube.com/watch?v=va0UCR5gMTA} 


\subsubsection{Interfaces subgroup}\label{interfaces-subgroup}

This subgroup is working on the following document: \emph{NIST Big Data
Interoperability Framework: Volume 8, Reference Architecture Interface}.

This document summarizes interfaces that are instrumental for the
interaction with Clouds, Containers, and HPC systems to manage virtual
clusters to support the NIST Big Data Reference Architecture (NBDRA).
The Representational State Transfer (REST) paradigm is used to define
these interfaces allowing easy integration and adoption by a wide
variety of frameworks. . This volume, Volume 8, uses the work performed
by the NBD-PWG to identify objects instrumental for the NIST Big Data
Reference Architecture (NBDRA) which is introduced in the NBDIF: Volume
6, Reference Architecture.

This presentation was given at the \emph{2nd NIST Big Data Public
Working Group (NBD-PWG) Workshop} in Washington DC in June 2017. It
explains our thoughts on deriving automatically a refernce architecture
form the Refernce Architecture Interface specifications directly from
the document.

The workshop Web page is located at

\begin{itemize}
\item
  \url{https://bigdatawg.nist.gov/workshop2.php}
\end{itemize}

The agenda of teh workshop is as follows:

\begin{itemize}
\item
  \url{https://bigdatawg.nist.gov/2017_NIST_Big_Data_PWG_WorkshopAgenda_with_Speakers_Bio.pdf}
\end{itemize}

The Web cas of the presentation is given bellow, while you need to fast
forward to a particular time

\begin{itemize}
\item
  Webcast: Interface subgroup:
  \url{https://www.nist.gov/news-events/events/2017/06/2nd-nist-big-data-public-working-group-nbd-pwg-workshop}

  \begin{itemize}
    \item
    see: Big Data Working Group Day 1, part 2 Time start: 21:00 min,
    Time end: 44:00
  \end{itemize}
\item
  Slides:
  \url{https://github.com/cloudmesh/cloudmesh.rest/blob/master/docs/NBDPWG-vol8.pptx?raw=true}
\item
  Document:
  \url{https://github.com/cloudmesh/cloudmesh.rest/raw/master/docs/NIST.SP.1500-8-draft.pdf}
\end{itemize}

You are welcome to view other presentations if you are interested.

\subsubsection{Requirements and Use Case Subgroup
Introduction}\label{requirements-and-use-case-subgroup-introduction}

The focus is to form a community of interest from industry, academia,
and government, with the goal of developing a consensus list of Big Data
requirements across all stakeholders. This includes gathering and
understanding various use cases from diversified application
domains.Tasks are gather use case input from all stakeholders; derive
Big Data requirements from each use case; analyze/prioritize a list of
challenging general requirements that may delay or prevent adoption of
Big Data deployment; develop a set of general patterns capturing the
`'essence'' of use cases (not done yet) and work with Reference
Architecture to validate requirements and reference architecture by
explicitly implementing some patterns based on use cases. The progress
of gathering use cases (discussed in next two units) and requirements
systemization are discussed.


\video{Usecases}{27:28}{Requirements}{https://www.youtube.com/watch?v=f_vxmx3CmMU} 


\subsection{51 Big Data Use Cases}\label{big-data-use-cases}

This units consists of one or more slides for each of the 51 use cases -
typically additional (more than one) slides are associated with
pictures. Each of the use cases is identified with source of parallelism
and the high and low level computational structure. As each new
classification topic is introduced we briefly discuss it but full
discussion of topics is given in following unit.


\slides{Usecases}{51}{100}{https://drive.google.com/open?id=0B8936_ytjfjmYUlKckhLSUQxMUk}


\subsubsection{Government Use Cases}\label{government-use-cases}

This covers Census 2010 and 2000 - Title 13 Big Data; National Archives
and Records Administration Accession NARA, Search, Retrieve,
Preservation; Statistical Survey Response Improvement (Adaptive Design)
and Non-Traditional Data in Statistical Survey Response Improvement
(Adaptive Design).


\video{Usecases}{17:43}{Government Use Cases}{https://www.youtube.com/watch?v=e0ks_BuYUVM} 


\subsubsection{Commercial Use Cases}\label{commercial-use-cases}

This covers Cloud Eco-System, for Financial Industries (Banking,
Securities \& Investments, Insurance) transacting business within the
United States; Mendeley - An International Network of Research; Netflix
Movie Service; Web Search; IaaS (Infrastructure as a Service) Big Data
Business Continuity \& Disaster Recovery (BC/DR) Within A Cloud
Eco-System; Cargo Shipping; Materials Data for Manufacturing and
Simulation driven Materials Genomics.


\video{Usecases}{17:43}{Commercial Use Cases}{https://www.youtube.com/watch?v=URy9u8_34ww} 


\subsubsection{Defense Use Cases}\label{defense-use-cases}

This covers Large Scale Geospatial Analysis and Visualization; Object
identification and tracking from Wide Area Large Format Imagery (WALF)
Imagery or Full Motion Video (FMV) - Persistent Surveillance and
Intelligence Data Processing and Analysis.


\video{Usecases}{15:43}{Defense Use Cases}{https://www.youtube.com/watch?v=FXFfE8zcco8} 


\subsubsection{Healthcare and Life Science Use
Cases}\label{healthcare-and-life-science-use-cases}

This covers Electronic Medical Record (EMR) Data; Pathology
Imaging/digital pathology; Computational Bioimaging; Genomic
Measurements; Comparative analysis for metagenomes and genomes;
Individualized Diabetes Management; Statistical Relational Artificial
Intelligence for Health Care; World Population Scale Epidemiological
Study; Social Contagion Modeling for Planning, Public Health and
Disaster Management and Biodiversity and LifeWatch.


\video{Usecases}{30:11}{Healthcare and Life  Science Use Cases}{https://www.youtube.com/watch?v=uGeYrXENlpU}


\subsubsection{Deep Learning and Social Networks Use
Cases}\label{deep-learning-and-social-networks-use-cases}

This covers Large-scale Deep Learning; Organizing large-scale,
unstructured collections of consumer photos; Truthy: Information
diffusion research from Twitter Data; Crowd Sourcing in the Humanities
as Source for Bigand Dynamic Data; CINET: Cyberinfrastructure for
Network (Graph) Science and Analytics and NIST Information Access
Division analytic technology performance measurement, evaluations, and
standards.


\video{Usecases}{14:19}{Deep Learning and Social Networks Use Cases}{https://www.youtube.com/watch?v=bdWyhT8bvE4}


\subsubsection{Research Ecosystem Use
Cases}\label{research-ecosystem-use-cases}

DataNet Federation Consortium DFC; The `Discinnet process', metadata
-big data global experiment; Semantic Graph-search on Scientific
Chemical and Text-based Data and Light source beamlines.


\video{Usecases}{9:09}{Research Ecosystem Use Cases}{https://www.youtube.com/watch?v=jjyv4RmMIUU} 


\subsubsection{Astronomy and Physics Use
Cases}\label{astronomy-and-physics-use-cases}

This covers Catalina Real-Time Transient Survey (CRTS): a digital,
panoramic, synoptic sky survey; DOE Extreme Data from Cosmological Sky
Survey and Simulations; Large Survey Data for Cosmology; Particle
Physics: Analysis of LHC Large Hadron Collider Data: Discovery of Higgs
particle and Belle II High Energy Physics Experiment.


\video{Usecases}{17:33}{Astronomy and Physics Use Cases}{https://www.youtube.com/watch?v=MPEe8yDVwAo}



\subsubsection{Environment, Earth and Polar Science Use
Cases}\label{environment-earth-and-polar-science-use-cases}

EISCAT 3D incoherent scatter radar system; ENVRI, Common Operations of
Environmental Research Infrastructure; Radar Data Analysis for CReSIS
Remote Sensing of Ice Sheets; UAVSAR Data Processing, DataProduct
Delivery, and Data Services; NASA LARC/GSFC iRODS Federation Testbed;
MERRA Analytic Services MERRA/AS; Atmospheric Turbulence - Event
Discovery and Predictive Analytics; Climate Studies using the Community
Earth System Model at DOE's NERSC center; DOE-BER Subsurface
Biogeochemistry Scientific Focus Area and DOE-BER AmeriFlux and FLUXNET
Networks.


\video{Usecases}{25:29}{Environment, Earth and Polar Science Use Cases}{https://www.youtube.com/watch?v=YJGk-uvaUCg} 



\subsubsection{Energy Use Case}\label{energy-use-case}

This covers Consumption forecasting in Smart Grids.


\video{Usecases}{4:01}{Energy Use Case}{https://www.youtube.com/watch?v=5y_O-a8_Fbg} 



\subsection{Features of 51 Big Data Use
Cases}\label{features-of-51-big-data-use-cases}

This unit discusses the categories used to classify the 51 use-cases.
These categories include concepts used for parallelism and low and high
level computational structure. The first lesson is an introduction to
all categories and the further lessons give details of particular
categories.


\slides{Usecases}{Features}{43}{https://drive.google.com/open?id=0B8936_ytjfjmREJTMHhjMktXRHc}


\subsubsection{Summary of Use Case Classification
I}\label{summary-of-use-case-classification-i}

This discusses concepts used for parallelism and low and high level
computational structure. Parallelism can be over People (users or
subjects), Decision makers; Items such as Images, EMR, Sequences;
observations, contents of online store; Sensors -- Internet of Things;
Events; (Complex) Nodes in a Graph; Simple nodes as in a learning
network; Tweets, Blogs, Documents, Web Pages etc.; Files or data to be
backed up, moved or assigned metadata; Particles/cells/mesh points. Low
level computational types include PP (Pleasingly Parallel); MR
(MapReduce); MRStat; MRIter (Iterative MapReduce); Graph; Fusion; MC
(Monte Carlo) and Streaming. High level computational types include
Classification; S/Q (Search and Query); Index; CF (Collaborative
Filtering); ML (Machine Learning); EGO (Large Scale Optimizations); EM
(Expectation maximization); GIS; HPC; Agents. Patterns include Classic
Database; NoSQL; Basic processing of data as in backup or metadata; GIS;
Host of Sensors processed on demand; Pleasingly parallel processing; HPC
assimilated with observational data; Agent-based models; Multi-modal
data fusion or Knowledge Management; Crowd Sourcing.


\video{Usecases}{23:39}{Summary of Use Case Classification}{https://www.youtube.com/watch?v=X0vEmbn1Ld8}



\subsubsection{Database(SQL) Use Case
Classification}\label{databasesql-use-case-classification}

This discusses classic (SQL) datbase approach to data handling with
Search\&Query and Index features. Comparisons are made to NoSQL
approaches.


\video{Usecases}{11:13}{Database (SQL) Use Case Classification}{https://www.youtube.com/watch?v=jIVdQID11Q4}


\subsubsection{NoSQL Use Case
Classification}\label{nosql-use-case-classification}

This discusses NoSQL (compared in previous lesson) with HDFS, Hadoop and
Hbase. The Apache Big data stack is introduced and further details of
comparison with SQL.


\video{Usecases}{11:20}{NoSQL Use Case Classification}{https://www.youtube.com/watch?v=uGL8cFPrhoE} 


\subsubsection{Use Case Classifications
I}\label{use-case-classifications-i}

This discusses a subset of use case features: GIS, Sensors. the support
of data analysis and fusion by streaming data between filters.


\video{Usecases}{12:42}{Use Case Classifications I}{https://www.youtube.com/watch?v=79IwNCNjVWU} 



\subsubsection{Use Case Classifications
II}\label{use-case-classifications-ii}

This discusses a subset of use case features: Pleasingly parallel,
MRStat, Data Assimilation, Crowd sourcing, Agents, data fusion and
agents, EGO and security.


\video{Usecases}{20:18}{Use Case Classifications II}{https://www.youtube.com/watch?v=b-olNbWCJyg} 



\subsubsection{Use Case Classifications
III}\label{use-case-classifications-iii}

This discusses a subset of use case features: Classification, Monte
Carlo, Streaming, PP, MR, MRStat, MRIter and HPC(MPI), global and local
analytics (machine learning), parallel computing, Expectation
Maximization, graphs and Collaborative Filtering.


\video{Usecases}{17:25}{Use Case Classifications III}{https://www.youtube.com/watch?v=ewqoFGxyQmc} 



\TODO{These resources have not all been checked to see if they still
  exist this is curretnly in progress}

\subsubsection{Resources}\label{resources}

\begin{itemize}
\item
  NIST Big Data Public Working Group (NBD-PWG) Process
  \url{https://www.nist.gov/el/cyber-physical-systems/big-data-pwg}
\item
  Big Data Definitions: \url{http://dx.doi.org/10.6028/NIST.SP.1500-1}
  (link is external)
\item
  Big Data Taxonomies: \url{http://dx.doi.org/10.6028/NIST.SP.1500-2}
  (link is external)
\item
  Big Data Use Cases and Requirements:
  \url{http://dx.doi.org/10.6028/NIST.SP.1500-3} (link is external)
\item
  Big Data Security and Privacy:
  \url{http://dx.doi.org/10.6028/NIST.SP.1500-4} (link is external)
\item
  Big Data Architecture White Paper Survey:
  \url{http://dx.doi.org/10.6028/NIST.SP.1500-5} (link is external)
\item
  Big Data Reference Architecture:
  \url{http://dx.doi.org/10.6028/NIST.SP.1500-6} (link is external)
\item
  Big Data Standards Roadmap:
  \url{http://dx.doi.org/10.6028/NIST.SP.1500-7} (link is external)
\end{itemize}

Some of the links bellow may be outdated. Please let us know the new
links and notify us of the outdated links.



\begin{itemize}
\item
  DCGSA Standard
  Cloud:~\url{https://www.youtube.com/watch?v=l4Qii7T8zeg}
\item
  On line 51 Use Cases \url{http://bigdatawg.nist.gov/usecases.php}
\item
  Summary of Requirements Subgroup
  \url{http://bigdatawg.nist.gov/_uploadfiles/M0245_v5_6066621242.docx}
\item
  Use Case 6 Mendeley
  \url{http://mendeley.com\%20http//dev.mendeley.com}
\item
  Use Case 7 Netflix
  \url{http://www.slideshare.net/xamat/building-largescale-realworld-recommender-systems-recsys2012-tutoria}
\item
  Use Case 8 Search
  \url{http://www.slideshare.net/kleinerperkins/kpcb-internet-trends-2013},
  \url{http://webcourse.cs.technion.ac.il/236621/Winter2011-2012/en/ho_Lectures.html},
  \url{http://www.ifis.cs.tu-bs.de/teaching/ss-11/irws},
  \url{http://www.slideshare.net/beechung/recommender-systems-tutorialpart1intro},
  \url{http://www.worldwidewebsize.com/}
\item
  Use Case 9 IaaS (Infrastructure as a Service) Big Data Business
  Continuity \& Disaster Recovery (BC/DR) Within A Cloud Eco-System
  provided by Cloud Service Providers (CSPs) and Cloud Brokerage Service
  Providers (CBSPs) \url{http://www.disasterrecovery.org/}
\item
  Use Case 11 and Use Case 12 Simulation driven Materials Genomics
  \url{https://www.materialsproject.org/}
\item
  Use Case 13 Large Scale Geospatial Analysis and Visualization
  \url{http://www.opengeospatial.org/standards},~
  \url{http://geojson.org/}~,
  \url{http://earth-info.nga.mil/publications/specs/printed/CADRG/cadrg.html}~
\item
  Use Case 14 Object identification and tracking from Wide Area Large
  Format Imagery (WALF) Imagery or Full Motion Video (FMV) - Persistent
  Surveillance
  \url{http://www.militaryaerospace.com/topics/m/video/79088650/persistent-surveillance-relies-on-extracting-relevant-data-points-and-connecting-the-dots.htm},
  \url{http://www.defencetalk.com/wide-area-persistent-surveillance-revolutionizes-tactical-isr-45745/}
\item
  Use Case 15 Intelligence Data Processing and Analysis
  \url{http://www.afcea-aberdeen.org/files/presentations/AFCEAAberdeen_DCGSA_COLWells_PS.pdf},
  \url{http://stids.c4i.gmu.edu/papers/STIDSPapers/STIDS2012/_T14/_SmithEtAl/_HorizontalIntegrationOfWarfighterIntel.pdf},
  \url{http://stids.c4i.gmu.edu/STIDS2011/papers/STIDS2011_CR_T1_SalmenEtAl.pdf},
  \url{https://www.youtube.com/watch?v=l4Qii7T8zeg},
  \url{http://dcgsa.apg.army.mil/}


\item
  Use Case 16 Electronic Medical Record (EMR) Data:
  \href{http://www.regenstrief.org/}{Regenstrief Institute},
  \href{http://loinc.org/}{Logical observation identifiers names and
  codes}, \href{http://www.ihie.org/}{Indiana Health Information
  Exchange},
  \href{http://www.iom.edu/Activities/Quality/LearningHealthcare.aspx}{Institute
  of Medicine Learning Healthcare System}
\item
  Use Case 17 Pathology Imaging/digital pathology;
  \url{https://web.cci.emory.edu/confluence/display/PAIS}~,~https://web.cci.emory.edu/confluence/display/HadoopGIS
\item
  Use Case 19 Genome in a Bottle Consortium:
  \href{https://bigdatacoursespring2015.appspot.com/www.genomeinabottle.org}{www.genomeinabottle.org}
\item
  Use Case 20 Comparative analysis for metagenomes and genomes
  \url{http://img.jgi.doe.gov/}
\item
  Use Case 25
  \href{https://www.biodiversitycatalogue.org/}{Biodiversity} and
  \href{http://www.lifewatch.eu/web/guest/home}{LifeWatch}
\item
  Use Case 26 Deep Learning: Recent popular press coverage of deep
  learning technology:
  \url{http://www.nytimes.com/2012/11/24/science/scientists-see-advances-in-deep-learning-a-part-of-artificial-intelligence.html}~,
  \url{http://www.nytimes.com/2012/06/26/technology/in-a-big-network-of-computers-evidence-of-machine-learning.html}~,
  \url{http://www.wired.com/2013/06/andrew_ng/},~

  A recent research paper on HPC for Deep Learning:
  \url{http://www.stanford.edu/~acoates/papers/CoatesHuvalWangWuNgCatanzaro_icml2013.pdf},
  Widely-used tutorials and references for Deep Learning:
  \url{http://ufldl.stanford.edu/wiki/index.php/Main_Page},
  \url{http://deeplearning.net/}
\item
  Use Case 27 Organizing large-scale, unstructured collections of
  consumer photos \url{http://vision.soic.indiana.edu/projects/disco/}
\item
  Use Case 28 Truthy: Information diffusion research from Twitter Data
  \url{http://truthy.indiana.edu/}~,~http://cnets.indiana.edu/groups/nan/truthy/~,~http://cnets.indiana.edu/groups/nan/despic/
\item
  Use Case 30 CINET: Cyberinfrastructure for Network (Graph) Science and
  Analytics \url{http://cinet.vbi.vt.edu/cinet_new/}
\item
  Use Case 31 NIST Information Access Division analytic technology
  performance measurement, evaluations, and standards
  \url{http://www.nist.gov/itl/iad/}
\item
  Use Case 32 DataNet Federation Consortium DFC:
  \href{http://datafed.org/}{The DataNet Federation Consortium},
  \href{http://irods.org/}{iRODS}
\item
  Use Case 33 The `Discinnet process', metadata \textless{} -
  \textgreater{} big data global experiment
  \url{http://www.discinnet.org/}
\item
  Use Case 34 Semantic Graph-search on Scientific Chemical and
  Text-based Data
  \url{http://www.eurekalert.org/pub_releases/2013-07/aiop-ffm071813.php}
 , \url{http://xpdb.nist.gov/chemblast/pdb.pl}
\item
  Use Case 35 Light source beamlines
  \url{http://www-als.lbl.gov/}~,~https://www1.aps.anl.gov/
\item
  Use Case 36 \href{http://crts.caltech.edu/}{CRTS survey},
  \href{http://www.lpl.arizona.edu/css/}{CSS survey} ; For an overview
  of the classification challenges, see, e.g.,
  \url{http://arxiv.org/abs/1209.1681}
\item
  Use Case 37 DOE Extreme Data from Cosmological Sky Survey and
  Simulations
  \url{http://www.lsst.org/lsst/}~,~http://www.nersc.gov/~,~http://www.nersc.gov/assets/Uploads/HabibcosmosimV2.pdf
\item
  Use Case 38 Large Survey Data for Cosmology \url{http://desi.lbl.gov/}
 , \url{http://www.darkenergysurvey.org/}

\item
  Use Case 39 Particle Physics: Analysis of LHC Large Hadron Collider
  Data: Discovery of Higgs particle
  \url{http://grids.ucs.indiana.edu/ptliupages/publications/Where\%20does\%20all\%20the\%20data\%20come\%20from\%20v7.pdf},
  \url{http://www.es.net/assets/pubs_presos/High-throughput-lessons-from-the-LHC-experience.Johnston.TNC2013.pdf}
\item
  Use Case 40 Belle II High Energy Physics Experiment
  \url{http://belle2.kek.jp/}
\item
  Use Case 41 EISCAT 3D incoherent scatter radar system
  \url{https://www.eiscat3d.se/}

\item
  Use Case 42 ENVRI, Common Operations of Environmental Research
  Infrastructure, 
  \href{http://envri.eu/}{ENVRI Project website},
  \href{http://confluence.envri.eu:8090/display/ERM/Start}{ENVRI  Reference Model},
  \href{http://confluence.envri.eu:8090/download/attachments/327687/D3.3\%20Analysis\%20of\%20Requirements\%20V1.0.pdf?version=1\&modificationDate=1366965933706\&api=v2}{ENVRI
    deliverable D3.2 : Analysis of common requirements of
    Environmental  Research Infrastructures}, 
\href{https://www.icos-ri.eu/}{ICOS},
  \href{http://www.euro-argo.eu/}{Euro-Argo},
  \href{https://www.eiscat3d.se/node}{EISCAT 3D},
  \href{http://www.lifewatch.com/}{LifeWatch},
  \href{http://www.epos-eu.org/}{EPOS},
  \href{http://www.emso-eu.org/}{EMSO}

\item
  Use Case 43 Radar Data Analysis for CReSIS Remote Sensing of Ice
  Sheets \url{https://www.cresis.ku.edu/}
\item
  Use Case 44 UAVSAR Data Processing, Data Product Delivery, and Data
  Services
  \url{http://uavsar.jpl.nasa.gov/}, \url{http://www.asf.alaska.edu/program/sdc}, \url{http://geo-gateway.org/main.html}
\item
  Use Case 47 Atmospheric Turbulence - Event Discovery and Predictive
  Analytics
  \url{http://oceanworld.tamu.edu/resources/oceanography-book/teleconnections.htm},
  \url{http://www.forbes.com/sites/toddwoody/2012/03/21/meet-the-scientists-mining-big-data-to-predict-the-weather/}
\item
  Use Case 48 Climate Studies using the Community Earth System Model at
  DOE's NERSC center
  \url{http://www-pcmdi.llnl.gov/}, 
  \url{http://www.nersc.gov/}, 
  \url{http://science.energy.gov/ber/research/cesd/}, 
  \url{http://www2.cisl.ucar.edu/}

\item
  Use Case 50 DOE-BER AmeriFlux and FLUXNET Networks
  \url{http://ameriflux.lbl.gov/},
  \url{http://www.fluxdata.org/default.aspx}
\item
  Use Case 51 Consumption forecasting in Smart Grids
  \url{http://smartgrid.usc.edu/},
  \url{http://ganges.usc.edu/wiki/Smart_Grid},

  \url{https://www.ladwp.com/ladwp/faces/ladwp/aboutus/a-power/a-p-smartgridla?_afrLoop=157401916661989\&_afrWindowMode=0\&_afrWindowId=null\#\%40\%3F_afrWindowId\%3Dnull\%26_afrLoop\%3D157401916661989\%26_afrWindowMode\%3D0\%26_adf.ctrl-state\%3Db7yulr4rl_17},
  \url{http://ieeexplore.ieee.org/xpl/articleDetails.jsp?arnumber=6475927}


\end{itemize}
