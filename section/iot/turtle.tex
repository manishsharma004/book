\FILENAME

\section{Turtle Graphics}\label{turtle-graphics}

\subsection{Demo}\label{demo}

Python comes with a nice demonstartion program that allows you to learn
some simple programming concepts while moving a turtle on the screen. It
can be started with

\begin{verbatim}
python -m turtledemo
\end{verbatim}

\subsection{Program example}\label{program-example}

You can also create programs with your favorite editor and run it. Let
us put the following code into the program \texttt{turtle\_demo.py}.
Never save a file with the name \texttt{turtle.py} because python will
import it instead of the built-in turtle import that you need.

\begin{verbatim}
import turtle

window = turtle.Screen() 
robot = turtle.Turtle() 

robot.forward(50)   # Moves forward 50 pixels
robot.right(90)     # Rotate clockwise by 90 degrees

robot.forward(50)
robot.right(90)

robot.forward(50)
robot.right(90)

robot.forward(50)
robot.right(90)

turtle.done()

window.mainloop()
\end{verbatim}

After saving it you can run it from a terminal with

\begin{verbatim}
$ python turtle_demo.py
\end{verbatim}

\subsection{Shape}\label{shape}

\begin{verbatim}
shapes: “arrow”, “turtle”, “circle”, “square”, “triangle”, “classic”
\end{verbatim}

You can change the shape of your turtle to any of these shapes with the
Turtle method \texttt{shape(name)}. For example, if you have an instance
of the Turtle class called \texttt{robot}, you can make it appear as a
turtle by calling \texttt{robot.shape("turtle")}.

You can add your own shapes with the following functions:

\begin{verbatim}
turtle.register_shape(name, shape=None)

turtle.addshape(name, shape=None)
\end{verbatim}

There are three different ways to call this function:

name is the name of a gif-file and shape is None: Install the
corresponding image shape.

\begin{verbatim}
window.register_shape("turtle.gif")
\end{verbatim}

Note: Image shapes do not rotate when turning the turtle, so they do not
display the heading of the turtle!

name is an arbitrary string and shape is a tuple of pairs of
coordinates: Install the corresponding polygon shape.

\begin{verbatim}
window.register_shape("triangle", ((5,-3), (0,5), (-5,-3)))
\end{verbatim}

name is an arbitrary string and shape is a (compound) Shape object:
Install the corresponding compound shape.

Add a turtle shape to TurtleScreen's shapelist. Only thusly registered
shapes can be used by issuing the command shape(shapename).

\subsection{Links}\label{links}

\begin{itemize}
\tightlist
\item
  http://openbookproject.net/thinkcs/python/english3e/hello\_little\_turtles.html
\item
  \url{https://docs.python.org/3/library/turtle.html}
\end{itemize}

\section{Robot Dance Simulator}\label{robot-dance-simulator}

\begin{verbatim}
cms robot dance dance.txt
\end{verbatim}

\section{Scratch}\label{scratch}

\begin{itemize}
\tightlist
\item
  \href{https://scratch.mit.edu/scratchr2/static/sa/Scratch-456.0.2.dmg}{Scratch}
\end{itemize}

\section{MBlock}\label{mblock}

\begin{itemize}
\tightlist
\item
  \href{http://www.mblock.cc/download/}{MBlock}
\end{itemize}
