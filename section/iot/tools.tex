

\chapter{Tools}\label{tools}

\FILENAME

\begin{itemize}
\item
  \textbf{Terminal}: On OSX, when you navigate to the search
  magnification glass, you can type in \emph{terminal} to start it. A
  terminal allows you to execute a number of commands to interact with
  the computer from a commandline interface, e.g.~the terminal.
\item
  \href{https://linuxconfig.org/bash-scripting-tutorial}{Bash} it the
  command language used in terminal.
\item
  \href{https://cloudmesh.github.io/classes/lesson/prg/pyenv.html?highlight=xcode\#install-pyenv-on-osxhttps://cloudmesh.github.io/classes/lesson/prg/pyenv.html?highlight=xcode\#install-pyenv-on-osx}{Pyenv}
  allows to manage multiple versions of python easily.
  \href{https://github.com/pyenv/pyenv\#how-it-works}{Pyenv link}
\item
  \href{https://cloudmesh.github.io/classes/lesson/prg/pyenv.html?highlight=xcode\#install-pyenv-on-osxhttps://cloudmesh.github.io/classes/lesson/prg/pyenv.html?highlight=xcode\#install-pyenv-on-osx}{XCode}
  is an integrated development environment for macOS containing a suite
  of software development tools developed by Apple for developing
  software for macOS, iOS, watchOS and tvOS.
\item
  \href{https://brew.sh}{Homebrew} is a \emph{package manager} for OS X
  which lets the user \emph{install software} from \emph{UNIX} and
  \emph{open source software} that is not included in OSX.
\item
  \href{https://www.jetbrains.com/pycharm/download/download-thanks.html?platform=mac\&code=PCC}{pyCharm}:
  is an Integrated Development Environment for Python.
\item
  \emph{Matplotlib}: Matplotlib is a libarry that allows us to create
  nice graphs in python. As we typically install python with virtualenv,
  we need to configure matplotlib properly to use it. The easiest way to
  do this is to execute the following commands. After you run them you
  can use matplotlib.

\begin{verbatim}
$ pip install numpy
$ pip install matplotlib
$ echo "backend : TkAgg" > ~/.matplotlib/matplotlibrc
\end{verbatim}
\item
  \href{https://macdown.uranusjr.com/}{Macdown} a macdown editor for OSX
\item
  \href{https://blog.ghost.org/markdown/}{Markdown} (from Markdown)
\item
  \href{http://oracc.museum.upenn.edu/doc/help/usingemacs/aquamacs/}{AquaEmacs}
  (from Aquaemacs)
\item
  \href{http://marvelmind.com/}{Marvelmind} (from Marvelmind if you have
  marvelmind positioning sensors which are optional)
\item
  \href{https://www.arduino.cc/en/guide/macOSX}{Arduino} (from Arduino
  if you like to use their interface to access the esp8266 boards)
\item
  \href{https://computers.tutsplus.com/tutorials/40-terminal-tips-and-tricks-you-never-thought-you-needed--mac-51192}{40
  OSX Terminal Tricks}
\end{itemize}

\section{Markdown}\label{markdown}

MarkDown is a format convention that produces nicely formated text with
simple ASCII text. Markdown has very good support for editors that
render the final output in a view window next to the editor pane. Two
such editors are

\begin{itemize}

\item
  \href{https://macdown.uranusjr.com/}{Macdown}: MacDown provides a nice
  integrated editor that works well.
\item
  \href{https://www.jetbrains.com/pycharm/download/download-thanks.html?platform=mac\&code=PCC}{pyCharm}:
  We have successfully used Vladimir Schhneiders
  \href{https://plugins.jetbrains.com/plugin/7896-markdown-navigator}{Markdown
  Navigator plugin}. Once installes you click on a .md file pycharm will
  automatically ask to install the plugins from Markdown for you.
\end{itemize}

A detailed set of syntax rules can be found at: \textbf{BUG: LINK TO
MARKDOWN SYNTAX MISSING}

The following are some basic examples

\begin{itemize}

\item
  To \emph{empazise} a text you use \texttt{*empasize*}
\item
  To make text \textbf{bold} use \texttt{**bold**}
\item
  To make text \textbf{\emph{bold-and-emphasize}} use
  \texttt{***bold-and-emphasize***}
\item
  To create a hyperlink use \texttt{{[}Google{]}(https://google.com)}
  which will result in \href{https://google.com}{Google}
\item
  To include an image use \texttt{!{[}Bracketed\ Text{]}(link)}
\end{itemize}

A list can be created by item starting with *, a - , or a + or a
number

\begin{verbatim}
1. one
2. two
\end{verbatim}

\begin{enumerate}
\def\labelenumi{\arabic{enumi}.}
\item
  one
\item
  two

\begin{verbatim}
* one
* two
\end{verbatim}
\end{enumerate}

\begin{itemize}

\item
  one
\item
  two
\end{itemize}

If you need to indent items underneath already bulleted items, precede
the indent items with four spaces and they will be nested under the item
above them.

To qoute textc precede it with a ``\textgreater{}''.

\begin{verbatim}
> Quote
\end{verbatim}

\begin{quote}
Quote
\end{quote}

Other syntax options can be found in the Format drop-down at the top of
the screen between View and Plug-ins of macdown.

\section{Aquamacs}\label{aquamacs}

There are many different versions of emacs available on OSX. Aquamacs is
often used as it integrates nicely with the OSX GUI interface.

\begin{itemize}

\item
  \href{http://aquamacs.org/download.shtml}{AquaEmacs}
\end{itemize}

\emph{Aquamacs} is a program for Mac devices which allows the user to
edit text, HTML, LaTeX, C++, Java, Python, R, Perl, Ruby, PHP, and more.
Aquaemacs integrates well with OSX and provides many functions through a
menu. You will mostly be using the File, Edit, menus or toolbar icons.

Emacs provides convenient keyboard shortcuts, most of which are
combinations with the Control or Meta key (The Meta key is the ESC key).
If you accidentally end up doing something wrong simply press
\texttt{CTRL-g} to get out without issue. Other Keyboard Shortcuts
include:

\begin{itemize}
\item
  \texttt{CTRL-x\ u} or File\textgreater{}Undo will cancel any command
  that you did not want done. (CHECK)
\item
  \texttt{ESC-g} will cancel any command you are in the middle of.
\item
  You can break paragraph lines with \texttt{Ctrl-x\ w}, where
  \texttt{w} will wrap text around word boundaries.
\item
  To delete text to the end of the current word, press \texttt{ESC-d}.
\item
  to delete the whole line from the position of the cursor to the end,
  press \texttt{CTRL-k}.
\end{itemize}

\section{Bash}\label{bash}

Bash is preinstalled in OSX. A \emph{bash} script contains
\emph{commands} in plain text. In order to create a bash script please
decide for a convenient name. Ltes assume we name our script
\emph{myscript}. Than you can create and edit such a script with

\begin{verbatim}
$ touch myscript.sh
$ emacs myscripts.sh
\end{verbatim}

Next you need to add the following line to the top ogf the script:

\begin{verbatim}
!# /bin/bash
\end{verbatim}

To demonstrate how to continue writing a script we will be using the
bash \texttt{echo} command that allows you to print text. Lets make the
second line

\begin{verbatim}
echo "Hello World"
\end{verbatim}

You can now save and start executing your script. Click ``File'' and
then ``Save''. Open Terminal and type in \texttt{cd} followed by the
name of the folder you put the document in. Now we need to execute the
script.

\emph{Executing} a Bash script is rather easy. In order to execute a
script, we need to first execute the \emph{permission set}. In order to
give Terminal permission to read/execute a Bash script, you have to type

\begin{verbatim}
chmod u+x myscript.sh
\end{verbatim}

After the script has been granted permission to be executed, you can
test it by typing

\begin{verbatim}
./myscript.sh
\end{verbatim}

into the terminal. You will see it prints

\begin{verbatim}
Hello World
\end{verbatim}

\section{Arduino}\label{arduino}

This instalation is optional. In the event that there is a TTY error,
you will need to install Arduino, since your Mac may be missing some
drivers that are included in Arduino. Simply go to
\href{https://www.arduino.cc/en/guide/macOSX}{Arduino} and follow the
instalation instructions.

\section{OSX Terminal}\label{osx-terminal}

\href{https://learn.sparkfun.com/tutorials/terminal-basics/coolterm-windows-mac-linux}{CoolTerm}

download \url{http://freeware.the-meiers.org/CoolTermMac.zip}
