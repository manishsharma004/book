\section{Data Formats}

\subsection{YAML}

The term \emph{YAML} stand for ``YAML Ain't Markup
Language''. According to the Web Page at 

\URL{http://yaml.org/}

``YAML is a human friendly data serialization standard for all
programming languages.'' There are multiple versions of YAML existing
and one needs to take care of that your software supports the right
version. The current version is YAML 1.2.

YAML is often used for configuration and in many cases can also be
used as XML replacement. Important is tat YAM in contrast to XML
removes the tags while replacing them with indentation. This has
naturally the advantage that it is mor easily to read, however, the
format is strict and needs to adher to proper indentation. Thus it is
important that you check your YAML files for correctness, either by
writing for example a python program that read your yaml file, or an
online YAML checker such as provided at 

\URL{http://www.yamllint.com/}

An example on how to use yaml in python is provided in
Figure~\ref{F:yaml}. Please note that YAML is a superset of
JSON. Originally YAML was designed as a markup language. However as it
is not document oriented but data oriented it has been recast and it
does no longer clasify itself as markup language. 

\begin{figure}[htb]
\begin{lstlisting}
import os
import sys
import yaml

try:
    yamlFilename = os.sys.argv[1]
    yamlFile = open(yamlFilename, "r")
except:
    print("filename does not exist")
    sys.exit()
try:
   yaml.load(yamlFile.read())
except:
   print("YAML file is not valid.")
\end{lstlisting}
\caption{Python YAML example}\label{F:yaml}
\end{figure}

Resources: 

\URL{http://yaml.org/}
\URL{https://en.wikipedia.org/wiki/YAML}
\URL{http://www.yamllint.com/}


\subsection{JSON}

The term JSON stand for \emph{JavaScript Object Notation}. It is
targeted as an open-standard file format that emphasizes on
integration of human-readable text to transmit data objects.  The data
objects contain attribute value pairs. Although it origiinates from
JavaScript, the format itself is language independent. It uses
brackest to allow organization of the data. PLease note that YAML is a
superset of JSON and not all YAML documents can be converted to
JSON. Furthermore JSON does not support comments. For these reasons we
ofthen prefer to us YAMl instead of JSON. However JSON data can easily
be translated to YAML as well as XML.

Resources:

\URL{https://en.wikipedia.org/wiki/JSON}
\URL{https://www.json.org/}


\subsection{XML}

XML stands for \emph{Extensible Markup Language}. XML allows to define
documents with the help of a set of rules in order to make it machine
readable. The emphasize here is on machine readable as document in XML can become
quickly complex and difficult to understnd for humans. XML is used for
documents as well as data structures.

A tutorial about XML is available at 

\URL{https://www.w3schools.com/xml/default.asp}

Resources:

\URL{https://en.wikipedia.org/wiki/XML}