

\chapter{Linux}
\label{C:linux}

\FILENAME

\section{History}

LINUX is a reimplementation by the community of UNIX which was
developed in 1969 by Ken Thompson and Dennis Ritchie of Bell
Laboratories and rewritten in C. An important part of UNIX is what is
called the {\em kernel} which allows the software to talk to
the hardware and utilize it. 

In 1991 Linus Trovalds started developing a Linux Kernel that was
initially targeted for PC's. THis made it possible to run it on
Laptops and was later on further developed by making it a full
Operating system replacement for UNIX. 

\section{Shell}

One of the most important features for us will be to access the
computer with the help of a {\em shell}. THe shell is typically run in
what is called a terminal and allows interaction to the computer with
commandline programs. 

There are many good tutorials out there that explain why one needs a
linux shell and not just a GUI. Randomly we picked the first one that
came up with a google query (This is not an endorsement for the material
we point to, but could be a worth while read for someone that has no
experience in Shell programming:

\URL{http://linuxcommand.org/lc3_learning_the_shell.php}

Certainly you are welcome to use other resources that may suite you
best. We will however summarize in table form a number of useful
commands that you may als find even as a RefCard.

\URL{http://www.cheat-sheets.org/\#Linux}

We provide in Table \ref{T:shell-commands} a number of useful commands
that you want to explore. For more information simply type man and the
name of the command.


\begin{center}
\begin{longtable}{|p{4cm}|p{8cm}|}
\caption{Common commands}\label{T:shell-commands}\\

\hline
\multicolumn{1}{|p{4cm}|}{\textbf{Example}} & \multicolumn{1}{p{8cm}|}{\textbf{Description}} \\ 
\hline 
\endfirsthead

\multicolumn{2}{p{12cm}}%
{{\bfseries \tablename\ \thetable{} -- continued from previous page}} \\
\hline 
\hline \multicolumn{1}{|p{4cm}|}{\textbf{Example}} & \multicolumn{1}{p{8cm}|}{\textbf{Description}} \\ 
\hline 
\endhead

\hline 
\multicolumn{2}{|r|}{{Continued on next page}} \\
\hline
\endfoot

%\hline 
\hline
\endlastfoot

  \multicolumn{2}{|l|}{\cellcolor{blue!15} Help Commands}\\
  \hline
  man \emph{command} & manual page for the \emph{command} \\
  apropos {\em text} & list all commands that have text in it\\
  & \\

  \hline
  \multicolumn{2}{|l|}{\cellcolor{blue!15} File Commands}\\
  \hline
  ls & Directory listing\\
  ls -lisa & list details \\
  tree & list the directories in graphical form \\
  cd \emph{dirname} & Change directory to \emph{dirname} \\
  mkdir \emph{dirname} & create the directory \\
  rmdir \emph{dirname} & delete the directory \\
  pwd & print working directory \\
  rm \emph{file} & remove the file \\
  cp \emph{a} \emph{b} & copy file \emph{a} to \emph{b} \\
  mv \emph{a} \emph{b} & move/rename file \emph{a} to \emph{b}\\
  cat \emph{a} & print content of file\emph{a}\\
  cat -n &  (assignment) \\
  less \emph{a} & print paged content of file \emph{a}\\
  head -5 \emph{a} & Display first 5 lines of file \emph{a}\\
  tail -5 \emph{a} & Display last 5 lines of file \emph{a}\\
  du -hs . & show in human readable form the space used by the current
             directory\\
  df -h & show the details of the disk file system \\
  wc {\em filename}&  counts the word in a file \\
  sort {\em filename} &  sorts the file \\
  uniq {\em filename} &  displays only uniq entries in the file \\
  tar -xvf {\em dir} &  tars up a compressed version of the directory \\
  rsync &  (assignment) \\
  gzip &  (assignment) \\
  gunzip &  (assignment) \\
  bzip2 &  (assignment) \\
  bunzip2 &  (assignment) \\
  clear & \\
  touch & \\
  who & \\
  whoami & \\
  echo & \\
  data & \\
  logout & \\
  exit & \\
  kill & \\
  ps & 
  sleep & \\
  uptime & \\
  time & \\
  date & \\
  find & \\
  diff & \\
  hostname & \\
  which & \\
  tail & \\
  head & \\
  top & \\
  locate {\em filename} & \\
  grep & \\
  chmod ug+rw {\em filename} & \\
  chown & \\
  history & \\
  sudo & \\
  su & \\
  uname & \\
  set -o emacs & \\
  & \\
  
  %\hline
  %\multicolumn{2}{|l|}{\cellcolor{blue!15} Search Commands}\\
  %\hline
  chmod go-rwx {\em file} & changes the permission of the file \\
  chown {\em username} {\em file} & changes the ownership of the file \\
  chgrp {\em group} {\em file} & changes the group of a file\\
  & \\

  \hline
  \multicolumn{2}{|l|}{\cellcolor{blue!15} Search Commands}\\
  \hline
  fgrep ``text'' filename &  searches the text in the given file \\
  grep -R ``xyz'' . & recursively searches for xyz in all files \\
  find . -name ``*.py'' &  find all files with .py at the end \\
  & \\

  %\hline
  %\multicolumn{2}{|l|}{\cellcolor{blue!15} Process Commands}\\
  %\hline
  ps & list the running processes \\
  kill -9 1234 & kill the process with the id 1234 \\
  at &  (assignment) \\
  cron &  (assignment) \\
  crontab &  (assignment) \\
  & \\

  \hline
  \multicolumn{2}{|l|}{\cellcolor{blue!15} Device Commands}\\
  \hline
  mount /dev/cdrom /mnt/cdrom & mount a filesystem from a cd rom to /mnt/cdrom\\
  & \\

  \hline
  \multicolumn{2}{|l|}{\cellcolor{blue!15} System Commands}\\
  \hline
  users &  (assignment) \\
  who &  (assignment) \\
  whoami &  (assignment) \\
  dmesg &  (assignment) \\
  last &  (assignment) \\
  free -tm &  (assignment) \\
  uname &  (assignment) \\
  date &  prints the current date and time \\
  time {\em command} &  prints the syste, real and user time \\
  shutdown -h ``shut down'' & (assignment) \\
  & \\

  \hline
  \multicolumn{2}{|l|}{\cellcolor{blue!15} Networking Commands}\\
  \hline
  ping &  (assignment) \\
  netstat &  (assignment) \\
  hostname &  (assignment) \\
  traceroute &  (assignment) \\
  ifconfig &  (assignment) \\
  & \\

  \hline
  \multicolumn{2}{|l|}{\cellcolor{blue!15} Internet Commands}\\
  \hline
  host &  (assignment) \\
  whois &  (assignment) \\
  dig &  (assignment) \\
  wget &  (assignment) \\
  curl &  (assignment) \\
  & \\

  \hline
  \multicolumn{2}{|l|}{\cellcolor{blue!15} Remote Access Commands}\\
  \hline
  ssh &  (assignment) \\
  scp &  (assignment) \\
  sftp &  (assignment) \\
  & \\


\end{longtable}
\end{center}

\section{Multi-command execution}

One of the important features is that one can execute multiple
commands in the shell.

To execute command 2 once command 1 has finished use

\begin{verbatim}
command1; command2
\end{verbatim}

To execute command 2 as soon as command 1 forwards output to stdout use

\begin{verbatim}
command1; command2
\end{verbatim}

To execute command 1 in the background use

\begin{verbatim}
command1 &
\end{verbatim}



\section{Keyboard Shortcuts}\label{keyboard-shortcuts}

These shortcuts will come in handy. Note that many overlap with emacs
short cuts.

\begin{tabular}{ll}
Keys     & Description  \\
\hline
Up Arrow & Show the previous command\\
Ctrl + z & Stops the current command  \\
         & Resume with fg in the foreground \\
         & Resume with bg in the background \\
Ctrl + c & Halts the current command\\
Ctrl + l & Clear the screen\\
Ctrl + a & Return to the start of the line\\
Ctrl + e & Go to the end of the line\\
Ctrl + k & Cut everything after the cursor to a special clipboard\\
Ctrl + y & Paste from the special clipboard \\
Ctrl + d & Logout of current session, similar to exit \\
\end{tabular}

\section{.bashrc and .bash\_profile}

Usage of a particular command and all the attributes associated with it,
use `man' command. Avoid using \verb|rm -r| command to delete files
recursively. A good way to avoid accidental deletion is to include the
following in your \verb|.bash_profile| file:

\begin{verbatim}
alias e=open_emacs
alias rm='rm -i'
alias mv='mv -i' 
alias h='history'
\end{verbatim}

More Information

\url{https://cloudmesh.github.io/classes/lesson/linux/refcards.html}

\section{Exercises}

\begin{exercise}
\label{E:Linux.1}
Familiarize yourself with the commands
\end{exercise}

\begin{exercise}
\label{E:Linux.2}
Find more commands that you find useful and add them to this page.
\end{exercise}

\begin{exercise}
\label{E:Linux.3}
Use the sort command to sort all lines of a file while removing
duplicates.
\end{exercise}

\begin{exercise}
\label{E:Linux.4} In Table \ref{T:shell-commands} you will find a number
  of commands with (assignment). Develop descriptions that you will
  contribute and add to the manual with a pull request. Work in a team
  so that only one pull request is issued. Do not only provide the
  description, but also a real example as showcased within the table.
\end{exercise}

\begin{exercise}
\label{E:Linux.4} Should there be other commands listed in the table. If
  so which? Create a pull request for them. 
\end{exercise}


