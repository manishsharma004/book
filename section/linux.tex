\FILENAME

\section{Linux Shell}\label{linux-shell}

There are many good tutorials out there that explain why one needs a
linux shell and not just a GUI. Randomly we picked the firts one that
came up with a google query (This is not an endorsement for the material
we point to, but could be a worth while read for someone that has no
experience in Shell programming:

\begin{itemize}
\tightlist
\item
  \url{http://linuxcommand.org/lc3_learning_the_shell.php}
\end{itemize}

Certainly you are welcome to use other resources that may suite you
best. We will however summarize in table form a number of useful
commands that you may als find in a link to a RefCard.

\begin{itemize}
\tightlist
\item
  \url{http://www.cheat-sheets.org/\#Linux}
\end{itemize}

\subsection{File commands}\label{file-commands}

Find included a number of commands related to file manipulation.

\begin{longtable}[]{@{}ll@{}}
\toprule
\begin{minipage}[b]{0.26\columnwidth}\raggedright\strut
Command\strut
\end{minipage} & \begin{minipage}[b]{0.50\columnwidth}\raggedright\strut
Description\strut
\end{minipage}\tabularnewline
\midrule
\endhead
\begin{minipage}[t]{0.26\columnwidth}\raggedright\strut
ls\strut
\end{minipage} & \begin{minipage}[t]{0.50\columnwidth}\raggedright\strut
Directory listing\strut
\end{minipage}\tabularnewline
\begin{minipage}[t]{0.26\columnwidth}\raggedright\strut
ls -lisa\strut
\end{minipage} & \begin{minipage}[t]{0.50\columnwidth}\raggedright\strut
list details\strut
\end{minipage}\tabularnewline
\begin{minipage}[t]{0.26\columnwidth}\raggedright\strut
cd \emph{dirname}\strut
\end{minipage} & \begin{minipage}[t]{0.50\columnwidth}\raggedright\strut
Change directory to \emph{dirname}\strut
\end{minipage}\tabularnewline
\begin{minipage}[t]{0.26\columnwidth}\raggedright\strut
mkdir \emph{dirname}\strut
\end{minipage} & \begin{minipage}[t]{0.50\columnwidth}\raggedright\strut
create the directory\strut
\end{minipage}\tabularnewline
\begin{minipage}[t]{0.26\columnwidth}\raggedright\strut
pwd\strut
\end{minipage} & \begin{minipage}[t]{0.50\columnwidth}\raggedright\strut
print working directory\strut
\end{minipage}\tabularnewline
\begin{minipage}[t]{0.26\columnwidth}\raggedright\strut
rm \emph{file}\strut
\end{minipage} & \begin{minipage}[t]{0.50\columnwidth}\raggedright\strut
remove the file\strut
\end{minipage}\tabularnewline
\begin{minipage}[t]{0.26\columnwidth}\raggedright\strut
cp \emph{a} \emph{b}\strut
\end{minipage} & \begin{minipage}[t]{0.50\columnwidth}\raggedright\strut
copy file \emph{a} to \emph{b}\strut
\end{minipage}\tabularnewline
\begin{minipage}[t]{0.26\columnwidth}\raggedright\strut
mv \emph{a} \emph{b}\strut
\end{minipage} & \begin{minipage}[t]{0.50\columnwidth}\raggedright\strut
move/rename file \emph{a} to \emph{b}\strut
\end{minipage}\tabularnewline
\begin{minipage}[t]{0.26\columnwidth}\raggedright\strut
cat \emph{a}\strut
\end{minipage} & \begin{minipage}[t]{0.50\columnwidth}\raggedright\strut
print content of file\emph{a}\strut
\end{minipage}\tabularnewline
\begin{minipage}[t]{0.26\columnwidth}\raggedright\strut
less \emph{a}\strut
\end{minipage} & \begin{minipage}[t]{0.50\columnwidth}\raggedright\strut
print paged content of file \emph{a}\strut
\end{minipage}\tabularnewline
\begin{minipage}[t]{0.26\columnwidth}\raggedright\strut
head -5 \emph{a}\strut
\end{minipage} & \begin{minipage}[t]{0.50\columnwidth}\raggedright\strut
Display first 5 lines of file \emph{a}\strut
\end{minipage}\tabularnewline
\begin{minipage}[t]{0.26\columnwidth}\raggedright\strut
tail -5 \emph{a}\strut
\end{minipage} & \begin{minipage}[t]{0.50\columnwidth}\raggedright\strut
Display last 5 lines of file \emph{a}\strut
\end{minipage}\tabularnewline
\bottomrule
\end{longtable}

\subsection{Search commands}\label{search-commands}

Find included a number of commands related to seraching.

\begin{longtable}[]{@{}ll@{}}
\toprule
\begin{minipage}[b]{0.46\columnwidth}\raggedright\strut
Command\strut
\end{minipage} & \begin{minipage}[b]{0.21\columnwidth}\raggedright\strut
Description\strut
\end{minipage}\tabularnewline
\midrule
\endhead
\begin{minipage}[t]{0.46\columnwidth}\raggedright\strut
fgrep\strut
\end{minipage} & \begin{minipage}[t]{0.21\columnwidth}\raggedright\strut
TBD\strut
\end{minipage}\tabularnewline
\begin{minipage}[t]{0.46\columnwidth}\raggedright\strut
grep -R ``xyz'' .\strut
\end{minipage} & \begin{minipage}[t]{0.21\columnwidth}\raggedright\strut
TBD\strut
\end{minipage}\tabularnewline
\begin{minipage}[t]{0.48\columnwidth}\raggedright\strut
find . -name ``*.py'' \textbar{} TBD \textbar{}\strut
\end{minipage} & \begin{minipage}[t]{0.48\columnwidth}\raggedright\strut
\strut
\end{minipage}\tabularnewline
\bottomrule
\end{longtable}

\subsection{Help}\label{help}

Find included a number of commands related to manual pages.

\begin{longtable}[]{@{}ll@{}}
\toprule
\begin{minipage}[b]{0.24\columnwidth}\raggedright\strut
Command\strut
\end{minipage} & \begin{minipage}[b]{0.44\columnwidth}\raggedright\strut
Description\strut
\end{minipage}\tabularnewline
\midrule
\endhead
\begin{minipage}[t]{0.24\columnwidth}\raggedright\strut
man \emph{command}\strut
\end{minipage} & \begin{minipage}[t]{0.44\columnwidth}\raggedright\strut
manual page for the \emph{command}\strut
\end{minipage}\tabularnewline
\bottomrule
\end{longtable}

\subsection{Keyboard Shortcuts}\label{keyboard-shortcuts}

These shortcuts will come in handy. Note that many overlap with emacs
short cuts.

\begin{longtable}[]{@{}ll@{}}
\toprule
\begin{minipage}[b]{0.17\columnwidth}\raggedright\strut
Keys\strut
\end{minipage} & \begin{minipage}[b]{0.77\columnwidth}\raggedright\strut
Description\strut
\end{minipage}\tabularnewline
\midrule
\endhead
\begin{minipage}[t]{0.17\columnwidth}\raggedright\strut
Up Arrow\strut
\end{minipage} & \begin{minipage}[t]{0.77\columnwidth}\raggedright\strut
Show the previous command\strut
\end{minipage}\tabularnewline
\begin{minipage}[t]{0.17\columnwidth}\raggedright\strut
Ctrl + z\strut
\end{minipage} & \begin{minipage}[t]{0.77\columnwidth}\raggedright\strut
Stops the current command\strut
\end{minipage}\tabularnewline
\begin{minipage}[t]{0.48\columnwidth}\raggedright\strut
\strut
\end{minipage} & \begin{minipage}[t]{0.48\columnwidth}\raggedright\strut
resume with fg in the foreground\strut
\end{minipage}\tabularnewline
\begin{minipage}[t]{0.48\columnwidth}\raggedright\strut
\strut
\end{minipage} & \begin{minipage}[t]{0.48\columnwidth}\raggedright\strut
resume with bg in the background\strut
\end{minipage}\tabularnewline
\begin{minipage}[t]{0.17\columnwidth}\raggedright\strut
Ctrl + c\strut
\end{minipage} & \begin{minipage}[t]{0.77\columnwidth}\raggedright\strut
Halts the current command\strut
\end{minipage}\tabularnewline
\begin{minipage}[t]{0.17\columnwidth}\raggedright\strut
Ctrl + l\strut
\end{minipage} & \begin{minipage}[t]{0.77\columnwidth}\raggedright\strut
Clear the screen\strut
\end{minipage}\tabularnewline
\begin{minipage}[t]{0.17\columnwidth}\raggedright\strut
Ctrl + a\strut
\end{minipage} & \begin{minipage}[t]{0.77\columnwidth}\raggedright\strut
Return to the start of the command you're typing\strut
\end{minipage}\tabularnewline
\begin{minipage}[t]{0.17\columnwidth}\raggedright\strut
Ctrl + e\strut
\end{minipage} & \begin{minipage}[t]{0.77\columnwidth}\raggedright\strut
Go to the end of the command you're typing\strut
\end{minipage}\tabularnewline
\begin{minipage}[t]{0.17\columnwidth}\raggedright\strut
Ctrl + k\strut
\end{minipage} & \begin{minipage}[t]{0.77\columnwidth}\raggedright\strut
Cut everything after the cursor to a special clipboard\strut
\end{minipage}\tabularnewline
\begin{minipage}[t]{0.17\columnwidth}\raggedright\strut
Ctrl + y\strut
\end{minipage} & \begin{minipage}[t]{0.77\columnwidth}\raggedright\strut
Paste from the special clipboard\strut
\end{minipage}\tabularnewline
\begin{minipage}[t]{0.17\columnwidth}\raggedright\strut
Ctrl + d\strut
\end{minipage} & \begin{minipage}[t]{0.77\columnwidth}\raggedright\strut
Log out of current session, similar to exit\strut
\end{minipage}\tabularnewline
\bottomrule
\end{longtable}

\subsection{.bashrc and .bash\_profile}\label{bashrc-and-.bash_profile}

Usage of a particular command and all the attributes associated with it,
use `man' command. Avoid using `rm -r' command to delete files
recursively. A good way to avoid accidental deletion is to include the
following in your .bash\_profile file:

\begin{verbatim}
alias e=open_emacs
alias rm='rm -i'
alias mv='mv -i' 
alias h='history'
\end{verbatim}

More Information

\url{https://cloudmesh.github.io/classes/lesson/linux/refcards.html}

\subsection{Exercise}\label{exercise}

\begin{description}
\item[Linux.1:]
Familiarize yourself with the commands
\item[Linux.2:]
Find more commands that you find useful and add them to this page.
\item[Linux.3:]
Use the sort command to sort all lines of a file while removing
duplicates.
\end{description}
