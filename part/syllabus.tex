\part{Syllabus}

%%%%%%%%%%%%%%%%%%%%%%%%%%%%%%%%%%%%%%%%%%%%%%%%%%%%%%%%%%%%%%%%%%%%%%
\chapter{Calendar}
%%%%%%%%%%%%%%%%%%%%%%%%%%%%%%%%%%%%%%%%%%%%%%%%%%%%%%%%%%%%%%%%%%%%%%

\begin{description}
\item[Week 1] Jan 12
\item[Week 2] Jan 15
\item[Week 3] Jan 22
\item[Week 4] Jan 29
\item[Week 5] Feb 5
\item[Week 6] Feb 12
\item[Week 7] Feb 19
\item[Week 8] Feb 26
\item[Week 9] Mar 5
\item[Week 10] Mar 11 - Mar 18 Sprinng break starts
\item[Week 11] Mar 19 
\item[Week 12] Mar 26
\item[Week 13] Apr 2
\item[Week 14] Apr 9
\item[Week 15] Apr 16
\item[Week 16] Apr 23
\item[Week 17] Apr 30
\item[Week 18] May 4  (Fri) - Semester ends
\item[Week 19]
\end{description}



%%%%%%%%%%%%%%%%%%%%%%%%%%%%%%%%%%%%%%%%%%%%%%%%%%%%%%%%%%%%%%%%%%%%%%
\chapter{E516: Introduction to Cloud Computing}
%%%%%%%%%%%%%%%%%%%%%%%%%%%%%%%%%%%%%%%%%%%%%%%%%%%%%%%%%%%%%%%%%%%%%%

We present the syllabus for the introduction to Cloud Computing course
taught at Indiana University that uses in part the material presented
in this document. The information includes the section or chapter
number, the name of the chapter or section, and the timeframe when it
is recommended to work through the material and the page number where
to find the material in this document. Please note that we will update
the document throughout the semester thus, pagenumbers will change. 

\section{Course Description}

This course describes the emerging cloud and big data technologies
within the world of distributed intelligent systems where each system
component has internal, and external sources of intelligence that are
subject to collective control. Examples are given from a wide variety
of applications. A project will allow students to dive into practical
issues after they have obtained a theoretical background.

\section{Course pre-requisites}

Python will be used as a Programming Languages. In some cases Java may
also be useful however it's use in class will be marginal. The class
will be using Linux commandline tools. Prior knowledge of Linux is of
advantage but not required.  Studnets are expected to have access to a
computer on which they can execute Linux easily. As the OS
requirements have recently increased we recommend a computer with 8GB
main memory and the ability to run virtualbox and/or containers. If it
turns out your machine is not capable enough we attempt to provide
access to IU linux machines.

\section{Registration Information}

\begin{verbatim}
ENGR-E 516  ENGINEERING CLOUD COMPUTING (3 CR)
       33699 RSTR     11:15A-12:30P   F      I2 150    Von Laszewski G
          Above class taught online
          Above class open to graduates only
          Discussion (DIS)


ENGR-E 516  ENGINEERING CLOUD COMPUTING (3 CR)
       33909 RSTR     ARR             ARR    WB WEB    Von Laszewski G
          This is a 100% online class taught by IU Bloomington. No
          on-campus class meetings are required. A distance education
          fee may apply; check your campus bursar website for more
          information
          Above class open to graduates only
\end{verbatim}


\section{Teaching and learning methods}


\begin{itemize}
\item Lectures
\item Assignments including specific lab activities
\item Final project 
\end{itemize}

\section{Covered Topics}

The class will cover a number of topics as part of tracks that are
executed in parallel throughout the class. Although we assume that at
a graduate course level the communication track has already been
covered elsewhere, we made sure we also include it here in case you dod
not yet have such experiences.

\begin{itemize}

\item {\bf Week 1: Course Overview:} Overview and status of cloud
\item {\bf Week 2-4: Introduction:} Overview and status of cloud
  computing. This includes the creation of REST services for Big Data
\item {\bf Week 6: Cloud Computing:} Basics of Cloud Computing 
\item {\bf Week 7: Containers:} Basics of Containers
\item {\bf Week 8: Virtual Machines:} Introduction to OpenStack
\item {\bf Week 9: Cloud 3.0:} Microservices, Events, Functions  
\item {\bf Week 10-13: Hadoop:} Introduction to Hadoop
\item {\bf Week 13-15: Project:} Delivery of a reproducible
  substantial student project
 
\end{itemize}

\section{Student learning outcomes}

When students complete this course, they should be able to:

\begin{itemize}
\item Have an elementary understanding of issues involved in designing
  and Software Defined Systems.
\item Understand the concepts of Cloud Computing
\item Gain hands-on laboratory experience with several examples.
\item Apply knowledge of mathematics, science, and engineering
\item	Have advanced skills in teamwork with peers.
\end{itemize}

\section{Grading}

\begin{tabular}{lr}
Grade Item	  & Percentage\\
\hline
Assignments	  & 40\% \\
Final Project	& 50\% \\
Participation	& 10\% \\
\hline
\end{tabular}




\section{Representative bibliography}

\begin{enumerate}
\item Cloud Computing for Science and Engineering By Ian Foster and
  Dennis
  B. Gannon\URL{https://mitpress.mit.edu/books/cloud-computing-science-and-engineering}
\item Ansible Tutorials
\item	\url{http://bigdataopensourceprojects.soic.indiana.edu/}
\item The backdrop for course is the ~350 software subsystems
  illustrated at \url{http://hpc-abds.org/kaleidoscope/}
\item	\url{http://cloudmesh.github.io/introduction_to_cloud_computing/}
\item	There are a huge number of web resources

\item (This document) {\bf Handbook of Clouds and Big Data}, Gregor von Laszewski,
  Geoffrey C. Fox, and Judy Qiu, Fall 2017,
  \url{https://tinyurl.com/vonLaszewski-handbook}

\item {\bf Use Cases in Big Data Software and
  Analytics Vol. 1}, Gregor von Laszewski, Fall 2017,
  \url{https://tinyurl.com/cloudmesh/vonLaszewski-i523-v1.pdf}

\item {\bf Use Cases in Big Data Software and
  Analytics Vol. 2}, Gregor von Laszewski, Fall 2017, \url{https://tinyurl.com/cloudmesh/vonLaszewski-i523-v2.pdf}

\item  {\bf Use Cases in Big Data Software and
  Analytics Vol. 3}, Gregor von Laszewski, Fall 2017, 
  \url{https://tinyurl.com/vonLaszewski-projects-v3}

\item (Draft) {\bf Big Data Software Vol 1.}, Gregor von Laszewski, Spring 2017,
\url{https://github.com/cloudmesh/sp17-i524/blob/master/paper1/proceedings.pdf}

\item (Draft) {\bf Big Data Software Vol 2.}, Gregor von Laszewski, Spring 2017,
\URL{https://github.com/cloudmesh/sp17-i524/blob/master/paper2/proceedings.pdf}

\item (Draft) {\bf Big Data Projects}, Gregor von Laszewski, Spring 2017,
\URL{https://github.com/cloudmesh/sp17-i524/blob/master/project/projects.pdf}

\end{enumerate}

\section{Lectures and Lecture Material}\label{S:lectures-516}

\begin{WARNING}
This section will change and be updated throughout the semester
\end{WARNING}


\begin{IU}

To start the class at IU, we require you to watch the following video:

% \slides{Overview}{45}{TBD}{https://drive.google.com/open?id=yyy}
\video{Class Management}{39:54}{Class Management -
  Overview}{https://youtu.be/azCXscVSZQc}

\video{Teaching Cloud and Big Data}{39:54}{Teaching Cloud and Big Data - Overview}{https://youtu.be/sJRMAbFz4hs}
\end{IU}

\subsection{Assignments}

All assignments are summarized in the Section \ref{s:616-assignments}

The current released assignments include:

\WHERE{\YES}{E:616-bigdata-collab}{Due: Jan 29, 2018}
\WHERE{\YES}{E:616-new-tech}{Due: Jan 29, 2018}
\WHERE{\YES}{E:616-new-tech-abstract}{Due: Jan 29, 2018}


\subsection{Communication Track}

\WHERE{\YES}{S:lectures-516}{Week 1}
\WHERE{\YES}{C:doc}{Week 1}
\WHERE{\YES}{S:plagiarism}{Week 1}
\WHERE{\YES}{S:results}{Week 1}

\WHERE{\YES}{S:markdown}{Week 1}
\WHERE{\YES}{C:emacs}{Week 2}

\WHERE{\YES}{S:writing}{Week 3}
\WHERE{\YES}{C:latex}{Week 3}
\WHERE{\YES}{C:bibtex}{Week 4}

\begin{description}
\item[Evaluation Paper1:] Create a paper about a cloud technology with
  our give class template in the git repository. If a paper is
  plagiarized you will receive an ``F'' and it is reported based on
  University policies.
\end{description}


\subsection{Theory Track}

\paragraph{Introduction to Cloud Computing}

\WHERE{\YES}{s:cloud-fundamentals-a}{Released, Feb 11, 2018}
\WHERE{\YES}{s:cloud-fundamentals-b}{Released, Feb 11, 2018}
\WHERE{\YES}{s:cloud-fundamentals-c}{Released, Feb 11, 2018}
\WHERE{\YES}{s:cloud-fundamentals-d}{Released, Feb 11, 2018}
\WHERE{\YES}{s:cloud-fundamentals-e}{Released, Feb 11, 2018}
\WHERE{\YES}{s:cloud-fundamentals-f}{Released, Feb 11, 2018}
\WHERE{\YES}{s:cloud-fundamentals-g}{Released, Feb 11, 2018}
\WHERE{\YES}{s:cloud-fundamentals-h}{Released, Feb 11, 2018}
\WHERE{\YES}{s:cloud-fundamentals-i}{Released, Feb 11, 2018}
\WHERE{\YES}{s:cloud-fundamentals-j}{Released, Feb 11, 2018}
\WHERE{\YES}{s:cloud-fundamentals-k}{Released, Feb 11, 2018}
\WHERE{\YES}{s:cloud-fundamentals-m}{Released, Feb 11, 2018}
\WHERE{\YES}{s:cloud-fundamentals-n}{Released, Feb 11, 2018}
\WHERE{\YES}{s:cloud-fundamentals-o}{Released, Feb 11, 2018}
\WHERE{\YES}{s:cloud-fundamentals-p}{Released, Feb 11, 2018}
\WHERE{\YES}{s:cloud-fundamentals-q}{Released, Feb 11, 2018}
\WHERE{\YES}{s:cloud-fundamentals-r}{Released, Feb 11, 2018}
\WHERE{\YES}{s:cloud-fundamentals-s}{Released, Feb 11, 2018}
\WHERE{\YES}{s:cloud-fundamentals-t}{Released, Feb 11, 2018}
\WHERE{\YES}{s:cloud-fundamentals-u}{Released, Feb 11, 2018}



\begin{itemize}
\item IaaS - OpenStack
\item PaaS - Hadoop
\item SaaS - SaaS with REST
\end{itemize}

\begin{description}
\item[Evaluation Paper1:] Create a paper about a cloud technology with
  our give class template in the git repository. If a paper is
  plagiarized you will receive an ``F'' and it is reported based on
  University policies. The paper is in a directory called paper1. All
  images are in the directory paper1/images, the report is in
  report.tex, the content is in content.tex. It follows the template
  we provided. Submission of report.pdf is not allowed. We will create
  the report and check completness that way.
\end{description}

\subsection{Programming Track}

\subsubsection{Development Environment}

\WHERE{\YES}{S:virtual-box}{Week 2}
\WHERE{\YES}{C:linux}{Week 2}
\WHERE{\NO}{C:ssh}{Week 2}
\WHERE{\NO}{C:github}{Week 3}


\begin{description}
\item[Evaluation Experiment 1:] Create a virtual machine and take a
  photo with your laptop or computer and the virtual box running on
  the screen. Showcase the virtual box interface and in non full
  screen mode at the same time the operating system you run. We
v  recommend yo use Ubuntu.
\end{description}

\subsubsection{Python}

\WHERE{\YES}{C:python}{Week 3 - 4}
\WHERE{\YES}{C:python-install}{Week ?}
\WHERE{\YES}{C:python-language}{Week ?}
\WHERE{\YES}{C:python-lib}{Week ?}
\WHERE{\NO}{C:python-cmd5}{Week ?}

\subsubsection{Cloud}

\WHERE{\YES}{s:rest-intro}{Week 2}
\WHERE{\YES}{s:eve-intro}{Week 2}
\WHERE{\YES}{c:swagger-codegen}{Week 3}


\paragraph{Assignments}

Develop a Big Data related REST service with swagger.

Chose two of them so you can see the difference between the APIs:

\begin{itemize}
\item Build a Multi cloud interface to OpenStack
\item Build a Multi cloud interface to AWS
\item Build a Multi cloud interface to Azure
\item Build a Multi cloud interface to Docker/Kubernetes
\end{itemize}


\begin{description}
\item[Evaluation Experiment 1:] Create a program in python that
  identifies a termination criteria for the Secchi disk
  problem. E.g. at what image can we no longer see the disk?
  Describe your solution in md and submit to the git repository in a
  directory called {\em secchi}. The program is called secchi.py, the
  description is in README.md. It uses cmd5 for creating a command
  shell that can load the data and analyze it. 
\item[Evaluation Assignment 1:] Create two a REST service for Big Data
  with swagger and provide a python implementation from the exported
  python code to provide real functionality.
\item[Evaluation Assignment 2:] Develop cloudmesh extensions to access
  cloud services from 2 different providers.
\end{description}

\subsubsection{Chameleon Cloud and OpenStack}

\WHERE{\YES}{C:chameleon}{Week 4}
\WHERE{\YES}{C:cc-charge}{Week 4}
\WHERE{\YES}{C:cc-hardware}{Week 4}
\WHERE{\YES}{C:cc-start}{Week 4}
\WHERE{\YES}{C:cc-horizon}{Week 5}
\WHERE{\YES}{C:cc-guide}{Week 4}
\WHERE{\NO}{C:cc-heat}{Optional}
\WHERE{\NO}{C:cc-baremetal}{Optional}


%%%%%%%%%%%%%%%%%%%%%%%%%%%%%%%%%%%%%%%%%%%%%%%%%%%%%%%%%%%%%%%%%%%%%%
\chapter{E616/I524: Advanced Cloud Computing}
%%%%%%%%%%%%%%%%%%%%%%%%%%%%%%%%%%%%%%%%%%%%%%%%%%%%%%%%%%%%%%%%%%%%%%

This class is also known under the name {\em I524 Big Data
Applications and Open Source Software.}
 

We present the syllabus for the E616 course taught at Indiana
University that uses in part the material presented in this
document. The information includes the section or chapter number, the
name of the chapter or section, and the timeframe when it is
recommended to work through the material and the page number where to
find the material in this document. Please note that we will update
the document throughout the semester thus, page numbers will change.

\section{Course Description }

This course describes Cloud 3.0 in which DevOps, Microservices, and
Function as a Service is added to basic cloud computing. The
discussion is centered around the Apache Big Data Stack and a major
student project aimed at demonstrating integration of cloud
capabilities.

\section{Course pre-requisites}

Python will be used as a Programming Languages. It is expected that
you know a programming language. ENGR-E516 or an introduction to cloud
computing is recommended. Studnets are expected to have access to a
computer on which they can execute Linux easily. As the OS
requirements have recently increased we recommend a computer with 8GB
main memory and the ability to run virtualbox and/or containers. If it
turns out your machine is not capable enough we attempt to provide
access to IU linux machines.

\section{Course Registration}

I524 and E616 are identical. In this courses we will be focussing on
Advanced Cloud COmputing and Big Data Applications and Analytics. 

Intelligent Systems Engeneering Residential:

\begin{verbatim}
	ENGR-E 616  ADVANCED CLOUD COMPUTING (3 CR)
              ***** RSTR     ARR             ARR    ARR       Von Laszewski G
                 Above class taught online
                 Above class open to graduates only
                 Discussion (DIS)
              33697 RSTR     09:30A-10:45A   F      I2 150    Von Laszewski G
\end{verbatim}

Info Residential:

\begin{verbatim}
INFO-I 524  BIG DATA SOFTWARE AND PROJECTS (3 CR) 
              ***** RSTR     ARR             ARR    ARR       Von Laszewski G          
                 Above class open to graduates only
                 Above class taught online
                 Discussion (DIS)
              13053 RSTR     09:30A-10:45A   F      I2 150    Von Laszewski G  
\end{verbatim}

Info Online:

\begin{verbatim}        
        INFO-I 524  BIG DATA SOFTWARE AND PROJECTS (3 CR)
              13054 RSTR     ARR             ARR    ARR       Von Laszewski G          
                 Above class open to graduates only
                 This is a 100% online class taught by IU Bloomington. No
                 on-campus class meetings are required. A distance education
                 fee may apply; check your campus bursar website for more
                 information
\end{verbatim}


\section{Teaching and learning methods}

\begin{itemize}
\item	Lectures
\item	Assignments including specific lab activities
\item	Final project
\item Class will use software mainly written in Python
  and Linux Shell.
\end{itemize}


\section{Covered Topics}

The class will cover a number of topics as part of tracks that are
executed in parallel throughout the class. Although we assume that at
a graduate course level the communication track has already been
covered elsewhere, we made sure we also include it here in case you did
not yet have such experiences.

Please note that we may change the order of the lectures. This is our
current plan

\begin{itemize}

\item {\bf Week 1-2: Introduction} Overview and status of cloud
  computing
\item {\bf Week 2: Cloud REST Services} Introduction to REST services
  to develop cloud services.
\item {\bf Weeb 3: Big Data Reference Architecture:} Using REST
  services to implement Big Data Services.
\item {\bf Week 4: Cloud and Big Data Applications:} Introduction to
  the Apache Big Data Stack. Selective presentation of the members of
  the Big Data Application set
\item {\bf 5: New Cloud Big Data Application Technologies:} Students will
  explore the Apache Web page and report on them. Continuation of
  earlier assigned homwework, including paper.
\item {\bf Week 6: DevOps:} Introduction to DevOps with Dokerfile and
  ansible
\item {\bf Week 7-14: Building a Kubernetes Cluster:} Residential
  students will be building a kubernetes cluster with 5
  servers. Online students could do that also, but need to simulate
  the cluster instead of using real hardware as it requires physical
  access to the hardware.

\item {\bf Week 7: Virtual Machines:} Introduction to OpenStack
\item {\bf Week 8-9: Containers:} Introduction to Container technology
\item {\bf Week 10: Advanced Containers:} Building clusters with containers.
\item {\bf Week 11: Cloud 3.0:} Microservices, Events, Functions  
\item {\bf Week 12-14: Project:} Delivery of a reproducible substantial
  student project (you are allowed to substantially enhance a project
  that you started from other cloud classes you took with us. Please ask.)
\end{itemize}

\section{Student learning outcomes}

When students complete this course, they should be able to:

\begin{itemize}
\item Have an advanced understanding of issues involved in designing
  and applying modern cloud technologies using the latest
  developments.
\item	Gain hands-on laboratory experience.
\item	Understand the Apache Big Data Software Stack.
\item	Apply knowledge of mathematics, science, and engineering.
\item Understand research challenges and important application areas
  of clouds
\item	Have advanced skills in teamwork with peers.
\item Be able to use DevOps technologies.
\end{itemize}

\section{Grading}


\begin{tabular}{lr}
Grade Item	  & Percentage\\
\hline
Assignments	  & 40\% \\
Final Project	& 50\% \\
Participation	& 10\% \\
\hline
\end{tabular}



\section{Representative bibliography}

\begin{enumerate}
\item Cloud Computing for Science and Engineering By Ian Foster and
  Dennis
  B. Gannon\URL{https://mitpress.mit.edu/books/cloud-computing-science-and-engineering}
\item	Machine to machine protocols \url{https://en.wikipedia.org/wiki/MQTT}
\item	Cloud software systems \url{http://hpc-abds.org/kaleidoscope/}
\item	Software Defined Networks \url{https://en.wikipedia.org/wiki/Software-defined_networking}
\item	There are a huge number of other web resources

\item (This document) {\bf Handbook of Clouds and Big Data}, Gregor von Laszewski,
  Geoffrey C. Fox, and Judy Qiu, Fall 2017,
  \url{https://tinyurl.com/vonLaszewski-handbook}

\item {\bf Use Cases in Big Data Software and
  Analytics Vol. 1}, Gregor von Laszewski, Fall 2017,
  \url{https://tinyurl.com/cloudmesh/vonLaszewski-i523-v1.pdf}

\item {\bf Use Cases in Big Data Software and
  Analytics Vol. 2}, Gregor von Laszewski, Fall 2017, \url{https://tinyurl.com/cloudmesh/vonLaszewski-i523-v2.pdf}

\item  {\bf Use Cases in Big Data Software and
  Analytics Vol. 3}, Gregor von Laszewski, Fall 2017,   
  \url{https://tinyurl.com/vonLaszewski-projects-v3}

\item (Draft) {\bf Big Data Software Vol 1.}, Gregor von Laszewski, Spring 2017,
\url{https://github.com/cloudmesh/sp17-i524/blob/master/paper1/proceedings.pdf}

\item (Draft) {\bf Big Data Software Vol 2.}, Gregor von Laszewski, Spring 2017,
\URL{https://github.com/cloudmesh/sp17-i524/blob/master/paper2/proceedings.pdf}

\item (Draft) {\bf Big Data Projects}, Gregor von Laszewski, Spring 2017,
\URL{https://github.com/cloudmesh/sp17-i524/blob/master/project/projects.pdf}

\end{enumerate}


\section{Lectures and Lecture Material}\label{S:lectures-616}

\begin{WARNING}
This section will change and be <updated throughout the semester
\end{WARNING}

\begin{IU}

To start the class at IU, we require you to watch the following video:

% \slides{Overview}{45}{TBD}{https://drive.google.com/open?id=yyy}
\video{Class Management}{39:54}{Class Management -
  Overview}{https://youtu.be/azCXscVSZQc}

\video{Teaching Cloud and Big Data}{39:54}{Teaching Cloud and Big Data - Overview}{https://youtu.be/sJRMAbFz4hs}

\end{IU}

\subsection{Assignments}

All assignments are summarized in the Section \ref{s:616-assignments}

The current released assignments include:

\WHERE{\YES}{a:616-bio}{Exercise \ref{E:616-bio-piazza},
  \ref{E:616-bio-googledocs}, \ref{E:616-iu-google} Due: Jan
  22, 2018}
\WHERE{\YES}{E:616-bigdata-collab}{Due: Jan 29, 2018}
\WHERE{\YES}{E:616-new-tech}{Due: Jan 29, 2018}
\WHERE{\YES}{E:616-new-tech-abstract}{Due: Jan 29, 2018}

\subsection{Communication Track}

\WHERE{\YES}{S:lectures-616}{Week 1}
\WHERE{\YES}{C:doc}{Week 1}
\WHERE{\YES}{S:plagiarism}{Week 1}
\WHERE{\YES}{S:results}{Week 1}

\WHERE{\YES}{S:markdown}{Week 1}
\WHERE{\YES}{C:emacs}{Week 2}

\WHERE{\YES}{S:writing}{Week 3}
\WHERE{\YES}{C:latex}{Week 3}
\WHERE{\YES}{C:bibtex}{Week 4}

\begin{description}
\item[Evaluation Paper1:] Create a paper about a cloud technology with
  our give class template in the git repository. If a paper is
  plagiarised you will receive an ``F'' and it is reported based on
  University policies.
\end{description}


\subsection{Theory Track}

\paragraph{Introduction to Cloud Computing}

\WHERE{\YES}{s:cloud-fundamentals-a}{Released, Feb 11, 2018}
\WHERE{\YES}{s:cloud-fundamentals-b}{Released, Feb 11, 2018}
\WHERE{\YES}{s:cloud-fundamentals-c}{Released, Feb 11, 2018}
\WHERE{\YES}{s:cloud-fundamentals-d}{Released, Feb 11, 2018}
\WHERE{\YES}{s:cloud-fundamentals-e}{Released, Feb 11, 2018}
\WHERE{\YES}{s:cloud-fundamentals-f}{Released, Feb 11, 2018}
\WHERE{\YES}{s:cloud-fundamentals-g}{Released, Feb 11, 2018}
\WHERE{\YES}{s:cloud-fundamentals-h}{Released, Feb 11, 2018}
\WHERE{\YES}{s:cloud-fundamentals-i}{Released, Feb 11, 2018}
\WHERE{\YES}{s:cloud-fundamentals-j}{Released, Feb 11, 2018}
\WHERE{\YES}{s:cloud-fundamentals-k}{Released, Feb 11, 2018}
\WHERE{\YES}{s:cloud-fundamentals-m}{Released, Feb 11, 2018}
\WHERE{\YES}{s:cloud-fundamentals-n}{Released, Feb 11, 2018}
\WHERE{\YES}{s:cloud-fundamentals-o}{Released, Feb 11, 2018}
\WHERE{\YES}{s:cloud-fundamentals-p}{Released, Feb 11, 2018}
\WHERE{\YES}{s:cloud-fundamentals-q}{Released, Feb 11, 2018}
\WHERE{\YES}{s:cloud-fundamentals-r}{Released, Feb 11, 2018}
\WHERE{\YES}{s:cloud-fundamentals-s}{Released, Feb 11, 2018}
\WHERE{\YES}{s:cloud-fundamentals-t}{Released, Feb 11, 2018}
\WHERE{\YES}{s:cloud-fundamentals-u}{Released, Feb 11, 2018}

\paragraph{Not yet released Lectures, they will be updated}

\WHERE{\NO}{S:o-workflow}{Week 1}
\WHERE{\NO}{S:o-application}{Week 2}
\WHERE{\NO}{S:o-programming}{Week 3}
\WHERE{\NO}{S:o-streams}{Week 4}
\WHERE{\NO}{S:o-prg-model}{Week 5}
\WHERE{\NO}{S:o-process-communication}{Week 5}
\WHERE{\NO}{S:o-db-memory}{Week 6}
\WHERE{\NO}{S:o-db-object}{Week 6}
\WHERE{\NO}{S:o-Tools}{Week 7}
\WHERE{\NO}{S:o-sql}{Week 6}
\WHERE{\NO}{S:o-NoSQL}{Week 6}
\WHERE{\NO}{S:o-file-management}{Week 8}
\WHERE{\NO}{S:o-data-transport}{Week 8}
\WHERE{\NO}{S:o-cluster}{Week 9}
\WHERE{\NO}{S:o-file-systems}{Week 8}
\WHERE{\NO}{S:o-interoperability}{Week 3}
\WHERE{\NO}{S:o-DevOps}{Week 10}
\WHERE{\NO}{S:o-hypervisors}{Week 11}
\WHERE{\NO}{S:o-cross-cutting-functions}{Week 12}
\WHERE{\NO}{S:o-protocols}{Week 13}
\WHERE{\NO}{S:o-todo}{Week 4}


\begin{description}
\item[Evaluation Paper1:] Create a paper about a cloud technology with
  our give class template in the git repository. If a paper is
  plagiarized you will receive an ``F'' and it is reported based on
  University policies. The paper is in a directory called paper1. All
  images are in the directory paper1/images, the report is in
  report.tex, the content is in content.tex. It follows the template
  we provided. Submission of report.pdf is not allowed. We will create
  the report and check completness that way.
\end{description}

\subsection{Programming Track}

\subsubsection{Development Environment}

\WHERE{\YES}{S:virtual-box}{Week 2}
\WHERE{\YES}{C:linux}{Week 2}
\WHERE{\NO}{C:ssh}{Week 3}
\WHERE{\NO}{C:github}{Week 3}


\begin{description}
\item[Evaluation Experiment 1:] Create a virtual machine and take a
  photo with your laptop or computer and the virtual box running on
  the screen. Showcase the virtual box interface and in non full
  screen mode at the same time the operating system you run. We
  recommend yo use Ubuntu.
\end{description}

\subsubsection{Python}

\WHERE{\YES}{C:python}{Week 3 - 4}
\WHERE{\YES}{C:python-install}{Week ?}
\WHERE{\YES}{C:python-language}{Week ?}
\WHERE{\YES}{C:python-lib}{Week ?}
\WHERE{\NO}{C:python-cmd5}{Week ?}

\WHERE{\NO}{c:numpy}{}
\WHERE{\NO}{c:scipy}{} 
\WHERE{\NO}{c:opencv}{}
\WHERE{\NO}{c:secchi-disk}{}

\begin{description}
\item[Evaluation Experiment 1:] Create a program in python that
  identifies a termination criteria for the Secchi disk
  problem. E.g. at what image can we no longer see the disk?
  Describe your solution in md and submit to the git repository in a
  directory called {\em secchi}. The program is called secchi.py, the
  description is in README.md. It uses cmd5 for creating a command
  shell that can load the data and analyze it. 
\end{description}

\subsection{DevOps}

\begin{itemize}
\item ssh for DevOps
\item Ansible
\item Dokerfile
\end{itemize}

\subsection{Cloud}

\WHERE{\YES}{s:rest-intro}{Week 2}
\WHERE{\YES}{s:eve-intro}{Week 2}
\WHERE{\YES}{c:swagger-codegen}{Week 3}

\begin{itemize}
\item OpenAPI REST Service
\item OpenAPI Big Data Services
\item Deploy cloud services with Ansible
\end{itemize}

\subsubsection{Chameleon Cloud and OpenStack}

\WHERE{\YES}{C:chameleon}{Week 4}
\WHERE{\YES}{C:cc-charge}{Week 4}
\WHERE{\YES}{C:cc-hardware}{Week 4}
\WHERE{\YES}{C:cc-start}{Week 4}
\WHERE{\YES}{C:cc-horizon}{Week 5}
\WHERE{\YES}{C:cc-guide}{Week 4}
\WHERE{\NO}{C:cc-heat}{Optional}
\WHERE{\NO}{C:cc-baremetal}{Optional}


\section{Containers}

\begin{itemize}
\item Introduction to Docker - Docker File
\item Advanced Docker - Doker Swarm -  Kubernetes
\item Deploy cloud services with Doker/Kubernetes
\item Build a Raspberry Pi based Kubernetes cluster with 5 nodes
  (residential student can work on this in teams up to 5 students, we
  have 50 Raspberry Pi's)
\item Benchmark a 144 node Raspberry PI kubernetes cluster after
  deploying it (open class work)
\end{itemize}

%%%%%%%%%%%%%%%%%%%%%%%%%%%%%%%%%%%%%%%%%%%%%%%%%%%%%%%%%%%%%%%%%%%%%%
\chapter{E222: Intelligent Systems Engineering II}
%%%%%%%%%%%%%%%%%%%%%%%%%%%%%%%%%%%%%%%%%%%%%%%%%%%%%%%%%%%%%%%%%%%%%%

\section{Course Description}

In this undergraduate course students will be familiarized with
different specific applications and implementations of intelligent
systems and their use in desktop and cloud solutions.

\section{Course pre-requisites}

One programming language, Intelligent Systems Engineering I or equivalent

\section{Course Registration}


\begin{verbatim}
ENGR-E 222  INTELLIGENT SYSTEMS II (3 CR)
              ***** RSTR     02:30P-03:45P   TR     Luddy 4063 Fox
                 Laboratory (LAB)
        E 222 : P - ENGR-E 221
              31434 RSTR     05:45P-06:35P   R      Luddy 4063 Fox
                 Above class for  Intelligent Systems Engineering students

\end{verbatim}        


\section{Teaching and learning methods}


\begin{itemize}
\item Lectures
\item Assignments including specific lab activities
\item Final project 
\end{itemize}

\section{Covered Topics}

The topics covered in thie class include.

\begin{itemize}
\item Introduction to REST: Theory and Practice - develop a REST service
\item Introduction to Clouds: Theory and Practice - create via a
  program virtual machines and start on them the REST service
\item Introduction to Kubernetes: Theory and Practice - create a
  container that runs a REST service
\item Introduction to Advanced AI: Integrate your AI engine into a
  REST service and run on a cloud and in Kubernetes 
\item Introduction to Hadoop: Theory and Practice - Run Hadoop in a
  container; run hadoop on a futuresystems cluster
\item Edge Computing: Theory and Practice - Integrate Sensordata into
  Cloud Services via REST and MQTT
\end{itemize}


\section{Student learning outcomes}

When students complete this course, they should be able to:

\begin{itemize}
\item Have an elementary understanding of issues involved in Cloud
  Computing as part of the intelligent systems effort.
\item Gain hands-on laboratory experience with several examples.
\item Apply knowledge of mathematics, science, and engineering
\item Understand research challenges and important issues with
  Software Defined Systems.
\item	Have advanced skills in teamwork with peers.
\item Have theoretical and practical knowledge about REST, Clouds,
  Containers, and Edge Computing.
\end{itemize}


\section{Grading}

\begin{tabular}{lr}
  Grade Item	  & Percentage\\
  \hline
  Assignments	  & 60\%\\
  Final Project	& 30\%\\
  Participation	& 10\%\\
  \hline
\end{tabular}


\section{Representative bibliography}

\begin{enumerate}

\item Cloud Computing for Science and Engineering By Ian Foster and
  Dennis
  B. Gannon\URL{https://mitpress.mit.edu/books/cloud-computing-science-and-engineering}


\item (This document) {\bf Handbook of Clouds and Big Data}, Gregor von Laszewski,
  Geoffrey C. Fox, and Judy Qiu, Fall 2017,
  \url{https://tinyurl.com/vonLaszewski-handbook}

\item {\bf Use Cases in Big Data Software and
  Analytics Vol. 1}, Gregor von Laszewski, Fall 2017,
  \url{https://tinyurl.com/cloudmesh/vonLaszewski-i523-v1.pdf}

\item {\bf Use Cases in Big Data Software and
  Analytics Vol. 2}, Gregor von Laszewski, Fall 2017, \url{https://tinyurl.com/cloudmesh/vonLaszewski-i523-v2.pdf}

\item  {\bf Use Cases in Big Data Software and
  Analytics Vol. 3}, Gregor von Laszewski, Fall 2017, 
  \url{https://tinyurl.com/vonLaszewski-projects-v3}


\item (Draft) {\bf Big Data Software Vol 1.}, Gregor von Laszewski, Spring 2017,
\url{https://github.com/cloudmesh/sp17-i524/blob/master/paper1/proceedings.pdf}

\item (Draft) {\bf Big Data Software Vol 2.}, Gregor von Laszewski, Spring 2017,
\URL{https://github.com/cloudmesh/sp17-i524/blob/master/paper2/proceedings.pdf}

\item (Draft) {\bf Big Data Projects}, Gregor von Laszewski, Spring 2017,
\URL{https://github.com/cloudmesh/sp17-i524/blob/master/project/projects.pdf}

\end{enumerate}

\section{Lectures and Lecture Material}


\subsection{Proposed Weekly Agenda}

\begin{WARNING}

The following information is related to our initial plans of
presenting material for E222 by Week. Please note that this could
change. These are proposed topics and we will update the Handbook
accordingly from week to week.

\end{WARNING}

\subsection{Week 1. Administration}

We have explained how to use piazza which we will be using for class
communication. Students that missed that lecture are responible for
working with TAs to catch up.

\subsection{Weeks 1, 2 and 4. Introdcution to Cloud Computing}

A number of presentations have been posted to introduce you to ths
Class. The following lectures are the video lectures from weeks 1-2.

\WHERE{\YES}{s:cloud-fundamentals-a}{Released, Feb 1, 2018}
\WHERE{\YES}{s:cloud-fundamentals-b}{Released, Feb 1, 2018}
\WHERE{\YES}{s:cloud-fundamentals-c}{Released, Feb 1, 2018}
\WHERE{\YES}{s:cloud-fundamentals-d}{Released, Feb 1, 2018}
\WHERE{\YES}{s:cloud-fundamentals-e}{Released, Feb 1, 2018}
\WHERE{\YES}{s:cloud-fundamentals-f}{Released, Feb 1, 2018}
\WHERE{\YES}{s:cloud-fundamentals-g}{Released, Feb 1, 2018}
\WHERE{\YES}{s:cloud-fundamentals-h}{Released, Feb 1, 2018}
\WHERE{\YES}{s:cloud-fundamentals-i}{Released, Feb 1, 2018}
\WHERE{\YES}{s:cloud-fundamentals-j}{Released, Feb 1, 2018}
\WHERE{\YES}{s:cloud-fundamentals-k}{Released, Feb 1, 2018}
\WHERE{\YES}{s:cloud-fundamentals-m}{Released, Feb 1, 2018}
\WHERE{\YES}{s:cloud-fundamentals-n}{Released, Feb 1, 2018}
\WHERE{\YES}{s:cloud-fundamentals-o}{Released, Feb 1, 2018}
\WHERE{\YES}{s:cloud-fundamentals-p}{Released, Feb 1, 2018}
\WHERE{\YES}{s:cloud-fundamentals-q}{Released, Feb 1, 2018}
\WHERE{\YES}{s:cloud-fundamentals-r}{Released, Feb 1, 2018}
\WHERE{\YES}{s:cloud-fundamentals-s}{Released, Feb 1, 2018}
\WHERE{\YES}{s:cloud-fundamentals-t}{Released, Feb 1, 2018}
\WHERE{\YES}{s:cloud-fundamentals-u}{Released, Feb 1, 2018}

This introduction has been continued in Week 4 of the class by
Geoffrey posting additional lectures.

\TODO{Tyler: include the links to Geoffreys lectures he presented, didn't see the slides presented on Thursday Feb 1,2018 about parallel computing, Amdahls Law I beleive was the first slide}


\subsection{Week 3. REST for Cloud computing }

We will be starting the class with introducing you to REST services
that provide a foundation for setting up services in the cloud and to
intercat with these services. As part of this class we will be
revisiting the REST services and use them to deploy them on a cloud as
well as develop our own AI based rest services in the second half of
the class. 

To get you started you need to read the follwoing sections:

An introduction to rest is provided in Section \ref{c:rest}. It also
provides a video recording of the material that was presented in
class.

\WHERE{\YES}{s:rest-intro}{}

Next, we present you with information on how to easily create a rest
service with FlaskRESTFUL a libraruy that makes the creation of web
services in python useing an object oriented approach easy. A Lab was
held that introduced you to developing such a service

\WHERE{\YES}{s:rest-flask}{}

A number of contributions from students have since been made including
the development effort for this lab. Links to this work can be found
at

\URL{https://github.com/cloudmesh-community/hid-sp18-505/tree/master/rest}
\URL{https://github.com/cloudmesh-community/hid-sp18-409/tree/master/rest}

Next, we introduced you to Python Eve which ia a framework that allows
you to define rest services with schemas. This is important as it
first allows you to easily define them while having just to do a very
minimal set of programming. Second, it allows you via Eve to make sure
you define a well designed data model that you can communicate to the
users of the REST service.

\WHERE{\YES}{s:eve-intro}{}

In the Lab that may spawn multiple weeks you will be developing a
Flask or Eve rest service.  The REST service retrieves information
form your computer and exposes it to the client. You can chose what
you like to present, but we want that all students in class do a
different information. TAs are working with you which information you
expose. A detailed Assignment is posted and coordinated in piazza
about this. YOu will be writing two services. One that uses flask
features, the other one that uses a database schema using Eve. The
entire class can openly collaborate with each other on this task. The
code is to be checcked in into your github repository. Information of
interrest include memory available of the computer, cpu type and so
on. This is not read from a text file but life queried.

\TODO{Tyler:Describe how each
  student gets a unique assignment. Coordinate that assignment.}

The link to the assignmnet in piazza is below, directions for each student will come out the week of Febuary 4, 2018. 
\URL{https://piazza.com/class/jc9dcfnbi045kv?cid=27}


\subsection{Weeks 2-3. Setting up your development environment}

It is important that you have a development environment to conduct the
class assignments. We recommend that you use virtual box and use
ubuntu. We have provided an extensive set of material for you to
achieve this. Please consult additional resources form the Web

The material includes:

\TODO{Tyler: please include links to the sections}

\begin{itemize}
\item Install virtual box

\item Install ubuntu on the virtual box

\item \WHERE{\YES}{S:virtual-box}{Week 2}

\item Install an editor to develop python programs. We recommend
  pyCharm or simply emacs.

\item \WHERE{\YES}{C:emacs}{Weeks 2-3}

\item Using pyenv for multi version python installs

\item \WHERE{\YES}{S:managing-multiple-python-versions-with-pyenv}{Weeks 2-3}

\end{itemize}

Technologies covered:	piazza,	git,	pycharm,	virtualbox,	pyenv,	python

Exersise: Continue to work on REST service.

\subsection{Week 5. Introduction to simple Containers}

We will be providing an introduction to containers and container
technologies. Excercises will include to run the REST services that we
developed earlier to start in containers and utilize them.

\WHERE{\YES}{s:motivation-microservices}{Released Feb 11, 2018}
\WHERE{\YES}{s:motivation-serverless}{Released Feb 11, 2018}

\subsection{Week 6. Introduction to Container Clusters}

We will be providing you with an introduction on how to not use one
server, but multiple servers to run containers on. This includes
docker swarm, kubernetes, (maybe mesos if time allows).
Exercises include the deployment of minikube that enables you to run
kubernetes on your computers. Alternatively access to a docker swarm
cluster may be provided.

\WHERE{\YES}{s:motivation-docker}{Released Feb 11, 2018}
\WHERE{\YES}{s:motivation-docker-kubernetes}{Released Feb 11, 2018}

\subsection{Week 7. Map Reduce}

In this section we will introduce you to the concept of Map reduce. We
will discuss systems such as Hadoop and Spark and how they differ. You
will be deploying via a container hadoop on your machine and use it to
gain hands on experience.

\subsection{Week 8. Overview of Cloud Services}

We will be introducing you how to use Cloud services offored via a
number of Cloud providers. Topics covered include: overview of AWS,
overview of Openstack, libcloud, and boto

\subsection{Week 9. Multi cloud environments}

We will teach you how to create a multi cloud shell while leveraging
an abstract programming interface to easily switch betwen multiple
clouds. You can practically participate in helping to develop
interfaces to AWS, Azure, and OpenStack. As you have also worked with
containers, you can develop such interfaces also for containers
including frameworks such as kubernetes. We will be using libcloud to
simplify the abstraction.
	
\subsection{Week 10. Cloud Data and Applications}

We will cover a number of Application examples for Cloud computing. In
the second part we will focus on CLoud Data Services and how we access
data on the cloud. Exercises will include moving data between data
services. THis includes your own computer, box and google which both
are offered at IU.


\subsection{Other weeks}

All excersises in these weeks will develop REST services that {\em
  expose} machine leraning algorithms as service. Data will be either
passed along directly through parameters to the call, or on case of
large data a URL to the data source. The lessons from the previous
weeks will be helping you to achive this. It is not sufficient to just
run the algorithms, but you must be integrating them into a REST service.

Other weeks are not yet included here but will cover Artificial
Intelligence.

\subsection{Assignments}

All assignments are summarized in the Section \ref{s:e222-assignments}
but are posted earlier in piazza.com in the pinned section.


\subsection{Communication Track}

\WHERE{\YES}{C:doc}{Week 1}
\WHERE{\YES}{S:plagiarism}{Week 1}
\WHERE{\YES}{S:results}{Week 1}

\WHERE{\YES}{S:markdown}{Week 1}
\WHERE{\YES}{C:emacs}{Week 2}

\WHERE{\YES}{S:writing}{Week 3}
\WHERE{\YES}{C:latex}{Week 3}
\WHERE{\YES}{C:bibtex}{Week 4}

\begin{description}
\item[Evaluation Paper1:] Create a paper about a cloud technology with
  our give class template in the git repository. If a paper is
  plagiarised you will receive an ``F'' and it is reported based on
  University policies.
\end{description}

\subsection{Theory Track}

\begin{itemize}
\item IaaS - OpenStack
\item PaaS - Hadoop
\item SaaS - SaaS with REST
\end{itemize}

\begin{description}
\item[Evaluation Paper1:] Create a paper about a cloud technology with
  our give class template in the git repository. If a paper is
  plagiarized you will receive an ``F'' and it is reported based on
  University policies. The paper is in a directory called paper1. All
  images are in the directory paper1/images, the report is in
  report.tex, the content is in content.tex. It follows the template
  we provided. Submission of report.pdf is not allowed. We will create
  the report and check completness that way.
\end{description}

\subsection{Programming Track}

\subsubsection{Development Environment}

\WHERE{\YES}{S:virtual-box}{Week 2}
\WHERE{\YES}{C:linux}{Week 2}
\WHERE{\NO}{C:ssh}{Week 3}
\WHERE{\NO}{C:github}{Week 3}

\begin{comment}
\begin{description}
\item[Evaluation Experiment 1:] Create a virtual machine and take a
  photo with your laptop or computer and the virtual box running on
  the screen. Showcase the virtual box interface and in non full
  screen mode at the same time the operating system you run. We
  recommend yo use Ubuntu.
\end{description}
\end{comment}


\subsubsection{Python}

\WHERE{\YES}{C:python}{Week 3 - 4}
\WHERE{\YES}{C:python-install}{Week 2}
\WHERE{\YES}{C:python-language}{Week 2}
\WHERE{\YES}{C:python-lib}{Week 3}
\WHERE{\NO}{C:python-cmd5}{Week 4}

\subsubsection{Cloud}

\WHERE{\YES}{s:rest-intro}{Week 2}
\WHERE{\YES}{s:eve-intro}{Week 2}
\WHERE{\YES}{c:swagger-codegen}{Week 3}

\begin{itemize}
\item Build a Rest Service
\item Build a program to create VMs on an OpenStack cloud.
\end{itemize}






\begin{comment}

\begin{longtable}{p{3cm}p{11cm}}
  \caption{Calendar} \\   
  \toprule
  Date & Activity \\
  \midrule
  \endfirsthead
  \toprule
  Date & Activity \\
  \endhead
  \hline
  \multicolumn{2}{c}{Continued}\\   \bottomrule
  \endfoot
  \bottomrule
  \endlastfoot

  Jan 8, Mon & Class Begins\\
  
  Jan 15, Mon 9am & Setup communication pathways for the class. (1)
  You must have created a github repository in our class repository.
  (2) You must be in the class Piazza.  (3) Motivation: if we can not
  communicate with you we can not conduct the class. Everyone must be
  in piazza and github timely.  \\

  Weekly & contribution to notebook.md \\
  Weekly & contribution to piazza and/or the Handbook \\

  Jan 15, Mon & MLK Jr. Day. Good day to work on projects, computer setup \\
  Jan 22, Mon 9am & Tutorial 1 \\
  Feb 5,  Mon 9am & Tutorial 2 \\
  Feb 26, Mon 9am & Paper 1 \\
  Mar 5, Mon 9am & Project draft paper due without panelty \\
  \hline
  Spring Break &\\
  Mar 11 - Mar 18.  & This is a good time to work ahead or catch up
  with things. We strongly advise to use this time wisely. \\
  \hline
  Mar 16, Mon 9am & Project reports due without penalty \\
  Mar 23, Mon 9am & Improvments to Projects and documents possible,
  but substential work must have been done before to not encounter a
  grade reduction \\
  May 1 & Any paper submitted after May 1st will get an
  incomplete and a grade reduction. \\
\end{longtable}
\end{comment}


