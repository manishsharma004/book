%---------------------------------------------------------------
%	PART
%---------------------------------------------------------------

\part{Preface}



%---------------------------------------------------------------
%	CHAPTER
%---------------------------------------------------------------

\chapterimage{water.png} % Chapter heading image

\chapter{Introduction}\label{S:p-intro}

This document is the first one in a series of documents that are
providing a comprehensive overview of Clouds, Big Data, and their
technologies and applications. Material from these volumes are used
selectively as part of this class.

\begin{IU}
They are developed within the Intelligent Systems Engineering department at
Indiana University. This courses are for students who are interested in
any component of the Masters or Ph.D. in Intelligent Systems
Engineering. It is an advanced elective class.
\end{IU}


The Series includes:

\begin{enumerate}
\item (This document) {\bf Handbook of Clouds and Big Data}, Gregor von Laszewski,
  Geoffrey C. Fox, and Judy Qiu, Fall 2017,
  \url{https://tinyurl.com/vonLaszewski-handbook}

\item {\bf Use Cases in Big Data Software and
  Analytics Vol. 1}, Gregor von Laszewski, Fall 2017,
  \url{https://tinyurl.com/cloudmesh/vonLaszewski-i523-v1.pdf}

\item {\bf Use Cases in Big Data Software and
  Analytics Vol. 2}, Gregor von Laszewski, Fall 2017, 
  \url{https://tinyurl.com/cloudmesh/vonLaszewski-i523-v2.pdf}

\item  {\bf Use Cases in Big Data Software and
  Analytics Vol. 3}, Gregor von Laszewski, Fall 2017, 
  \url{https://tinyurl.com/vonLaszewski-projects-v3}

\item (Draft) {\bf Big Data Software Vol 1.}, Gregor von Laszewski, Spring 2017,
\url{https://github.com/cloudmesh/sp17-i524/blob/master/paper1/proceedings.pdf}

\item (Draft) {\bf Big Data Software Vol 2.}, Gregor von Laszewski, Spring 2017,
\url{https://github.com/cloudmesh/sp17-i524/blob/master/paper2/proceedings.pdf}

\item (Draft) {\bf Big Data Projects}, Gregor von Laszewski, Spring 2017,
\url{https://github.com/cloudmesh/sp17-i524/blob/master/project/projects.pdf}

\end{enumerate}

Our plan for this semester also includes the reorganization of some of
the volumes while sorting the contributions nto chapters with papers
from the same topic.

\section{Citation}

The citation for this document is 

\begin{quote}
Gregor von Laszewski, Geoffrey C. Fox, and Judy Qiu, Handbook of
Clouds and Big Data -- Theory and Practice, Indiana University,
Jan., 2018
Smith Research Center, Bloomington, IN 47408
\url{https://tinyurl.com/cloudmesh/vonLaszewski-bigdata.pdf} 
\end{quote}
% \URL{https://github.com/laszewski/laszewski.github.io/raw/master/vonLaszewski-bigdata.pdf}
% \URL{https://laszewski.github.io/papers/vonLaszewski-bigdata.pdf}

The bibtex entry for this document is

\begin{verbatim}
@TechReport{las17handbook,
  author =       {Gregor von Laszewski and Geoffrey C. Fox and Judy Qiu},
  title =        {Handbook of Clouds and Big Data -- Theory and Practice},
  institution =  {Indiana University},
  year =         {2018},
  OPTtype =      {Draft},
  address =      {Smith Research Center, Bloomington, IN 47408},
  month =        jan,
  url={https://tinyurl.com/cloudmesh/vonLaszewski-bigdata.pdf} 
}
\end{verbatim}



\section{Authors}

\FILENAME

\begin{description}

\item[Gregor von Laszewski] \index{von Laszewski, Gregor} Gregor von
  Laszewski is an Assistant Director DSC in the School of Informatics
  and Computing and Engineering at Indiana University. He holds also a
  position as Adjunct Professor in the Intelligent Systems Engineering
  Department. Previously he held Adjunct Professor positions at the
  Computer Science Department at Indiana University and University of
  North Texas. He received a Ph.D. from Syracuse University in
  computer science.

  At IU He served as the architect of the FutureGrid project. His
  current interest and projects include cloud computing, big data, and
  scientific impact metrics, and edge computing.  He initiated the
  Cloudmesh project which is a toolkit to enable cloud computing
  across various Cloud and Container IaaS such as OpenStack, AWS,
  Azure, docker, docker swarm, and kubernetes.


\item[Geoffrey C. Fox] \index{Fox, Geoffrey C} Fox received a Ph.D. in
  Theoretical Physics from Cambridge University and is now
  distinguished professor of Informatics and Computing, and Physics at
  Indiana University where he is director of the Digital Science
  Center, Chair of Department of Intelligent Systems Engineering and
  Director of the Data Science program at the School of Informatics,
  Computing, and Engineering.

  He currently works in applying computer science from infrastructure
  to analytics in Biology, Pathology, Sensor Clouds, Earthquake and
  Ice-sheet Science, Image processing, Deep Learning, Manufacturing,
  Network Science and Particle Physics. The infrastructure work is
  built around Software Defined Systems on Clouds and Clusters. The
  analytics focuses on scalable parallelism.

\item [Dr. Judy Qiu] is an Associate Professor in the School of
  Informatics and Computing at Indiana University. Her research
  interests focus on data-intensive computing at the intersection of
  cloud and multicore technologies, with an emphasis on life science
  applications using MapReduce as well as traditional parallel and
  distributed computing approaches. Her contributions are focused on
  Hadoop and Introduction into Cloud Computing. 

\end{description}


\section{About}

The material in this document is covering material that is or has been
used in the following classes at Indiana University.

\begin{itemize}
\item Undergraduate: E222 Intelligent systems II
\item Graduate: E516 Introduction to Cloud Computing
\item Graduate: E616 Advanced Cloud Computing
\item Graduate: E534 Big Data Applications
\item Graduate: I524 Big Data Applications and Open Source Software
\item Graduate: I523 Big Data Applications and Analytics
\end{itemize}

The format has been influenced by experiences gained from teachin I523
and I524 by Gregor von Laszewski.  The collection of this material is
updated continuously and new versions will be made available
throughout the semester. A timestamp on the front page indicates the
last time the document was updated.

\section{Other Results}\label{S:results}

Besides the documents posted in Section~\ref{S:p-intro} a small set of
additional results exist in form of a video presentation.  However,
these results are older and we are no longer using such
presentations. To gurantee completness, we include them however in
this section

\video{Cloud}{8:48}{Student Work 1}{https://www.youtube.com/watch?v=DYG6_bUGsqY}

\slides{Cloud}{Page 7}{Student Work 1}{https://drive.google.com/open?id=0B88HKpainTSfQU1uQmxZWHdWQ1k}

\slides{Cloud}{Page 7}{Student Work 1 - pptx}{https://drive.google.com/open?id=0B88HKpainTSfb1ZhWG4zTEg0SVk}


\video{Cloud}{10:03}{Student Work 2}{https://www.youtube.com/watch?v=DqaQ0kemmaw}

\slides{Cloud}{Page 12}{Student Work 2}{https://drive.google.com/open?id=0B88HKpainTSfQU1uQmxZWHdWQ1k}

\slides{Cloud}{Page 12}{Student Work 2 - pptx}{https://drive.google.com/open?id=0B88HKpainTSfb1ZhWG4zTEg0SVk}

\section{Contributing}
\index{Contributing}

We encourage students of the classes to contribute to this document or
other volumes, provide corrections, and additions. This document is
managed on \verb|github.com| at

\URL{https://github.com/cloudmesh/book/}

\subsection{Class Contributions}

The current release version is held in the master branch.
Development versions will be held under a number of branches:

\begin{description}
\item[e222] Branch with contributions from students of the e222
  class. Merges to and from the \textit{latex} branch will be conducted
  on a daily bases by TA's.
\item[e516] Branch with contributions from students of the e616
  class. Merges to and from the \textit{latex} branch will be conducted
  on a daily bases by TA's.
\item[e616] Branch with contributions from students of the e616
  class. Merges to and from the \textit{latex} branch will be conducted
  on a daily bases by TA's.
\item[dev] Branch managed by Gregor and the TA's
\item[master] Branch that contains the current released version. This
  version is updated once a week from the branch \textit{latex}.
\end{description}

Contributions are to be conducted as pull requests. It is important to
keep the pull requests small and on a section or even paragraph
base. This helps avoiding conflicts at time of checkin and is a common
practice in large communities. It is not a good practice to work for
weeks on improvements and than issue the pull request. For sure things
will have changed and it will take you a long time to catch up.

The original document is based on selected material published and edited by Gregor von
Laszewski at the following Web page

\URL{https://cloudmesh.github.io/classes/}

Based on feedback from students the desire to have the material in
book format available was raised and we have implemented it.


\subsection{Community Contributions}
\index{Contributing}

It is easy to contribute to this document and we invite everyone to
improve the material. To do so you need to fork the repository from 

\URL{https://github.com/cloudmesh/book/}

and clone it. Make sure you check out the \textit{dev} branch. Then you
can modify information in the different files or add new sections. It
is important that you make changes based on sections and than for them
create a new pull request. This simplifies the review process. We will
typically want only one file to be changed. Aslo before you issue your
pull request make sure that no one else has already made changes. In
that case we ask you to integrate them into your document.


\subsection{Contributors}

We like to acknowledge the following contributors that helped on this
document. Please notify us with your name and a brief commend on what
you contributed:

Descriptions provided in Section~\ref{S:BD-tech} were contributed by the
following people that are either listed by full name or their
github.com id:

\begin{quotation}{\em
Abhijit Thakre, Abhishek Gupta, Abhishek Naik, Ajit Balaga, Anurag
Kumar Jain, Avadhoot Agasti, Badi' Abdul-Wahid, Cmbays, DIKSHA,
Dimitar Nikolov, Govind, Govind Mishra, Grace Li, Gregor von
Laszewski, Harshit Krishnakumar, Hyungro Lee, Jerome Mitchell, Jimmy
Ardiansyah, Jon, Jon Montgomery, Jordan Simmons, Juliette Zerick,
Karthik, Kumar Satyam, Mark McCombe, Matthew Lawson, Methkupalli
Vasanth, Miao Jiang, Miao Zhang, Milind Suryawanshi,
MilindSuryawanshi, Nandita Sathe, Naveen, Niteesh01, Piyush Rai,
Piyush Shinde, Prashanth, Pratik Jain, Rahul Raghatate, Rahul Singh,
Ribka Rufael, Ronak Parekh, Saber Sheybani, Sabyasachi Roy Choudhury,
Sagar Vora, Sahiti Korrapati, Scott McClary, Sean Shiverick,
SilviaKarim14, Sivaprasad Sushmita, Snehal Chemburkar, Sowmya Ravi,
Srikanth Ramanam, Sunanda Unni, SushmitaSivaprasad, Tony Liu, Vasanth
Methkupalli, Veera Marni, Vibhatha Abeykoon, Vibhatha Lakmal Abeykoon,
Vishwanath Kodre, William H Knapp III, acastrob, ak.15, alyez,
anveling, argetlam115, athakre, bhavesh37, cacoulte, cglmoocs,
elenadesigner, eunosm3, harkrish1, jemitchell, justbbusy, jzerick,
kartanba, karthick, karthick venkatesan, karthik-anba, kpvenkat,
ksrivatsav, lmundia, miaozhan, michaelsmith1983, mmccombe, nsathe,
piyurai, pratik11jain, ronak1182, sabyasachi087,
shah0112, sriramsitharaman, suunni, tifabi, tonythomascn, vasanth,
vibhatha, vkodre, vlabeyko, xl41, yatinsharma7
}\end{quotation}

\subsection{Section Contributors}

\section{Chameleon Cloud}

\textit{Contributed by the Chamelon Team and Gregor von Laszewski}

Chapter~\ref{C:chameleon} focussing on Chameleon Cloud was coppied
with permission from the chameleoncloud.org website on Jan 1st, 2018 and
represents the contribution of the Chameleon project team, developed
under the NSF grant 1743358, Collaborative Research: Chameleon: A
Large-Scale, Reconfigurable Experimental Environment for Cloud
Research. Some content has been modified and added by Gregor von
Laszewski to specifically target classes and education material
targeted for Indiana University.

\section{MQTT}

\textit{Contributed by the Arnav and Gregor von Laszewski}

Chapter~\ref{c:mqtt} was contributed by Arnav and Gregor von
Laszewski. The original version was written as part of i523 in Fall
2017, but was signifcantly modified in Spring 2018.

\section{Conventions}
\index{Convention}

\subsection{Videos}

Videos to the class are referred to with embedded links into the PDF
document as follows: 

\video{About}{25:36}{Test Video}{https://www.youtube.com/watch?v=yC3PNkb_9mI}

An index will also be available in the index page
that lists on which page the video has been added.

\subsection{Slides}

Sides
\slides{About}{10}{Test slides}{PUT URL HERE}

\subsection{Images}

The video icon was copied from \url{http://www.freeiconspng.com/img/8039}.

\subsection{URLs}

The online version of this document contains a significant number of
links. The links are either embedded or are directly visible. The
color of the links is blue.

\begin{description}
\item[Direct URL:] This is an example for a
  \url{https://github.com/cloudmesh/book/}
\item[Embedded URL:] This is an example for an embedded URL that
  points to the \href{https://github.com/cloudmesh/book/}{source on github}
\end{description}

\subsection{Boxes}

\begin{NOTE}
Notes are written in blue boxes and indicate a clarification or some
important information that you do not want to overlook.
\end{NOTE}

\begin{WARNING}
Warnings are written in red  boxes and indicate a piece of information
that you must not ignore.
\end{WARNING}

\begin{IU}
Red boxes with Indiana University are information that relate to
students that use this material as part of the courses offered at
Indiana University.
\end{IU}

To dos are highlighted in boxes with the keyword TODO. The offer
opportunities for student to gather points for the discussion grade.

\TODO{An example todo}

\subsection{Copy and Paste}


First, we do not recommend to just copy and paste ever without reading
the content. We recommend that you first read the entire section or
sometimes even the chapter. Only after you understand what the command
does, use it.

Second, to make it simple the handbook contains now a hyperlink to the
LaTeX source of a page if you look at the lower right of the page
itself. You can click on it and will be redirected to github. You can
than locate the section that defines the listing and copy from
there. This helps in cases you are unsure of spaces or minus signs or
other characters including line breaks.

Third, It is natural that some issues may exist, but be reminded that
the examples we have in the handbook have actually been tried
before. Spelling errors naturally can occur, and we would be greatful
in case you see an error help us fixing it. THisis especially the case
when it comes to version numbers. Often you may want to visit the
original source and check if the recommended version numbers listed in
the handbook re still up to date. This naturally has also the
consequence that if a new version is available that you must double
check the rest of the content in order not to overlook updates that
have been provided by the developers.



\section{Exercises}
\bigskip

\begin{exercise}
\label{E:Preface.1} 
Inspect the PDF documents produced by previous
classes. Note the differences between technology and application
reviews and projects. 
\end{exercise}

\begin{exercise}
\label{E:Preface.2}
 Contribute to this document while finding a
  single spelling error. Before you do the pull request, make sure the
  document compiles.
\end{exercise}


