\MDNAME\
%%%%%%%%%%%%%%%%%%%%%%%%%%%%%%%%%%%%%%%%%%%%%%%%%%%%%%%%%%%%%%%%%%%%%%%%%%%%%%%
% DO NOT MODIFY THIS FILE
%%%%%%%%%%%%%%%%%%%%%%%%%%%%%%%%%%%%%%%%%%%%%%%%%%%%%%%%%%%%%%%%%%%%%%%%%%%%%%%

\subsection{MongoDB Tutorial}

\textbf{THIS IS A DRAFT AND NOT COMPLETED YET}

\begin{itemize}
\item
  status: 0
\item
  feedback: 1
\end{itemize}

Some steps are copied from
\url{https://www.tutorialspoint.com/mongodb/mongodb_environment.htm/}

\subsubsection{Overview}

\begin{longtable}[]{@{}ll@{}}
\toprule
RDBMS & MongoDB\tabularnewline
\midrule
\endhead
Database & Database\tabularnewline
Table & Collection\tabularnewline
Tuple/Row & Document\tabularnewline
column & Field\tabularnewline
Table Join & Embedded Documents\tabularnewline
Primary Key & Primary Key\tabularnewline
\bottomrule
\end{longtable}

MongoDB is an open-source document database and NoSQL database, which is
written in C++. A document is a set of key-value pairs. The table shows
the difference between RDBMS terminology and MongoDB.

\subsubsection{Advantages of MongoDB over RDBMS}

\begin{itemize}
\item
  Schema less - MongoDB is a document database in which one collection
  holds different documents. Number of fields, content and size of the
  document can differ from one document to another.
\item
  Structure of a single object is clear.
\item
  No complex joins.
\item
  Deep query-ability. MongoDB supports dynamic queries on documents
  using a document-based query language that's nearly as powerful as
  SQL.
\item
  Tuning.
\item
  Ease of scale-out - MongoDB is easy to scale.
\item
  Conversion/mapping of application objects to database objects not
  needed.
\item
  Uses internal memory for storing the (windowed) working set, enabling
  faster access of data.
\end{itemize}

\subsubsection{Install MongoDB on Ubuntu}

Run the following command to import the MongoDB public GPG key

\begin{lstlisting}
sudo apt-key adv --keyserver hkp://keyserver.ubuntu.com:80 --recv 7F0CEB10
\end{lstlisting}

Create a /etc/apt/sources.list.d/mongodb.list file using the following
command.

\begin{lstlisting}
echo 'deb http://downloads-distro.mongodb.org/repo/ubuntu-upstart dist 10gen' | sudo tee /etc/apt/sources.list.d/mongodb.list
\end{lstlisting}

Now issue the following command to update the repository

\begin{lstlisting}
sudo apt-get update
\end{lstlisting}

Next install the MongoDB by using the following command

\begin{lstlisting}
apt-get install mongodb-10gen = 2.2.3
\end{lstlisting}

Start MongoDB

\begin{lstlisting}
sudo service mongodb start
\end{lstlisting}

Stop MongoDB

\begin{lstlisting}
sudo service mongodb stop
\end{lstlisting}

Restart MongoDB

\begin{lstlisting}
sudo service mongodb restart
\end{lstlisting}

