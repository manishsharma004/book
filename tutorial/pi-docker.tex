\MDNAME\
%%%%%%%%%%%%%%%%%%%%%%%%%%%%%%%%%%%%%%%%%%%%%%%%%%%%%%%%%%%%%%%%%%%%%%%%%%%%%%%
% DO NOT MODIFY THIS FILE
%%%%%%%%%%%%%%%%%%%%%%%%%%%%%%%%%%%%%%%%%%%%%%%%%%%%%%%%%%%%%%%%%%%%%%%%%%%%%%%

\section{VERSION A: Install Docker on a Raspberry Pi}

hid-sp18-503

Docker is now supported on ARM processors. Thus the instalation of
docker on Raspbery PI's is extremely simple and we do not need any
special setup to get docker running.

To install docker on the raspberry pi, please execute the following
steps:

First, ssh into the raspberry pi (or login using a monitor and open
terminal) Second, run the following command

\texttt{curl\ -sSL\ https://get.docker.com\ \textbar{}\ sh}

If you dont want to run docker using sudo add pi to the docker user
group as recommended using

\texttt{sudo\ usermod\ -aG\ docker\ pi}

Now any docker image built for ARM can be run. Naturally it must be
small enough to fit on the PI. PLease remember it has a very small
memory.

Please report back to us if you found useful packages and tools for the
PI.

\section{VERSION B: Pi Docker}

Docker is a tool that allows you to deploy applications inside of
software containers. It is a method of packaging software, to include
not only your code, but also other components such as a full file
system, system tools, services, and libraries. This can be useful for
the Raspberry Pi because it allows users to run applications without lot
of steps, as long as the application is packaged inside of a Docker
image. You simply install Docker and run the container.

\subsection{Preparing the SD card}

Download the latest Raspbian Jessie Lite image from

\begin{lstlisting}
https://www.raspberrypi.org/downloads/raspbian/
\end{lstlisting}

Please note that Raspbian Jessie Lite image contains the only the bare
minimum amount of packages.

\subsection{Download Etcher here:}

\begin{lstlisting}
https://etcher.io/
\end{lstlisting}

Now follow the instructions in Etcher to flash Raspbian image on the SD
card. Plbefore ejecting the SD card.

\subsection{Enable SSH on the SD Card}

To prevent Raspberry Pis from being hacked the RPi foundation have now
disabled SSH on the default image. So, create a text file in /boot/
called ssh - it can be empty file or you can type anything you want
inside it.

Please note that you have renamed the ssh.txt to ssh i.e.~without
extension.

Now insert the SD card, networking and power etc.

\subsection{Starting Pi}

Once you boot up the Raspberry Pi, Connect using SSH

\begin{lstlisting}
    $ ssh pi@raspberrypi.local
\end{lstlisting}

The password is raspberry.

For security reasons, please change the default password of the user pi
using the passwd command.

Note : If you want to change the hostname of the Pi, Use an editor and
change the hostname raspberrypi in:

\begin{lstlisting}
    * /etc/hosts
    * /etc/hostname
\end{lstlisting}

\subsection{Docker Installation}

\subsection{Run apt-get update}

Since Raspbian is Debian based, we will use apt to install Docker. But
first, we need to update.

\begin{lstlisting}
    sudo apt-get update
            
\end{lstlisting}

\subsection{Install Docker}

An automated script maintained by the Docker project will create a
systemd service file and copy the relevant Docker binaries into
/usr/bin/.

\begin{lstlisting}
    $ curl -sSL https://get.docker.com | sh
\end{lstlisting}

\subsection{Configure Docker}

\begin{lstlisting}
* Set Docker to auto-start
        
    $ sudo systemctl enable docker
            
* Reboot the Pi, or start the Docker daemon with:

    $ sudo systemctl start docker
\end{lstlisting}

\subsection{Enable Docker client}

The Docker client can only be used by root or members of the docker
group. Add pi or your equivalent user to the docker group using :

\begin{lstlisting}
    $ sudo usermod -aG docker pi
            
\end{lstlisting}

After executing the above command, log out and reconnect with ssh.

\subsection{Test Docker}

To test docker was installed successfully, run the hello-world image.

\begin{lstlisting}
    $ docker run hello-world
            
\end{lstlisting}

If Docker is installed properly, you'll see a ``Hello from Docker!''
message.

