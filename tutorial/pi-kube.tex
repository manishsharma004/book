\MDNAME\
%%%%%%%%%%%%%%%%%%%%%%%%%%%%%%%%%%%%%%%%%%%%%%%%%%%%%%%%%%%%%%%%%%%%%%%%%%%%%%%
% DO NOT MODIFY THIS FILE
%%%%%%%%%%%%%%%%%%%%%%%%%%%%%%%%%%%%%%%%%%%%%%%%%%%%%%%%%%%%%%%%%%%%%%%%%%%%%%%

\section{Tutorial: Raspberry Pi Kubernetes Cluster}

Authored by rashray (hid-sp18-417) -
https://github.com/cloudmesh-community/hid-sp18-417/tree/master/tutorial

\section{Setting up Raspberry Pi Kubernetes Cluster}

This tutorial will guide the steps towards setting up a Kubernetes Pi
cluster. The steps are: a. Setting up static IP b. Secure ssh key setup
for communication c. Kubenetes cluster setup d. Automating the process

Cluster needs: * One head/master Pi * number of follower nodes Pi *
router {[}optional{]}

Pis can be connected directly to the home's Internet router. Please note
that a router is needed when portability is a criteria. \#\# Initial
Setup Some Pi kits come with pre installed SD card if not then: 1.
format the SD card:
https://www.sdcard.org/downloads/formatter\_4/eula\_windows/ 2. Download
the package from : https://www.raspberrypi.org/downloads/noobs/ 3.
Download and unzip the package and copy it to the SD card {[}Copy only
the files inside NOOBS\_\{version\}{]} 4. Connect the power cable,
keybboard and mouse to the Pi 5. Insert the SD card and the installed
will walk you through the installation process 6. Once the installation
is through make sure the time and keyboard setting are updated according
to your local settings Normally they come in UK settings

\subsection{Setting up Static IP and HostName}

HostName: Hostname can be given by clicking the top right wifi
icon\textgreater{} network setting\textgreater{} The windown can be
launched by

\begin{lstlisting}

command i the terminal or by :

    sudo nano /etc/hostname

static IP similarly can be given by
Hostname can be given by clicking the top right wifi icon> network setting>
make sure you give both etho and wlan0 setting for both LAN and Wifi communication

or by ensuring the following in for LAN and Wifi config``/etc/dhcpcd.conf``

I have added the following aletrnatively eth0 block can be added if wired setup is preferred
\end{lstlisting}

interface wlan0 static ip\_address=\{desired IP\}/24 static
routers=\{router IP\} static domain\_name\_servers=\{DNS server IP\}

\begin{lstlisting}

Ensure that the desired IP falls with in the assigned IP range of your router.
As we will be automating the process later, its advised to follow a naming sequence for the hostname [IP address]
Example: kub00[192.168.56.100], kub01 [192.168.56.101], kub02 [192.168.56.102]...

Its essential to ``reboot`` the system for the changes to take effect.

## SSH setup

* Ensure that ssh is enabled in the Pi:

  - Click on the ```Raspberry Pi Configuration``` from the ```Preferences``` on run the command 
      
    ```sudo raspi-confi``` 
        
    in the terminal. Go to Interface tab and enable SSH
  - In the terminal run:
    ```
    sudo systemctl enable ssh
    sudo systemctl start ssh
    ```

  With the static ip setup and ssh enabled you should be able to ssh in to the Pi. For passwordless acess setup the SSH key as   per the following step
* Generate the ssh key; make sure you give a passcode:
        ```
          ssh-keygen -t rsa 
        ```
  Copy the generated public key ``~/.ssh/id_rsa.pub` to the other computers for passwordless acess
  ``ssh-copy-id`` or ``ssh-import-id`` can be used for the purpose as well
  
## Cluster setup

  * Inastall Docker
  
  ``
  curl -sSL get.docker.com | sh && \
  sudo usermod pi -aG docker
  ``
  
  [optional] Command to run Docker as a non root user:
  ``
  sudo usermod -aG docker pi
  ``
  
  * Turn off swap:
\end{lstlisting}

sudo dphys-swapfile swapoff \&\&\\
sudo dphys-swapfile uninstall \&\&\\
sudo update-rc.d dphys-swapfile remove
`\texttt{*\ Edit\ /boot/cmdline.txt\ and\ add\ the\ following\ with\ a\ space\ {[}no\ new\ line{]}}
cgroup\_enable=cpuset cgroup\_memory=1 `` * setup kubeadm

\begin{lstlisting}
curl -s https://packages.cloud.google.com/apt/doc/apt-key.gpg | sudo apt-key add - && \
echo "deb http://apt.kubernetes.io/ kubernetes-xenial main" | sudo tee /etc/apt/sources.list.d/kubernetes.list && \
sudo apt-get update -q && \
sudo apt-get install -qy kubeadm
\end{lstlisting}

The process till now stays the same for both workes and master

\section{Master setup}

In the master node initiate the master:

\begin{lstlisting}
sudo kubeadm init --token-ttl=0 --apiserver-advertise-address=<internal master ip>
\end{lstlisting}

This will initiate the kubectl head. At the end of the this process,
there should be an echo with the following text. Save\\
this join token as you will joining the workers later:

\begin{lstlisting}
kubeadm join --token <token> --discovery-token-ca-cert-hash <ca hash>
\end{lstlisting}

Once the master is initiated, now set up the kubectl admin config

\begin{lstlisting}
mkdir -p $HOME/.kube
sudo cp -i /etc/kubernetes/admin.conf $HOME/.kube/config
sudo chown $(id -u):$(id -g) $HOME/.kube/config
\end{lstlisting}

\begin{itemize}
\item
  The final step is setting up the networking. I have used weave.
\end{itemize}

\begin{lstlisting}
kubectl apply -f \
"https://cloud.weave.works/k8s/net?k8s-version=$(kubectl version | base64 | tr -d '\n')"
\end{lstlisting}

\section{Worker setup}

After Kubernetes installation, join the workers using the saved join
token.

Use \texttt{get\ nodes} in the master to check the status

\texttt{kubectl\ get\ pods\ -\/-namespace=kube-system} can be used to
check the pod status of the cluster

\# Troubleshooting During the development of this tytorial its been
experienced that Kubernetes cluster needs atleast one master and three
worker nodes. Using less resources will lead to slow processor

