\MDNAME\
%%%%%%%%%%%%%%%%%%%%%%%%%%%%%%%%%%%%%%%%%%%%%%%%%%%%%%%%%%%%%%%%%%%%%%%%%%%%%%%
% DO NOT MODIFY THIS FILE
%%%%%%%%%%%%%%%%%%%%%%%%%%%%%%%%%%%%%%%%%%%%%%%%%%%%%%%%%%%%%%%%%%%%%%%%%%%%%%%

\section{Enable remote access}

enable vnc enable ssh

\section{Install GrovePi}

\textbf{WARNING:} Do not install the Grovepi shield yet

\begin{lstlisting}
sudo apt-get update
sudo apt-get install emacs -y
cd /home/pi/Desktop
sudo git clone https://github.com/DexterInd/GrovePi
cd /home/pi/Desktop/GrovePi/Script
sudo chmod +x install.sh
sudo ./install.sh
sudo reboot
\end{lstlisting}

After the reboot we install the following

\begin{lstlisting}
sudo pip install grovepi
sudo shutdown -h now
\end{lstlisting}

\textbf{ONlY NOW put the grove shield on and power on}

check

\begin{lstlisting}
sudo i2cdetect -y 1
\end{lstlisting}

If everything goes right you will see:

\begin{lstlisting}
    0  1  2  3  4  5  6  7  8  9  a  b  c  d  e  f
00:          -- 04 -- -- -- -- -- -- -- -- -- -- -- 
10: -- -- -- -- -- -- -- -- -- -- -- -- -- -- -- -- 
20: -- -- -- -- -- -- -- -- -- -- -- -- -- -- -- -- 
30: -- -- -- -- -- -- -- -- -- -- -- -- -- -- -- -- 
40: -- -- -- -- -- -- -- -- -- -- -- -- -- -- -- -- 
50: -- -- -- -- -- -- -- -- -- -- -- -- -- -- -- -- 
60: -- -- -- -- -- -- -- -- -- -- -- -- -- -- -- --
70: -- -- -- -- -- -- -- --        
\end{lstlisting}

\section{Test a joystick}

\begin{lstlisting}
cd ~/Desktop/GrovePi/Software/Python/   
python grove_thumb_joystick.py
\end{lstlisting}

You will see output such as

\begin{lstlisting}
('x =', 497, ' y =', 513, ' Rx =', 10.583501006036217, ' Ry =', 9.941520467836257, ' click =', 0)
('x =', 496, ' y =', 513, ' Rx =', 10.625, ' Ry =', 9.941520467836257, ' click =', 0)
('x =', 497, ' y =', 513, ' Rx =', 10.583501006036217, ' Ry =', 9.941520467836257, ' click =', 0)
\end{lstlisting}

\section{tkinter}

You can install tkinter as follows

\begin{lstlisting}
sudo apt-get install python-tk
sudo apt autoremove
\end{lstlisting}

\section{install cloudmesh.pi}

\begin{lstlisting}
sudo apt-get install python-dev python-setuptools
sudo apt-get install libjpeg-dev -y
pip install Pillow
mkdir ~/github
cd ~/github
git clone git@github.com:cloudmesh/cloudmesh.pi.git
git checkout dev
pip install -e .
\end{lstlisting}

The last command allows us to modify the code and have it directly
working. This will take quite a while as it installs numpy also.

\section{motor}

\url{https://cdn-learn.adafruit.com/downloads/pdf/adafruit-16-channel-servo-driver-with-raspberry-pi.pdf}

\begin{lstlisting}
cd ~/github
git clone https://github.com/adafruit/Adafruit_Python_PCA9685.git
cd Adafruit_Python_PCA9685
sudo python setup.py install
sudo python3 setup.py install
\end{lstlisting}

\section{Making the board LED blink}

The green LED can be made blinking as follows in root

\begin{lstlisting}
echo 1 > /sys/class/leds/led0/brightness
echo 0 > /sys/class/leds/led0/brightness
\end{lstlisting}

THis can also be done via a file and if combined witth an uploades key
into authorized\_keys, we can controll them via ssh

\begin{lstlisting}
ssh pi@red03 "echo 1 > led; sudo cp led /sys/class/leds/led0/brightness"        
ssh pi@red03 "echo 0 > led; sudo cp led /sys/class/leds/led0/brightness"
\end{lstlisting}

\section{Find PIs on the network}

\begin{lstlisting}
arp -a | fgrep "b8:27:eb"
\end{lstlisting}

\section{(obsolete as the above works) New kernel has a bug}

Run this

\begin{lstlisting}
sudo rpi-update 52241088c1da59a359110d39c1875cda56496764.
\end{lstlisting}

\section{Enable remote access}

\section{Set up Grove PI}

Setup the rasperry grove pi software

\begin{lstlisting}
sudo curl -kL dexterindustries.com/update_grovepi | bash
sudo reboot
\end{lstlisting}

Setup the examples

\begin{lstlisting}
cd /home/pi/Desktop
sudo git clone https://github.com/DexterInd/GrovePi
\end{lstlisting}

\section{GrovePI Firmware update}

\url{https://www.dexterindustries.com/GrovePi/get-started-with-the-grovepi/updating-firmware/}

\begin{lstlisting}
cd Desktop/
git clone https://github.com/DexterInd/GrovePi.git
cd /home/pi/Desktop/GrovePi/Firmware
sudo chmod +x firmware_update.sh
sudo ./firmware_update.sh
\end{lstlisting}

say \texttt{y}

You will see

\begin{lstlisting}
avrdude: safemode: Fuses OK
\end{lstlisting}

which indicates the firmware update was successfully completed.

check the firmware version

\begin{lstlisting}
python ~/Desktop/GrovePi/Software/Python/grove_firmware_version_check.py
\end{lstlisting}

\section{Install 4 inch monitor}

We assume you have VNC and ssh enabled as services in the prefernces.

The documentation to the monitor is located at

\url{https://www.waveshare.com/wiki/4inch_HDMI_LCD}

The best way to install it is to use ssh pi@

and use the terminal of you laptop. Frst download the driver with

\begin{lstlisting}
sudo su
wget http://www.waveshare.com/w/upload/0/00/LCD-show-170703.tar.gz
cd /boot
hdmi_group=2
hdmi_mode=87
hdmi_cvt 480 800 60 6 0 0 0
dtoverlay=ads7846,cs=1,penirq=25,penirq_pull=2,speed=50000,keep_vref_on=0,swapxy=0,pmax=255,xohms=150,xmin=200,xmax=3900,ymin=200,ymax=3900
display_rotate=3
\end{lstlisting}

This will reboot the pi. Next install the touch screen driver with

\begin{lstlisting}
cd /boot
wget http://www.waveshare.com/w/upload/0/00/LCD-show-170703.tar.gz
sudo tar xvf LCD-show-170703.tar.gz 
cd LCD-show
chmod +x LCD4-800x480-show
./LCD4-800x480-show
\end{lstlisting}

\section{GUI}

\begin{lstlisting}
sudo pip3 install guizero
\end{lstlisting}

