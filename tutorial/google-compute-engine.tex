\MDNAME\
%%%%%%%%%%%%%%%%%%%%%%%%%%%%%%%%%%%%%%%%%%%%%%%%%%%%%%%%%%%%%%%%%%%%%%%%%%%%%%%
% DO NOT MODIFY THIS FILE
%%%%%%%%%%%%%%%%%%%%%%%%%%%%%%%%%%%%%%%%%%%%%%%%%%%%%%%%%%%%%%%%%%%%%%%%%%%%%%%

\section{Creating a Virtual Machine on Google Compute Engine}

\begin{itemize}
\item
  Bertolt Sobolik (hid-sp18-419) \#\# Introduction
\end{itemize}

Google Compute Engine is an Infrastructure as a Service (IaaS) offering
from Google. The company offers a wide variety of virtual machines,
operating systems, and storage options to serve a wide variety of
customers, including many high-end options and configurations designed
for specific applications.

\subsection{Requirements}

In order to create a VM on Google Compute Engine one must have a Google
Account. You can sign up for one
\href{https://accounts.google.com/SignUp?hl=en}{here}.

Next, you will need to
\href{https://console.cloud.google.com/freetrial}{sign up for Google
Cloud Platform}. As of this writing, Google is offering a free trial
that includes \$300 worth of credits that expire one year from signup,
but you will need to provide billing information (either a credit card
or bank details) to sign
up\textasciitilde{}\cite{hid-sp18-419-tutorial-gce-signup}.

When you sign up, Google will create a project for you called \emph{My
First Project} with an autogenerated project ID. The page after the
welcome screen contains links to a number of easy to follow tutorials
and \emph{Quick Starts} that will guide you through the steps required
to start various Google Cloud Platform services in the console.

\subsection{Automated Creation of a VM on Ubuntu 16.04}

The following instructions are adapted from
\href{https://cloud.google.com/sdk/docs/quickstart-debian-ubuntu?authuser=1}{here}\textasciitilde{}\cite{hid-sp18-419-tutorial-gce-setup}.

You will need to have \emph{curl} installed. You can install it with:

\begin{lstlisting}
sudo apt install curl
\end{lstlisting}

First you need to install the gcloud SDK:

\begin{lstlisting}
export CLOUD_SDK_REPO="cloud-sdk-$(lsb_release -c -s)" 
echo "deb http://packages.cloud.google.com/apt $CLOUD_SDK_REPO main" \
    | sudo tee -a /etc/apt/sources.list.d/google-cloud-sdk.list 
curl https://packages.cloud.google.com/apt/doc/apt-key.gpg \
    | sudo apt-key add - 
sudo apt-get update && sudo apt-get install google-cloud-sdk 
\end{lstlisting}

Next, initialize the SDK with:

\begin{lstlisting}
gcloud init 
\end{lstlisting}

This will open a web browser and prompt you to log in to your account.
If you don't have a browser installed or are doing a headless setup, you
can use the following command instead:

\begin{lstlisting}
gcloud init --console-only 
\end{lstlisting}

This will display a link in your terminal that you must copy and paste
into a browser. The browser will return a verification code for you to
type into the terminal.

After you have logged in, the terminal will prompt you to select a
project or create a new one. Project IDs must be unique. If you pick one
like \emph{test-vm} it will fail. If you start again with:

\begin{lstlisting}
gcloud init 
\end{lstlisting}

You will be prompted to pick up where you left off or create a new
configuration.

After selecting a project, you will be asked if you want to configure a
default Compute Region and Zone. If you do not, configuration is
complete.

\subsection{Create an Instance}

You can create an instance with the following command (where is the name
of the instance you want to create):

\begin{lstlisting}
gcloud compute instances create <name of instance> 
\end{lstlisting}

If you have not set up a default Compute Region and Zone, you will be
prompted to select one from the 45 possibly zones, so it is probably
better to either set a default or decide which zone you want the
instance in before you type the command. For example, if you want to
create an instance called \emph{foo} in \emph{us-central1-a} (which is
in Iowa), you would enter:

\begin{lstlisting}
gcloud compute instances create foo --zone=us-central-1a 
\end{lstlisting}

This will create an \emph{n1-standard-1} instance with one CPU and 3.75
GB of memory, which costs about \$25 a month. If you leave it running,
you will burn through a lot of your free credits for nothing. You can
stop it with:

\begin{lstlisting}
gcloud compute instances stop foo --zone=us-central-1a 
\end{lstlisting}

A full list of all the options for the \textbf{gcloud compute instances}
command is
\href{https://cloud.google.com/sdk/gcloud/reference/compute/instances/}{here}\textasciitilde{}\cite{hid-sp18-419-tutorial-gce-reference}.

\subsection{Creating and Downloading Credentials for the Default Service Account}

To create a service account, use the following command:

\begin{lstlisting}
gcloud iam service-accounts create <account> 
\end{lstlisting}

A display name can be created for the account with the
\texttt{-\/-display-name} option. The account created will be associated
with an iam email address in the form: *
\texttt{\textless{}account\textgreater{}@\textless{}project\textgreater{}.iam.gserveiceaccount.com}
To download credentials for this account use the command:

\begin{lstlisting}
gcloud iam service-accounts keys create <file> --iam-account <iam email address>
\end{lstlisting}

followed by the name and path where you want the key (e.g.
\textasciitilde{}/key.json will create a file named \texttt{key.json} in
the home directory.

Google's documentation of the account creation process is
\href{https://cloud.google.com/iam/docs/creating-managing-service-accounts}{here}.
Documentation of the key generation process is
\href{https://cloud.google.com/iam/docs/creating-managing-service-account-keys}{here}.
Accounts and keys can also be created on the Google Cloud Platform site
or via the API.

