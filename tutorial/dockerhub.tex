\FILENAME\

\TODO{Min: please check the entire chapter for proper citation and quoting}


\section{Docker Hub}

\label{s:dockerhub}\index{Dockerhub}

Docker Hub~\cite{hid-sp18-405-tutorial-dockerhub-overview} ``is a
cloud-based registry service which allows users to link to code
repositories, build their own images and test them, stores manually
pushed images, and links to Docker Cloud so they can deploy images to
their hosts.''  There are both public and private
repositories. Companies could have a private repository for use within
their own organization, whereas public images can be used by anyone.

``There are thousands of images published on DockerHub. DockerHub is
hardcoded into Docker as the default registry, so when you run the
docker pull command against any image, it will be downloaded from Docker
Hub''~\cite{hid-sp18-405-tutorial-dockerhub-blog-use}. ``It provides a
centralized resource for container image discovery, distribution and
change management, user and team collaboration, and workflow automation
throughout the development pipeline''
\cite{hid-sp18-405-tutorial-dockerhub-overview}.

\subsection{Create Docker ID and Log In You could create a Docker ID on the}

Log-in is not necessary for pulling Docker images from the Hub but
necessary for push images. However when you want to store images on
Docker hub you need to create an account.  To create an account on
Docker Hub, please visit the \href{https://hub.docker.com/}{Docker Hub
  main page}. The free service will give you only one private Docker Hub
Repository. In case you need more, you will need to upgrade to a paid plan.

For the rest of the tutorial we assume that you use the environment
variable DUSER to indicate you username. It is easiset if you set it
in your shell with

\begin{lstlisting}
export DUSER=<PUT YOUR DOCKER USERNAME HERE> 
\end{lstlisting}

\subsection{Searching for Docker Images}

There are two ways to search for Docker images on Docker Hub:

One way is to use the Docker command line tool. You could open an
terminal and run the \emph{docker search} command. For example, the
following command searches for centOS images:

\begin{lstlisting}
docker search centos
\end{lstlisting}

you will see output similar to:

\begin{lstlisting}
| NAME            | DESCRIPTION        | STAR | OFFICIAL | AUTOMATED |
|-----------------|--------------------|------|----------|-----------|
| centos          | Official CentOS    | 4130 | [OK]     |           |
| ansible/centos7 | Ansible on Centos7 | 105  |          | [OK]      |
...
\end{lstlisting}

Official repositories are public, certified repositories from vendors
and contributors to Docker. They contain Docker images from vendors
like Canonical, Oracle, and Red Hat that you can use as the basis to
build your applications and services. There is one official repository
in this list, the first one, \emph{centos}.

The other way is to search via the \emph{Web Search Box} at the top of the
Docker web page by typing the keyword. The search results can be sorted
by number of stars, number of pulls, and whether it is an official
image. Then for each search result, you can verify the information of
the image by clicking the \emph{details} button to make sure this is the
right image that fits your needs.

\subsection{Pulling Images}

A particular image (take centos as an example) could be pulled using the
following command:

\begin{lstlisting}
docker pull centos
\end{lstlisting}

Tags could be used to specify the image to pull. By default the tag is
latest, therefore the previous command is the same as the following:

\begin{lstlisting}
docker pull centos:latest
\end{lstlisting}

You could use a different tag:

\begin{lstlisting}
docker pull centos:6
\end{lstlisting}

To check the existing local docker images, run the following command:

\begin{lstlisting}
docker images
\end{lstlisting}

The results show:

\begin{lstlisting}
| REPOSITORY | TAG    | IMAGE ID     | CREATED     | SIZE  |
|------------|--------|--------------|-------------|-------|
| centos     | latest | 26cb1244b171 | 2 weeks ago | 195MB |
| centos     | 6      | 2d194b392dd1 | 2 weeks ago | 195MB |
\end{lstlisting}

\subsection{Create Repositories}

In order to push images to Docker Hub, you need to have a repository
created.

When you first create a Docker Hub user, you see a \emph{Get started
with Docker Hub} screen, from which you can click directly into
\emph{Create Repository}. You can also use the \emph{Create} menu to
\emph{Create Repository}. When creating a new repository, you can choose
to put it in your Docker ID namespace, or that of any organization that
you are in the owners team
\cite{hid-sp18-405-tutorial-dockerhub-repository}.

As an example, we created a repository cloudtechnology with the name
space \texttt{\$DUSER}. Hence the full name is
\texttt{\$DUSER}/cloudtechnology

\subsection{Pushing Images}

To push an image to the repository created, the following steps could be
followed:

\begin{itemize}
\item
  Log into Docker Hub from the command line by specifying the username

\begin{lstlisting}
  docker login --username=$DUSER
\end{lstlisting}

  Enter the password when prompted. If everything worked you will get a
  message similar to:

\begin{lstlisting}
  Login Succeeded
\end{lstlisting}
\item
  Check image ID using:

\begin{lstlisting}
  docker images
\end{lstlisting}

  the result looks similar to:

\begin{lstlisting}
  | REPOSITORY    | TAG    | IMAGE ID     | CREATED     | SIZE   |
  |---------------|--------|--------------|-------------|--------|
  | cloudmesh-nlp | latest | 1f26a5f7a1b4 | 10 days ago | 1.79GB |
  | centos        | latest | 26cb1244b171 | 2 weeks ago | 195MB  |
  | centos        | latest | 2d194b392dd1 | 2 weeks ago | 195MB  |
\end{lstlisting}

  and the image with ID 1f26a5f7a1b4 is the one to push to Docker Hub.
\item
  Tag the image

\begin{lstlisting}
  docker tag 1f26a5f7a1b4 $DUSER/cloudmesh:firsttry
\end{lstlisting}

  In general, a good choice for a tag is something that will help you
  understand what this container should be used in conjunction with, or
  what it represents.
\item
  Now the list of images will look something like

\begin{lstlisting}
  | REPOSITORY       | TAG      | IMAGE ID     | CREATED  | SIZE   |
  |------------------|----------|--------------|----------|--------|
  | cloudmesh-nlp    | latest   | 1f26a5f7a1b4 | 10 d ago | 1.79GB |
  | $DUSER/cloudmesh | firsttry | 1f26a5f7a1b4 | 10 d ago | 1.79GB |
  | centos           | latest   | 26cb1244b171 | 2 w ago  |  195MB |
  | centos           | latest   | 2d194b392dd1 | 2 w ago  |  195MB |
\end{lstlisting}
\item
  Push the image to the repository

\begin{lstlisting}
  docker push $DUSER/cloudmesh
\end{lstlisting}

  It shows something similar to:

\begin{lstlisting}
  The push refers to repository [docker.io/$DUSER/cloudmesh]
  18f9479cfc2c: Pushed 
  e9ddee98220b: Pushed 
  1d3522002590: Pushed 
  e3ab85ae555e: Pushed 
  bae105d9c555: Pushed 
  6b0c2fb2fe92: Pushed 
  c33cd8954775: Pushed 
  ecafeebb22db: Pushed 
  e0dbd107774a: Pushed 
  8cb07daea6f6: Pushed 
  db584c622b50: Mounted from library/ubuntu 
  52a7ea2bb533: Mounted from library/ubuntu 
  52f389ea437e: Mounted from library/ubuntu 
  88888b9b1b5b: Mounted from library/ubuntu 
  a94e0d5a7c40: Mounted from library/ubuntu 
  firsttry: digest: sha256:305b0f911077d9d6aab4b447b... size: 3463
\end{lstlisting}

  Now the image is available on Docker Hub and everyone can pull it
  since it is a public repository by using command:

\begin{lstlisting}
  docker pull $DUSER/cloudmesh
\end{lstlisting}
\end{itemize}

\subsection{Resources}

\begin{itemize}
\item
  The offical
  \href{https://docs.docker.com/docker-hub/\#use-official-repositories}{Overview
  of Docker Hub} \cite{hid-sp18-405-tutorial-dockerhub-overview}
\item
  Information about using docker repositories can be found at
  \href{https://docs.docker.com/docker-hub/repos/}{Repositories on
  Docker Hub} \cite{hid-sp18-405-tutorial-dockerhub-repository}
\item
   \href{https://www.linux.com/blog/learn/intro-to-linux/2018/1/how-use-dockerhub}{How
  to Use DockerHub} \cite{hid-sp18-405-tutorial-dockerhub-blog-use}

\item \href{https://rominirani.com/docker-tutorial-series-part-4-docker-hub-b51fb545dd8e}{Docker
  Tutorial Series}\cite{docker-tutorial-series-part-4}
\end{itemize}


