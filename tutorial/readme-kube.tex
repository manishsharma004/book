\MDNAME\
%%%%%%%%%%%%%%%%%%%%%%%%%%%%%%%%%%%%%%%%%%%%%%%%%%%%%%%%%%%%%%%%%%%%%%%%%%%%%%%
% DO NOT MODIFY THIS FILE
%%%%%%%%%%%%%%%%%%%%%%%%%%%%%%%%%%%%%%%%%%%%%%%%%%%%%%%%%%%%%%%%%%%%%%%%%%%%%%%

\section{Raspberry Pi Kubernetes Cluster}

\begin{itemize}
\item
  Tim Whitson @whitstd (hid-sp18-526)
\item
  Juliano Gianlupi @JulianoGianlupi (hid-sp18-601)
\end{itemize}

\subsection{Note}

This tutorial is for Raspbian Stretch.

\subsection{Hardware}

Each cluster consists of:

\begin{itemize}
\item
  1 head node (\href{head}{setup}) (recommend following the instructions
  here first)
\item
  4 compute nodes
\end{itemize}

\subsection{Configuration}

\subsubsection{Flash Raspbian}

\begin{enumerate}
\def\labelenumi{\arabic{enumi}.}
\item
  Download Raspbian image \url{https://www.raspberrypi.org/downloads/}.
\item
  Download Etcher \url{https://etcher.io/}.
\item
  Using Etcher, flash Raspbian onto SD card.
\end{enumerate}

\subsubsection{Keyboard Layout}

The default keyboard layout may need to be changed to US.

Menu -\textgreater{} Preferences -\textgreater{} Mouse and Keyboard
Settings -\textgreater{} Keyboard tab -\textgreater{} Variant
-\textgreater{} English (US)

\subsubsection{Change password}

Change password:

\begin{lstlisting}
passwd
\end{lstlisting}

Enter new password via prompts.

\subsubsection{Change hostnames}

Change hostname of each raspberry pi (in descending order).

\begin{enumerate}
\def\labelenumi{\arabic{enumi}.}
\item
  rp0
\item
  rp1
\item
  rp2
\item
  rp3
\item
  rp4
\end{enumerate}

This can be done on the command line using:

\begin{lstlisting}
sudo raspi-config
\end{lstlisting}

Or on the desktop by going to Menu -\textgreater{} Preferences
-\textgreater{} Raspbery Pi Configuration

Or by modifying \textbf{/etc/hostname}

\subsubsection{Configure Head Node}

Install Dependencies:

\begin{lstlisting}
apt-get update
apt-get install -qy dnsmasq clusterssh iptables-persistent
\end{lstlisting}

\paragraph{Create Static IP}

Copy old config (-n flag prevents overwrite):

\begin{lstlisting}
\cp -n /etc/dhcpcd.conf /etc/dhcpcd.conf.old
\end{lstlisting}

To update DHCP configuration, add the following to
\textbf{/etc/dhcpd.conf}:

\begin{lstlisting}
interface wlan0
metric 200

interface eth0
metric 300
static ip_address=192.168.50.1/24
static routers=192.168.50.1
static domain_name_servers=192.168.50.1
\end{lstlisting}

\paragraph{Configure DHCP Server:}

Copy old config (-n flag prevents overwrite):

\begin{lstlisting}
\cp -n /etc/dnsmasq.conf /etc/dnsmasq.conf.old
\end{lstlisting}

To update DNS configuration, add the following to
\textbf{/etc/dhcpd.conf}

\begin{lstlisting}
interface=eth0
interface=wlan0

dhcp-range=eth0, 192.168.50.1, 192.168.50.250, 24h
\end{lstlisting}

\paragraph{NAT Forwarding}

To Setup NAT Forwarding, uncomment the following line in
\textbf{/etc/sysctl.conf}:

\begin{lstlisting}
net.ipv4.ip_forward=1
\end{lstlisting}

\paragraph{IP Tables}

Create IP Tables:

\begin{lstlisting}
sudo iptables -t nat -A POSTROUTING -o wlan0 -j MASQUERADE
sudo iptables -t nat -A POSTROUTING -o eth0 -j MASQUERADE
sudo iptables -A FORWARD -i $INTERNAL -o wlan0 -j ACCEPT
sudo iptables -A FORWARD -i $EXTERNAL -o eth0 -j ACCEPT
\end{lstlisting}

Make rules permanent:

\begin{lstlisting}
iptables-save > /etc/iptables/rules.v4
\end{lstlisting}

\subsubsection{SSH Configuration}

\textbf{Note: Gregor says this is not best practice}

Generate SSH keys:

\begin{lstlisting}
ssh-keygen -t rsa
\end{lstlisting}

Copy key to each compute node:

\begin{lstlisting}
ssh-copy-id <hostname>
\end{lstlisting}

For hostnames rp1-4 (final node names will be: rp0, rp1, rp2, rp3, rp4).

\subsubsection{Configure Cluster SSH}

To update Cluster SSH configuration, add the following to
\textbf{/etc/clusters}:

\begin{lstlisting}
rpcluster rp1 rp2 rp3 rp4
\end{lstlisting}

Now you can run commands to all clusters by:

\begin{lstlisting}
cssh rpcluster
\end{lstlisting}

\section{Intructions for installing kubernetes}

\subsection{First install doker, disable swap, install kubeadm}

All the following steps are made automatically by the
docker\_kubernetes\_install.sh script.

\subsubsection{Install docker}

In order to install kubernetes you first need to have docker installed.
This is very strait foward.

\subsubsection{Disable swap memory}

Docker has an issue (in my opinion severe) in that it is \textbf{not
compatible with SWAP memory}, therefore it is neeeded to disable it.
This might create some issues, if you encounter them you should try to
rebbot the cluster again, if that fails change line 16 in the script
from

\begin{lstlisting}
orig="$(head -n1 /boot/cmdline.txt) cgroup_enable=cpuset cgroup_memory=1"
\end{lstlisting}

to

\begin{lstlisting}
orig="$(head -n1 /boot/cmdline.txt) cgroup_enable=cpuset cgroup_memory=memory".
\end{lstlisting}

\subsubsection{Installing kubernetes administrator}

Finally, to configure kubernetes you'll need kubeadm. Now the Pi needs
to be rebooted.

\subsubsection{\texorpdfstring{\emph{All of this will be done by the script, don't worry} (maybe worry)}{All of this will be done by the script, don't worry (maybe worry)}}

\subsection{For the nodes}

All of the above needs to be done in each node aswell. The script
copy\_dk\_kub\_install\_script\_to\_nodes.sh should copy the needed
script to each of them and run it. It is set up to work with 4 nodes
named rp\textless{}number\textgreater{} with pi as the username (the
numbers start at 1 because the head node is rp0). Changing the number of
nodes is trivial, if all of your nodes have the same username it is also
trivial.

If your nodes are not configured like that you'll need to change this
script or copy docker\_kubernetes\_install.sh to each of the nodes
manually.

\subsection{Now some more explanations on the scripts}

First docker needs to be installed

\begin{lstlisting}
curl -sSL get.docker.com \
  | sh \
  && sudo usermod pi -aG docker
\end{lstlisting}

Now, as docker does not work with SWAP memory it needs to be disabled,
this is easy enough

\begin{lstlisting}
sudo swapoff -a 
echo Adding " cgroup_enable=cpuset cgroup_enable=1" to /boot/cmdline.txt
sudo cp /boot/cmdline.txt /boot/cmdline_backup.txt
\end{lstlisting}

If after running this script your raspberry pi starts to not work
properly reboot it once again, if thar does not solve the issue replace
cgroup\_memory=1 with cgroup\_enable=memory.

\begin{lstlisting}
orig="$(head -n1 /boot/cmdline.txt) cgroup_enable=cpuset cgroup_memory=1"
echo $orig | sudo tee /boot/cmdline.txt
\end{lstlisting}

Now kubernetes admin will be installed

\begin{lstlisting}
curl -s https://packages.cloud.google.com/apt/doc/apt-key.gpg | sudo apt-key add - &&\
  echo "deb http://apt.kubernetes.io/ kubernetes-xenial main" | \
  sudo tee /etc/apt/sources.list.d/kubernetes.list && \
  sudo apt-get update -q && \
  sudo apt-get install -qy kubeadm
sudo reboot 
\end{lstlisting}

\subsubsection{On the nodes}

Docker and kubernetes need to be installed on the nodes, as well as the
SWAP memory needs to be disabled. This is handled by another script that
simply copies the installation script to the nodes and runs it. First
copy the script

\begin{lstlisting}
for number in {1..4}
    do 
        scp /home/pi/docker_kubernites_install.sh \
            pi@rp$number:/home/pi/docker_kubernites_install.sh
    done
exit 0
\end{lstlisting}

Now the installation script will be run on the nodes using cssh

\begin{lstlisting}
cssh -a "sh docker_kubernites_install.sh"
\end{lstlisting}

